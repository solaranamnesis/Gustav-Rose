\documentclass[a4paper, 11pt, oneside]{article}
\usepackage[utf8]{inputenc}
\usepackage[T1]{fontenc}
\usepackage[ngerman]{babel}
\usepackage{yfonts}
%\usepackage{fbb} %Derived from Cardo, provides a Bembo-like font family in otf and pfb format plus LaTeX font support files
\usepackage{amssymb}
\usepackage{booktabs}
\setlength{\emergencystretch}{15pt}
\usepackage{fancyhdr}
\usepackage{graphicx}
\usepackage{microtype}
\graphicspath{ {./} }
\usepackage[titles]{tocloft}
\usepackage{sectsty}

\allsectionsfont{\frakfamily}
\sectionfont{\frakfamily\Huge}
\subsectionfont{\frakfamily\LARGE}
\subsubsectionfont{\frakfamily\LARGE}
\paragraphfont{\frakfamily\LARGE}

\begin{document}
\renewcommand{\contentsname}{
\frakfamily{Inhaltsverzeichnis}
}

\renewcommand{\cftsecfont}{\frakfamily}
\renewcommand{\cftsubsecfont}{\frakfamily}
\renewcommand{\cftsubsubsecfont}{\frakfamily}

% fix toc page numbers
\let\origcftsecfont\cft
\let\origcftsecpagefont\cftsecpagefont
\let\origcftsecafterpnum\cftsecafterpnum
\renewcommand{\cftsecpagefont}{\frakfamily{\origcftsecpagefont}}
\renewcommand{\cftsecafterpnum}{\frakfamily{\origcftsecafterpnum}}
\let\origcftsubsecpagefont\cftsubsecpagefont
\let\origcftsubsecafterpnum\cftsubsecafterpnum
\renewcommand{\cftsubsecpagefont}{\frakfamily{\origcftsubsecpagefont}}
\renewcommand{\cftsubsecafterpnum}{\frakfamily{\origcftsubsecafterpnum}}
\let\origcftsubsubsecpagefont\cftsubsubsecpagefont
\let\origcftsubsubsecafterpnum\cftsubsubsecafterpnum
\renewcommand{\cftsubsubsecpagefont}{\frakfamily{\origcftsubsubsecpagefont}}
\renewcommand{\cftsubsubsecafterpnum}{\frakfamily{\origcftsubsubsecafterpnum}}

\renewcommand\thefootnote{\frakfamily{\arabic{footnote}}}

\frakfamily
\begin{titlepage} % Suppresses headers and footers on the title page
	\centering % Centre everything on the title page
	%\scshape % Use small caps for all text on the title page

	%------------------------------------------------
	%	Title
	%------------------------------------------------
	
	\rule{\textwidth}{1.6pt}\vspace*{-\baselineskip}\vspace*{2pt} % Thick horizontal rule
	\rule{\textwidth}{0.4pt} % Thin horizontal rule
	
	\vspace{1.5\baselineskip} % Whitespace above the title
	
	{\scshape\Huge Beschreibung und Einteilung}
	
	\vspace{1\baselineskip} % Whitespace after the title block

	{\scshape\Huge der Meteoriten auf Grund der Sammlung}

	\vspace{1\baselineskip} % Whitespace after the title block

	{\scshape\Huge im Mineralogischen Museum zu Berlin.}

	\vspace{1.5\baselineskip} % Whitespace above the title

	\rule{\textwidth}{0.4pt}\vspace*{-\baselineskip}\vspace{3.2pt} % Thin horizontal rule
	\rule{\textwidth}{1.6pt} % Thick horizontal rule
	
	\vspace{1\baselineskip} % Whitespace after the title block
	
	%------------------------------------------------
	%	Subtitle
	%------------------------------------------------
	
	{\LARGE Von Gustav Rose.} % Subtitle or further description
	
	\vspace*{1\baselineskip} % Whitespace under the subtitle
	
    {Aus den Abhandlungen der K"onigliche Akademie\\ der Wissenschaften zu Berlin 1863.\\Mit vier Kupfertafeln.} % Subtitle or further description
    
	%------------------------------------------------
	%	Editor(s)
	%------------------------------------------------
    \vspace*{\fill}

	\vspace{1\baselineskip}

	{\small\scshape Berlin. 1864.}
	
	{\small\scshape{In Kommission bei F. D"ummlers Verlags-Buchhandlung\\Harrwitz und Gossmann.}}
	
	\vspace{0.5\baselineskip} % Whitespace after the title block

    \scshape Internet Archive Online Edition  % Publication year
	
	{\scshape\small Namensnennung Nicht-kommerziell Weitergabe unter gleichen Bedingungen 4.0 International} % Publisher
\end{titlepage}
\setlength{\parskip}{1mm plus1mm minus1mm}
\clearpage
\pagestyle{fancy}
\fancyhf{}
\cfoot{\frakfamily{\thepage}}
\frakfamily{
\tableofcontents
}
\clearpage
\LARGE
\section*{\frakfamily{Geschichte der Meteoritensammlung in dem mineralogischen Museum von Berlin auf Grund deren die neue Einteilung der Meteoriten gemacht ist.}}
\paragraph{}
Die Meteoritensammlung macht einen besonderen Teil des mineralogischen Museums der Berliner Universit"at aus. Als bei der Gr"undung der Universit"at im Jahre 1810 auch das mineralogische Museum durch "ubernahme der Mineraliensammlung der fr"uheren General-Bergbau-Direktion gegr"undet wurde, waren die wenigen Meteoriten, die sich in derselben befanden, noch nicht getrennt und mit den "ubrigen Mineralien vereinigt. Wie viele Meteoriten sich schon damals in ihr befanden, l"asst sich nicht angeben, da dar"uber die Nachweisungen fehlen, indessen enthielt sie doch schon manche kostbare St"ucke, wie ein gro"ses prachtvolles Exemplar von dem Pallas-Eisen, das in einer Sammlung Russischer Mineralien enthalten war, die Kaiser Alexander 1 dem K"onige Friedrich Wilhelm 3 im Jahre 1803 zum Geschenk gemacht hatte, so wie ein gro"ses St"uck von dem Durango-Eisen, welches Al. von Humboldt aus Mexiko mitgebracht und dem damaligen Direktor Dietrich Karsten f"ur die Sammlung "ubergeben hatte. Wei"s, der nach Karstens Tode Direktor des mineralogischen Museums wurde, hatte ein gro"ses Interesse f"ur die Meteoriten und lie"s keine Gelegenheit vor"ubergehen, die sich ihm zur Erwerbung von Meteoriten darbot, doch fand sich dieselbe im Anfang, wo das Interesse f"ur die Meteoriten "uberhaupt noch nicht so lebhaft war wie jetzt, nicht h"aufig. Die erste gr"o"sere Bereicherung erhielt das Museum erst durch den Ankauf der Mineraliensammlung von Klaproth nach dessen im Jahre 1817 erfolgten Tode, indem sich darin nach Weglassung aller Arten, die sich sp"ater als unecht erwiesen haben, Steinmeteorit von 12 Fund"ortern und Eisenmeteorit von 5 Fund"ortern befanden, und mehrere derselben in mehreren Exemplaren vertreten waren. In dem im Jahre 1826 vollendeten Kataloge des Museums sind Meteorit von 31 Fund"ortern aufgef"uhrt, und zwar Steinmeteorit von 21 und Eisenmeteorit von 9 Fund"ortern. Aber schon im n"achsten Jahre vermehrte sich die Sammlung um mehr als das Doppelte durch das Verm"achtnis Chladni's, wodurch die ganze ber"uhmte Meteoritensammlung dieses um die Meteoritenkunde so verdienten Gelehrten dem Berliner Museum zufiel.\footnote{\frakfamily{Chladni starb den 3. April 1827 auf einer Reise in Breslau. Er stand in den freundschaftlichsten Beziehungen zu dem damaligen Direktor des Berliner Museums, wie "uberhaupt zu den Berliner Gelehrten, und dies Verh"altnis hatte ihn bei dem Wunsche seine Sammlung gemeinn"utzig zu machen, den er in seine Testamente ausdr"ucklich ausgesprochen hatte, wohl besonders bewogen, seine Sammlung dem Berliner Museum zu vermachen.}} Sie bestand in Steinmeteoriten von 31 und Eisenmeteoriten von 10 verschiedenen Fund"ortern, unter denen 18 neue Meteorit sich befanden.

Durch den Ankauf der Sammlung des Medizinal-Raths Bergemann im Jahre 1837 erhielt das Museum einen Zuwachs an Steinmeteoriten von 9 und von Eisenmeteoriten von 4 Fund"ortern, doch waren darunter nur 2 neue Fund"orter. Die sp"ateren Erwerbungen geschahen nun nur durch Kauf, Tausch oder Schenkung einzelner Meteorit, wobei vor Allem das gro"se Verdienst hervorzuheben ist, welches sich Al. von Humboldt durch die Schenkung so vieler ausgezeichneter Meteorit um das Museum erworben hat. Bei dem Tode des Professor Wei"s im Jahre 1856 belief sich die Zahl der verschiedenen Meteoriten auf 90; sie ist seit dieser Zeit auf 176 gestiegen.\footnote{\frakfamily{Wobei die mir zweifelhaften Eisenmeteoriten von Scriba, Hemalga, Newstead, Livingstone und Melrose, die in den Katalogen von der Wiener, G"ottinger und Londoner Sammlung aufgef"uhrt werden, nicht gerechnet sind.}} Einen gro"sen Zuwachs erhielt sie noch in der neuesten Zeit\footnote{\frakfamily{Noch vor der Lesung des dritten Teils dieser Abhandlung.}} durch den Ankauf einer ganzen Meteoritensammlung vom Prof. Shepard in New Haven in den Vereinigten Staaten, zu welchem die Akademie auf das liberalste die Mittel bewilligte.\footnote{\frakfamily{Vergl. die Monatsberichte der Akademie von 1862, S. 644.}} Die Sammlung stammte zum Teil aus der gro"sen Meteoritensammlung des Prof. Lawrence Smith in Louisville, V. St., und enthielt neben vielem Neuen einzelne St"ucke von bedeutender Gr"o"se wie ein fast vollst"andiges Exemplar von dem M. von New Concord von 26 Pfund und 24,3 Loth und eine gro"se Platte von dem Toluca-Eisen mit vielen Einschl"ussen, die "uber einen Fu"s lang und einen halben Fu"s breit ist.

Von vielen Seiten aufgefordert, ein Verzeichnis der in dem Berliner mineralogischen Museum befindlichen Meteoriten bekannt zu machen, schien es mir zweckm"a"sig in diesem Verzeichnis nicht, wie man gew"ohnlich zu tun pflegt, die Eisen- und Steinmeteorit in der zuf"alligen Ordnung ihrer Fall- oder Fundzeit aufzuf"uhren, sondern das Gleichartige zusammenstellend, sie nach ihrer mineralogischen Beschaffenheit zu ordnen. Ich fand zu einer solchen Anordnung umso mehr Veranlassung, als ich beabsichtigte, mit einem solchen Verzeichnis eine neue Aufstellung der Meteoriten des Berliner Museums vorzunehmen. Ich habe deshalb s"amtliche Meteoriten des Museums genau untersucht und dazu s"amtliche Stein- und Eisenmeteorit anschleifen lassen, und letztere ge"atzt, da man nur auf diese Weise bei den ersteren einen "uberblick "uber die Gemengteil erhalten, bei den letzteren die Struktur erkennen kann, eine Arbeit, die lange aufhielt. Au"serdem hatte ich von einem gro"sen Teil der ersteren d"unne Platten schleifen lassen, und von den ge"atzten Fl"achen der letzteren Hausenblasenabdr"ucke gemacht, um sie unter dem Mikroskop zu beobachten, und bin nun dadurch zu den Resultaten gelangt, die ich mir erlaube, hiermit der Akademie vorzulegen.

Systematische Anordnungen der Meteoriten sind schon von Partsch,\footnote{\frakfamily{Die Meteoriten oder die vom Himmel gefallenen Steine und Eisenmassen im k. k. Hof-Mineralien-Kabinette zu Wien 1843, S. 162.}} Shepard\footnote{\frakfamily{\emph{Report on American meteorites} (from the Amer. Journ. of Science and arts, 2. Ser). New Haven 1848 p. 16.}} und in der neusten Zeit von Reichenbach\footnote{\frakfamily{Anordnung und Einteilung der Meteoriten in Pongendorffs Annalen 1859, B. 107, S. 155.}} versucht worden. Partsch teilt die Meteoriten ein zuerst in Stein- und Eisenmeteorit, und letztere in normale und anomale; eine Einteilung und Benennung, die Reichenbach sehr tadelt, da man bei Meteoriten von normal und anomal nicht reden k"onne. Vergleicht man aber die Reihung selbst, so ist diese sehr naturgem"a"s. Zu den anomalen rechnet er nur 4, die von Alais, Capland, Chassigny und Simonod, von denen die drei ersteren allerdings auch von besonderer Art sind. Den von Simonod kenne ich nicht, er wird von Reichenbach als Meteorit ganz verworfen.\footnote{\frakfamily{A. a. O. S. 163.}} Die normalen werden in 2 Abteilungen geteilt, in solche die kein metallisches Eisen enthalten, wie a) die von Juvenas, Stannern, Konstantinopel, Jonzac, die nach der damaligen Annahme aus Augit und Labrador bestehen und b) die von Bialystock, Loutolax, Nobleborough und M"assing, die au"serdem noch Olivin enthalten und ein breccienartiges Ansehen haben, worauf dann die gro"se Schaar derer folgt, die Eisen eingesprengt enthalten. H"atte Partsch die ganze erstere Abteilung noch zu den anomalen gerechnet, so w"are die Einteilung noch passender gewesen, und h"atte er die Meteoriten der ersteren Abteilung ungew"ohnliche, die der letzteren gew"ohnliche genannt, so w"urden diese Ausdr"ucke Reichenbach vielleicht weniger Gelegenheit zum Ansto"s gegeben haben. Aber die Einteilung ist doch immer noch nicht n"aher gerechtfertigt und zu unbestimmt.

Das Meteoreisen teilt Partsch in "astiges und derbiges; das erstere erh"alt durch eingemengten Olivin eine schwammige Gestalt, das letztere hat eine unbestimmte Form und geringe Beimengungen. Zu den ersteren geh"oren die Meteoriten von Atacama, Krasnojarsk (das Pallas-Eisen) und von Brahin, zu den letzteren alle "ubrigen, die nun noch weiter, je nachdem sie durch "atzung mehr oder weniger deutliche oder auch gar keine Widmanst"attenschen Figuren geben, eingeteilt werden.

Die Einteilung von Shepard bezieht sich zwar haupts"achlich nur auf den amerikanischen Meteoriten, nimmt aber doch auch auf einige ausl"andische R"ucksicht. Er teilt den Meteoriten ebenfalls zuerst ein in Eisen- und Steinmeteorit und beide dann weiter wie folgt:

\begin{center}
I. Klasse: Eisenmeteorit.
\end{center}
\begin{enumerate}
  \item Ordnung: Dehnbare und gleichartige.
  \begin{enumerate}
    \item Sekt.: Reine (Scriba, Walker County).
    \item Sekt.: Legierte.
    \begin{enumerate}
      \item Feink"ornige (Green County, Texas, Dickson County, Burlington).
      \item Grobk"ornige (De Kalb, Ashville, Guildford, Carthago).
    \end{enumerate}
  \end{enumerate}
  \item Ordnung: Dehnbare und ungleichartige.
  \begin{enumerate}
    \item Sekt.: Blasig-olivinige (Krasnojarsk).
    \item Sekt.: Blasig-pyritische (Cambria).
    \item Sekt.: Pyritisch-graphitische (Cocke County).
  \end{enumerate}
  \item Ordnung: Spr"ode.
  \begin{enumerate}
    \item Sekt.: Reine (Redford County, Randolph County).
    \item Sekt.: Legierte (Otsego County).
  \end{enumerate}
\end{enumerate}

\begin{center}
2. Klasse: Steinmeteorit.
\end{center}
\begin{enumerate}
  \item Ordnung: Trachytische.
  \begin{enumerate}
    \item Sekt.: Olivinige.
    \begin{enumerate}
      \item Grobk"ornige (Weston, Richmond).
      \item Feink"ornige (Nobleboro, Little Piney).
    \end{enumerate}
    \item Sekt.: Augitisch (Juvenas).
    \item Sekt.: Chadnitisch (Bishopville).
    \item Sekt.: Kohligen (Cold Bokkeveld).
  \end{enumerate}
  \item Ordnung: Trappartige.
  \begin{enumerate}
    \item Sekt.: Gleichartige (Chantonnay).
    \item Sekt.: Porphyrartige (Renazzo).
  \end{enumerate}
  \item Ordnung: Bimmsteinartige (Waterville).\footnote{\frakfamily{Nicht meteorisch.}}
\end{enumerate}
\paragraph{}
Reichenbach gibt wohl den, meiner Meinung nach einzig richtigem Weg zur Einteilung der Meteoriten an, befolgt ihn aber selbst nicht. Er sagt, es w"are am nat"urlichsten, der Meteorit nach der Verschiedenheit der Mineralspezies, die sie enthalten, einzuteilen, da wir diese aber noch zu wenig kennen, so m"ussen wir es so machen wie die Botaniker bei den nat"urlichen Systemen der Pflanzen und die Meteoriten nach der allgemeinen "ahnlichkeit reihen. Er teilt demnach dieselben in 9 Sippen und jede wieder in verschiedene Gruppen, indem er mit den eisenfreien Steinmeteoriten von Langres (Chassigny), Bishopville, Jonzac, Juvenas, Stannern, Konstantinopel anf"angt, durch die eisenhaltigen zu den Eisenmeteoriten fortgeht, die noch Olivin enthalten, und mit den aus fast reinem Eisen bestehenden Meteoriten schlie"st. Es kann nicht fehlen, dass ein so scharfblickender Kenner der Meteoriten, wie Baron Reichenbach, nicht eine Menge neuer interessanter Zusammenstellungen macht und Beziehungen zwischen Meteorsteinen hervorhebt, die fr"uher nicht beachtet waren; sein System ist aber doch nur, wie er selbst das Verfahren der Botaniker nennt, ein „geistreiches Tatonnement“, es ist ihm derselbe Vorwurf zu machen, den er dem Partschen Systeme macht, es fehlt ihm ein Einteilungsprinzip; wir scheinen doch hinreichend in der Kenntnis der Meteoriten vorger"uckt, um vollst"andig die strengen Grunds"atze in Anwendung bringen zu k"onnen, die uns bei der Einteilung der Gebirgsarten, mit denen die Meteoriten doch vollst"andig zu vergleichen sind, leiten. Wie man dort aus einem jeden selbstst"andigen Gemenge eine besondere Gebirgsart macht, so muss man es auch hier tun, und wenn man allerdings auch noch nicht vollst"andig alle Gemengteil der Meteoriten genau kennt, so wei"s man davon doch so viel, um das Zusammengeh"orige zusammenstellen zu k"onnen.

Ich behalte zuerst die alte Einteilung in Eisen- und Steinmeteorit bei, je nachdem der Meteorit nur oder vorzugsweise aus Eisen, und zwar Nickeleisen, oder vorzugsweise aus einem Gemenge von Silicaten bestehen, in denen das Nickeleisen nur untergeordnet oder gar nicht enthalten ist.

1. Die Eisenmeteorit machen 3 Arten aus, Meteoritenarten kann man sie nennen, wie man in der Petrographie Gebirgsarten oder Felsarten sagt.

Die 1. Art besteht aus Nickeleisen, das nur in geringer Menge mit einigen Eisenverbindungen gemengt ist; ich nenne sie Meteoreisen.

Die 2. Art besteht aus demselben Meteoreisen, worin Kristalle von Olivin porphyrartig eingewachsen sind. Der von Pallas am Jenisei gefundene Eisenmeteorit war der erste der Art, den man kennen lernte; er ist bekannt unter dem Namen Pallas-Eisen und bildet noch immer einen Hauptrepr"asentanten dieser Art; ich schlage daher vor, die ganze Art Pallasit zu nennen.

Die 3. Art ist ein k"orniges Gemenge von Meteoreisen und Magnetkies mit Olivin und Augit. Ich nenne sie Mesosiderit von $\mu\varepsilon\sigma$o$\varsigma$ in der Mitte stehend und $\sigma\iota\delta\eta\varrho$o$\varsigma$ Eisen, da sie aus einer ziemlich gleichen Menge von metallischen Eisenverbindungen und Silicaten besteht und so gewisserma"sen in der Mitte zwischen den Eisen- und Steinmeteoriten steht.

2. Die Steinmeteorit sind in 7 Arten zu teilen, f"ur die ich die folgenden Namen vorschlage:

1. Chondrit (von $\chi$o$\nu\delta\varrho$o$\varsigma$, die kleine Kugel). Sie ist die erste und haupts"achlichste Art, die den gr"o"sten Teil der Steinmeteorit enth"alt. Sie ist durch kleine Kugeln ausgezeichnet, die aus einem noch nicht bestimmten Magnesia-Silicat bestehen und in einem feink"ornigen Gemenge eingemengt sind, das aus Olivin, Chromeisenerz, einer schwarzen noch zu bestimmenden Substanz, sowie aus Nickeleisen und Magnetkies besteht.

2. Howardit zu Ehren Howards benannt, dem wir die erste Analyse eines Meteorsteins verdanken; ein feink"orniges Gemenge von Olivin mit einem wei"sen Silicat, m"oglicher Weise Anorthit, und mit einer geringen Menge von Chromeisenerz und Nickeleisen. Sie enth"alt die Meteorsteine von Loutolax, Bialystock und M"assing u. s. w.

3. Chassignit, von Chassigny, dem Fallorte des einzigen bekannten Meteoriten dieser Art; ein kleink"orniger eisenreicher Olivin mit sparsam eingemengten kleinen K"ornern von Chromeisenerz.

4. Chladnit nach Chladni benannt; ein Gemenge von Shepardit (Mg$_{2}$Si$_{3}$) mit einem noch n"aher zu bestimmenden Tonerde-haltigen Silicate mit geringen Mengen von Nickeleisen, Magnetkies und einigen anderen noch zu bestimmenden Substanzen. Hierher geh"ort auch nur ein Meteorit, der von Bishopville.\footnote{\frakfamily{Mit Chladnit hat zwar Shepard, der diesen Meteorit zuerst untersucht und beschrieben hat, schon das in ihm vorkommende Magnesia-Silicat bezeichnet, doch schien es mir zweckm"a"siger, nach Chladni, der sich um die Meteoritenkunde so viele Verdienste erworben hat, einen Meteoriten, als ein Mineral zu benennen, wenn sich dieses auch bis jetzt nur in einem Meteoriten gefunden hat. Ich m"ochte dann weiter vorschlagen, den bisherigen Chladnit: Shepardit zu nennen, der zwar auch schon einer in diesen Meteoriten sparsam vorkommenden Substanz gegeben ist, die nach Shepard Schwefelchrom ist, die indessen doch in ihren Eigenschaften noch erst sehr wenig gekannt ist; Vorschl"age, mit denen Hr. Shepard selbst sich einverstanden gegen mich erkl"art hat.}}

5. Shalkit, der Meteorstein von Shalka, ein k"orniges Gemenge von vorwaltendem Olivin mit Shepardit und Chromeisenerz.

6. Die kohligen Meteorit, wie von Bokkeveld und Alais, die ich nicht genauer untersucht habe und f"ur die ich daher noch einen eigenen Nannen aus Setze.

7. Eukrit von $\varepsilon\nu\kappa\varrho\iota\tau$o$\varsigma$ wohl bestimmbar, da die mineralogische Beschaffenheit dieser Art bis auf einige Nebendinge ganz klar ist, und ihre wesentlichen Gemengteil vollkommen bestimmbar sind. Ein haupts"achlich aus Augit und Anorthit bestehendes k"orniges Gemenge mit einer geringen Menge Magnetkies, meistens noch geringerer Menge Nickeleisen, zuweilen mit kleinen, n"aher zu bestimmenden tafelartigen Kristallen (Juvenas) und mit etwas Olivin (Petersburg, V. St.). Es geh"oren hierhin die Meteoriten von Juvenas, Stannern, Jonzac und Petersburg.
\clearpage
\section{\frakfamily{Eisenmeteorit.}}
\paragraph{}
Die ersten bestimmten Angaben "uber die Natur des Meteoreisens haben wir von Howard im Jahre 1802 erhalten, der bei Gelegenheit der chemischen Analyse des 1798 bei Benares in Bengalen gefallenen Meteorsteins die merkw"urdige Entdeckung machte, dass das in demselben eingesprengte Eisen Nickel enthalte.\footnote{\frakfamily{Philosophical transactions von 1802 und daraus in Gilberts Annalen.}} Er fand darin 35 pC. und einen "ahnlichen Gehalt in andern gediegenen Eisenmassen, die f"ur meteorisch angesehen wurden, in dem Eisen von Otumpa in Brasilien, von Krasnojarsk in Sibirien (dem Pallas-Eisen), dem Eisen aus B"ohmen und vom Senegal.

Klaproth best"atigte sp"ater den Nickelgehalt bei der Untersuchung des Meteoreisens von Agram und Ellbogen, wenngleich er die Menge darin weit geringer fand (nur 3,5 und 2,5 pC.), und man war nun seit der Zeit gewohnt, den Nickelgehalt als ein charakteristisches Kennzeichen des Meteoreisens zu betrachten, und bei zuf"allig auf der Oberfl"ache der Erde gefundenen Eisenmassen ihren meteorischen Ursprung erst dann anzunehmen, wenn die chemische Untersuchung einen Gehalt an Nickel nachgewiesen hatte, eine Annahme, wozu man auch jetzt noch berechtigt ist, da man noch nie ein Meteoreisen ohne Nickelgehalt gefunden hat. Tellurisches gediegenes Eisen enth"alt keinen Nickel, ist "uberhaupt eine au"serordentliche Seltenheit und wohl zuf"allig nur durch einen Reduktionsprozess entstanden.

Sp"ater fand Stromeyer in dem Meteoreisen neben dem Nickel etwas Kobalt, das ja auch in den tellurischen Mineralien so h"aufig das Nickel zu begleiten pflegt und darauf auch etwas Kupfer, und Laugier in dem Meteoreisen von Krasnojarsk und Brahin etwas Chrom, 0,5 pC., das aber wie das in den Meteorsteinen vorkommende, wo es Laugier schon fr"uher gefunden hatte, von eingemengtem Chromeisenerz herr"uhrt.

Am meisten bereichert wurde unsere Kenntnis von der chemischen Beschaffenheit des Meteoreisens durch die genauen Analysen derselben von Berzelius, die er zuerst auf Veranlassung des Grafen Caspar Sternberg mit dem in B"ohmen aufgefundenen Eisen von Bohumilitz\footnote{\frakfamily{Pongendorffs Annalen von 1833 B. 27, S. 118.}} und dann bei seiner gro"sen Arbeit "uber die Meteoriten "uberhaupt, die durch die "ubersendung der 1833 bei Blansko in M"ahren gefallenen Meteorsteine durch Baron von Reichenbach veranlasst wurde, mit dem Meteoreisen von Krasnojarsk (dem Pallas-Eisen) und dem von Elbogen\footnote{\frakfamily{A. a. O. von 1834 B. 33, S. 123.}} angestellt hatte. Er schied durch Behandlung des Meteoreisens mit verd"unnter Salpeters"aure einen darin l"oslichen und einen anderen darin unl"oslichen Teil, und fand auf diese Weise bei dem Eisen von Bohumilitz (a), Krasnojarsk (b), und Elbogen (c):
\begin{center}
\begin{tabular}{ |l|r|r|r| }
    \hline
     & \emph{a} & \emph{b} & \emph{c}\\
    \hline\hline
    Eisen & 93,775 & 88,042 & 88,231\\\hline
    Nickel & 3,812 & 10,732 & 7,517\\\hline
    Kobalt & 0,213 & 0,455 & 0,762\\\hline
    Magnesium & -,- & 0,050 & 0,279\\\hline
    Mangan & -,- & 0,132 & Spur\\\hline
    Zinn und Kupfer & -,- & 0,066 & Spur\\\hline
    Kohle & -,- & 0,043 & -,-\\\hline
    Schwefel & -,- & Spur & Spur\\\hline
    Unl"osliches & 2,200 & 0,480 & 2,211\\\hline
    & 100,000 & 100,000 & 100,000\\
    \hline
\end{tabular}
\end{center}
\paragraph{}
Der in verd"unnter Salpeters"aure unl"osliche R"uckstand bestand aus metallischen K"ornern und Sch"uppchen, die schwer zu Boden liegen, und aus einer feinen verteilten schwarzen kohle"ahnlichen Masse, die sich leicht in der Fl"ussigkeit aufschl"ammen l"asst. Die ersteren waren merkw"urdiger Weise Phosphormetalle und bestanden aus:
\begin{center}
\begin{tabular}{ l r r r }
    Eisen & 65,987 & 48,67 & 68,11\\
    Nickel & 15,008 & 18,33 & 17,72\\
    Magnesium & -,- & 9,66 & 17,72\\
    Phosphor & 14,023 & 18,47 & 14,17\\
    Kiesel & 2,037 & -,- & -,-\\
    Kohle & 1,422 & -,- & -,-\\
     & 98,477 & 95,13 & 100,00\\
\end{tabular}
\end{center}
\paragraph{}
Die letztere, die beim Erhitzen rauchte und sodann verglimmt, wurde nur bei dem Pallas-Eisen quantitativ untersucht und bestand hier aus:
\begin{center}
\begin{tabular}{ l r }
    Eisen & 57,18\\
    Nickel & 34,00\\
    Magnesium & 4,52\\
    Zinn und Kupfer & 3,75\\
    Kohle & 0,55\\
\end{tabular}
\end{center}
\paragraph{}
Eine Spur von Phosphor, die Berzelius fand, glaubt er umschlossenen Teilen der Phosphorverbindung zu schreiben zu m"ussen. Bei dem R"uckstand aus dem Bohumilitz-Eisen wurden auch noch etwas Kiesel und Chromeisenerz gefunden. Dieser feinere Teil des R"uckstands ist daher von dem schwereren wesentlich verschieden zusammengesetzt.

Durch Berzelius wurden also in dem Meteoreisen 6 neue Stoffe aufgefunden: Phosphor, Zinn, Mangan, Magnesium, Kiesel und Kohle, von denen der Phosphor ganz besonders bemerkenswert ist, da solche Phosphormetalle, wie sie in dem Meteoreisen hiernach enthalten sind, unter den tellurischen Mineralien nicht bekannt sind.

Nach Berzelius wurden nun noch Analysen von anderen Meteoreisenmassen und von andern Chemikern nach denselben oder "ahnlichen Methoden gemacht, die aber ganz "ahnliche Resultate gegeben haben.\footnote{\frakfamily{Vergl. die Aufz"ahlung derselben in Rammelsbergs Mineralchemie, S. 902.}} In allen wurde ein in verd"unnter S"aure unl"oslicher, haupts"achlich aus Phosphornickeleisen bestehender R"uckstand erhalten, derselbe war wie bei Berzelius stets nur in sehr geringer und in sehr ver"anderlicher Menge enthalten, und au"serdem waren die Verh"altnisse von Phosphor gegen Eisen und Nickel so verschieden, dass sich eine gemeinschaftliche Formel f"ur die chemische Zusammensetzung dieser Verbindung nicht aufstellen l"asst.\footnote{\frakfamily{Es geht dies aus den Berechnungen von Rammelsberg hervor, wonach bei den verschiedenen Analysen auf 1 Atom Phosphor 2, 3 1/2, 5, 6, 8, 14, 15, 18, 30 Atome Metall kommen. (A. a. O. S. 948) Es scheint aber nicht, dass man die beiden Arten des R"uckstandes, die Berzelius wohl unterschieden, getrennt hat.}} Diesem unl"oslichen R"uckstand hat Haidinger bei Gelegenheit der von Patera ausgef"uhrten Analyse des Meteoreisens von Arva, worin derselbe in verh"altnism"a"sig gro"ser Menge enthalten ist, den Namen Schreibersit gegeben\footnote{\frakfamily{"osterreich. Bl"atter f"ur Lit. 1847, N. 175, S. 644 und N. Jahrbuch f"ur Min. von 1848, S. 698.}} zu Ehren des fr"uheren Direktors des kaiserlichen Mineralienkabinetts in Wien, der sich um die Meteoritenkunde durch die Herausgabe seiner Beitr"age zur Geschichte und Kenntnis meteorischer Stein- und Metallmassen verdient gemacht hat.

Aber schon viel fr"uher, als Berzelius durch seine chemischen Untersuchungen die gemengte Beschaffenheit des Meteoreisens dartat, hatte sie von Widmanst"atten in Wien auf eine andere Weise bewiesen. Derselbe zeigte n"amlich schon 1808, dass wenn man an dem Meteoreisen angeschliffene und polierte Fl"achen mit einer schwachen S"aure "atzt, gewisse Figuren hervortreten, die man seitdem die Widmanst"attenschen Figuren genannt hat. Die Fl"ache, die vor dem "atzen ganz gleichartig aussieht, oder nur bei h"ochster Politur und nach dem Anhauchen schwache Andeutungen der Figuren gibt, erscheint nun "uberall mit schmalen, glanzlosen, unter einander parallelen Streifen bedeckt, die nach verschiedenen Richtungen gehend, sich unter verschiedenen schiefen Winkeln durchneiden, von d"unnen, hervortretenden, metallisch gl"anzenden Leisten eingefasst werden und dunklere matte Felder einschlie"sen, was alles eine sehr komplizierte Struktur des Meteoreisens anzeigt. Besonders sch"on fielen diese Figuren auf dem gro"sen St"ucke aus, das 1812 von der Elbogener Eisenmasse abgeschnitten und nach Wien gebracht war. Da die schmalen einfassenden Leisten, die wenig oder gar nicht von der verd"unnten S"aure angegriffen werden, bei der "atzung aus der "ubrigen Masse etwas hervortreten, so kam Widmanst"atten auf die Idee, die ge"atzten Eisenmassen wie einen Schriftsatz in der Buchdruckerpresse abdrucken zu lassen, was auch vollkommen gelang. Er konnte dadurch vollkommen naturgetreue Abbildungen liefern, wie sie die Kunst nicht darzustellen vermag. V. Schreibers beschrieb in dem eben genannten Werke\footnote{\frakfamily{Beitr"age etc. S. 70, vergl. auch Partsch Meteoriten, S. 100.}} die einzelnen Teile des Meteoreisens, die Streifen, Einfassungsleisten und Zwischenfelder, und gab auch einen Abdruck von der ge"atzten Fl"ache der gro"sen Elbogener Masse des Wiener Mineralien-Kabinetts; die tieferen Stellen sind darin wei"s und nur die h"oheren Stellen, die Leisten, schwarz, zum Teil auch die Zwischenfelder, die oft wieder gestreift erscheinen. Auch von andern Eisenmassen lie"s er Abdr"ucke machen, die sp"ater herausgegeben werden sollten, wozu es aber nicht gekommen ist, und die dann nur an einzelne Personen verteilt wurden. Nach v. Schreibers wurden dergleichen Abdr"ucke nun auch von andern hergestellt, von Partsch in seine Werke die Meteoriten, von Haidinger in den Sitzungsberichten der Wiener Akademie, von mir selbst in Pongendorffs Annalen u. s. w. Au"serordentlich sch"on sind die Abdr"ucke in den neusten Abhandlungen von Haidinger, die die Figuren des Meteoreisens von Sarepta und Arva darstellen.\footnote{\frakfamily{Sitzungsberichte der kaiserl. Akad. d. Wiss. vom 24. Juli 1862.}}

Indessen geben nicht alle Eisenmeteorit Widmanst"attenschen Figuren, und namentlich ist dies der Fall bei der im Jahre 1847 bei Braunau gefallenen Eisenmasse, bei welcher Haidinger\footnote{\frakfamily{Berichte der Versammlungen der Freunde der Naturwissenschaften in Wien, 1847 und daraus in Pongendorffs Ann. 1847 B. 72, S. 580.}} die merkw"urdige Entdeckung machte, dass sie in ihrer ganzen Masse nach denselben drei untereinander rechtwinkligen Richtungen parallel den Fl"achen des Hexaeders spaltbar sei. Es waren zwei Massen gefallen, die beide in die H"ande des Pr"alaten vom Kloster zu Braunau Hrn. Rotter gelangten, der die gr"o"sere, 42 Pfd. 6 Lth. schwere Masse zerschneiden lie"s und einzelne St"ucke davon den verschiedenen Museen als Geschenk "ubersandte.\footnote{\frakfamily{Auch das Berliner Museum erhielt auf diese Weise ein ausgezeichnet sch"ones St"uck, 2 Pfund 21,3 Loth schwer.}} An dem St"ucke, welches das k. Mineralien-Kabinett in Wien erhielt, machte Haidinger die obige Beobachtung. Es war wie die "ubrigen zum Teil durchschnitten und die weitere Trennung durch Zerrei"sung hervorgebracht, so dass also stellenweise der nat"urliche Bruch sichtbar war. Da die Spaltungsfl"achen auf der ganzen Bruchfl"ache und demnach auch wahrscheinlich durch das ganze St"uck in gleicher Richtung fortgehen, so ist das ganze St"uck und so auch die ganze Masse, von der es abgeschnitten, ein St"uck eines Individuums, eines Kristalls, dessen "au"sere Form nicht mehr wahrgenommen werden kann, weil er beim Durchzuge durch die Luft zerplatzt und die einzelnen St"ucke an der Oberfl"ache abgeschmolzen sind, dessen innere Struktur in den St"ucken aber erhalten ist. Haidinger lie"s das erhaltene St"uck anschleifen und "atzen; es entstanden nun keine Widmanst"attenschen Figuren, aber andere gerade und untereinander parallele Linien nach mehreren Richtungen wurden sichtbar, die nachher von Neumann\footnote{\frakfamily{Naturwissenschaftliche Abhandl. gesammelt von Haidinger 1849 B. 3, Abt. 2, S. 45.}} ihrer Richtung nach sorgf"altig beschrieben und gedeutet wurden, worauf ich sp"ater zur"uckkommen werde.

Sehr wichtige und interessante Untersuchungen "uber die Struktur des Meteoreisens hat nun in der letzten Zeit der Baron von Reichenbach,\footnote{\frakfamily{Pongendorffs Ann. 1861 B. 114, S. 99, 250, 264,477.}} gemacht. Er unterscheidet bei den Eisenmeteoriten, die Widmanst"attenschen Figuren geben, vier Gemengteil, die durch die "atzung einer angeschliffenen Fl"ache sichtbar werden und die er mit dem Namen Balkeneisen oder Kamazit, Bandeisen oder T"anit, F"ulleisen oder Plessit und Glanzeisen oder Lamprit bezeichnet. Das Balkeneisen bildet auf der ge"atzten Fl"ache die unter einander parallelen Streifen, die sich unter schiefen Winkeln (von 30, 60 und 120 Graden) durchschneiden und nimmt somit den gr"o"sten Raum ein; es wird durch "atzung grau und glanzlos und zeigt sich nun mit einer Menge unter einander paralleler Linien nach Art des Braunauer Eisens bedeckt, die Reichenbach Schraffirungslinien nennt und f"ur Andeutungen von Spaltungsfl"achen h"alt; in vielen F"allen erscheint es aber selbst wieder k"ornig, wie namentlich in dem Eisen von Ruffs mountain. Das Bandeisen fasst die Streifen des Balkeneisens ein und bedeckt sie in papierd"unnen Bl"attern zu beiden Seiten; es wird von der verd"unnten S"aure schwach r"otlichgelb gef"arbt, sonst wenig oder gar nicht angegriffen, und ragt daher auf der ge"atzten Fl"ache "uber dem Balkeneisen leistenartig etwas hervor. Das F"ulleisen erf"ullt die drei oder vierseitigen Felder, die von dem Balkeneisen eingeschlossen werden; es wird von der "atzung wie das Balkeneisen angegriffen und erh"alt dabei eine noch dunkle graue Farbe, wie dieses. Es ist in manchen Ab"anderungen wie in dem Eisen von Ruffs mountain gar nicht vorhanden, f"ullt auch h"aufig die Felder nicht allein aus, sondern enth"alt oft noch eine gro"se Menge Bl"attchen von Bandeisen, die in untereinander paralleler Richtung enge nebeneinander und bei vierseitigen Feldern gew"ohnlich zwei parallelen Seiten, oft aber auch zum Teil den beiden andern parallel liegen, in welchem letzteren Fall die Bl"atter in einer Diagonale des Vierecks aneinandergrenzen. Reichenbach nennt diese die Zwischenfelder ausf"ullenden Bl"atter des Bandeisens K"amme. Das Glanzeisen liegt in einzelnen l"anglichen K"ornern und Streifen in der Mitte des Balkeneisens; es wird durch die verd"unnte S"aure gar nicht angegriffen und beh"alt den vollen Glanz und die lichte, stahlgraue fast zinnwei"se Farbe, die es durch die Politur der Fl"ache erhalten hat. Es findet sich nicht in allen Eisenmeteoriten, sehr ausgezeichnet in dem von Lenarto und Arva.

Reichenbach pr"ufte diese 4 Eisenarten noch weiter nach einer Methode, die schon Widmanst"atten angewandt hatte, durch das Anlaufen in der Hitze. Er zeigte, dass das Balkeneisen zuerst anl"auft, dann das F"ulleisen und zuletzt das Band- und Glanzeisen. Da nun auch das erstere von der S"aure am leichtesten, die letzteren am schwersten angegriffen werden, so sieht man, dass die Wirkungen der Hitze und der S"aure gleichen Schritt halten, wie denn auch beide Erscheinungen auf st"arkerer oder schw"acherer Verwandtschaft zum Sauerstoff beruhen. Bei einer Hitze, bei welcher das Balkeneisen schon dunkelblau geworden ist, erscheint das F"ulleisenbl"aulichrot und das Bandeisen goldgelb. Stahl l"auft aber bekanntlich bei 230$^{\circ}$ C. gelb, bei 263$^{\circ}$ purpurrot, bei 290$^{\circ}$ blau an. Die Hitze also, die das Balkeneisen schon blau macht, f"arbt erst das F"ulleisen purpurrot und das Bandeisen goldgelb.

Eine vollst"andige Trennung s"amtlicher Gemengteil f"ur die chemische Untersuchung konnte Reichenbach nicht bewerkstelligen, doch gl"uckte es ihm wenigstens einigerma"sen f"ur einen derselben, f"ur das Bandeisen. Manche dieser Eisenmeteoriten, wie namentlich der von Cosby Creek, die vor ihrer Auffindung vielleicht lange Zeit in der feuchten Erde gelegen haben, sind n"amlich an der Oberfl"ache sehr stark oxydiert und zerteilen sich hier parallel den Bl"attern des Bandeisens in Platten, welche Zerteilung durch leises H"ammern noch vollst"andiger bewirkt werden kann. Die oxydierten Platten des Balkeneisens sind aber hier mit papierd"unnen Bl"attern des Bandeisens bedeckt, die sich nun mit Leichtigkeit von dem Balkeneisen abl"osen und so in hinreichender Menge zur Analyse gewinnen lassen.\footnote{\frakfamily{Diese Bl"attchen von Bandeisen k"onnen so zuweilen von bedeutender Gr"o"se erhalten werden; so beschreibt Reichenbach ein St"uck von dem Cosby-Eisen in seiner Sammlung das mit einem Blatte Bandeisen bedeckt ist, das eine L"ange von 3 Zoll bei einer Breite von 2 Zoll hat.}} Reichenbach verfuhr so mit dem Eisen von Cosby; das gesammelte Bandeisen untersuchte er zuerst in R"ucksicht des spezifischen Gewichtes, er fand dasselbe 7,428, etwas gr"o"ser als das spezifische Gewicht der ganzen Masse, das 7,260 betr"agt, es wurde sodann von seinem Sohne Reinold v. Reichenbach analysiert, der zugleich auch eine Analyse der ganzen Masse machte. Er fand\footnote{\frakfamily{Vergl. Pongendorffs Ann. 1861 B. 114, S. 258.}} in dem Bandeisen (a) und in der ganzen Masse nach 2 Analysen (b) und (c):
\begin{center}
\begin{tabular}{ |l|r|r|r| }
    \hline
     & \emph{a} & \emph{b} & \emph{c}\\
    \hline\hline
    Eisen & 85,714 & 90,125 & 89,324\\\hline
    Nickel & 13,215 & 9,786 & 10,123\\\hline
    Kobalt & 0,550 & 9,786 & 0,422\\\hline
    Schwefel & 0,226 & Spur & Spur\\\hline
    Phosphor & 0,295 & 0,089 & 0,131\\\hline
     & 100 & 100 & 100\\
    \hline
\end{tabular}
\end{center}
\paragraph{}
Die Analyse gab also in dem Bandeisen einen etwas gr"o"ser Nickelgehalt, auch etwas mehr Schwefel und Phosphor und daf"ur weniger Eisen als in der ganzen Masse, welches Verh"altnis gegen die "ubrigen Gemengteil sich noch etwas gr"o"ser stellen w"urde, wenn man bei der Analyse der ganzen Masse das Bandeisen h"atte entfernen k"onnen. Reichenbach glaubte indessen durch diese Analyse noch keine v"ollige Aufkl"arung "uber die chemische Beschaffenheit des Bandeisens erhalten zu haben, da er bei der Besichtigung desselben unter dem Mikroskop fand, dass noch eine Menge anders gearteter K"orperchen in demselben eingelagert waren.

In den Eisenmeteoriten, die keine Widmanst"attenschen Figuren geben, hat nach Reichenbach die eine oder die andere dieser Eisenarten "uberhandgenommen und die andern kommen dann nur ganz unregelm"a"sig und untergeordnet und zum Teil auch gar nicht darin vor. So besteht das Eisen von Braunau fast nur aus Balkeneisen, und das Eisen vom Cap der guten Hoffnung und von Rasgatà ist Reichenbach geneigt, als fast ganz aus F"ulleisen bestehend anzunehmen.

Die meisten Ab"anderungen des Meteoreisens enthalten aber nun noch einen andern Gemengteil, feine nadelf"ormige Kristalle oder Nadeln, wie sie Reichenbach kurzweg nennt. W"ohler\footnote{\frakfamily{Annalen der Chem. u. Pharm. B. 81, S. 254.}} beobachtete sie zuerst bei dem Meteoreisen von einem unbekannten Fundort\footnote{\frakfamily{Reichenbach h"alt dies Meteoreisen f"ur das von Santa Rosa in Columbien, da es aber Widmanst"attenschen Figuren gibt, stimmt es wenigstens nicht mit dem "uberein, welches Boussingault von dort mitgebracht und an v. Humboldt geschenkt hat.}} sowohl auf dessen polierter und ge"atzter Fl"ache, als auch in dem R"uckstande bei seiner Behandlung mit verd"unnter Salpeters"aure, wo sie unter dem Mikroskop kenntlich wurden. Reichenbach zeigte,\footnote{\frakfamily{Vergl. Poggendorffs 1862, B. 115, S. 148.}} dass sie in den meisten Eisenmeteoriten enthalten sind und bei der "atzung einer polierten Fl"ache derselben zum Vorschein kommen, wobei sie einen ausgezeichneten Parallelismus durch die ganze Masse zeigen. Ihre L"ange "uberschreitet selten 2 Linien. Reichenbach h"alt sie f"ur eine vollkommenere Ausbildung des Bandeisens.\footnote{\frakfamily{Zu diesen Einmengungen w"urden auch noch die kleinen Eisenk"ugelchen zu z"ahlen sein, die ich zuerst und dann ausf"uhrlich Reichenbach beschrieben (Pongendorffs Ann. 1861 B. 113, S. 187 und B. 115, S. 152), und die auch auf der geschliffenen Fl"ache in ihren Durchschnitten sichtbar werden sollen. Die Annahme von solchen K"ugelchen beruht aber, wie ich mich jetzt "uberzeugt habe, auf einem Irrtum. Die angeblichen runden Kugeln sind nichts anderes als Stellen, die beim "atzen durch eine ansitzende Luftblase vor dem Angriff der S"aure gesch"utzt waren. Die Luftblase bildete sich durch die Art, wie ich das Meteoreisen "atzte und die darin bestand, dass ich dasselbe mit der polierten wohl gereinigten Fl"ache in die verd"unnte S"aure tauchte und darin eine halbe bis eine ganze Minute hielt, wobei dann "ofter eine Luftblase an der Fl"ache sitzen blieb, die den Angriff der S"aure verhinderte. Wenn man die Fl"ache vorher mit Wasser nass macht oder die Fl"ache nicht mit einem Male unter die Oberfl"ache der S"aure bringt, sondern erst mit einer Seite und sie dann mehr und mehr neigt, bis sie ganz in Wasser eingetaucht ist, so bleiben keine Luftblasen h"angen. Daher kommt es, dass, wie Reichenbach erw"ahnt, er die Eisenk"ugelchen nur bei den St"ucken des Berliner Museums und nicht in den St"ucken seiner eigenen Sammlung gesehen hatte.}}

Au"ser den genannten, vorzugsweise aus metallischem Eisen bestehenden Einmengungen kommen in den Eisenmeteoriten noch andere, teils gr"obere teils feinere, mehr oder weniger h"aufig vor. Zu den ersteren geh"oren Schwefeleisen, Grafit und besonders Olivin.

Das Schwefeleisen ist nach den Untersuchungen von Smith und Rammelsberg kein Magnetkies, wie man bisher angenommen hatte, sondern einfach Schwefeleisen, FeS, Troilit, wie es Haidinger zu nennen vorgeschlagen hat\footnote{\frakfamily{Zur Erinnerung an den Berichterstatter des Meteoritenfalls von Albareto bei Modena 1766, Domenico Troilit, der schon lange vor Chladni die Tats"achlichkeit der Meteoritenf"alle bewies, freilich ohne seiner Meinung Geltung verschaffen zu k"onnen. Vergl. Sitzungsbericht d. k. Akad. der Wiss. M"arz 1863.}}; also eine Verbindung, die unter den Mineralien der Erde bisher noch nicht bekannt ist. Es besteht nach Rammelsberg\footnote{\frakfamily{Monatsber. d. k. Pr. Akad. d. Wiss. 1864, S. 29.}} in zwei Ab"anderungen aus dem Eisen von Seel"asgen (a) und Sevier County (b) und nach der Berechnung nach der Formel (c) aus:
\begin{center}
\begin{tabular}{ |l|r|r|r| }
    \hline
     & \emph{a} & \emph{b} & \emph{c}\\
    \hline\hline
    Eisen & 63,41 & 62,22 & 63,64\\\hline
    Nickel & -,- & 1,76 & -,-\\\hline
    Mangan & 0,64 & -,- & -,-\\\hline
    Schwefel & 35,91 & 36,01 & 36,36\\\hline
     & 99,96 & 99,99 & 100,0\\
    \hline
\end{tabular}
\end{center}
\paragraph{}
Es enth"alt also bald Nickel (Schwefelnickel) bald ist es, obgleich mitten in dem nickelhaltigen Meteoreisen vorkommend, davon frei, wie der Olivin in dem Pallas-Eisen (s. weiter unten), doch scheint das erstere h"aufiger zu sein, da auch Smith in dem Troilit aus dem Meteoreisen von Tazewell etwas Nickel angibt. 

Der Troilit ist bis jetzt in dem Meteoreisen nur derb vorgekommen, in mehr oder weniger gro"sen K"ornern und unregelm"a"sigen Massen, zuweilen in der Form von Zylindern von mehr als 1 Zoll Gr"o"se und mehreren Linien Dicke, wie Reichenbach beobachtet hat. Er ist im Bruch uneben, zeigt aber "ofter d"unnschalige Zusammensetzungsst"ucke, wie dies "ofter beim Magnetkies, z. B. von Bodenmais vorkommt; tombakbraun, metallisch gl"anzend, spezifisches Gewicht 4,787 (Seel"asgen), 4,817 (Sevier County). Diess hohe spezifische Gewicht unterscheidet ihn von dem Magnetkiese, dessen Gewicht nicht "uber 4,623 hinauszieht. Ebenso unterscheidet er sich durch sein Verhalten gegen Chlorwasserstoffs"aure, indem er sich darin ohne einen R"uckstand von Schwefel aufl"ost. Nickeleisen, Chromeisenerz und Grafit kommen "ofter in ihm eingemengt vor, doch ist er zuweilen auch davon ganz frei, wie die Analyse des Troilit von Seel"asgen durch Rammelsberg beweist.

Wenn es so erwiesen ist, dass einfach Schwefeleisen in dem Meteoreisen vorkommt, so bleibt es doch noch auszumachen "ubrig, ob alles Schwefeleisen in demselben von derselben Art sei oder ob neben diesen nicht auch Magnetkies vorkommt. In den Steinmeteoriten ist, wie bekannt, das Vorkommen dieses letzter nicht zweifelhaft, da, wenn dar"uber auch noch keine chemischen Untersuchungen angestellt sind, das Schwefeleisen in dem Meteorstein von Juvenas kristallisiert vorkommt, und an der Kristallform als Magnetkies erkannt werden kann. Ist es daher m"oglich, dass dieser auch in den Eisenmeteoriten vorkommt, so ist dies doch noch nicht erwiesen, wie auf der anderen Seite auch das Vorkommen von einfach Schwefeleisen in den Steinmeteoriten nicht bewiesen ist; ich werde daher bis auf Weiteres das Schwefeleisen der Eisenmeteorit, auch wo es noch nicht untersucht ist, als Troilit und das der Steinmeteorit als Magnetkies auff"uhren. Der Grafit findet sich in kleinen abgerundeten, im Innern aus dicht zusammengeh"auften Sch"uppchen bestehenden Parthien bis zu der Gr"o"se einer Haselnuss oder Walnuss, zuweilen aber auch, wie Haidinger bei dem Eisen von Arva beobachtete\footnote{\frakfamily{Pongendorffs Ann. 1846 B. 67, S. 437.}} in Pseudomorphosen.\footnote{\frakfamily{Haidinger glaubte darin die Form einer Kombination des Hexaeders mit dem Pentagondodekaeder zu erkennen und nimmt daher an, dass die Pseudomorphosen aus Eisenkies entstanden w"aren, eine Ansicht, die ich jedoch nicht teilen m"ochte, da Eisenkies mit Sicherheit in den Meteoriten bis jetzt nicht beobachtet ist, und die Pseudomorphosen selbst, die Hr. Haidinger die G"ute hatte, mir zur Ansicht zu schicken, mir mehr die Form eines Hexaeders mit zu gesch"arften als mit schief abgestumpften Kanten zu haben schienen. Man kann nun aber fragen, woraus die Pseudomorphosen dann entstanden w"aren? Am n"achsten liegt hier nun wohl die Annahme, dass dies der Diamant gewesen sei; wenn aber auch diese Annahme durch die Form der Pseudomorphose und die M"oglichkeit der Bildung gerechtfertigt wird, so findet sie doch darin eine gro"se Schwierigkeit, dass eben Diamanten in den Meteoriten bisher noch nicht beobachtet sind.}} Der Olivin findet sich in einzelnen abgerundeten K"ornern in manchen Ab"anderungen in gro"ser Menge, wie namentlich in dem Meteoreisen, das Pallas 1776 am Jenisei im "ostlichen Sibirien gefunden hatte. Biot\footnote{\frakfamily{Bulletin des sciences, par la soc. philomatique. 1820 p. 89.}} schloss aus dem optischen Verhalten der Olivin-K"orner, dass dieselben wirkliche Kristalle w"aren, und ich beobachtete,\footnote{\frakfamily{Pongendorffs Ann. 1825 B. 4, S. 186.}} dass die K"orner, wie wohl meistenteils, ganz rund, wo sie frei im Eisen liegen, doch schon einzelne sehr gl"anzende Fl"achen und in seltenen F"allen sogar in gro"ser Menge enthalten. Au"serdem Pallas-Eisen enthalten auch noch das Meteoreisen von Brahin (Gouv. Minsk) und von der W"uste Atacama und andere solche Kristalle.

Wo aber diese Einschl"usse vorkommen, sind sie stets, wie Reichenbach hervorhob, von einer H"ulle von Balkeneisen umgeben, und wenn sie sich in einem Meteoriten, der Widmanst"attenschen Figuren gibt, finden, so fangen diese immer erst in einer gewissen Entfernung, die 1 bis 3 Linien betr"agt, an, sich regelm"a"sig zu entwickeln. Dies zeigt sich besonders sch"on bei den Olivin-Einschl"ussen. Sind sie in gro"ser Menge vorhanden, wie in Pallas-Eisen und in dem Eisen von Brahin und Atacama, so dass sie oft nur wenig Raum zwischen sich lassen, so wird dieser von dem Balken-, Band- und F"ulleisen meistenteils ganz ausgef"ullt, und zwar so, dass zuerst an dem Olivin sich eine d"unne Lage von Balkeneisen anlegt, dann eine viel d"unnere Lage von dem Bandeisen folgt und zuletzt das F"ulleisen den inneren Raum einnimmt, wie man dies auf einer durch ein solches Meteoreisen gelegten Schnittfl"ache, die man ge"atzt hat, sehr gut sehen kann. Sind die R"aume zwischen den Olivinkristallen gr"o"ser, so bilden sich in dem F"ulleisen die Widmanst"attenschen Figuren; bei den genannten Meteoriten sieht man jedoch diese nur selten, aber bei dem Eisen von Steinbach und Rittersgr"un, wo die Olivine kleiner sind und die Eisenmasse zwischen ihnen gr"o"ser ist, sind auch die Widmanst"attenschen Figuren gr"o"ser, umso mehr als auch nun die Einfassung des Olivins durch das Balkeneisen schmaler ist.

Die Widmanst"attenschen Figuren waren fr"uher in den Olivin-haltigen Eisenmeteoriten ganz "ubersehen, bis sie Partsch in dem Eisen von Steinbach entdeckte,\footnote{\frakfamily{Die Meteoriten, S. 91.}} der dadurch auf den gleichen Ursprung von vielen St"ucken Meteoreisen, die in den verschiedenen Sammlungen mit der Angabe von verschiedenen Fund"ortern aufgef"uhrt waren, schloss. Sie wurden nachher auch von Reichenbach beschrieben.

Zu den feiner Einmengungen, die sich in den Eisenmeteoriten finden, geh"oren eine Menge kleiner mikroskopischer, meist farbloser, doch auch farbiger gl"anzender Steinchen von mehr als Quarzh"arte, die W"ohler au"ser dem Schreibersit als R"uckstand bei der Aufl"osung des Eisens von Rasgatà erhielt, jedoch nicht weiter untersuchte und einige kleine Quarzkristalle, die ich in dem Eisen von Toluca beobachtete\footnote{\frakfamily{Pongendorffs Ann. 1861 B. 113, S. 184.}} und die noch so gro"s und gl"anzend waren, dass ich ihre Winkel mit Genauigkeit bestimmen konnte. Sie steckten zwar nur in der "au"sern oxydierten Rinde, doch so, dass man nicht daran zweifeln konnte, dass sie sich in dem Meteoreisen gebildet hatten und nicht erst sp"ater hineingekommen waren.

Alle diese Eisenmeteorit, die man zuf"allig auf der Oberfl"ache der Erde findet, sind mit solcher Rinde von Eisenoxydhydrat umgeben, die sich erst durch Oxydation gebildet hat. Diese Oxydation geht aber in der Regel nicht weit, und die so entstandene Rinde sch"utzt die Eisenmeteorit vor ihrer Zerst"orung und bewirkt, dass sie sich Jahrtausende weiter unversehrt erhalten. Sie ist die Ursache, dass die Eisenmeteorit, die nur so selten fallen, dass man nur die Fallzeit von dreien kennt, doch h"aufig gefunden werden, so dass man jetzt in den Sammlungen mehr als halb so viel Eisen- wie Steinmeteoriten hat. Die Eisenmeteoriten, welche man hat fallen sehen und unmittelbar nach ihrem Falle gesammelt hat, wie die von Agram und Braunau, haben keine solche Rinde von Brauneisenerz; die rundlichen Erhabenheit und Vertiefungen, die sich "uberall an der Oberfl"ache derselben finden, sind, wie Reichenbach gezeigt hat, mit einem d"unnen "uberzuge von Magneteisenerz bedeckt, der sich bei dem Durchzuge durch die Luft an der Oberfl"ache durch Schmelzung und Oxydation bildet,\footnote{\frakfamily{Pongendorffs Ann. 1858 B. 103, S. 637. Reichenbach spricht hier von Eisenoxydul, was wohl nur hei"sen soll Oxydoxydul oder Magneteisenerz.}} und so "au"serst d"unn ist, weil das gebildete Magneteisenerz beim Schmelzen abtropft und nur das wenigste durch Adh"asion haften bleibt. Unter diesem "uberzuge findet sich dann eine 1 bis 1 Linien dicke Lage, in welcher das Eisen ganz k"ornig geworden ist, wie auch schon Reichenbach bei dem Eisen von Braunau beobachtet hat\footnote{\frakfamily{A. a. O. 1862 B. 115, S. 135.}} und auch in den Abdr"ucken dieses Eisens von Haidinger zu sehen ist,\footnote{\frakfamily{Sitzungsberichte der math.-naturw. Klasse d. k. Akad. der Wissenschaft. 1855 B. 15, S. 354, Fig. 5.}} was beweist, dass das Eisen vor der Oxydation seinen Aggregatzustand "andert. Dass sich auf der Oberfl"ache dieser Eisenmassen beim Liegen in und auf der feuchten Erde nicht blo"s Eisenoxydhydrat, sondern auch Magneteisenerz bildet, hat Krantz gezeigt,\footnote{\frakfamily{Pongendorffs Ann. 1857, B. 101, S. 152.}} der auf der Oberfl"ache des Eisens von Toluca oktaedrische Kristalle dieser Substanz beobachtet hat.
\subsection{\frakfamily{Meteoreisen.}}
\paragraph{}
Das Meteoreisen von den verschiedenen Orten, wo man es zuf"allig gefunden oder hat fallen sehen, ist von verschiedener Art. Diese Massen sind:
\begin{itemize}
  \item[a)] nur St"ucke eines Individuums oder eines Kristalls ohne schalige Zusammensetzung,
  \item[b)] Aggregate grobk"orniger Individuen, ebenfalls ohne schalige Zusammensetzung,
  \item[c)] Individuen mit schaligen Zusammensetzungsst"ucken parallel den Fl"achen des Oktaeders (Meteoreisen, das durch "atzung Widmanst"attenschen Figuren gibt),
  \item[d)] Aggregate mit gro"sk"ornigen, schalig zusammengesetzten Individuen,
  \item[e)] Aggregate mit feink"ornigen Zusammensetzungsst"ucken.
\end{itemize}
\paragraph{}
\emph{a)} Meteoreisenmassen, welche St"ucke eines Kristalls ohne schalige Zusammensetzung sind.
\vspace{\medskipamount}
\paragraph{}
Hierher geh"ort vor Allen:

1) das Eisen von Braunau in B"ohmen (gefallen am 14. Juni 1847), das schon oben S. 34 erw"ahnt ist und an welchem Haidinger die Beobachtung machte, dass es Spaltungsfl"achen ganz gleicher Art nach drei untereinander rechtwinkligen Richtungen, also nach den Fl"achen des Hexaeders hat. Das Meteoreisen hat also dieselbe Form, wie das k"unstlich dargestellte reine Eisen, bei welchem sich auch oft Massen mit k"ornigen Zusammensetzungsst"ucken finden, die die Spaltungsfl"achen des Hexaeders deutlich wahrnehmen lassen.\footnote{\frakfamily{Z. B. bei dem Eisen, das lange Zeit als Rostbalken gedient hat.}}

Neumann zeigte,\footnote{\frakfamily{Naturwissenschaftliche Abhandlungen, herausgegeben von Haidinger, 1849, B. 3, Abt. 2, S. 45.}} dass, wenn man eine polierte Schnittfl"ache dieses Meteoreisens mit verd"unnter Salpeters"aure "atzt, sich auf derselben eine Menge gerader und unter einander paralleler Linien oder linienf"ormiger Furchen bilden, die meistenteils eng neben einander liegen und nach mehreren sich unter verschiedenen Winkeln durchschneidenden Richtungen gehen, von denen aber die Linien einer oder zweier Richtungen stets vorwalten, die Linien der andern Richtungen nur untergeordnet und mehr stellenweise vorkommen. Er bestimmte mit gro"ser Sorgfalt die Lage dieser Linien und zeigte, dass sie auf einer Hexaederfl"ache nach sechs Richtungen gehen, nach den zwei Diagonalen der Hexaederfl"ache, \emph{ac} und \emph{bd} (Taf. 1 Fig. 2) und nach 4 andere Richtungen, die den Linien \emph{af}, \emph{ag}, \emph{df}, und \emph{de}, die aus 2 benachbarten Winkeln nach den Mitten der gegen"uberliegenden Seiten gezogen werden k"onnen, parallel gehen. Die beiden Diagonalen \emph{ac} und \emph{db} schneiden sich also unter rechten Winkeln und ebenso je 2 der "ubrigen Linien, die aus verschiedenen Winkeln der Hexaederfl"ache auslaufen, wie \emph{ag} und \emph{df} oder \emph{af} und \emph{de}, dagegen die beiden Linien, die aus einem Winkel auslaufen, wie \emph{de} und \emph{df}, gegen einander Winkel von 36$^{\circ}$ 52’ und jede dieser Linien mit der benachbarten Seite der Hexaederfl"ache einen Winkel von 26$^{\circ}$ 34’ macht.\footnote{\frakfamily{Neumann gibt wohl nur aus Versehen den ersten Winkel zu 35$^{\circ}$ 14’ an, und die Winkel, welche die Linien \emph{af} und \emph{ag} oder \emph{de} und \emph{df} mit den sie durchschneidenden Diagonalen machen, zu 72$^{\circ}$ 23’ und zu 107$^{\circ}$ 37’ statt zu 70$^{\circ}$ 34’ und zu 108$^{\circ}$ 26’.}} Man sieht diese Linien in Fig. 1 (Taf. 1), die eine Zeichnung in vergr"o"sertem Ma"sstabe von der Hexaederecke \emph{o} an einem St"ucke des Braunauer Eisens Fig. 4 ist, an welchem die drei Fl"achen \emph{A}, \emph{B}, \emph{C} m"oglichst genau parallel den 3 Spaltungsfl"achen des Eisens geschliffen und darauf ge"atzt sind. Die Atzungslinien sind auf Fig. 1 m"oglichst getreu nach der Natur gezeichnet. Man sieht hier, wie die Linien, die parallel der Richtung \emph{ag} (Fig. 2) gehen, vorherrschen und gruppenweise wiederkehren, w"ahrend andere nur stellenweise vorkommen und die erster bald durchschneiden, bald nicht. Noch besser sieht man sie in Fig. 5, die eine kleine Stelle in der Gegend von \emph{r} auf der Fl"ache \emph{C} (Fig. 1) 140-mal vergr"o"sert darstellt.\footnote{\frakfamily{Diese Fig. ist auf die Weise gezeichnet, dass von der Fl"ache \emph{C} Fig. 4 zuerst ein Hausenblasenabdruck gemacht, von der Stelle \emph{r} auf ihr, sodann ein unter dem Mikroskop vergr"o"sertes photographisches Bild angefertigt und dasselbe darauf abgezeichnet wurde. Die Richtung der Linien ist daher genau die der Natur, und die Winkel w"urden bis auf die Sekunde genau sein, wenn die geschliffene Fl"ache h"atte der Hexaederfl"ache genau parallel geschliffen werden k"onnen. Es ist ein gro"ser "ubelstand, der die Untersuchung sehr erschwert, dass man das Meteoreisen nicht spalten kann, sondern alle Hexaederfl"achen, die man haben will, erst anschleifen lassen muss, was immer m"uhsam ist, und mit gro"ser Genauigkeit doch nicht ausgef"uhrt werden kann. Indessen sind in diesem Fall die Abweichungen von den Winkeln, die man auf der Zeichnung messen kann, nicht sehr abweichend von den berechneten, woraus hervorgeht, dass die geschliffene Fl"ache in ihrer Lage wenigstens nicht sehr von der Spaltungsfl"ache abweicht. Das photographische Bild zu dieser Figur hat Hr. Dr. Hrn. Vogel freundlichst dargestellt.}} Die Linien derselben Richtung sind in Fig.5 mit denselben Buchstaben bezeichnet wie in Fig. 2; es fehlen also in Fig. 5 nur die Linien nach den Richtungen \emph{ac} und \emph{af}. Die Linien erscheinen hier oft unterbrochen, die einen dicker und die andern d"unner, und die einen erscheinen bald von den andern durchsetzt, bald durchsetzen sie die andern.

Es ist schwer zu sagen, wof"ur man diese Linien halten soll. Sie charakterisieren keineswegs das Meteoreisen allein, sie finden sich vollkommen ebenso bei dem k"unstlich dargestellten Eisen, wie Prestel gezeigt hat\footnote{\frakfamily{Sitzungsberichte der math. naturw. Klasse der k. k. Akad. der Wiss. B. 15, S. 355.}} und, wie ich mich selbst "uberzeugt habe, unter andern bei einem sch"onen Spaltungsst"uck von solchem k"unstlich dargestellten Eisen, das ich noch Mitscherlich verdanke, bei welchem eine Hexaederkante 1 1/2 Zoll lang ist; die Linien sind feiner als bei dem Meteoreisen, sonst von derselben Art. Da sie auf der Hexaederfl"ache parallel gehen den Durchschnittslinien von einem Hexaeder mit 4 andern, die in Zwillingsstellung mit dem erster stehen, so dass mit den 4 Eckenaxen des erster immer eine Eckenaxe der 4 andere Individuen parallel ist, um welche diese um 180 gedreht erscheinen, so hielt sie Neumann bei dem Braunauer Eisen auch f"ur die Durchschnittslinien von 5 auf diese Weise regelm"a"sig verwachsenen Individuen. Andere halten diese Linien f"ur Anzeigen von versteckten Spaltungsfl"achen, und allerdings w"urden die Durchschnitte des Hexaeders mit dem Ikositetraeder (\emph{a}:\emph{a}:1/2 \emph{a}) oder dem Triakisoktaeder (1/2 \emph{a}:1/2 \emph{a}:\emph{a}) ganz dieselben Linien geben. Ich m"ochte sie am liebsten mit den Eindr"ucken vergleichen, die bei allen Kristallen durch die "atzung entstehen und die namentlich Leydolt, der sie Vertiefungsgestalten nennt, beim Quarz und Aragonit so sch"on dargestellt und beschrieben hat. Sie haben bei diesen zwar keine Linienform, sondern erscheinen wie vertiefte Ecken, sind aber doch h"aufig linienartig aneinandergereiht. Man hat diese Eindr"ucke erst bei so wenigen K"orpern genau untersucht; sie werden sich gewiss in den verschiedenen F"allen sehr verschieden verhalten.

Ich habe an dem St"ucke Fig. 4 versucht, die Fortsetzung der Linien einer Fl"ache auf den benachbarten zu verfolgen, um zu entscheiden, ob die Linien den Durchschnitten des Hexaeders mit dem Triakisoktaeder oder dem Leucitoaeder parallel gehen, aber ich fand, dass bald das eine, bald das andre der Fall war, wie aus Fig. 1 zu ersehen ist. Die meisten dieser Linien, wie \emph{lm} und \emph{mn}, \emph{rs} und \emph{st}, \emph{ps} und \emph{sq}, gehen allerdings parallel den Durchschnitten mit dem Leucitoaeder, doch andere wie \emph{uv} und \emph{vw} parallel den Durchschnitten mit dem Triakisoktaeder. Es ist freilich oft schwer die zusammengeh"origen Linien zu erkennen, doch glaube ich mich nicht zu irren, wenn ich annehme, dass die Durchschnittslinien nach beiden Formen vorkommen.

Neben den "atzungslinien sieht man auf der ge"atzten Schnitt- oder Spaltungsfl"ache bei einiger Aufmerksamkeit "uberall zerstreut noch kleine nadelf"ormige, metallisch gl"anzende Kristalle, wie sie W"ohler und v. Reichenbach auch bei anderen Meteoreisen beobachtet haben (vgl. oben S. 38), hervorragen. Beide Beobachter sahen schon, dass sie h"aufig untereinander parallel, und die Ursache des Schillerns der ge"atzten Fl"achen in bestimmten Richtungen sind. Bei dem Braunauer Eisen kann man nun auf das bestimmte sehen, dass sie eine untereinander und in Bezug auf das Eisen, worin sie eingemengt sind, ganz bestimmte Lage haben. Ich konnte dies recht gut beobachten bei einem kleinen St"ucke, das durch zwei nat"urliche Spaltungsfl"achen und durch eine Schnittfl"ache begrenzt ist, die ungef"ahr die Richtung einer Dodekaederfl"ache hat. Es ist in Taf. 1. Fig. 3 in etwas vergr"o"sertem Ma"sstabe dargestellt; die zwei nat"urlichen Spaltungsfl"achen sind mit \emph{A} und \emph{C}, die angeschliffene Fl"ache mit \emph{D} bezeichnet; auf der Hinterseite ist das St"uck durch die nat"urliche Oberfl"ache begrenzt. Man kann hier deutlich sehen, dass kleine prismatische Kristalle auf jeder Spaltungsfl"ache mit ihren Hauptaxen in zwei den Kanten der Spaltungsfl"achen parallelen Richtungen, die Kristalle also in dem ganzen St"ucke nach den dreierlei Hexaederkante d. i. nach drei untereinander rechtwinkligen Richtungen liegen. Auf der Schlifffl"ache, der Dodekaederfl"ache, sieht man nur Kristalle, die parallel der Hexaederkante liegen, als deren Abstumpfungsfl"ache die Dodekaederfl"ache erscheint und die auch hier wie die Kristalle auf den Spaltungsfl"achen in ihrer wahren L"ange erscheinen. Dreht man die Fl"ache \emph{D} um die ihr parallele Hexaederkante, so reflektieren eine gro"se Menge der kleinen Kristalle zu gleicher Zeit das Licht, so wie man in die Richtung der zu der Kante geh"orenden Hexaederfl"achen kommt, also der Fl"achen \emph{C} und der dritten Hexaederfl"ache \emph{B}, von der bei dem St"ucke auf \emph{A} immer noch Spuren zu sehen sind; die Seitenfl"achen dieser kleinen Kristalle sind also selbst wie die Hexaederfl"achen rechtwinklig gegeneinander geneigt, die Kristalle also quadratische Prismen, die nicht nur mit ihren Hauptaxen parallel einer der 3 Hexaederkante, sondern auch mit ihren Seitenfl"achen parallel den 2 Hexaederfl"achen dieser Kante liegen. Indessen schillert das St"uck noch in anderen Richtungen dazwischen, so dass sich hieraus nur ergibt, dass die Kristalle entweder noch mehrere Seitenfl"achen haben oder wenn sie nur quadratische Prismen mit ihren Seitenfl"achen nicht "uberall untereinander parallel liegen.

In Fig. 3 sind diese kleinen Kristalle auf den Fl"achen \emph{A}, \emph{C}, \emph{D} durch kleine Striche und Punkte bezeichnet, je nachdem man sie im L"angs- oder Querschnitt sieht; doch soll durch sie nur ihre Lage im Allgemeinen angegeben werden, sie sind willk"urlich hineingezeichnet, in der Natur sind sie sehr ungleich verteilt und liegen bald einzeln, bald gruppenweise beisammen. Diess letztere sieht man besonders bei einem St"ucke des Braunauer Eisens der Berliner Sammlung, an welchem eine Fl"ache, ungef"ahr in der Richtung einer Oktaederfl"ache, angeschliffen ist; man sieht hier nur die Querschnitte der Kristalle; durch das gruppenweise Beisammenliegen erscheint aber die Fl"ache wie gesprenkelt. Bei dem St"ucke Fig. 3 sind auf der Fl"ache \emph{D}, die der Kombinationskante mit \emph{C} parallel liegenden Kristalle in gro"ser Menge vorhanden.

Noch deutlicher als mit blo"sen Augen erscheinen die kleinen eingewachsenen Kristalle, wenn man von der ge"atzten Fl"ache einen Hausenblasenabdruck macht und diesen unter dem Mikroskop betrachtet. In Fig. 5 (Taf. 1) sind diese nadelf"ormigen Kristalle nur klein, und es finden sich zuf"allig hier meistenteils auch nur solche, die aus der Fl"ache \emph{C} senkrecht stehen, also nur ihre Querschnitte zeigen. Viel gr"osser als in dem Eisen von Braunau sind sie in dem von Seel"asgen, von dem Fig. 4 Taf. 2 die Zeichnung einer ge"atzten Schnittfl"ache in nat"urlicher Gr"o"se ist. Die Fig. 6-8 Taf. 1 sind Zeichnungen von Stellen der Fl"ache einer anderen "ahnlichen Platte dieses Eisens in 140-maliger Vergr"o"serung, die wie Fig. 5 Taf. 1 dargestellt sind.\footnote{\frakfamily{Das Eisenkorn, welches die Fig. 6 Taf. 1 gezeichnete Stelle, sowie den sp"ater zu erw"ahnenden dreieckigen Einschluss enth"alt, ist Fig. 7 in nat"urlicher Gr"o"se gezeichnet.}} Man sieht in denselben Quer- und L"angsschnitten der Prismen, und letztere in zwei untereinander rechtwinkligen Richtungen, woraus sich ergibt, dass die Schnittfl"ache parallel einer Hexaederfl"ache geht. Die Prismen erscheinen hier auf das bestimmte als quadratische, aber man sieht zugleich, dass, wenn auch ihre Hauptaxen einer der Hexaederkante, doch ihre Seitenfl"achen nicht immer den Fl"achen derselben parallel gehen, und dies scheint "uberall der Fall zu sein, wie man dies auch beim Drehen der Fl"ache \emph{D} Fig. 3 beobachten kann, wie eben gezeigt ist. In Fig. 6 sind zuf"allig die einen der horizontal liegenden Kristalle au"serordentlich gro"s und scheinen hier an den Enden mit der geraden Endfl"ache begrenzt zu sein, w"ahrend sie in Fig. 9 Taf. 1, welche eine Stelle von der Fl"ache \emph{D} Fig. 3 darstellt, doch eine andere Endkristallisation zu haben scheinen. In Fig. 5 und 8 Taf. 1 erscheinen die Kristalle nicht mit ganz parallelen Kanten, was hier offenbar davon herr"uhrt, dass die Schnittfl"ache nicht genau parallel einer Hexaederfl"ache geht.

Es scheint mir zweckm"a"sig, diese kleinen eingewachsenen Kristalle mit einem besonderen Namen zu bezeichnen, ich werde sie daher in dem Folgenden mit dem Namen Rhabdit, von $\rho\alpha\beta\delta$o$\varsigma$ [rhabdos] der Stab, benennen.

Au"ser diesen feinen Rhabdit-Kristallen finden sich in dem Braunauer Eisen noch andere etwas gr"o"sere unregelm"a"sig, zum Teil auch regelm"a"sig begrenzte Einmengungen, die von der verd"unnten Salpeters"aure auch nicht angegriffen werden und beim "atzen des Eisens ihren metallischen Glanz und ihre stahlgraue Farbe behalten. Bei der Aufl"osung des Eisens in Salpeters"aure oder Chlorwasserstoffs"aure bleiben diese Einmengungen zur"uck, und man erkennt in dem R"uckstande Kristalle mit regelm"a"sigen Formen. Fischer sah darin l"angliche, rechtwinklige Tafeln\footnote{\frakfamily{Pongendorffs Ann. B. 73. S. 592.}}; ich sah unter dem Mikroskop diese und andere Formen, doch war die angewandte Menge zu gering, um gen"ugende Beobachtungen zu machen.

Von gr"o"ser Einmengungen finden sich kleine runde oder l"angliche Parthien von Troilit, auf dessen ge"atzter Fl"ache kleine gl"anzende Punkte erscheinen, der also wohl Nickeleisen in feinen Teilen beigemengt enth"alt. In der N"ahe dieses Troilit werden auf der ge"atzten Fl"ache die "atzunglinien wohl feiner, gehen aber in v"ollig unver"anderter Richtung bis zu ihm fort. Das erstere ist wohl nur eine Folge davon, dass der Troilit viel leichter aufl"oslich ist als das Meteoreisen, und die Salpeters"aure in der N"ahe des Troilit durch seine Aufl"osung noch mehr verd"unnt wird, so dass sie auf das Meteoreisen in der N"ahe des Troilit nur eine schw"achere Einwirkung aus"uben kann.

Die beiden oben S. 34 und S. 46 erw"ahnten St"ucke des Braunauer Eisens in dem Berliner Museum haben beide zum Teil noch ihre nat"urliche Oberfl"ache mit ihrer rundlichen Erhabenheit und Vertiefungen und ihrer d"unnen Decke von Magneteisenerz, und ebenso kann man unter dieser die 1 bis 1 1/2, Linien dicke Lage erkennen, in welcher das Eisen ganz k"ornig geworden ist.\footnote{\frakfamily{Vergl. oben S. 42.}}

2) Claiborne, County Alabama V. St., gefunden 1838. Eine 3 Zoll lange und 1 Zoll breite geschnittene Platte von Hrn. v. Reichenbach in Tausch erhalten. Sie gleicht der Platte des Braunauer Eisens, die parallel der Oktaederfl"ache geschnitten ist und zeigt daher au"ser den "atzungslinien die Querschnitte des eingemengten Rhabdit. Au"serdem kommen noch einige gr"o"sere graue, metallisch gl"anzende Einmengungen vor, die teils eine K"ornerteils eine Nadelform haben. Eine Bruchfl"ache befindet sich daran nicht.

3) Saltillo (Santa Rosa), Neu-Mexico, gefunden 1860. Zwei kleine Platten vom Prof. Shepard durch Tausch erhalten. Sie zeigen die "atzelinien und Rhabdit-Kristalle sehr deutlich.
\vspace{\medskipamount}
\paragraph{}
\emph{b)} Meteoreisenmassen, welche Aggregate grobk"orniger Individuen ohne schalige Zusammensetzung sind,
\vspace{\medskipamount}
\paragraph{}
Zu diesen geh"ort besonders:

4) das Meteoreisen von Seel"asgen bei Schwiebus im Reg.-Bezirk Frankfurt in Preu"sen, gefunden 1847. Dasselbe besteht aus einer Menge gr"o"serer und kleinerer anscheinend unregelm"a"sig begrenzter und unregelm"a"sig verbundener Zusammensetzungsst"ucke, die auf den ge"atzten Schnittfl"achen die gr"u"ste "ahnlichkeit mit den Schnittfl"achen des Braunauer Eisens zeigen. Fig. 4 Taf. 2 ist die schon oben S. 47 erw"ahnte Zeichnung einer ge"atzten Schnittfl"ache dieses Eisens aus der Berliner Sammlung. Man sieht auf den einzelnen Zusammensetzungsst"ucken die "atzungslinien sehr sch"on, meistenteils die, weiche den Linien \emph{df}, \emph{ef} und \emph{db} in Fig. 2 Taf. I entsprechen, doch auch die andern angegebenen. Die Linien einer Richtung herrschen gew"ohnlich vor, aber diese vorherrschenden Linien und somit alle "ubrigen liegen in den meisten Zusammensetzungst"ucken untereinander verschieden und nur in einigen fast oder ganz gleich. Man sieht ferner auch ohne Lupe die kleinen Kristalle des Rhabdit sowohl in ihren L"angs- als Querschnitten als kleine Striche oder Punkte und erstere auf den Zusammensetzungsst"ucken, bei denen die Schnittfl"ache ungef"ahr parallel einer ihrer Spaltungsflachen geht, in zwei ungef"ahr aufeinander rechtwinkligen Richtungen. Noch besser sieht man sie auf dem Hausenblasenabdruck unter dem Mikroskop, wie sie in den schon oben beschriebenen Fig. 6--5 Tal. I dargestellt sind, Masche Stellen erscheinen auch gesprenkelt, kurz man sieht alle Erscheinungen, die das Eisen von Braunau dargeboten hat.\footnote{\frakfamily{Die "atzelinien und Rhabdit-Kristalle sind in der Zeichnung Fig. 4 nur zum Teil angegeben, es war nicht m"oglich, alle Einzelheiten rollst"andig wiederzugeben.}} Bei einer bestimmten Beleuchtung erscheinen die einen Zusammensetzungsst"ucke gl"anzend und von fast dunkel bleigrauer Farbe, w"ahrend die andern matt und von lichterer stahlgrauer Farbe sind. Diese bekommen dann in anderer Richtung mehr Glanz, wenngleich der Unterschied der Farbe bleibt. Es erh"alt auf diese Weise die ge"atzte Fl"ache das Ansehen von Damast, womit man dieselbe schon "ofter verglichen hat. In der Zeichnung Fig, 4 Taf. 2 ist dies dadurch darzustellen versucht, dass den letzteren Zusammensetzungsstucken ein etwas grauer Ton gegeben ist, Die gl"anzenden St"ucke haben zuweilen eine "ubereinstimmende Streifung, aber keineswegs immer, was man oft bei zwei dicht aneinander Grenzenden sehen kann.\footnote{\frakfamily{Bei den mit \emph{n} und \emph{l} bezeichneten Individuen ist die Streifung und die Lage der eingeschlossenen Kristalle ganz verschieden, w"ahrend sie Joch bei den Individuen \emph{n} und \emph{m} ganz gleich ist; und doch haben aller gleiche Glanz.}} Worauf also diese gleiche Spiegelung beruht, kann ich nicht angeben.

Troilit ist h"aufig in dem Eisen von Seel"asgen eingemengt und findet sich zuweilen in gr"o"seren Partien, die teils eine zylinderf"ormige, teils kuglige, teils unf"ormliche Gestalt, doch immer eine ziemlich ebene Oberfl"ache haben. Sie sind mit einer etwa eine halbe bis eine ganze Linie dicken Schicht von einer in verd"unnter Salpeters"aure unl"oslichen Substanz, wahrscheinlich einem nickelreicheren Nickeleisen als das Meteoreisen ist, umgeben, die wohl beim "atzen etwas br"aunlichgelb anl"auft und dadurch wohl einige "ahnlichkeit mit Eisenkies hat, Joch nicht damit zu verwechseln ist, wie dies "ofter geschehen ist.\footnote{\frakfamily{Eine solche Troilitpartie mit ihrer Umgebung von dem schwerl"oslichen Nickeleisen und den Zusammensetzungsst"ucken des Meteoreisens ist Fig. 5 Taf. 2 dargestellt.}} Graphit kommt zuweilen in kleinen Parthien in dem Troilit eingemengt vor; aber kein Nickeleisen, und Nickel ist "uberhaupt nicht einmal chemisch verbunden in dem Troilit des Eisens von Seel"asgen enthalten, wie oben angegeben (vergl. S. 39).

5) Nelson County, Kentucky, Ver. St. N. A., gefunden 1856, "ahnlich dem vorigen.

6) Union Cty, Georgia, Ver. St. N. A., gefunden 1853, ebenso.

7) Tucuman (Otumpa), Argentinische Rep. 5, A., gefunden 1788. Mit Angaben dieses Fundorts besitzt das mineralogische Museum 4 St"ucke, die demselben auf verschiedene Weise zugekommen und von verschiedenem Ansehen sind. Das Hauptst"uck befand sich in einer Sammlung von Mineralien, die von dem preu"sischen Reisenden Sello, der auf Kosten der Regierung Brasilien und die s"udlich angrenzenden Freistaaten bereiste, aber auf der Reise starb, geschickt waren; es hatte den beiliegenden Zettel: Meteoreisen aus der Provinz Gr. Chaco, Geschenk des Canonego Dr. Bartholo Muños zu Buenos Aires; es hat ge"atzt ein "ahnliches Ansehen wie das Meteoreisen von Seel"asgen, und nach ihm ist die Stelle in dem Verzeichnis bestimmt.

Die beiden folgenden St"ucke, 0,56 und 0,05 Loth schwer, stammen aus der Chladnischen Sammlung und haben den Zettel: Bezirk St. Jago del Estero, Prov. Chaco Gualambo in 5. Amerika; sie sind klein, besonders das eine, k"onnen aber dem Ansehen nach wohl mit dem erster vereinigt werden.

Das vierte St"uck, 3,11 Loth, war auch in einer der Selloschen Sendungen enthalten und hat auf dem Zettel keine andere Angabe als: Meteoreisen aus Tucuman; es ist ein flaches St"uck mit feink"ornigem Bruch, geh"ort also zur 5. Abteilung. Diese Beschaffenheit. scheint mit der des Wiener St"uckes aus Tucuman "ubereinzustimmen, da Partsch (Meteoriten, S. 129) von diesem anf"uhrt, dass es dem Eisen vom Senegal, welches zu dieser Variet"at geh"ort, "ahnlichsehe. Es muss daher noch unentschieden bleiben, ob die beiden Selloschen St"ucke von einer und derselben Eisenmasse stammen oder ob unter Gran Chaco und Tucuman zwei ganz verschiedene Fundorte gemeint sind. Nimmt man an, dass die St"ucke von einer Eisenmasse abstammen, und zwar von der gro"sen Masse, die Rubin de Celis besucht, und deren Gewicht er auf 300 Ztr gesch"atzt hatte, so w"urde daraus folgen, dass auch die Eisenmeteorit an einer Stelle feink"ornig und an einer andern gro"sk"ornig sein k"onnen, was bei den Steinmeteoriten zwar h"aufig vorkommt, bei den Eisenmeteoriten aber noch nicht beobachtet ist.
\vspace{\medskipamount}
\paragraph{}
\emph{c)} Meteoreisenmassen, welche St"ucke eines Kristalls mit schaliger Zusammensetzung parallel den Fl"achen des Oktaeders sind, d. h. Eisenmassen, die Widmanst"attensche Figuren geben.
\vspace{\medskipamount}
\paragraph{}
Meteoreisenmassen dieser Art sind die gew"ohnlichsten, wenngleich die Erscheinung nicht "uberall gleich regelm"a"sig und deutlich ist. Sie besteht darin, dass das Eisen in der Form des Oktaeders aus lauter "ubereinander liegenden Schalen parallel den Fl"achen des Oktaeders zusammengesetzt erscheint, zwischen denen sich d"unne Bl"attchen von dem in verd"unnter Salpeters"aure unl"oslichen Nickeleisen, welches Reichenbach T"anit genannt hat, befinden, Sie beweist, dass die Kristallbildung ruckweise vor sich gegangen ist; sie hat von Zeit zu Zeit aufgeh"ort, w"ahrend welcher Zeit sich dann der T"anit abgelagert hat, nat"urlich in so geringer Menge, dass er die Anziehung und somit die weitere regelm"a"sige Ablagerung des Meteoreisens nicht verhindert hat. Es ist also eine Bildung, wie sie sowohl bei aufgewachsenen als auch eingewachsenen Kristallen h"aufig vorkommt, bei aufgewachsenen z. B. beim sogenannten Cap-Quarz aus Devonshire, wo eine geringe Menge von Eisenoxyd die Ursache der Schalenbildung ist; bei eingewachsenen z. B. beim Leucit in den Laven vom Vesuv oder bei dem Magneteisenerz in dem Schwedischen Eisenglanz von Norberg in Westmanland, bei welchem letztere die Schalen wie bei dem Meteoreisen parallel den Fl"achen des Oktaeders gehen. Hat das Meteoreisen lange Zeit in feuchter Erde gelegen, so oxydiert es sich hier an der Oberfl"ache und "andert sich in Eisenoxydhydrat um; die Oxydation folgt den Schalen, und es l"osen sich oft ganz deutlich oktaedrische Teile ab, wie man dies sehr sch"on bei dem Eisen von Cosby und von Arva sehen kann (vergl. S. 37). Legt man nun Schnittfl"achen durch solche Massen, poliert und "atzt man dieselben, so ragen die T"anitbl"attchen mit gl"anzenden Kanten aus dem matten Grunde der Schnittfl"ache hervor, und es bilden sich die Widmanst"attenschen Figuren, Auf diesen ge"atzten Schnittfl"achen kann man die St"arke und gegenseitige Stellung dieser Schalen am bester erkennen, wenn erstere auch von der Lage des Schnitts gegen die Schalen abh"angig ist. Man sieht dann, dass sie in der Regel nur eine halbe Linie, zuweilen aber auch 1 bis 2 Linien dick sind, wie z. B. bei dem Eisen von Bohumilitz. Wo sie aber diese Dicke erreichen, haben sie nicht so ebene Fl"achen, und die T"anitbl"attchen zwischen den Schalen sind in dem Maa"sen unebener. Zuweilen sind sie aber "uberaus geradfl"achig, wie z. B. bei dem Eisen von Elbogen, Agram, Texas, Tazewell u. s. w., so dass man: bei den Abdr"ucken von den ge"atzten Schnittfl"achen derselben mit ziemlicher Genauigkeit die Winkel, die die Schalen untereinander machen, messen und danach die Lage des Schnitts in der Eisenmasse bestimmen kann.\footnote{\frakfamily{Bei dem Abdruck z. B. von der ge"atzten Schnitte der Elbogener Masse, der sich in die oben angef"uhrten Werke von v. Schreibers befindet (vergl. oben S. 34), machen \emph{M} die drei schmalsten und geradlinigsten Streifen Winkel von 64$^{\circ}$, 60 1/2$^{\circ}$ und 55 1/2$^{\circ}$ (in der beistehenden Zeichnung mit \emph{a}, \emph{"s}, \emph{y} bebe zeichnet), die von den Winkeln eines gleichseitigen Dreiecks nicht viel abweichen, daher der Schnitt der Eisenmasse beinahe \emph{E} parallel einer Oktaederfl"ache gef"uhrt ist. Die Schalen, die der vierten Oktaederfl"ache parallel geben, schneiden daher die Schnittfl"ache unter einem sehr spitzen Winkel, ihre Durchschnitte sind "s breiter und unregelm"a"siger, und ihre genaue Richtung ist nun auch schwerer zu messen. Sie haben eine solche Richtung, dass ihre Durchschnitte mit der Schnittfl"ache einer Linie parallel gehen, die von \emph{a} aus der gegen"uberliegenden Seite so trifft, dass sie mit diesem Winkel von ungef"ahr 80$^{\circ}$ und 100$^{\circ}$ bildet und der spitze Winkel \emph{d} dem Winkel \emph{"s} gegen"uber liegt.}}

Durch Einwirkung der verd"unnten Salpeters"aure zeigen sich auf den Querschnitten der Schalen die "atzungslinien mehr oder weniger deutlich und mehr oder weniger eng nebeneinander liegend. Dadurch dass gewisse Richtungen bei diesen Linien vorherrschen und diese verschieden liegen in den verschiedenen Schalen, erhalten auch die Schnittfl"achen dieser Meteoriten das damast"ahnliche Ansehen, wie die der vorigen Abteilung. Zuweilen haben aber nebeneinander und auch auf angrenzenden Schnittfl"achen ganz gleich gelegene Schalen ganz verschieden liegende "atzungslinien, wie ich dies z. B. ganz bestimmt bei dem Eisen von Schwetz beobachtet habe und zuweilen erscheinen selbst dieselben Schalen mit denselben "atzungslinien auf der einen H"alfte gl"anzend und auf der andern matt; der Glanz richtet sich oft gar nicht nach den Schalen, die ge"atzte Fl"ache ist streifenweise gl"anzend und streifenweise matt, wie dies bei dem Eisen von Bohumilitz zu beobachten ist, und zuweilen erscheint jede Schale k"ornig und die k"ornigen Zusammensetzungsst"ucke verschieden gestreift, wie bei Ruffs mountain, wo diese Zusammensetzung auch Reichenbach hervorhebt. Es sind dies alles Verh"altnisse, die noch der Erkl"arung bed"urfen, die aber doch zum Teil von derselben Art sind wie bei dem Eisen von Seel"asgen, wo die Zusammensetzungsst"ucke, wie es scheint, unregelm"a"sig nebeneinander liegen, daher es doch wohl sein kann, dass ungeachtet der unregelm"a"sigen Form der einzelnen Zusammensetzungst"ucke ihre Lage gegeneinander doch eine gewisse Regelm"a"sigkeit haben kann.

Wie die "atzungslinien, so sieht man auch die Rhabdit-Kristalle teils in ihren zwei aufeinander rechtwinkligen L"angsschnitten, h"ochstens liniengro"s, teils in ihren Querschnitten als Punkte; recht deutlich z. B. bei dem Eisen von Misteca, und es findet hier auch dasselbe statt, was bei den "atzungslinien erw"ahnt ist, dass ihre Stellung in zwei benachbarten Schalen nicht immer gleich, ja oft ganz entgegengesetzt ist.

Au"ser dem T"anit und Rhabdit enthalten die Schalen noch eine andere bei der "atzung gl"anzend bleibende Substanz von stahlgrauer Farbe eingeschlossen. Sie findet sich meistens in plattenf"ormigen St"ucken stets: in der Mitte der Schalen und diesen mit ihren breiten Fl"achen parallel, und erscheint in manchem Meteoreisen recht h"aufig, wie z. B. in dem Eisen von Arva, Sarepta, Cosby und Lenarto, und wie dies in den sch"onen, schon oben S. 33 erw"ahnten Abdr"ucken, die Haidinger von den ge"atzten Schnittfl"achen der beiden ersten Eisenmassen bekannt gemacht hat, zu sehen ist. Auf der Schnittfl"ache erscheinen die K"orner und Pl"attchen oft voller kleiner Vertiefungen, was dadurch entsteht, dass sie spr"oder als die umgebende Masse sind, und daher beim Schleifen des Meteoreisens einzelne Teile von ihnen leicht ‚herausgerissen werden. Reichenbach hat diese K"orner und Pl"attchen wegen ihres starken Glanzes, den sie auch nach der "atzung behalten, Lamprit genannt; sie sind aber offenbar dasselbe, was Haidinger schon fr"uher bei dem Arva-Eisen Schreibersit genannt hat,\footnote{\frakfamily{Vergl. dar"uber auch Haidinger in den Sitzungsber. d. math. naturw. Kl. d. k. Akad. d. Wiss. von 1862 B. 46.}} daher der "altere Name gr"o"sere Anspr"uche hat, beibehalten zu werden.

Der Schreibersit und die Rhabdit-Kristalle finden sich jedoch nicht in jedem dieser Eisenmeteoriten. W"ahrend der Schreibersit in den genannten Meteoriten in verh"altnism"a"sig gro"ser Menge vorkommt; findet er sich in den Meteoriten von Schwetz, Misteca, Bohumilitz u. s. w. gar nicht, dagegen in diesen letzteren wiederum der Rhabdit: in gro"ser Menge erscheint, Beide Substanzen scheinen sich’ demnach einander f"ormlich auszuschlie"sen, In dem Arva-Eisen kommen zwar beide auf eine ausgezeichnete Weise vor, aber sie finden sich doch auch hier nicht beide in einem und: demselben St"ucke, denn die einen enthalten Schreibersit, die andern nicht, eine Ungleichheit, die schon Partsch, Reichenbach und Haidinger bei diesem Eisen angegeben haben. Die, welche keinen Schreibersit haben, enthalten daf"ur den Rhabdit. Es w"are demnach wohl m"oglich, dass beide nur verschiedene Arten des Vorkommens einer und derselben Substanz w"aren, und was man von der chemischen Zusammensetzung dieser Massen kennt, ist dem nicht entgegen. Beide m"ussen wenigstens Phosphor enthalten, denn dies ergibt sich daraus, dass derselbe sowohl von Reinhold von Reichenbach in der Schreibersit f"uhrenden Ab"anderung des Arva-Eisens,\footnote{\frakfamily{Vergl. Pongendorffs Ann. 1863 B. 119, S. 172. Es ist hier zwar nicht besonders angegeben, dass das untersuchte St"uck eine Schreibersit f"uhrende Ab"anderung des Arva-Eisens ist, doch kann ich dies insofern bezeugen, als das von Reichenbach, dem Sohne, analysierte St"uck aus dem Berliner Museum stammt, Reichenbach, der Vater, n"amlich, dem bei seiner letzten Anwesenheit in Berlin die vielen eingeschlossenen Schreibersit-K"orner in mehreren St"ucken des Arva-Eisens des Berliner Museums auffielen, w"ahrend eine gro"se Masse des Arva-Eisens in seiner Sammlung diese gar nicht enthielten, hat sich eins dieser Berliner St"ucke in Austausch gegen ein anderes aus seiner Sammlung aus, um es von seinem Sohne analysieren zu lassen, und so m"oglicher Weise die Zusammensetzung des darin eingeschlossenen Schreibersits zu erfahren. So entstand die oben angef"uhrte Abhandlung in Pongendorffs Ann., die nun zwar nicht die genaue Zusammensetzung des Schreibersits, doch aber bestimmt ausmachte, dass der Phosphorgehalt dieses Meteoreisens von ihm herr"uhre.}} als auch von Berzelius und Rammelsberg in dem Rhabdit f"uhrenden Meteoreisen von Bohumilitz gefunden ist. Da nun diese Eisenmassen von den in verd"unnter Salpeters"aure schwer l"oslichen Substanzen, so viel man wei"s, nur noch den T"anit enthalten, derselbe aber nach Reinhold von Reichenbach keinen oder nur eine unbedeutende Menge von Phosphor enth"alt (vergl. oben S. 37), so muss er sowohl in dem Schreibersit als auch in dem Rhabdit enthalten sein, Sollte durch fortgesetzte Untersuchungen es sich best"atigen, dass Schreibersit und Rhabdit dieselbe Substanz sind‚ so muss nat"urlich der Name Rhabdit fortfallen.

Von den nach verschiedenen Richtungen gehenden Schalen des Meteoreisens werden "ofter eckige R"aume eingeschlossen (Zwischenfelder, wie sie v. Schreibers nennt), die mit einem Meteoreisen erf"ullt sind, das mit ganz d"unnen, untereinander parallelen Bl"attchen einer in verd"unnter Salpeters"aure unl"oslichen Substanz durchsetzt wird. Die Bl"attchen haben in demselben Raum bald eine bald mehrere Richtungen und gehen bald den Schalen, die sie einschlie"sen, parallel, bald nicht. Es kann fraglich sein, ob sie aus T"anit oder Schreibersit bestehen, doch m"ochte das erstere wahrscheinlicher sein, was auch die Meinung von Reichenbachs ist (vergl. oben S. 36). Solche R"aume sind dieser Abteilung besonders eigen; ich habe zwei solcher nicht weit voneinander liegender R"aume, die bei dem Eisen von Bohumilitz vorkommen, mit \emph{a} und \emph{b} bezeichnet, in Fig. 6 Taf. 1 in nat"urlicher Gr"o"se und in Fig. 7 25-mal vergr"o"sert dargestellt. In der letzteren Fig. sieht man, dass der Raum \emph{a} eigentlich aus 3 R"aumen besteht, deren jeder seine besonderen Bl"attchen hat, die in jedem nach mehreren Richtungen gehen, die sich gegenseitig durchschneiden. Auch der Raum 2 ist, soweit er gezeichnet ist, aus zweien zusammengesetzt, die aber beide nur Bl"attchen in einer Richtung enthalten. Die R"aume \emph{a} und \emph{b} sind von den Schalen des Meteoreisens \emph{c}, \emph{d}, \emph{e}, \emph{f} und \emph{g} umgeben, die die "atzelinien und die Rhabdit-Kristalle zeigen, die aber nur unvollst"andig in der Fig. wiedergegeben sind.

In der Ecke von \emph{a} Fig.7 befindet sich noch eine Bildung h, die eine besondere Beschaffenheit hat; sie ist in Fig. 8 besonders und 140-mal vergr"o"sert dargestellt. Sie zeigt auf der Schnittfl"ache ungef"ahr parallellaufende gegliederte Streifen, zwischen denen sich kleine K"orper befinden, die wie Kristalle aussehen. Dergleichen Bildungen finden sich in dem Eisen von Bohumilitz h"aufig, oft von noch bedeutenderer Gr"o"se, und kommen mit "ahnlichen "uberein, die sich auch in dem Eisen von Seel"asgen finden, wo nur die kleinen Kristall"ahnlichen K"orper fehlen. Eine solche ist in Fig. 7 und 6 Taf. I in nat"urlicher und 140-maliger Vergr"o"serung dargestellt.

In dem Folgenden ist das Meteoreisen dieser Abteilung in der Ordnung aufgef"uhrt, dass zuerst die Ab"anderungen mit den dicken Schalen und dann die mit den d"unner folgen.

8) Bohumilitz, Prachimer Kreis in B"ohmen, gefunden 1829. Eine gro"se dicke Platte, an 3 Seiten mit nat"urlicher Oberfl"ache begrenzt und eine kleine d"unne Platte. Fig. 6 Taf. 2. Die Schalen sind 1 bis 2 Linien dick, ziemlich geradfl"achig, die "atzelinien auf denselben sehr deutlich, der eingemengte Rhabdit im Allgemeinen nicht gro"s, in einigen ziemlich h"aufig, in andern weniger. Die mit T"anit erf"ullten R"aume zwischen den Schalen des Meteoreisens (\emph{a} und \emph{b} in Fig. 7 Taf. 2) nicht selten, gew"ohnlich aber nur klein. Die gro"se Platte enth"alt mehrere Parthien von Grafit eingemengt, die mit einer bei der "atzung gl"anzend gebliebenen, stahlgrauen Rinde umgeben sind.

9) Brazos, Texas, V. St. 1856, kleines St"uck, "ahnlich dem vorigen.

10) Denton County, Texas, V. St. 1856. Kleines St"uck ebenso.

11) Arva (Szlanicza) Ungarn. 1844. Sechs meistens ziemlich gro"se St"ucke, teils Platten, teils St"ucke, die noch gr"o"sere Teile der oxydierten nat"urlichen Oberfl"ache zeigen. Die St"ucke sind interessant durch ihre ungleiche Struktur (vgl. oben S. 55). Vier derselben enthalten in der Mitte der Schalen, die bei allen von gleicher und von derselben St"arke, wie bei dem Eisen von Bohumilitz sind, eine gro"se Menge von K"ornern und Platten von Schreibersit, die nach dem "atzen der Masse sehr gl"anzend hervortreten, w"ahrend zwei andere deren gar nicht, dagegen den Rhabdit und diesen von einer Gr"o"se der Kristalle enthalten, wie ich sie kaum bei einem anderen Meteoreisen. gesehen habe; sie sind zuweilen "uber eine Linie lang und die quadratische Gestalt ihrer Durchschnitte ist mit der Lupe oft recht deutlich zu sehen. Die "atzelinien sind bei beiden Ab"anderungen auf: den ge"atzten Fl"achen vorhanden, aber in beiden meistenteils nur schwach. Das Meteoreisen ist nach dem "atzen nur matt, wenngleich auch hier bei einer bestimmten Beleuchtung bei gewissen Schalen gl"anzender als bei andern. An einer Platte der zweiten Ab"anderung findet sich eine platt-zylinderf"ormige, durch die Dicke der Platte hindurchgehende Masse von Troilit, die feinen K"ornchen von Nickeleisen (?) eingemengt enth"alt, und mit einer H"ulle einer auch nach dem "atzen metallisch gl"anzend bleibenden, spiesgelben Substanz umgeben ist.

12) Cosby Creek, Coke County, East-Tennessee, V. St. 1840. Mehrere auf der Oberfl"ache oxydierte oktaedrische Bruchst"ucke von Hrn. Prof. Troost als Geschenk erhalten, zugleich mit einzelnen kleinen St"ucken Grafit und Troilit aus demselben. Die St"ucke gleichen der ersten Ab"anderung des Arva-Eisens au"serordentlich, die Menge des eingemengten Schreibersits ist an einem St"ucke noch etwas gr"o"ser. An einem St"ucke Grafit ist etwas Troilit eingemengt.

Verschieden von diesem Meteoreisen ist ein anderes, welches Hr. Ehrenberg ohne n"ahere Angabe des Fundorts als aus Tennessee von Hrn. C. T. Adae in Cincinnati Ver. St. erhalten\footnote{\frakfamily{Vergl. Monatsber. der k. Pr. Akad. d. Wiss. von 1861, S. 517.}} und dem mineralogischen Museum "ubergeben hatte. Den Erscheinungen der "atzung nach w"urde sich dieses vielmehr den grobk"ornigen Ab"anderungen, wie der von Seel"asgen anreihen lassen. Dies erkannte auch Hr. Shepard, als er das St"uck im hiesigen Museum sah, wusste es aber doch keinem anderen bekannten Meteoreisen der Ver. Staaten zu vergleichen, daher ich dies St"uck nur als Anhang hier auff"uhre.

13) Sarepta, Gouv. Saratow, Russland 1854. Drei St"ucke, eine gro"se Platte, die 8 Zoll lang, 3 1/2 Zoll breit und einen Zoll dick ist, ein kleines St"uck und eine 2 Zoll lange und 1/4 Zoll dicke Platte. Die erste Platte ist am Rande rund herum mit nat"urlicher Oberfl"ache begrenzt und l"asst hier auf der einen Seite die runde W"olbung der gefundenen Masse sehr gut erkennen.\footnote{\frakfamily{Vergl. die Abbildung und Beschreibung derselben in den Sitzungsberichten der k. Akad. der Wiss. von 1862 B. 46. Das in der Wiener Sammlung befindliche St"uck dieses Eisens, wovon Haidinger seiner Abhandlung einen so vortrefflich geratenen Abdruck hinzugef"ugt hat, ist die H"alfte einer "ahnlichen Platte, wie die des Berliner Museums.}} Die nat"urliche Oberfl"ache hat stellenweise ganz das Ansehen des Braunauer Eisens, und ist hier nur mit einer d"unnen Rinde von Eisenoxyd-oxydul bedeckt, daher die Masse nicht lange Zeit in dem Boden, wenigstens nicht in einem feuchten gelegen haben kann. --- Das zweite St"uck hat keine, die kleine Platte an den R"andern nur zum Teil nat"urliche Oberfl"ache. Die drei St"ucke zeigen aber vollkommen denselben Unterschied wie das Arva-Eisen. W"ahrend die beiden ersten St"ucke der ersten Ab"anderung des Arva-Eisens gleichen, enth"alt die dritte Platte gar keinen Schreibersit, dagegen den Rhabdit in gro"ser Menge.\footnote{\frakfamily{Die St"ucke sind indessen unzweifelhaft von derselben Masse. Dieselbe wurde in der Kalm"uckensteppe in der N"ahe der Herrnhuterkolonie Sarepta an der Wolga gefunden, und von Hrn. Glitsch nach Moskau gesandt, wo Hr. Dr. Auerbach sie abformen und sodann zerschneiden lie"s. Mehrere St"ucke wurden darauf zum Verkaufe nach Herrnhut gesandt; von hier, und zwar durch Hrn. M"oschler habe ich die beiden ersten St"ucke, die dritte Platte von Hrn. Auerbach selbst erhalten.}} Die Schalen sind bei diesem St"ucke noch dicker als bei den beiden andern, und die eine Schnittfl"ache derselben hat ganz das Ansehen einer Schnittfl"ache des Seel"asgen-Eisens, und scheint hier ganz aus unregelm"a"sigen Zusammensetzungsst"ucken zu bestehen. Einzelne nach der "atzung gl"anzend bleibende K"ornchen, die hier eingewachsen sind, haben eine lichte speisgelbe Farbe, wie die Substanz, die bei dem Eisen von Bohumilitz den Grafit umgibt. Die "atzelinien sind "uberall sehr deutlich Bei der gro"sen Platte findet sich eine "uber einen halben Zoll gro"se Troilitpartie, die aus d"unnschaligen Zusammensetzungsst"ucken ohne merkliche Umh"ullung durch eine andere Substanz besteht.

14) Sevier County, Tennessee V, St., 1840. Eine 3 1/2 Zoll lange schmale Platte. Sie enth"alt Schreibersit in ungef"ahr gleicher Menge wie die ersten Ab"anderungen: der beiden vorigen Meteoriten, doch sind die Schalen des Meteoreisens etwas schm"aler. Die Platte wurde von Hrn. v. Reichenbach in Tausch erhalten. Shepard und Haidinger f"uhren dieses Eisen nicht besonders auf, sondern vereinigen es mit dem Cosby-Eisen. Wesentliche Unterschiede in dem Ansehen der beiden Meteoriten finde ich nicht, doch habe ich die von Reichenbach angenommene Trennung einstweilen beibehalten.

15) Rio Bemdegó, Capitanie Bahia, Brasilien 1816. Kleines St"uck, Geschenk von Al. v. Humboldt, der es von A. F. Mornay erhalten hatte. Die Schalen zeigen "atzunglinien und Rhabdit, keinen Schreibersit.

16) Schwetz, Reg.-Bezirk Marienwerder, Preu"sen 1850. Das Berliner Museum, durch den Baurat Knoblauch, auf die bei dem Durchstich eines Sandh"ugels f"ur die Ostbahn gefundene Eisenmasse, und deren wahrscheinlich meteorischen Ursprung aufmerksam gemacht, erhielt durch den Geh. Rath Wernich fast die ganze der etwas "uber 43 Pfund schweren Eisenmasse, Sie war, als sie vom Museum erhalten wurde, schon in 3 St"ucke zerschlagen, wobei der Sprung nat"urlichen Kl"uften folgte. Das gr"o"ste jetzt noch im Museum befindliche St"uck, das noch 10 Pfund 1,21 Loth wiegt, ist zum Teil von der nat"urlichen Oberfl"ache, zum Teil von der Kluftfl"ache begrenzt, die an einer Seite unter spitzem Winkel zusammensto"sen. Die erstere ist sehr oxydiert, die letztere matt und schwarz, sonst aber wohl erhalten; sie ist hakig und zeigt schon die schalige Zusammensetzung der Masse sehr deutlich. Dieselbe gibt beim "atzen sehr sch"one Widmanst. Figuren, wie in dem Abdruck , den ich meiner Beschreibung dieses Eisens in Pongendorffs Ann.\footnote{\frakfamily{B. 83, S. 594.}} hinzugef"ugt habe, zu sehen ist. Die Querschnitte der Schalen sind darin etwas krumm, was wohl eine Folge, der beim Zerschlagen der Masse angewandten Gewalt ist. In den ge"atzten Schalen des Meteoreisens sieht man die "atzelinien sowie auch den Rhabdit sehr deutlich. Troilit ist nur in kleinen Partien eingemengt. An einem St"ucke der Sammlung findet sich mitten im Eisen ein kleines Korn von Chromeisenerz. Schreibersit ist nicht zu sehen.

17) Ruffs Mountain, Newberry, S"ud-Carolina, Ver. St. 1850. Eine dicke Platte, an einer Seite mit nat"urlicher Oberfl"ache. Sehr merkw"urdig durch die k"ornige Beschaffenheit der Eisenschalen, wie schon oben S. 35 bemerkt. Die k"ornigen Zusammensetzungsst"ucke einer und derselben Schale sind bei einer bestimmten Beleuchtung teils gl"anzend, teils matt; aber es sind nicht einzelne Zusammensetzungsst"ucke, die auf diese Weise miteinander abwechseln, sondern meistenteils immer wieder k"ornige Partien, eine schwer zu erkl"arende Erscheinung. Schreibersit in nicht gro"ser Menge eingemengt.

18) Seneca River, New York, Ver. St. 1851. Kleines St"uck. Die Meteoreisenschalen ebenfalls k"ornig, in den dickeren Schreibersit. Mit d"unnen T"anitbl"attchen ausgef"ullte Zwischenfelder h"aufig.

19) Toluca-Thal, Mexico 1784. Von den vielen Eisenmassen, die in dem fast 3 Meilen langen Raume des Toluca-Thales von einigen Unzen bis zu mehreren Ctrn. schwer, gefunden sind, und wahrscheinlich alle von einem Falle herr"uhren,\footnote{\frakfamily{Vgl. Burkart in Leonhard u. Bronns n. Jahrb. f. Mineral. etc. von 1856, S. 297.}} besitzt das Berliner Museum mehrere sch"one St"ucke; eine gro"se Platte, einen Fu"s lang, 7 Zoll breit und 10 Pfund 12,8 Loth wiegend, von Hrn. Shepard erhalten; eine kleinere Platte 5 1/2, Zoll lang und beinahe 4 Zoll breit, von Hrn. G. A. Stein erhalten, der die 230 Pfund schwere Masse, von der sie abgeschnitten, selbst aus Mexiko mitgebracht halte; ein "uberall mit Rinde umgebenes, gr"o"seres St"uck von der Form einer plattgedr"uckten Kugel, von Hrn. Krantz erhalten, vor ihrer Teilung in 2 platte St"ucke 3 Pfund 1,814 Loth schwer; ein kleineres, mehr zylinderf"ormiges St"uck 24,93 Loth schwer, von dem k. Pr. Minister-Residenten in Washington Hrn. von Gerolt zum Geschenk erhalten, und mehrere andere kleinere St"ucke, darunter eins aus der Chladnischen Sammlung, bei welchem freilich nur als Fundort Mexico angegeben war, das "uber den durch "atzung erhaltenen Figuren nach hierhergeh"ort. Alle diese St"ucke verhalten sich bei der "atzung sehr "ahnlich; sie geben alle sehr sch"one, "uberall gleiche Widmanst. Figuren und beweisen dadurch, dass sie alle einem Falle angeh"oren. Die Schalen sind mehr als liniendick, sehr gerade und deutlich gestreift. Der Rhabdit ist darin sehr fein, in den meisten St"ucken der Sammlung sieht man gr"o"stenteils nur die Querschnitte der kleinen Kristalle, was ein geflecktes Ansehen der Schalen hervorbringt. Bei gewisser Beleuchtung sind dieselben 1heils gl"anzend teils matt, wie besonders bei dem St"ucke des Dr. Krantz, und zwar sind auch hier nach allen Richtungen laufende Schalen teils gl"anzend, teils matt, Schreibersit findet sich nicht, dagegen noch Partien von Troilit und Grafit. Die gro"se Platte enth"alt ersteren in gro"ser Menge und in Partien, die "uber 2 Zoll lang sind. Gew"ohnlich kommen Troilit und Grafit miteinander verbunden vor, doch finden sie sich auch voneinander getrennt, und ebenso kommen beide mit einer H"ulle einer speisgelb gef"arbten, metallischen Substanz umgeben vor, finden sich aber auch ohne dieselbe. Das Steinsche St"uck enth"alt den Troilit nur in kleineren Partien; Grafit habe ich darin nicht bemerkt. Dass in diesem Eisen zuweilen auch Quarz in sehr kleinen Kristallen vorkommt, zeigt das St"uck im Besitz des Dr. Nagel in Berlin (vgl. oben S. 42), wo er allerdings nur in der oxydierten Rinde beobachtet ist, und endlich haben W"ohler und v. Reichenbach in einem von Stein erhaltenen, urspr"unglich 19 Pfund wiegenden St"ucke etwas Olivin beobachtet, der indessen in den St"ucken der Berliner Sammlung nicht zu sehen ist. Das Krantzsche und Geroltsche St"uck, welche beide rundum mit nat"urlicher Oberfl"ache begrenzt sind, zeigen auf ihr dieselben runden Vertiefungen wie das Braunauer Eisen, die Oberfl"ache ist indessen meistenteils schon in Eisenoxydhydrat umge"andert, und die durch Schmelzung entstandene Rinde von Magneteisenerz nur stellenweise erhalten. Dass an der Oberfl"ache des Toluca-Eisens Krantz kleine Oktaeder von Magneteisenerz beobachtet hat, ist oben angegeben (S. 43).

Das Toluca- Eisen ist vielfach chemisch untersucht. Die Analysen, die in W"ohlers Laboratorium nach denselben Methoden mit Teilen von sehr verschiedenen St"ucken angestellt sind,\footnote{\frakfamily{Vgl. Sitzungsb. der math.-naturw. Kl. d. k. Akad. d. Wiss. von 1856 B. 20, S. 217 etc.}} haben kein v"ollig "ubereinstimmendes Resultat gegeben; die Analysen von Uricoechea und Pugh z. B. Abweichungen im Nickelgehalt von 5,02 bis 9,05 pC., doch bemerkt dar"uber schon W"ohler, dass bei der gemengten Beschaffenheit des Meteoreisens, wovon immer nur ein sehr kleiner Teil untersucht werden k"onnte, dies kein Beweis w"are, dass die analysierten St"ucke von verschiedenen Meteoriten ber"uhrten. Die "ubereinstimmung der Widmanst, Figuren beweist in diesem Fall besser den gleichen Ursprung aller dieser Massen als. die chemische Analyse. Sind aber die vielen in dem Toluca-Thal gefundenen Eisenmassen eines und desselben Ursprungs, so geh"oren sie einem der bedeutendsten Meteoritenf"alle an, von denen man Kunde hat. Bemerkenswert ist, dass Pugh nur in der oxydierten Rinde, die er auch untersucht hat, Kiesels"aure, und zwar 7,47 pC. gefunden hat, In dem R"uckstande von Teilen der vom Dr. Stein mitgebrachten St"ucke fanden Uricoechea und Pugh auch K"orner von Olivin und einigen anderen nicht bestimmten rubinroten und himmelblauen Mineralien. Dergleichen fand auch B"oking, ohne sie n"aher zu bestimmen, und au"serdem noch etwas Grafit.

20) Misteca, im Staate Oajaca, Mexico, 1834. Ein parallelepipedisches St"uck von 2 Pfd. 13,7 Lth., Abschnitt von einem, dem Geh. Bergrath Burkart in Bonn geh"origen St"ucke.\footnote{\frakfamily{Vgl. Burkart a. a. O. S. 305.}} Es ist von zwei Schnittfl"achen, einer Bruchfl"ache und im "ubrigen von nat"urlicher Oberfl"ache begrenzt. Die Schlifffl"achen zeigen sehr sch"one Widmanst, Figuren, die denen des Toluca-Eisens sehr "ahnlich, doch dadurch ausgezeichnet sind, dass auf ihnen die eingewachsenen Rhabdit-Kristalle viel gr"o"ser und zahlreicher sind. Die "atzunglinien sind sehr schwach und nur an manchen Stellen sichtbar. Troilit immer in sehr kleinen Partien, doch findet sich auf der einen Schnittfl"ache eine kleine, krumm zylinderf"ormige H"ohlung, worin Troilit gesessen zu haben scheint. Eine gr"o"sere runde Rinne auf der Oberfl"ache war vielleicht fr"uher auch damit gef"ullt. Die Bruchfl"ache ist schwach angelaufen, doch sind darauf die Schalen des Meteoreisens noch gut zu sehen. Die nat"urliche Oberfl"ache ist an einer Stelle durch Hammerschl"age verletzt, an andern zeigt sie die d"unne Rinde von Magneteisenerz noch deutlich.

21) Sierra blanca, unweit Villa nueva de Huajuquilla in Mexiko, 1784; ein St"uck aus der Sammlung des Medizinal-Raths Bergemann,\footnote{\frakfamily{Vergl. Burkart a. a. O. S. 278.}} 15,28 Loth schwer. Die Widmanst. Figuren sind denen des Toluca-Eisens sehr "ahnlich, die "atzunglinien deutlicher, der Rhabdit undeutlicher, kein Schreibersit.

22) Tula (Netschaevo) Russland 1856. Ein drei Zoll langes, ziemlich quadratisches Prisma mit lauter Schnittfl"achen bis auf eine der Endfl"achen, die nat"urliche Oberfl"ache ist; von Hrn. Dr. Auerbach in Moskau erhalten. Die Widmanst. Fig. sehr sch"on, die Schalen so dick wie im Toluca-Eisen, die "atzunglinien fein, Rhabdit nicht recht sichtbar. Auf einer der Seiten des Prismas sieht man den Durchschnitt einer Einmengung, die sich von einem Ende, fast 5 Linien breit, bis fast zum andern fortzieht, wo sie sich auskeilt, Sie besteht aus einem Gemenge von Nickeleisen mit einem harten, schwarzen, matten und undurchsichtigen Silicate. Haidinger\footnote{\frakfamily{Sitzungsberichte der math.-naturw. Kl. d. k. Akad. d. Wiss. von 1860.}} hat von dem St"ucke der Wiener Sammlung, das mehrere dergleichen Einmengungen enth"alt, einen Abdruck gegeben, und Auerbach\footnote{\frakfamily{Poggendorffs Ann. von 1863, B. 118, S. 363.}} hat neuerdings das Silicat analysiert. Er fand darin:
\begin{center}
\begin{tabular}{ l r }
    Eisenoxydul & 31,64\\
    Magnesia & 16,63\\
    Kalk & 0,77\\
    Natron & 1,22\\
    Kali & 0,20\\
    Tonerde & 9,93\\
    Kiesels"aure & 37,76\\
\end{tabular}
\end{center}
h"alt es aber selbst noch f"ur ein Gemenge, da es sich von Chlorwasserstoffs"aure nur zum Teil zersetzen l"asst.
\paragraph{}
Was es mit dieser Einmengung f"ur eine Bewandtnis hat, ist schwer zu sagen. Dergleichen Einmengungen sind bei andern Eisenmeteoriten noch nicht vorgekommen. Da die urspr"unglich an 600 russische Pfunde schwere Masse, um sie zu zerkleinern, in ein Schmiedefeuer gebracht worden ist, so k"onnte man glauben, dass die Einmengung sich erst durch die Behandlung im Feuer gebildet hat, doch ist das Gemenge des Silicats und des Eisens in demselben so fein, die chemische Zusammensetzung des Silicats durch den gro"sen Magnesiagehalt so verschieden von den gew"ohnlichen Eisenschlacken, die Widmanst. Fig. in dem Eisen sind so regelm"a"sig, dass diese Annahme doch ihre Schwierigkeit hat.

23) Robertson County, Tennessee, V. St. 1861. Kleines St"uck, die Widmanst. Fig. "ahnlich denen des Tula-Eisens.

24) Carthago, Smith County, Tennessee, V. St. 1844. Ein gro"ses, 1 Pfd. 16,44 Lth. schweres St"uck, in Form eines dreiseitigen Prismas, von Hrn. von Reichenbach erhalten. Zwei Seitenfl"achen sind Schnittfl"achen, die dritte eine nat"urliche Kluftfl"ache. Die Widmanst. Fig. sind sehr sch"on, "ahnlich denen des Toluca-Eisens. Troilit ist in kleinen Partien eingemengt. Er hat stets eine kleine Einfassung von Meteoreisen, jenseits welcher erst die Widmanst. Fig. anfangen. In der einen Partie von Troilit ist ein kleines Korn von Chromeisenerz eingemengt (s. oben S. 39).

25) Burlington, Otsego County, New York, Ver. St. 1819. Mehrere kleine St"ucke, Widmanst. Fig. wie Toluca.

26) Marshall County, Kentucky, V. St. 1856, sch"one Widmanst. Fig.; darin ziemlich viel Schreibersit.

27) St. Rosa bei Tunja, Columbien, ein ganz kleines St"uck, 0,025 Loth schwer, das von Prof. Karsten aus Columbien mitgebracht ist. So klein es ist, so konnte ich doch noch eine Fl"ache anschleifen lassen und "atzen, und daran ganz deutliche Widmanst. Fig. erkennen.

28) Orange Fluss, S"ud-Afrika, 1856, d"unne Platte, "ahnlich dem von Marshall Cty, auch mit etwas Schreibersit.

29) Texas (Red River), Ver. St. 1814. Eine 4 Zoll lange, d"unne Platte. Geschenk von Hrn. B. Silliman. Sch"one Widmanst. Figuren. Die Schalen etwas schm"aler als bei Toluca. Ein wohlgeratener Abdruck von einer ge"atzten Platte des Texas-Eisens befindet sich in Sillimans Journal.

30) Lenarto, Scharosch, Ungarn, 1815. Dicke der Schalen wie bei Texas; sehr feine "atzlinien, in einigen St"ucken ziemlich viel Schreibersit, in andern weniger.

31) Durango, Mexico, 1804. Ein gr"o"seres St"uck, 1 Pfd. 2,67 Lth. schwer, und mehrere kleinere, Geschenk von Al. v. Humboldt. Die Widmanst. Fig. noch schm"aler als bei dem Texas-Eisen. Troilit in kleinen Partien eingemengt.

32) Werchne-Udinsk, West-Sibirien, 1854. Eine dicke Platte, 1 Pfd. 4,17 Lth. schwer, an den R"andern bis auf eine Schnittfl"ache mit nat"urlicher Oberfl"ache begrenzt. Letztere ist uneben und mit nur sehr d"unner Rinde von Magneteisenerz bedeckt, das in kleinen Vertiefungen, "uberall aber undeutlich kristallisiert ist. Die Hauptfl"achen zeigen ge"atzt sehr sch"one Widmanst. Fig. Troilit in kleinen Partien hier und da eingeschlossen.

33) Elbogen, B"ohmen, 1811. Ein 9,92 Lth. schweres St"uck, eine d"unne Platte und ein aus der Masse geschmiedetes Federmesser; s"amtliche St"ucke aus der Chladnischen Sammlung. Die Widmanst. Fig. desselben sind aus dem sch"onen Abdruck in die Werke des Hrn. von Schreibers bekannt.

34) Nebraska Territory, Ver. St. 1856.

35) Madoc, Ober-Canada 1854.

36) Black mountain, Buncumbe Cty, Nord-Carolina, V. St. 1835.

37) Caille, Grasse, Var, Frankreich, 1828. Eine drei Zoll lange, 5,69 Lth. schwere Platte, Die Widmanst. Fig. sehr geradlinig und sch"on.

38) Agram (Hraschina), Kroatien, gefallen den 26. Mai 1751.

39) Ashville, Buncumbe County, Nord-Carolina, Ver. St. 1839. Sch"one Widmanst. Fig., aber sehr schnell rostend.

40) Guildford, Nord-Carolina, Ver. St. 1830.

41) L"owenfluss, Gro"s Namaqualand, S"ud-Afrika 1853.

42) Lockport (Cambria), New-York, V. St. 1845. Eine fast zwei Zoll lange Platte. Feine Widmanst. Fig. mit vielen rundlichen, haselnussgro"ssen Partien von Troilit, die oft zusammengeh"auft, aber stets mit einer d"unnen H"ulle einer lichten speisgelben Legierung umgeben sind.

43) Jewell Hill, Madison County, Nord-Carolina, Ver. St. 1856. Eine "ahnliche, nur dickere Platte: ebenfalls sehr feine Widmanst. Fig.

44) Oldham County (Lagrange), Kentucky, Ver. St. 1856. Zwei gro"se, dicke Platten, an den R"andern zum Teil mit nat"urlicher Oberfl"ache begrenzt, und eine kleinere Platte. Die Widmanst. Fig. sehr fein.

45) Putnam County, Georgia, V. St. 1854. Die Widmanst. Fig. sehr gerade, fein und zierlich. In dem kleinen St"ucke findet sich ein Einschluss von Troilit, in welche kleinen Partien von Nickeleisen eingemengt sind.

46) Tazewell, Claiborne County, Tennessee, Ver. St. 1854. Die Widmanst. Fig. sehr geradlinig und in ihnen d"unne Streifen von Schreibersit, die sich aber merkw"urdiger Weise nur in den Schalen einer Richtung finden, was das Meteoreisen von Tazewell vor allen "ubrigen Ab"anderungen auszeichnet. Auch hier kleine Einmengungen von Troilit, der mit einer speisgelben metallischen Einfassung umgeben ist.
\vspace{\medskipamount}
\paragraph{}
\emph{d)} Meteoreisenmassen, die Aggregate gro"sk"orniger, schalig zusammengesetzter Individuen sind.
\vspace{\medskipamount}
\paragraph{}
47) Zacatecas, Mexico, 1792. Ein Meteoreisen von sehr eigent"umlicher Struktur, die nur in einem gr"o"seren St"ucke erkannt werden kann. Das Berliner Museum besitzt davon neben mehreren kleinen St"ucken eine ungef"ahr rechtwinklige zolldicke Platte, die ungef"ahr 3 Zoll breit und 3 1/2 Zoll lang, von einem gr"o"seren St"ucke abgeschnitten ist, das der Geh. Bergrath Burkart aus Mexico selbst mitgebracht hat.\footnote{\frakfamily{Vergl. N. Jahrbuch f"ur Min. etc. von Leonhard und Bronn von 1856, S. 288.}} Die Seiten der Platte werden zum Teil durch nat"urliche Oberfl"ache gebildet. Auf der ge"atzten Hauptfl"ache, die in Fig. 1 Taf. 2 dargestellt ist, sieht man, dass dies Meteoreisen aus grobk"ornigen Zusammensetzungsst"ucken besteht, die "uber zollgro"s und unregelm"a"sig begrenzt sind, und selbst wieder aus Zusammensetzungsst"ucken bestehen, die parallel den Fl"achen des Oktaeders liegen, wie bei dem Meteoreisen, das Widmanst. Fig. gibt. Diese schaligen Zusammensetzungsst"ucke sind auch nicht sehr regelm"a"sig begrenzt, ihre Richtung ist aber doch sehr gerade, was man aus dem eingemengten Schreibersit sehen kann, der in der Mitte derselben enthalten ist, und auf der Schnittfl"ache oft fast zusammenh"angende, wenn auch meistenteils sehr d"unne Streifen bildet. Unter dem Mikroskop sieht man jedoch, dass diese Streifen stets aus einzelnen St"ucken bestehen, die in einer oder mehreren Reihen nebeneinander liegen. Die einzelnen St"ucke sind zum Teil regelm"a"sig begrenzt und liegen in paralleler Stellung nebeneinander, sind also unvollkommen ausgebildete Kristalle. Fig. 2 stellt die Stelle bei \emph{a} (Fig. 1), Fig. 3 bei \emph{b} (Fig. 1) bei 90-maliger Vergr"o"serung in einen der gr"o"sten Zusammensetzungsst"ucke der Platte dar. In Fig. 1 sind die Schalen, woraus die Zusammensetzungsst"ucke bestehen, nur unvollst"andig wiedergegeben, da sie sehr fein sind, der Schreibersit ist aber m"oglichst vollst"andig dargestellt. Die Schalen zeigen feine, doch deutliche "atzlinien; eingewachsene Kristalle von Rhabdit kommen nicht vor, wenn auch der eingemengte Schreibersit in manchen Zusammensetzungsst"ucken so h"aufig und fein ist, dass man ihn leicht damit verwechseln kann. Auch hier sieht man bei einer bestimmten Beleuchtung einen Teil der schaligen Zusammensetzungsst"ucke gl"anzen, einen andern nicht, der dann bei anderer Beleuchtung gl"anzt, w"ahrend es nun bei dem ersten nicht der Fall ist. Troilit in kleinen unregelm"a"sigen Partien ist h"aufig eingemengt und in den Figuren durch schwarze Farbe bezeichnet; er hat h"aufig eine d"unne, metallisch gl"anzende Einfassung, die, soweit man es erkennen kann, von dem Schreibersit nickt verschieden erscheint; auch kleine Partien von Grafit kommen darin vor. Auf der nat"urlichen Oberfl"ache sieht man zwei runde, rinnenf"ormige Eindr"ucke, der eine davon 3 Zoll lang und 3/8 Zoll breit, die Baron Reichenbach, der sie hier gesehen, f"ur H"ohlungen h"alt, die durch ausgewitterten Troilit entstanden sind.
\vspace{\medskipamount}
\paragraph{}
\emph{e)} Meteoreisenmassen, welche Aggregate feink"orniger Zusammensetzungst"ucke sind.
\vspace{\medskipamount}
\paragraph{}
Diese Eisenmassen erscheinen im Bruche fein- bis kleink"ornig, zeigen geschliffen und ge"atzt keine Widmanst. Fig., doch treten dann nadel- oder tafelf"ormige Kristalle von Rhabdit oder Schreibersit hervor, die gew"ohnlich gar keine regelm"a"sige Lage haben. Troilit ist in einzelnen gr"o"seren oder kleineren Partien eingemengt.

48) St. Rosa (Tocavita) bei Tunga, 20 spanische Meilen NO. von Bogota in Columbien 1823.

49) Rasgatà, bei der Saline Zipaquira bei Bogota, Columbien 1823. Ich f"uhre beide Eisenmassen zusammen auf, weil beide nach den St"ucken des Berliner Museums die gr"o"ste "ahnlichkeit miteinander haben. Das St. Rosa-Eisen wurde mir 1824 bei meinen Aufenthalten in Paris von Al. von Humboldt, der sie selbst von Boussingault erhalten, f"ur das Berliner Museum "ubergeben. Es waren urspr"unglich 2 vollst"andige kleine Eisenmassen von platt-kugelf"ormiger Gestalt, wenn auch mit vielen flachen Vertiefungen auf der Oberfl"ache. Von beiden wurden sp"ater einige St"ucke abgeschnitten; sie wiegen jetzt noch 35,26 und 29,93 Loth; ein kleines von dem einen abgeschnittenes St"uck ist 3,04 Loth schwer. Die Oberfl"ache derselben ist sehr oxydiert, und das Eisen hier in Eisenoxydhydrat umge"andert. Das Eisen selbst ist au"serordentlich hart, im Bruche feink"ornig und nimmt geschliffen, eine sehr gute Politur an. Ge"atzt wird es im Allgemeinen matt, des eine St"uck etwas fleckig; man sieht nur mit der Lupe kleine runde oder vielmehr noch in die L"ange gezogene Erhabenheit, die auf ihrer H"ohe noch kleinere runde, l"angliche, oft ganz linienartige, gl"anzend gebliebene Teile zeigen.

Von dem Eisen von Rasgatà besitzt das Museum eine 4,79 Loth schwere Platte, die an den Seiten von 2 Schnittfl"achen, im "ubrigen von der nat"urlichen Oberfl"ache begrenzt ist. Das Berliner Museum hatte sie vom Direktor Partsch aus dem Wiener Kabinette erhalten, wo es von einem St"ucke abgeschnitten war, das aus der Meteoritensammlung von Heuland stammt, der es selbst von Mariano de Rivero erhalten hatte.\footnote{\frakfamily{Partsch Meteoriten, S. 127.}} Die ge"atzte Schnittfl"ache gleicht denen der vorigen St"ucke; die kleinen gl"anzend gebliebenen Teile sind vielleicht noch h"aufiger und vorzugsweise linienartig, zum Teil auch untereinander parallel. Sie sind meistenteils sehr fein, liegen aber wie bei den vorigen St"ucken auf schon etwas erhabenen Teilen der Grundmasse.

Die St"ucke stammen demnach s"amtlich von Boussingault und Mariano de Rivero, weiche beide zusammen an den Fund"ortern derselben waren und dar"uber die erste Nachricht gegeben haben.\footnote{\frakfamily{Ann. de Chimie 1824, t. 25, p. 438.}} Sie sahen in St. Rosa bei Tunja, 20 spanische (?) Meilen NO. von Bogota, eine gro"se Eisenmasse, deren Gewicht sie auf 750 Kilogramme sch"atzten, bei einer Schmiede, der sich ihrer als Amboss bediente. Dieselbe hatte sich auf einem H"ugel Tocavita, 1/4 Meile von St. Rosa, mit anderen kleineren St"ucken in der N"ahe gefunden. Andere Eisenmassen sahen sie in dem Dorfe Rasgatà in der N"ahe der Saline Zipaquira bei Bogota, darunter Massen von 41 und 22 Kilogramme.

Das von Prof. Karsten mitgebrachte und oben (S. 64) erw"ahnte Eisen von Santa Rosa ist ein St"uck der gro"sen Eisenmasse, die bei der Schmiede liegt. Prof. Karsten ist zwar nicht selbst in Santa Rosa gewesen, sondern hatte die St"ucke von einem Bewohner Santa Rosas erhalten, den er in Bogota kennen gelernt hatte. Da demselben das Vorhandensein der Eisenmasse bei der Schmiede wohl bekannt war, so hatte Prof. Karsten ihm eine Metalls"age mitgegeben, um mittelst derselben ein St"uckchen von der Eisenmasse abzul"osen und ihm dasselbe bei seiner R"uckkehr nach Bogota zu bringen, was er nun auch getan hatte, wenn auch das St"uck bei der H"arte der gro"sen runden Eisenmasse, der er nicht recht beikommen konnte, nur sehr klein ausgefallen ist. Prof. Karsten hatte aber keinen Zweifel an der Echtheit des St"uckes; da es nun aber ganz anderer Art ist, als die St"ucke von St. Rosa, die Boussingault an Humboldt gegeben, so m"usste man annehmen, dass die verschiedenen auf dem H"ugel Tocavita bei St. Rosa gefundenen Eisenmassen eine so gro"se Verschiedenheit der Struktur gezeigt haben. Bei der "ahnlichkeit der Boussingaultschen St"ucke mit denen von Rasgatà k"onnte man an eine Verwechselung der Fundorte denken, die indessen anzunehmen ich keine Berechtigung habe. Vielleicht k"onnte noch Hr. Boussingault selbst dar"uber Auskunft geben. Auch w"urde es w"unschenswert sein, noch andere Beschreibungen der St"ucke von St. Rosa und Rasgatà zu erhalten. --- Dass W"ohler bei der Aufl"osung des Eisens von Rasgatà eine Menge kleiner mikroskopischer harter Steinchen als R"uckstand neben dem Schreibersit erhielt, ist schon oben angef"uhrt (vgl. oben S. 41).

50) Chesterville, S"ud-Carolina, Ver. St. 1849. Drei St"ucke, eins davon in zwei St"ucke zerbrochen, urspr"unglich 14,93 Loth, ein zweites 8,92 Loth schwer, alle tafelartig geschnitten, eins noch zum Teil mit nat"urlicher Rinde, die schwarz, d"unn und uneben ist. Das zerbrochene St"uck zeigt einen feink"ornigen Bruch mit stahlgrauer, ins Bleigraue sich ziehender Farbe. In der feink"ornigen Masse liegen etwas gr"o"sere, gl"anzende K"orner. Die ge"atzte Schnittfl"ache ist matt, mit der Lupe sichte man darauf lauter kleine rundliche Erhabenheit, und dazwischen einzelne unregelm"a"sig gestaltete, oder runde, auch ganz geradlinige, gl"anzende Teile. Bei dem durchgebrochenen St"ucke eins der Bruch gerade durch ein darin befindliches St"uck Troilit von Haselnussgro"ssen, so dick wie die Platte selbst; derselbe ist braun, feink"ornig und wird von einer d"unnen H"ulle gl"anzender K"orner umgeben, die etwas gr"o"ser sind als die "ubrigen gl"anzenden Teile der Masse.

51) Tucuman (Otumpa), Argent. Rep. in S"ud-Amerika, 1788. Das vierte der oben S. 51 erw"ahnten St"ucke mit feink"ornigem Bruch.

52) Senegal im Lande Siratik und Bambuk, Afrika, 1763. Ein 4,462 Loth schweres St"uck, zum Teil mit zwei Schnittfl"achen und einer Bruchfl"ache, zum Teil mit der nat"urlichen Oberfl"ache begrenzt. Letztere ist uneben und schwarz, die Bruchfl"ache ist etwas gr"ober k"ornig als bei Chesterville; die ge"atzte Schnittfl"ache zeigt auch die feine rundliche Erhabenheit von Chesterville, au"serdem aber andere d"unne, geradlinige, oft 2 Linien lange, die unregelm"a"sig durcheinanderlaufen, ohne sich zu schneiden, aber nur schwach gl"anzen, etwa wie bei Chesterville die erhabenen Teile in der n"achsten Umgebung der stark gl"anzenden.

53) Salt River, Kentucky, V. St. 1851. Eine d"unne, quadratzollgro"se, 1,197 Loth schwere Platte. Graue matte Grundmasse, worin h"aufige lichtere l"angliche Teile schon etwas regelm"a"sig nach den Seiten eines ungef"ahr gleichseitigen Dreiecks liegen, in deren Mitte sich gl"anzende Teile befinden, die auch meistenteils von einer l"anglichen Form sind. Die eine H"alfte der ge"atzten Schnittfl"ache ist bei einer bestimmten Beleuchtung von lichter grauer Farbe, die andere dunkler; bei anderer Beleuchtung umgekehrt, die erste H"alfte dunkel und die andere licht. Das Eisen ist dem vom Senegal zu vergleichen, unterscheidet sich aber durch die regelm"a"sige Lage der lichtern Teile, die sehr r"atselhaft ist.\footnote{\frakfamily{Um die bei diesen wie den vorigen Meteoriten angegebenen Erscheinungen zu sehen, ist es n"otig, dass die Schnittfl"ache gut poliert und dann nur schwach ge"atzt wird; bei starker "atzung bildet sich nur eine k"ornige, graue Fl"ache, an der nichts weiter zu erkennen ist.}} Es w"are wichtig, den Bruch zu sehen, der bei der d"unnen Platte nicht zu beobachten ist.

54) Cap der guten Hoffnung (zwischen Sonntags- und Boschmanns-Fluss), S. A. 1801. Eine dicke Platte (Fig. 9 und 10 Taf. 3) von Hrn. Prof. Breda in Harlem erhalten, und ein kleines St"uck aus der Chladnischen Sammlung. Die Platte ist ein Abschnitt der gro"sen, in dem Museum von Harlem aufbewahrten, angeblich 171 Pfd. schweren Masse.\footnote{\frakfamily{Vergl. Partsch Meteoriten, S. 132.}} Sie ist an einer Seite mit einer Schnittfl"ache, an den "ubrigen mit nat"urlicher Oberfl"ache begrenzt, die nur eine sehr d"unne braune Rinde hat. Um den Bruch zu sehen, wurde, parallel der Fl"ache \emph{AF}, an der gegen"uberliegenden Seite der Platte eine Scheibe abgeschnitten und zum Teil abgebrochen, was nicht ohne Schwierigkeiten ausgef"uhrt werden konnte, da das Eisen sehr weich und sehr dehnbar ist, und darin einen gro"sen Gegensatz mit dem Eisen von St. Rosa (S. 67) bildet. Es musste von zwei Seiten tief eingeschnitten werden, ehe sich die d"unne Scheibe abbrechen lie"s, so dass nur ein 3 Linien dicker Streifen mit Bruch entstand, an dem man jedoch deutlich seine Beschaffenheit wahrnehmen kann. Er ist ganz feink"ornig und lichte stahlgrau. Das "ubrig gebliebene, gro"se St"uck war wohl an einigen Stellen der Oberfl"ache angerostet, zeigte aber sonst keine merkliche Verschiedenheit. Als die Schnittfl"achen \emph{AF} und \emph{AC} der Platte geschliffen und poliert wurden, rosteten sie in sehr kurzer Zeit an den 3 Stellen \emph{B}, \emph{D}, \emph{G}, wie in der Fig. angegeben ist, und als sie darauf schwach ge"atzt wurden, zeigte sich der Rost auf denselben Stellen sehr bald wieder, ohne aber sp"ater merklich weiter fortzuschreiten; die "ubrigen Teile der Fl"achen wurden aber dabei merkw"urdig ver"andert. Man sieht nun verschiedene, abwechselnd lichte und dunkelstahlgraue, mehr oder weniger breite Streifen in paralleler Richtung und mit scharfer Grenze "uber dieselben fortlaufen, die ihren Farbenton umtauschen, je nachdem das Licht in der einen oder die andere Richtung auf das St"uck f"allt. H"alt man dasselbe so, dass die Kante \emph{AB} dem Beobachter zugekehrt und ungef"ahr horizontal ist, so erscheinen die Streifen so, wie sie in Fig. 9 angegeben sind: \emph{a}, \emph{c}, \emph{e}, \emph{g} sind hell, mit Ausnahme einiger feiner, dunkler Streifen in \emph{e} und \emph{g}; \emph{a}, \emph{d}, \emph{f}, \emph{h} dunkel, mit Ausnahme des hellen, feinen Streifen in \emph{d}. H"alt man dagegen die Fl"ache \emph{AC} so, dass die Streifen dem Beobachter parallel gehen, wie in Fig. 10, so sind die Streifen, die fr"uher hell waren, dunkel; dabei erscheint diese Fl"ache, wie auch schon in der fr"uheren Lage, gefleckt und auf diesen Flecken teilweise mit feinen Streifen, die die breiten, schr"ag durchsetzen, gestreift. H"alt man die Fl"ache so, dass sie etwas nach hinten geneigt ist, so ist die Stelle bei \emph{B}, wie in der Zeichnung angegeben, hell; ist die Fl"ache \emph{AC} ganz horizontal, so wird die Ecke bei \emph{B} dunkel, ist sie nach vorn geneigt, wieder hell.

Unebenheiten, die diese Ver"anderungen in dem Ton der Streifen bedingen k"onnten, sind mit den blo"sen Augen und auch kaum mit den Mikroskopen wahrzunehmen. Ich habe von der Fl"ache AC einen Hausenblasenabdruck gemacht; unter dem Mikroskop erschien der lichte Streifen \emph{h} Fig. 10 ganz gek"ornt, wie in Fig. 11; in dem dunklen \emph{g} waren diese K"orner ebenfalls, aber mit andern gemengt, die in die L"ange gezogen waren. Die die breiten Streifen durchsetzenden feinen Streifen entstehen dadurch, dass die K"orner hier gedr"angter liegen. Wie dadurch aber der Wechsel von hell und dunkel in den Streifen bewirkt wird, ist nicht einzusehen. Es m"ussten dazu noch weitere Untersuchungen angestellt, und namentlich noch Schnitte parallel und rechtwinklig auf den breiten Streifen gemacht, auch nachgesehen werden, ob die Lagen, die durch die Streifen auf den Schnittfl"achen des St"uckes Fig. 9 angezeigt werden, durch die ganze Masse des gro"sen St"uckes in dem Museum von Harlem, wovon das beschriebene abgeschnitten ist, hindurchgehen.

Die chemische Beschaffenheit dieses so eigent"umlichen Eisens ist schon mehrfach untersucht, zuletzt noch durch Uricoechea und B"oking\footnote{\frakfamily{Ann. d. Chem. und Pharm. B. 91, S. 252 und B. 95, S. 246.}} in W"ohlers Laboratorium, wodurch der gro"se Nickelgehalt desselben bis 15 pC., dem schon Holger und Wehrle gefunden hatten, best"atigt wurde. Als W"ohler im Mai d. J. in dem Berliner Museum das ge"atzte St"uck sah, veranlasste ihn das eigent"umliche Ansehen desselben, noch eine neue Analyse zu machen, mit St"ucken, die ich ihm mitteilen konnte. Er fand darin:\footnote{\frakfamily{Nach einer brieflichen Mitteilung, die ich mit seiner Erlaubnis hier bekannt mache.}}
\begin{center}
\begin{tabular}{ l r }
    Nickel & 16,215\\
    Kobalt & 0,727\\
    Phosphor & 0,148\\
\end{tabular}
\end{center}
au"serdem noch Spuren von Kupfer und Chrom, welches letztere auch schon Stromeyer darin nachgewiesen hatte. Doch ist die Untersuchung noch nicht abgeschlossen. Das schnelle Rosten an manchen Stellen setzt hier doch eine besondere Beschaffenheit voraus. --- Noch ist zu bemerken, dass man in dem Streifen \emph{g} der Fl"ache \emph{AF}, was man so selten zu beobachten Gelegenheit hat, den sechsseitigen Durchschnitt eines Kristalls von Troilit, so wie in dem untern Streifen \emph{h} etwas Schreibersit von der in der Zeichnung angegebenen Gestalt sieht.
\paragraph{}
55) Babbs Mill, Greenville, Green County, Tennessee, Ver. St. 1845. Zwei kleine flache St"ucke; bei dem gr"o"seren zwei Schnittfl"achen, die "ubrige Begrenzung nat"urliche Oberfl"ache, die in Eisenoxydhydrat umge"andert ist. Die gro"se Schnittfl"ache ist ge"atzt, matt, die eine H"alfte dunkelgrau, die andere viel heller. Beide Schattierungen verlaufen ineinander, wie dies auch bei den Flecken in dem Eisen vom Cap vorkommt; ebenso derselbe Wechsel des Farbentons bei dem Wechsel in der Lage des St"ucks; aber die geradlinigen Streifen sind in dem kleinen St"ucke nicht sichtbar; ein Bruch auch nicht wahrnehmbar. In dem kleineren St"ucke sieht man gl"anzende Einmengungen in Gestalt von gebogenen Linien, die bei dem gr"o"seren nicht sichtbar sind.

Wie in dem "au"sern Ansehen, so gleicht dies Eisen auch dem vom Cap in der chemischen Beschaffenheit. Es hat ebenfalls einen sehr hohen Nickelgehalt, der nach der von Clark in W"ohlers Laboratorium angestellten Analyse sogar noch etwas h"oher ist als bei dem Cap-Eisen. Er fand n"amlich Eisen 80,590, Nickel 17,104, Kobalt 2,037, schwerl"osliche Phosphormetalle 0,124; au"serdem noch Spuren von Mangan, Silicium und Magnesium.

56) De Kalb County, Tennessee, V. St. 1845. Zwei kleine St"uckchen, wonach dies Eisen dem der beiden vorigen Fund"orter sehr "ahnlich zu sein scheint.
\begin{center}
Anhang. 
\end{center}
\paragraph{}
57) Tucson, Sonora, Mexico, 1850. Kleines St"uckchen.

58) Cranburne, Melbourne, Australien, 1861. Drei kleine St"uckchen.
\subsection{\frakfamily{Pallasit.}}
\paragraph{}
Gemenge von Meteoreisen mit Olivin. Das Meteoreisen bildet hier eine Grundmasse, in welcher Olivin-Kristalle porphyrartig eingewachsen sind.\footnote{\frakfamily{"astig und schwammig, wie es gew"ohnlich beschrieben wird, erscheint es nur da, wo die Olivin-Kristalle herausgefallen sind, was bei der Trennung kleinerer St"ucke von Gr"o"seren mit dem Hammer h"aufig der Fall ist.}} Es geh"oren hierher die Eisenmeteorit von Krasnojarsk (das Pallas-Eisen), von Brahin, Atacama, Steinbach, Rittersgr"un, Breitenbach\footnote{\frakfamily{Den Eisenmeteorit von Breitenbach stelle ich vorl"aufig hierher, da das darin enthaltene Meteoreisen dieselben Widmanst. Fig. gibt, wie das in den Pallasiten von Steinbach und Rittersgr"un, ohne aber die neben dem Meteoreisen vorkommenden Gemengteil schon untersucht zu haben.}} und Bitburg.

Die Olivin-Kristalle sind bei dem Pallas-Eisen am sch"onsten ausgebildet. Sie sind hier 2 bis 4 Linien gro"s und zuweilen noch gr"osser, und liegen entweder ganz frei in dem Eisen oder zu mehreren nebeneinander, sich gegenseitig in der Ausbildung st"orend. Im ersteren Falle sind sie rund und n"ahern sich der Kugelgestalt mehr oder weniger; oft sind sie in die L"ange gezogen und an einem Ende konisch zulaufend, wie man an ihren Durchschnitten auf den Schnittfl"achen des Eisens, oder an den bei der Lostrennung einzelner St"ucke von der gr"o"seren Masse mit dem Hammer herausgefallenen Kristallen sehen kann. Ihre Oberfl"ache ist aber glatt und stark gl"anzend. Sie sind gelblichgr"un und vollkommen durchsichtig, so dass man auf der geschliffenen Fl"ache des Pallasit auch bei den Durchschnitten gr"o"serer Kristalle die hintere Seite deutlich erkennen kann; indessen sind sie doch h"aufig mit Spr"ungen durchsetzt, und auf diesen und in der N"ahe derselben braun gef"arbt, und dann mehr oder weniger undurchsichtig. Ungeachtet ihrer Abrundung zeigen sie aber noch einzelne Fl"achen, die sich gew"ohnlich nicht in Kanten schneiden und runde Umrisse haben, aber an den Winkeln, die sie miteinander bilden, zu bestimmen sind. Am h"aufigsten fand ich die Fl"achen des L"angsprisma \emph{k} (80$^{\circ}$ 54’) und die Zone der L"angsprismen ausgebildet, wie z. B. in Fig. 1 Taf. 4, wo diese Zone rund um den Kristall zu verfolgen ist und sich auf der einen Seite die Fl"achen \emph{P}, \emph{h}, \emph{k}, \emph{i}, \emph{T}, \emph{i}, \emph{k} (Ha"uy), auf der entgegengesetzten Seite die Fl"achen \emph{h}, \emph{T} befinden; er ist in der den Fl"achen \emph{P} und \emph{T} parallelen Axe besonders ausgedehnt.\footnote{\frakfamily{Bei \emph{d} befindet sich keine Kristall- sondern eine Zusammensetzungsfl"ache, in welcher der Kristall an einen andern angrenzte.}} An andern Kristallen fand ich aber auch die Fl"achen anderer Zonen; ja ich fand sogar einen Kristall, den ich schon bei einer fr"uheren Gelegenheit beschrieben,\footnote{\frakfamily{Pongendorffs Ann. 1825 B. 4, S. 185. Er ist indessen dort ganz vollst"andig gezeichnet, obgleich die Fl"achen nur auf einer Seite ausgebildet sind; "uber \emph{T} die Fl"achen \emph{i}, \emph{k}, \emph{P} (letztere sehr gro"s), unter \emph{T}: \emph{i'}; neben \emph{T}: \emph{r}, \emph{s}, \emph{n}, "uber diesen \emph{l}, \emph{f}, \emph{e}, \emph{d}, unter ihnen \emph{l'}, \emph{f'}, \emph{e'}.}} bei welchem mehrere Reihen von Fl"achen "ubereinander vorkommen. Wo sich mehrere Fl"achen finden oder dieselben schon einige Gr"o"se haben, schneiden sie sich auch "ofter in scharfen Kanten. So findet sich bei einem St"ucke der Sammlung ein in dem Eisen festsitzender Kristall, an welchem 2 Fl"achen sichtbar sind, die eine 3 Linien lange Kante bilden. Die Fl"achen sind gr"o"stenteils "uberaus glatt und gl"anzend, so dass sich die Kristalle zu den sch"arfsten Messungen eignen, die man bei dem Olivin anstellen kann. Nur die gerade Endfl"ache \emph{P} ist bei dem oben erw"ahnten, sehr ausgebildeten Kristalle parallel der Kante mit dem L"angsprisma gefurcht, und Fl"achen mit solchen Furchen, die also auch wahrscheinlich die Fl"achen \emph{P} sind, habe ich auf ansitzenden Kristallen noch "ofter bemerkt. Seiten sind die Kristalle um und um ausgebildet, gew"ohnlich liegen zwei oder mehrere Kristalle nebeneinander. Sie begrenzen sich dann mit Zusammensetzungsfl"achen, die sich von den Kristallfl"achen gleich dadurch unterscheiden, dass sie immer etwas uneben und bei weitem nicht so gl"anzend wie jene sind.\footnote{\frakfamily{Sie haben "ofter eine f"unfseitige Gestalt, wir in dem St"ucke, welches Chladni beschreibt, wo drei solcher f"unfseitigen Fl"achen zusammensto"sen, „so dass der Kristall einem Pentagondodekaeder sehr "ahnlich ist,” wie Chladni sagt, diese Zusammensetzungsfl"achen mit Kristallfl"achen verwechselnd. (S. dessen Feuermeteore, S. 322.)}} Zuweilen sind die Kristalle nur durch eine schmale Lage von Eisen oder auch Troilit voneinander getrennt.

Betrachtet man die Kristalle mit einer Lupe, so sieht man h"aufig in ihnen ganz feine, haarf"ormige Einschl"usse, die ganz geradlinig und untereinander parallel, mehr oder weniger lang in verschiedenen H"ohen des Kristalls liegen, und "ofter Farben spielen. Besser erkennt man diese Einschl"usse noch, wenn man die Kristalle in d"unn geschliffenen Platten unter dem Mikroskop betrachtet, wo sie bei 140-maliger Vergr"o"serung wie in Fig. 10 Taf. I erscheinen.\footnote{\frakfamily{Diese Figur ist die Vergr"o"serung der kleinen, rechts liegenden hellen Stelle in der Platte Fig. 11, die aus einem sehr kl"uftigen Olivin-Kristall des Pallas-Eisens geschliffen, und in nat"urlicher Gr"o"se dargestellt ist.}} Sie machen im Allgemeinen den Eindruck von R"ohren, haben aber untereinander eine etwas verschiedene Beschaffenheit und erscheinen bei 360-maliger Vergr"o"serung, wie in der Fig. 2 Taf. 4 dargestellt ist. Am h"aufigsten erscheinen sie, wie in \emph{a} Fig. 2, als zwei nebeneinander liegende, gerade Linien, dann sieht man in der Mitte dieser eine st"arkere und schw"arzere b; dann erscheinen die beiden Linien von \emph{a} in zwei schw"achere geteilt \emph{c}, so dass man vier Linien sieht. In Innern sind sie teils ungef"arbt, teils lichte grau oder dunkelschwarz. Zuweilen sind die R"ohren unterbrochen und fangen in einiger Entfernung wieder an, \emph{b} Fig. 2, oder es ist nur die F"arbung in der R"ohre unterbrochen, wie bei \emph{e}. Eine ungew"ohnlich starke R"ohre \emph{f} erschien der ganzen L"ange nach dunkel und nur an den Enden eine kleine Strecke etwas lichter und an dem einen Ende zuletzt ganz licht. Gew"ohnlich erscheinen die R"ohren scharf abgeschnitten, zuweilen aber hatten sie eine Endigung wie in \emph{b} unten angegeben. Fig. 12 Taf. I stellen schiefe Durchschnitte dieser R"ohren in einer aus einen solchen Olivin-Kristalle geschliffenen Platte dar.

Es ist schwer zu sagen, wof"ur man diese Einschl"usse halten soll. Wenn ich sie R"ohren genannt habe, so soll damit nur der Eindruck bezeichnet werden, den sie auf mich gemacht haben. Sie sind aber alle parallel, wenn sie auch nur in geringer Menge und vereinzelt in dem Kristalle liegen, und m"ussen also, da sie sich untereinander nicht ber"uhren, eine ganz bestimmte Lage in dem Kristalle haben, worin sie liegen. Welche diese aber ist, war schwer auszumachen, da man gew"ohnlich nur so wenige Fl"achen bei den Kristallen sieht, doch konnte ich bei einigen Kristallen nicht zweifeln, dass sie eine gegen die Endfl"ache \emph{P} rechtwinklige, also eine der Hauptaxe parallele Lage haben. Bei einem Kristall z. B., an welchem sich zwei Fl"achen \emph{k} und dazwischen die Fl"ache \emph{T} befindet, kann man bei hellem Lampenlichte deutlich sehen, dass die Fl"ache \emph{T} und die R"ohren zu gleicher Zeit das Licht reflektieren, und letztere zugleich rechtwinklig gegen die Axe der Zone \emph{kT} liegen.

Der Olivin in dem Eisen von Brahin gleicht in Farbe, Durchsichtigkeit und Gr"o"se der K"orner sehr dem Eisen von Krasnojarsk; einzelne herausgefallene K"orner zeigten auch einzelne glatte Fl"achen wie bei diesem; auf der geschliffenen Fl"ache waren die Durchschnitte der Kristalle noch eckiger, so dass die Fl"achen sich vielleicht in noch viel gr"o"serer Anzahl finden als bei dem Sibirischen Eisen. Sie sind meistenteils sehr kl"uftig, zeigen aber auch die eingewachsenen r"ohrenartigen Einschl"usse sehr ausgezeichnet und in gro"ser Zahl.

Die Olivin-K"orner in dem Eisen von Atacama sind meistenteils noch gr"osser als die in dem Pallas-Eisen, oft 5/8 Zoll gro"s, aber sie sind noch viel kl"uftiger, vielleicht auch schon mehr oder weniger zersetzt, daher sie auf der geschliffenen Fl"ache keine Politur annehmen.

Der Olivin in dem Eisen von Rittersgr"un findet sich in kleineren K"ornern, die untereinander noch mehr zusammengeh"auft und zu k"ornigen Partien verbunden sind, als dies bei dem Sibirischen Eisen der Fall ist; wo sie aber an das Eisen angrenzen, sind sie kristallisiert, und noch deutlicher als bei diesem. Bei einem Korne konnte ich die Fl"achen in der Zone der L"angsprismen mit Sicherheit bestimmen, bei andern zeigten sich einzelne Fl"achen, die aber nicht bestimmt werden konnten; gewiss w"urde man bei bessern St"ucken, als mir bis jetzt zu Geboten stehen, recht ausgebildete Kristalle finden k"onnen. Der gemessene Kristall ist aber hinreichend, um zu beweisen, dass die Kristalle die Winkel des Olivins haben. Andere K"orner waren an der gegen das Eisen angrenzenden Seite rund, aber dabei stets etwas drusig; so glatte K"orner wie in dem Pallas-Eisen habe ich hier nicht bemerkt. Die Farbe dieses Olivins ist in den K"ornern gr"uner als bei dem Pallas-Eisen, doch erscheinen sie h"aufiger durch anfangende Zersetzung braun.

Der Olivin in dem Steinbach-Eisen gleicht dem von Rittersgr"un vollkommen; an dem kleinen St"ucke des Berliner Museums fand ich ein Korn, das ganz von Eisen umschlossen und v"ollig rund war, ein anderes, welches mehrere Fl"achen zeigte, die aber nicht bestimmt werden konnten.

Der Olivin in dem Eisen von Bitburg gleicht, nach dem kleinen St"ucke des Berliner Museums zu urteilen, den beiden vorigen; Kristallfl"achen habe ich nicht beobachtet.

Das spezifische Gewicht des Olivins aus dem Pallas-Eisen wird von Stromeyer zu 3,332 angegeben in v"olliger "ubereinstimmung mit dem der meisten der in den Basalten vorkommenden Olivine, das von dem Olivin von Steinbach nach demselben Chemiker zu 3,276.\footnote{\frakfamily{Jahrb. d. Chem. u. Phys. 1825 B. 14, S. 275. Stromeyer nennt zwar das Eisen, worin dieser Olivin vorkommt, von Grimma, doch hat schon Chladni gezeigt, dass darunter das Eisen von Steinbach zu verstehen sei. Vergl. Chladni Feuermeteore, S. 326 und Partsch Meteoriten, S. 91.}}

Vor dem L"otrohr und gegen S"auren verh"alt sich dieser Olivin wie der der Basalte. Vor dem L"otrohr schmilzt er nicht und ver"andert sich nicht.

In R"ucksicht der chemischen Zusammensetzung ist der aus dem Pallas-Eisen au"ser Stromeyer und Walmstedt von Berzelius (1), der von Atacama von Schmid (2) und der von Steinbach (Grimma) von Stromeyer (3) analysiert:
\begin{center}
\begin{tabular}{ |l|r|r|r| }
    \hline
     & 1 & 2 & 3\\
    \hline\hline
    Magnesia & 47,35 & 43,16 & 25,83\\\hline
    Eisenoxydul & 11,72 & 17,21 & 9,12\\\hline
    Manganoxydul & 0,43 & 1,81 & 0,31\\\hline
    Kiesels"aure & 40,86 & 36,92 & 61,88\\\hline
    Zinns"aure & 0,17 & -,- & -,-\\\hline
    Chromoxyd & -,- & -,- & 0,33\\\hline
    Gl"uhverlust & -,- & -,- & 0,45\\\hline
     &100,53 & 99,10 & 97,92\\
    \hline
\end{tabular}
\end{center}
\paragraph{}
Die beiden ersteren Ab"anderungen haben also vollkommen die Zusammensetzung des terrestrischen Olivins, die erste enth"alt etwas weniger Eisenoxydul als die zweite und ist:
\begin{center}
(7Mg + Fe)$^{2}$Si,
\end{center}
\paragraph{}
die zweite:
\begin{center}
(4Mg + Fe)$^{2}$Si.\footnote{\frakfamily{Vergl. Rammelsberg Mineralchemie, S. 438 und S. 503.}}
\end{center}
\paragraph{}
Beide sind dadurch ausgezeichnet, dass sie, obgleich mitten in einem nickelhaltigen Eisen vorkommend, kein Nickeloxyd enthalten, welches doch Stromeyer, wenn auch nur in geringer Menge, in allen terrestrischen Olivinen nachgewiesen hat, was aber nur beweist, wie auch schon Stromeyer anf"uhrt, dass das Nickeloxyd so leicht reduzierbar ist und weniger Verwandtschaft zur Kiesels"aure als das reine Metall zum Eisen hat. Merkw"urdig ist ferner die, wenn auch nur geringe Menge von Zinns"aure in dem Olivin des Pallas-Eisens, die aber Berzelius neben etwas Kupferoxyd auch in dem Olivin von Boscovich bei Aufsig in B"ohmen und in einem aus dem Dep. Puy de Dome gefunden hat. Sie ersetzt eine geringe Menge der Kiesels"aure.

Der Olivin von Steinbach (Grimma) hat dagegen nach Stromeyer eine ganz andere Zusammensetzung als der terrestrische Olivin. Da nun aber seine "au"sern Charaktere mit denen des "ubrigen Olivins stimmen, wenn auch die Winkel der Kristalle noch nicht bestimmt sind, das spezifische Gewicht auch nicht merklich verschieden ist, und so auch bei andern "achten Olivinen vorkommt, so muss hier offenbar ein Irrtum stattgefunden haben, wenn ich gleich nicht angeben kann, wodurch derselbe veranlasst ist.\footnote{\frakfamily{Stromeyer hat noch einen andern Olivin, angeblich aus dem Eisen von Olumba (soll wohl hei"sen Otumpa) aus der Prov. Chaco Gualamba, analysiert (Schweigger, Journ f. Chem. u. Phys. 1825 B. 44, S. 275) Da dieses Meteoreisen aber keinen Olivin enth"alt, so m"ochte auch hier ein Irrtum stattgefunden haben, und da dieser analysierte Olivin in der Zusammensetzung ganz mit dem Olivin aus dem Pallas-Eisen nach Stromeyers Analyse stimmt, so w"are es m"oglich, dass eine Verwechselung mit diesem die Ursache davon ist.}}

Das Eisen, welches die Grundmasse bildet, worin die Olivin-Kristalle eingeschlossen sind, findet sich bei den verschiedenen Pallasiten in gr"o"serer oder geringerer Menge. Das erstere ist der Fall bei den Pallasiten von Steinbach und Rittersgr"un, und bei diesen kann daher seine Struktur am besten erkannt werden. Es zeigt auf den Schnittfl"achen, ge"atzt sehr sch"one Widmanst"attensche Figuren, deren Streifen sich auf den Schnittfl"achen bei den einzelnen St"ucken des Berliner Museums stets "uberall parallel bleiben, also beweisen, dass das ganze Eisen jedes dieser St"ucke zu einem Individuum besteht. Ob dies auch bei gr"o"seren St"ucken der Fall ist, werden diese lehren; m"oglich, dass diese aus mehreren St"ucken bestehen. Dennoch sind immer die einzelnen K"orner oder die k"ornigen Partien des Olivins von einer hier nur sehr d"unnen Einfassung von dem Meteoreisen umgeben, jenseits welcher erst die Widmanst"attenschen Figuren anfangen. Das Eisen von Bitburg gleicht dem vorigen, doch sind die Widmanst. Figuren noch feiner. Bei den Pallasiten von Krasnojarsk, Brahin und Atacama ist der Olivin gr"osser, der Raum zwischen ihm geringer. Die Einfassung des Olivins von dem Meteoreisen ist im Verh"altnis der Gr"o"se der K"orner dicker, "uber dieser sieht man die d"unne Lage des T"anit, wie dies Reichenbach ausf"uhrlich beschrieben (vgl. oben S. 41), worauf nun ein etwas dunkler gef"arbtes Meteoreisen erscheint, das Reichenbach zu seinem F"ulleisen (Plessit) rechnet. Wo diese R"aume etwas gr"osser als gew"ohnlich sind, erscheinen durch "atzung darin noch Widmanst. Figuren. Wenn sich mehrere solcher R"aume mit diesen Figuren auf einer geschliffenen Fl"ache finden, wie dies besonders bei dem Eisen von Atacama vorkommt, so sieht man selten, dass diese an den verschiedenen Stellen eine gleiche Lage haben, was beweist, dass gr"o"sere Massen dieser Pallasit aus mehreren Eisenindividuen, wie das Meteoreisen von Seel"asgen, bestehen. Diess wird noch dadurch best"atigt, dass die in den Olivin-Kristallen des Pallas-Eisens vorkommenden r"ohrenartigen Einschl"usse, die in einem Kristall alle untereinander parallel sind, wenn man verschiedene Kristalle miteinander vergleicht, keine parallele Lage haben, was doch wahrscheinlich der Fall w"are, wenn das Eisen, worin sie liegen, ein Individuum w"are.

Die Einfassung des Olivins durch das Meteoreisen ist recht merkw"urdig. Sie scheint zu beweisen, dass, nachdem der Olivin sich in dem fl"ussigen Eisen ausgeschieden hat, das den Olivin zun"achst Umgebende zuerst fest wurde und denselben mit einer d"unnen H"ulle umgab, auf welche sich sogleich etwas T"anit legte und nun der innere Raum mit schaligen Lagen von Meteoreisen und T"anit, die die Widmanst"attenschen Figuren bilden, oder nur mit dem sogenannten Plessit von Reichenbach ausgef"ullt wurde. Aber dieser Plessit scheint selbst nichts anderes zu sein als ein schaliges Meteoreisen, in welchem nur die Schalen recht d"unn und die T"anit-Lagen verh"altnism"a"sig dick sind, so dass sie im Ganzen die Erscheinung darstellen, die Reichenbach unter dem Namen der K"amme beschrieben hat, und deren oben S. 36 Erw"ahnung getan ist. Denn wenn man von der ge"atzten Fl"ache des Pallas-Eisens einen Hausenblasenabdruck macht und den Abdruck des F"ulleisens unter dem Mikroskop betrachtet, so sieht man das feine Gemenge deutlich und die sich durchschneidenden Lagen, ganz "ahnlich denen, die die Widmanst"attenschen Figuren bilden.

Au"ser dem Olivin und dem Meteoreisen, die die wesentlichen Gemengteil des Pallasit bilden, finden sich in demselben einige unwesentliche, die nur in mehr oder weniger geringen Menge darin vorhanden sind. Zu diesen geh"ort:

1. Troilit oder Magnetkies. Er ist wie in dem Meteoreisen von tombakbrauner Farbe, nur derb, und auch eigentlich nur auf den angeschliffenen Fl"achen zu erkennen; er findet sich gew"ohnlich nur in kleinen Partien, aber in allen Pallasiten, am gr"o"sten noch in dem von Krasnojarsk und Brahin, wo er doch K"orner von 3 bis 4 Linien Durchmesser bildet. Bei einem St"ucke des P. von Krasnojarsk des Berliner Museums findet sich ein Olivin-Korn, das ganz von Troilit umschlossen ist, bei einem andern ein anderes, das zu 3/4 des Umfangs von Troilit und nur zu 1/4 von Meteoreisen umschlossen ist. Zuweilen sind einzelne K"orner, wie sie "ofter von haarbreiten Lagen von Meteoreisen getrennt sind, auch durch solche Lagen von Troilit voneinander getrennt.

2. Chromeisenerz von samtschwarzer Farbe, unvollkommenem Metallglanz, braunem Strich, und vor dem L"otrohr mit Phosphorsalz ein smaragdgr"unes Glas gebend. Es ist in der Regel vollkommen in dem Meteoreisen eingewachsen und grenzt an dasselbe in geraden Fl"achen, ist also kristallisiert; zuweilen grenzt es aber auch an den Olivin, es bildet dann mit ihm eine unregelm"a"sige Grenze und nimmt Eindr"ucke von diesem an, scheint also sp"ater als dieser kristallisiert zu sein. Man erkennt das Chromeisenerz auch nur auf geschliffenen Fl"achen; es scheint aber "uberall nur sparsam vorzukommen; ich habe es bestimmt gesehen nur in den Pallasiten von Brahin und Atacama. Doch muss es auch in den "ubrigen, wenigstens in dem von Krasnojarsk und Steinbach vorkommen, weil Laugier in dem ersteren und Stromeyer in dem Steinbach- (Grimma-) Olivin etwas Chromoxyd angibt, was doch wahrscheinlich von eingemengtem Chromeisenerz herr"uhrt.\footnote{\frakfamily{In den angeschliffenen St"ucken des Pallas-Eisens von der Berliner Sammlung ist das Chromeisenerz nicht sichtbar, und in der Analyse des Pallas-Eisens von Berzelius wird auch kein Chrom angegeben. Indessen ist beides doch kein Grund, dass nicht Chromeisenerz in dem Pallas-Eisen vorkommen kann, da die angeschliffenen Fl"achen der Berliner St"ucke nicht gro"s sind, und es zuf"allig auf diesen fehlen kann, und Berzelius das Pallas-Eisen vor der Analyse geh"ammert und dadurch alle spr"oden Gemengteil, wie den Olivin und also auch das etwa vorhandene Chromeisenerz, entfernt hat.}}
\subsection{\frakfamily{Mesosiderit.}}
\paragraph{}
Ein k"orniges Gemenge von Meteoreisen, Troilit, Olivin und Augit. Es geh"oren hierher die Eisenmeteorit der Sierra de Chaco und von Hainholz.

1) Sierra de Chaco in der W"uste Atacama, N. von Chile 1862. Ein urspr"unglich 28,87 Loth schweres, mit nat"urlicher Oberfl"ache und Bruch begrenztes St"uck, Geschenk des Prof. Domeyko in St. Yago in Chile.\footnote{\frakfamily{Vergl. Monatsberichte der k. Akad. der Wissensch. 1863 S. 30.}} Davon wurde ein Teil abgeschnitten; die Sammlung enth"alt jetzt noch ein gro"ses St"uck von 23,90 und ein kleines von 1,4 Loth. Es sieht im Bruche k"ornig und im Allgemeinen gr"unlichschwarz und glanzlos aus; man erkennt nur einzelne gr"o"sere K"orner von r"otlichgelbem Olivin und kleinere schw"arzlichgr"une von Augit; das "uberall fein eingesprengte Eisen ist hier fast gar nicht wahrzunehmen. Vollkommen aber unterscheiden sich die Gemengteil auf einer geschliffenen und polierten Fl"ache; das Eisen tritt nun gleich durch seine stahlgraue Farbe und seinen starken Metallglanz hervor, und man sieht nun erst, in welcher Menge es vorhanden ist. Es ist in feinen Teilen "uberall mit kleinen Teilen der Silicate gemengt, die "uberall mit ganz unregelm"a"sigen, eckigen und zackigen Oberfl"achen ineinandergreifen, und zwischen deren der Troilit "uberall, aber in noch feineren Teilen, durch seine tombakbraune Farbe kenntlich, enthalten ist. Dazwischen treten nun in einzelnen gr"o"seren K"ornern Nickeleisen, Olivin und Augit auf. Ge"atzt zeigen die gr"o"seren K"orner des Nickeleisens sehr feine und zierliche Widmanst"attensche Figuren von einem eigent"umlichen Verhalten; man erkennt darin nicht ein System von Streifen, die einem aus schaligen Zusammensetzungsst"ucken parallel den Fl"achen des Oktaeders bestehenden Individuum entsprechen, sondern stets mehrere; bei einem am Rande des kleinen St"uckes befindlichen Korne von 4 Linien Durchmesser, das aber nur zum Teil auf dem St"ucke enthalten ist, sind deren drei zu erkennen, die durch eine halbe Linie dickes nicht gestreiftes Nickeleisen getrennt sind, in welchen nur hier und da kleine K"orner oder k"ornige Partien von Augit liegen. Die kleineren K"orner des Nickeleisens zeigen keine Widmanst"attensche Figuren, sondern enthalten in ihrer Mitte nur unregelm"a"sig gestaltete Teile von der in verd"unnter Salpeters"aure nicht angegriffenen Substanz. Nach einer vorl"aufig mitgeteilten Nachricht von Hrn. Domeyko enth"alt dasselbe nach seinen Untersuchungen 88,55 pC. Eisen und 11,5 Nickel, ist also an dem letzteren Bestandteil sehr reich. Der Olivin ist von gr"unlichgelber bis r"otlichgelber und brauner Farbe und zuweilen von betr"achtlicher Gr"o"se; auf der "au"sern Fl"ache befindet sich ein Korn von 3/4 Zoll im Durchmesser. Er ist zerkl"uftet und nimmt im Allgemeinen keine so gute Politur an, wie das Augit, vielleicht weil er schon etwas zersetzt ist. Er schmilzt und ver"andert sich vor dem L"otrohr nicht, ist also wie der gew"ohnlich in den Meteoriten vorkommende Olivin nicht Eisenreich. Das Augit ist olivengr"un, auf der geschliffenen und polierten Fl"ache ganz schwarz und gl"anzend, in sehr d"unnen Splittern aber doch mit gr"unlichweissem Lichte durchsichtig; er ist deutlich spaltbar nach den Fl"achen des vertikalen Prismas und seiner Quer- und L"angsfl"ache, und so vollkommen, dass sich die Spaltungsfl"achen ziemlich genau mit dem Reflexionsgoniometer messen lassen. An deutlichsten lie"s sich bei einem Splitter die Neigung der Fl"ache des vertikalen Prismas zur L"angsfl"ache messen, ich fand sie zu 136$^{\circ}$ 4’. Vor dem L"otrohr ist dieses Augit nur in d"unnen Splittern an den "au"sersten Kanten zu einem schwarzen Glase schmelzbar; mit Phosphorsalz bildet er unter Abscheidung der Kiesels"aure ein Glas, das, solange es hei"s ist, gr"unlichwei"s ist, das aber beim Erkalten ganz ausblasst. Nickeleisen wie auch in geringerer Menge Troilit kommen in diesem Augit wie auch in dem Olivin gew"ohnlich in sehr feinen Teilen eingemengt vor, wie man auf der geschliffenen Fl"ache des Meteoriten, wenn man sie mit der Lupe betrachtet, ganz deutlich sehen kann, daher man zu den L"otrohrversuchen diese Silicate erst pulvern, und die anziehbaren Teile mit dem Magnete ausziehen muss. Troilit ist in gr"o"seren K"ornern in dem Meteorit nicht eingemengt.

Die nat"urliche Oberfl"ache ist nur wenig uneben; das Nickeleisen ist hier wohl etwas mit braunem Eisenoxydhydrat bedeckt, doch nicht sehr stark; die gro"sen K"orner von Augit und Olivin sind ganz deutlich zu erkennen, wenn sie auch aus der Oberfl"ache nicht hervortreten.

2) Hainholz, Reg.-Bez. Minden, Preu"sen, 1856. Zwei gr"o"sere platte und zwei kleinere St"ucke; erstere durch Zerschneiden eines einzigen St"uckes erhalten, sind au"ser den Schnittfl"achen mit nat"urlicher Oberfl"ache begrenzt. Dem vorigen sehr "ahnlich; das Nickeleisen findet sich jedoch nicht in so gro"sen K"ornern, und diese zeigen auch beim "atzen keine Wid. Fig., sondern sind matt, und enthalten wohl gl"anzend gebliebene Teile, die aber nicht in der Mitte der K"orner liegen, sondern ganz an den Rand derselben gedr"angt sind. Der Troilit ist ferner nicht in so gro"ser Menge und das Augit nicht in so gro"sen K"ornern enthalten, dagegen kommt der Olivin hier in noch gr"o"seren Individuen vor. An dem einen St"ucke der Sammlung findet sich ein Kristall von etwas "uber Zolll"ange, und Reichenbach gibt an, dass ist einem in seinen Besitz befindlichen St"ucke sich ein Kristall von 1 3/4 Zoll L"ange und 1 1/2 Zoll Breite befindet.\footnote{\frakfamily{Vgl. Poggendorffs Ann. 1857, B. 101, S. 312.}} Reichenbach beobachtete ferner in den St"ucken seiner Sammlung Kugeln und Knollen, die, ohne sich in der Beschaffenheit zu unterscheiden, abgesondert in der Masse vorkommen\footnote{\frakfamily{A. a. O. S. 312.}}; bei den St"ucken der Berliner Sammlung finden sie sich nicht, doch kommen sie vielleicht auch bei dem Eisen der Sierra de Chaco vor, da man auf der Bruchfl"ache bei diesem ein Viertel bis einen halben gro"se rundliche Vertiefungen sieht, die wie Eindr"ucke von solchen Kugeln aussehen.

An der Oberfl"ache ist dieser Meteorit schon st"arker zersetzt, besonders stellenweise. Dieselbe ist im Allgemeinen braun, einzelne Eisenk"ornchen ragen daraus hervor, auch erkennt man auf ihr zum Teil noch die gro"sen Olivin-Kristalle. Als das St"uck der Sammlung durchschnitten wurde, bildeten sich nach nicht langer Zeit auf gewissen Stellen kleine Bl"aschen von Eisenchlorid. Ich lie"s daher die St"ucke auf den Rat des Baron Reichenbach einige Zeit in reinem Wasser liegen. Dadurch wurde das Eisen an diesen Stellen noch st"arker oxydiert und braun, aber obgleich dies schon vor l"anger als anderthalb Jahren geschah, so hat doch das Rosten seit der Zeit nicht weiter zugenommen. Bei dem Eisen der Sierra de Chaco zeigt sich keine Neigung zum Rosten.
\clearpage
\section{\frakfamily{Steinmeteorit.}}
\subsection{\frakfamily{Chondrit.}}
\paragraph{}
Diese Art ist unter den verschiedenen Arten der Steinmeteorit die bei weiten zahlreichste und zugleich diejenige, mit der sich Berzelius vorzugsweise besch"aftigt hat, der, wenn er auch vorzugsweise nur die chemische Beschaffenheit ermittelte, doch dadurch zugleich die wichtigsten Anhaltspunkte f"ur die Beurteilung der mineralogischen Beschaffenheit gegeben hat.
\begin{center}
"au"sere Beschaffenheit.
\end{center}
\paragraph{}
Diese Meteoriten sind besonders durch ihre kugliche Struktur ausgezeichnet, worauf sich ihr Name bezieht. Sie bestehen n"amlich aus einer mehr oder weniger feink"ornigen Grundmasse, in der mehr oder weniger h"aufig kleine Kugeln neben vielen andern Gemengteilen, wie Olivin, Nickeleisen, Magnetkies, Chromeisen und anderen schwarzen K"ornern liegen.

Die Grundmasse ist teils bedeutend fest, wie bei dem Chondrit von Erxleben, Chantonnay etc., teils weniger fest und fast zerreiblich, wie bei dem von Mauerkirchen, Iowa, Bachmut etc. Sie hat am h"aufigsten eine lichte aschgraue Farbe, die einesteils ins graulichwei"se bis schneewei"se, auf der andern Seite doch seltener ins dunkelgraue und selbst graulichschwarze "ubergeht, ist aber selten gleichm"a"sig gef"arbt; in der Regel kommen Massen von graulichwei"ser und von aschgrauer oder selbst graulichschwarzer Farbe an einem und demselben St"ucke vor, und grenzen dann aneinander teils mit unbestimmter, ineinander, wenn auch schnell verlaufender, teils mit sehr bestimmter scharfer Grenze. Das erstere findet z. B. bei den Chondriten von Chantonnay und G"uterslohe, das letztere bei den von Ensisheim und Weston, ganz besonders aber bei einem St"ucke von Siena (aus der Klaprothschen Sammlung stammend) statt, von dem in Fig. 9 Taf. 2 eine Zeichnung einer angeschliffenen Fl"ache desselben gegeben ist, und bei dem auch der Unterschied in der Farbe recht stark hervortritt. Zuweilen kommen scharfe und unbestimmte Grenzen an einem und demselben St"ucke vor, wie bei dem Ch. von Bremerv"orde. Die verschiedenfarbigen Massen liegen teils in gr"o"seren Partien nebeneinander (Ch. von G"uterslohe und Chantonnay), teils durchzieht die eine wie in Adern die andere; die wei"se die schwarze in dem Ch. von Ensisheim, die schwarze die wei"se in dem von Agen; die wei"se breitet sich stellenweise aus und schlie"st dann St"ucke der schwarzen ein, wie in dem Ch. von Ensisheim, so dass das Gestein hier ein anscheinend breccienartiges Ansehen erh"alt. Wo die Grundmasse fest ist, ist sie auch so hart, dass sie sich mit dem Messer nicht ritzen l"asst; auch erh"alt sie in diesem Fall schon einigen Glanz, der ihr sonst fehlt.

Die Kugeln, die in dieser Grundmasse auf eine "ahnliche Weise wie in den Varioliten oder vielen roten Porphyren eingewachsen vorkommen und die die kuglige Struktur dieser Meteorit bedingen, sind gew"ohnlich nur so klein wie Schrotk"orner oder Hirsek"orner, zuweilen aber auch gr"osser, selbst 3 bis 4 Linien gro"s, wie bei dem Ch. von Ausson und New-Concord.\footnote{\frakfamily{In dem von Mez"o-Madaras beobachtete Reichenbach sogar eine Kugel eines halben Zolls Durchmesser (Poggendorffs Ann. 1860 B. 111, S. 366).}} Sie sind ferner mit Ausnahme der gr"o"seren, die mehr unregelm"a"sig gerundet und in die L"ange gezogen sind, gew"ohnlich regelm"a"sig gerundet, sind aber selten an einem St"ucke von gleicher oder ungef"ahr gleicher Gr"o"se, besonders wenn darunter solche von der Gr"o"se wie in dem Ch. von Ausson und New-Concord vorkommen. Ihre Oberfl"ache ist rau und selbst drusig (Richmond), seltener glatt (Poltava), und im Bruche erscheinen sie teils uneben, teils fasrig, im letzteren Fall jedoch stets nur sehr feinfasrig, indessen doch immer bestimmt erkennbar fasrig, besonders unter der Lupe; was mir aber dabei sehr bemerkenswert scheint und sie von den Kugeln der irdischen Gebirgsarten, namentlich der Diorit unterscheidet, nie radial, sondern immer exzentrisch fasrig; so bei den Ch. von Erxleben, Stauropol, Forsyth, Bachmut, Ausson etc. Ihre Farbe ist wie die der Grundmasse, unterscheidet sich aber doch immer etwas von ihr; sie sind gew"ohnlich gr"unlichgrau oder braun, dabei von einem nur geringen Glanze, der etwas fettartig ist und stets nur "au"serst schwach an den Kanten durchscheinend, fast undurchsichtig. Sie sind wie die Grundmasse bald heller, bald dunkler, und gew"ohnlich finden sich beide Arten zusammen in einem und demselben Meteoriten, wo denn bald die helleren, bald die dunkleren vorherrschen. Das erstere, was der gew"ohnlichere Fall ist, findet z. B. statt bei dem Ch. von Mez"o Madaras, Okniny, Cabarras, das letztere bei den von G"uterslohe, Ausson. In dem Ch. von Krasnoi-Ugol sah ich auch eine graue Kugel eine kleinere wei"se einschlie"sen,\footnote{\frakfamily{Etwas "ahnliches beobachtete Reichenbach auch bei Kugeln von Tabor und Parma (Pongendorffs Ann. 1860 B. 111, S. 377).}} und in dem von Mauerkirchen, wo meistenteils nur lichtere Kugeln vorkommen, sind dieselben doch "ofter nach der Oberfl"ache zu etwas dunkler gef"arbt. Sie sind ferner mit der Grundmasse mehr oder weniger fest verwachsen, und ersteres gew"ohnlich da, wo die Grundmasse fester, letzteres, wo sie zerreiblich ist. Im ersteren Fall fallen sie beim Zerschlagen des Gesteins nicht heraus, und man sieht dann auf der Bruchfl"ache des Gesteins auch ihren Bruch, in letzteren Fall fallen sie heraus oder l"osen sich zum Teil von der Grundmasse, so dass man auf dem Bruch die halbkugelf"ormigen H"ohlungen oder Erhabenheit der sich herausgel"osten oder sitzengebliebenen Kugeln sieht. Die Kugeln finden sich in der Regel h"aufig, in manchen F"allen jedoch nur sparsam, wie in dem Ch. von Erxleben, Klein-Wenden, Ensisheim und Chantonnay, in andern aber wiederum so h"aufig, dass sie ganz gedr"angt nebeneinander liegen und sich "ofter gegenseitig in der Ausbildung st"oren, wie in dem Ch. von Timochin, Richmond, Benares und Mez"o-Madaras.\footnote{\frakfamily{Hausmann h"alt die lichtern Kugeln in dem Bremerv"orde f"ur undeutliche Kristalle, deren Form nicht n"aher zu bestimmen ist (G"ottinger Nachrichten von 1856, S. 151). „Nach den Durchschnitten derselben, welche selten die Gr"o"se von ein paar Linien erreichen, zu urteilen” sagt er, „scheinen sie teils rechteckige teils irregul"ar sechsseitige Prismen, zu sein, wonach auf ein trimetrisches Kristallisationssystem zu schlie"sen sein d"urfte.” Ich habe von Kristallformen bei diesen Mineralen nichts bemerkt und kann auch einer anderen Angabe von Hausmann, dass es vor dem L"otrohr „ruhig und nicht eben schwer zu Email schmelze” nicht beistimmen. Ich fand es vor dem L"otrohr unschmelzbar und stimme nur darin mit Hausmann "uberein, dass es beim Erhitzen dunkelbraun gef"arbt wird. Ich kann daher auch nicht das Mineral, wie Hausmann weiter unten in seiner Abhandlung getan hat, f"ur feldspatartig halten.}}

Der Olivin ist nur selten in der Grundmasse erkennbar und findet sich dann in kleinen, nur h"ochstens liniengro"sen, gelblichgr"unen, durchsichtigen K"ornern, wie in dem Ch. von Erxleben, Klein-Wenden, Krasnoi-Ugol, Timochin und Pultava. Durchschnitte von ausgebildeten Kristallen, und zwar von rektangul"arer Form, habe ich nur etwa bei dem Olivin in dem Ch. von Pultava und Erxleben gesehen.

Das Nickeleisen ist dagegen ein best"andiger und h"aufiger Gemengteil. Es findet sich in den Meteorsteinen gew"ohnlich nur fein eingesprengt, doch kommen unter den feineren K"ornern mitunter einzelne gr"o"sere vor; so sieht man in einem der St"ucke des Ch. von Barbotan ein l"angliches Korn von 3 Linien L"ange,\footnote{\frakfamily{Partsch spricht von linsen- und h"ohnen gro"sen K"ornern in den St"ucken dieses Meteoriten in dem Wiener Mineralien-Kabinette (Meteoriten, S. 77).}} ein "ahnliches ragt bei dem von Klein-Wenden aus der Oberfl"ache hervor, und andere "ahnliche, wenn auch nicht ganz so gro"se K"orner enthalten die Ch. von Lucé (Toulouse?), Seres und Macao. Die feinen K"orner sind zackig und eckig, die gr"o"seren haben meistenteils eine rundliche Oberfl"ache, kristallisiert findet es sich in den St"ucken der Berliner Sammlung nicht, daher ich auch die unvollkommenen Hexaeder, die Partsch bei dem Nickeleisen der Wiener St"ucke des Ch. von Barbotan beobachtet haben will (Meteoriten, S. 77), nur f"ur solche rundliche K"orner erkl"aren m"ochte. Indessen Individuen sind diese K"orner doch, denn das oben erw"ahnte Korn in dem Ch. von Barbotan, wie auch ein anderes in dem von Ausson zeigen geschliffen und ge"atzt die Linien des Meteoreisens von Braunan.\footnote{\frakfamily{Partsch (Meteoriten, S. 82) und Reichenbach (Pongendorffs Annalen B. 111, S. 365) erhielten bei den Eisenk"ornern in den Meteoriten von Macao und Blansko selbst Widmanst"attensche Figuren.}} Es umschlie"st auch, wie man auf der geschliffenen Fl"ache sehen kann, kleine Teile der Grundmasse. Die Menge des eingesprengten Eisens ist oft sehr betr"achtlich, wie in dem Ch. von Erxleben und Kl. Wenden; wie gro"s aber die Menge desselben eigentlich ist, erkennt man erst, wenn die St"ucke angeschliffen und poliert sind, wo sein Metallglanz und seine stahlgraue in die silberwei"se "ubergehende Farbe erst recht stark hervortreten und es auf diese Weise kenntlich machen. Die ganz feinen K"orner sind da "uberhaupt erst zu erkennen, und man sieht dann, dass solche auch in den Kugeln enthalten sind, wo in solchen feinen Teilen das Nickeleisen nie fehlt, wenn es auch gew"ohnlich nur in sehr geringer Menge vorhanden ist In der Umgebung der Kugeln ist es dagegen h"aufig in gr"o"serer Menge angeh"auft, wie z. B. in dem Ch. von Mez"o-Madaras, Krasnoi-Ugol und Ausson. Der feuchten Luft ausgesetzt, oxydiert sich das Nickeleisen leicht und das gebildete Eisenoxyd f"arbt die Umgebung braun, wie man dies bei so vielen St"ucken in den Sammlungen sehen kann.\footnote{\frakfamily{Wie schnell diese Oxydation wo sich geht, zeigen die beiden St"ucke von dem Ch. von G"uterslohe des Berliner Museums. Von den gefallenen Steinen wurde einer schon am folgenden Tage gefunden, ein anderer erst ein Jahr sp"ater, und St"ucke von beiden, der erstere fast vollst"andig, wurden von Hrn. Dr. Stohlmann in G"uterslohe durch g"utige Vermittlung des Hrn. Prof. Dove dem Berliner mineralogischen Museum verehrt. Das erste St"uck ist im Bruch aber vollkommen frisch, das andre dagegen durch und durch voller Rostflecke. H"atte es in der feuchten Erde noch l"anger gelegen, so w"urde es ganz zerfallen oder unkenntlich geworden sein. Diese schnelle Verwitterung ist auch der Grund, weshalb man noch nie einen Meteorstein gefunden hat, den man nicht hat, fallen sehen, w"ahrend man doch so viele von den nur so selten fallenden Eisenmassen zuf"allig gefunden hat oder noch findet, die durch die entstehende oxydierte Rinde vor weiterem Angriff der Atmosph"are gesch"utzt werden (siehe oben S. 42).}}

Magnetkies\footnote{\frakfamily{Vergl. oben S. 40.}} ist ebenfalls ein nie fehlender Gemengteil dieser Meteorit. Er findet sich wie das Nickeleisen gew"ohnlich fein eingesprengt, doch nicht in solcher Menge als dieses,\footnote{\frakfamily{Nach Partsch ist in Richmond mehr Magnetkies als Nickeleisen; bei dem kleinen St"ucke der Berliner Sammlung ist dies nicht der Fall; es enth"alt wohl Magnetkies, aber nur in geringer Menge.}} seltener kommt er in gr"o"seren K"ornern vor, doch auf diese Weise selbst noch h"aufiger als das Nickeleisen. In solchen findet er sich in den Ch. von Barbotan, Parma, Zaborcica, besonders aber in dem von Gr"uneberg, wo in dem gr"o"seren St"ucke des Berliner Museums ein Korn von ihm enthalten ist, das einen halben Zoll im Durchmesser hat. Wo er fein eingesprengt vorkommt, ist er auf der Bruchfl"ache des Gesteins kaum oder schwer zu erkennen, aber recht gut auf der geschliffenen Fl"ache, wo ihn tombakbraune Farbe und geringerer Glanz gleich vor dem Nickeleisen auszeichnen. Man sieht dann auch, dass der Magnetkies bald mit dem Nickeleisen verbunden ist, bald in getrennten K"ornern vorkommt. In dem St"ucke von Zaborcica der Berliner Sammlung schlie"st ein erbsengro"ses Korn von Magnetkies ein Korn von Nickeleisen ein und in einem St"ucke von Gr"uneberg sowie von Krasnoi-Ugoi wird umgekehrt ein Korn Magnetkies von Nickeleisen vollst"andig umschlossen. Der Magnetkies dieser Meteorit ist gar nicht oder nur "au"serst schwach magnetisch.

In geringer Menge, aber doch vielleicht "uberall kommen in diesen Meteoriten schwarze K"orner vor, die aber gew"ohnlich nur sehr klein, und nur zuweilen etwas gr"osser, und so gro"s wie etwa ein Hirsekorn sind, so dass man sie herausnehmen und besonders untersuchen kann; sie erweisen sich dann als Chromeisenerz, da sie zerrieben ein braunes Pulver und mit Borax oder Phosphorsalz vor dem L"otrohr geschmolzen ein chromgr"unes Glas geben. Ob aber nun s"amtliche schwarze K"orner aus Chromeisenerz bestehen oder neben diesen noch andere schwarze K"orner vorkommen, muss ich dahingestellt sein lassen. Die gr"o"sten schwarzen K"orner beobachtete ich in dem Ch. von Château Renard, doch sind sie auch noch deutlich erkennbar in dem von Erxleben, Klein-Wenden, Tabor, Richmond. W"ohler beobachtete sie auch in dem von Bremerv"orde. Wo sie sehr klein sind, erkennt man sie am besten auf einer angeschliffenen Fl"ache, denn auf einer Bruchfl"ache des Gesteins k"onnen sie leicht mit dem Nickeleisen verwechselt werden, da dies auch schwarz erscheint, wenn man es nicht gerade im vollen reflektieren Lichte betrachtet. Sie sind auch teils mit dem Nickeleisen verbunden, teils nicht. Das Chromeisenerz der Meteorit ist "uberall nicht magnetisch.\footnote{\frakfamily{Andre Gemengteil als die angegebenen habe ich nicht bemerkt, dennoch sind aber solche von anderen Beobachtern "ofter beschrieben. Hausmann nahm an, dass der Bremerv"orde aus einem „feldspatartigen K"orper (vergl. oben S. 86), einem K"orper der Pyroxen-Substanz und Olivin” bestehe, und W"ohler f"ugte diesen Gemengteilen au"ser Chromeisenerz noch etwas Grafit in feinen Bl"attchen hinzu (G"ottinger Nachrichten 1856, S. 153), welcher letztere wohl darin enthalten sein kann, da derselbe ja auch in den Meteoreisen vorkommt. Eichwald f"uhrt bei der Beschreibung des Ch. von Lixna an: „Von den nicht metallischen K"ornern k"onnte man die helleren, fast wei"sen f"ur kleine abgerundete Kristalle von Anorthit oder Labrador, die gelblichbraunen f"ur Olivin oder sehr kleine Granatkristalle halten und die viel Gr"o"seren und Selteneren f"ur Augit nehmen. Diese letzteren sind etwa 3/4 Linien gro"s und dennoch zehn Mal gr"osser als die Kristalle des Olivins und Anorthits. Sie bilden das nicht metallische Gemenge des Meteorsteins, in dem die Augitkristalle deutlich eingesprengt erscheinen, w"ahrend die anderen kleinen Kristalle seine Hauptmasse ausmachen” (Pongendorffs Ann. 1852 B. 85, S. 577). Dufrenoy spricht bei dem Ch. von Château Renard von einem unvollkommen bl"attrigen Mineral, das in einigen Punkten analoge Streifen zeigt, wie sie den hemitropen Massen von Albit oder Labrador eigen sind. Das darin vorkommende Chromeisenerz h"alt er f"ur ein dem Perlit "ahnliches Mineral (Poggend. Ann. 1841 B. 53, S. 413). Abich nennt das Gef"uge des Ch. von Stauropol psammitisch und spricht bei der Beschreibung auch von einem Mineral, das sich auf Labrador (oder Saussurit) zur"uckf"uhren lie"se. Es h"atte eine gr"unlichgraue F"arbung, lie"se bei g"unstiger Zerspaltung deutlichen Bl"atterdurchgang erkennen und finde sich in rundlichen, aber auch stumpfkantig vorkommenden Fragmenten von gew"ohnlich 2, zuweilen aber auch von 8 Millimetern Gr"o"se. Eine auf einer Bruchfl"ache sichtbare Labradormasse von 14 Millimeter im Durchmesser enthielte in ihrem Innern ein fremdartiges Aggregat, aus einem durch Zersetzung unkenntlichen wei"slichen Mineral, feinen Teilen von Meteoreisen und kleinen, wei"sgelblichen, mehr fett- als glasgl"anzenden Kristallfragmenten eines besonderen Minerals bestehend (bulletin de l'acad. imp. des sc. d. St. Petersbourg 1860 t. 2, p. 412).}}

"au"serlich hat der Meteorit dieser Abteilung wie die "ubrigen eine durch Schmelzung der Oberfl"ache entstandene d"unne schwarze Rinde, die indessen hier stets matt und "ofters durch das hervorragende, schwer schmelzbare Nickeleisen h"ockerig ist, wie z. B. bei Aigle u. s. w. Nicht selten sind sie auch im Innern mit Spr"ungen durchsetzt, auf welchen etwas von der geschmolzenen Oberfl"ache w"ahrend des Zuges durch die Atmosph"are durch den Druck der Luft hineingepresst ist,\footnote{\frakfamily{Vergl. Reichenbach und Haidinger Ber. d. Wiener Akad. 1859 B. 34 (S.10).}} wie bei den Ch. von Lissa, Ensisheim, Politz, Château Renard u. s. w. Auch finden sich "ofter gl"anzende schwarze Abl"osungs- oder Rutschfl"achen, auf welchen das Eisen breit gefletscht ist, wie z. B. bei den Ch. von Lixna und Aigle.

Ich will hier nur noch die Beschreibung einiger ausgezeichneten Ab"anderungen dieser Abteilung von Meteoriten folgen lassen, die gewisserma"sen als Typen von ganzen Gruppen von Ab"anderungen in dieser Abteilung betrachtet werden k"onnen, und w"ahle dazu diejenigen, die in guten Exemplaren in der Berliner Sammlung vertreten sind.

1) Der Chondrit von Erxleben, gef. d. 15. April 1812. Er geh"ort zu den am deutlichsten kristallinischen der Sammlung, und ist daher allen "ubrigen voranzustellen. Derselbe hat eine lichte, graulichwei"se, feink"ornige Grundmasse, die gl"anzend von Glasglanz, hart und fest ist, und daher geschliffen eine gute Politur annimmt. Darin liegen ziemlich h"aufig ungef"ahr liniengro"se Kugeln, die gew"ohnlich mit der festen Grundmasse fest, in manchen F"allen doch auch weniger stark verwachsen sind und sich von ihr beim Zerschlagen des Gesteins abl"osen, so dass man auf der Bruchfl"ache, wenn auch gew"ohnlich ihren Bruch, doch auch zuweilen einen Teil ihrer kugligen Oberfl"ache oder die konkaven H"ohlungen, worin sie gesessen haben, sieht. Die Oberfl"ache der Kugeln, sowie auch die der Vertiefungen ist feink"ornig, die Farbe der Kugeln teils etwas gr"unlichgrau und etwas dunkler als die der Grundmasse, teils gelblichgrau und lichter als diese. Die Kugeln der ersteren Art sind h"aufiger und im Bruche uneben, die der letzteren weniger h"aufig und feinfasrig. Die Farben treten noch besser als im Bruche auf einer geschliffenen Fl"ache hervor. Olivin kommt in den St"ucken der Sammlung mehr oder weniger deutlich vor, zuweilen noch auf der Bruchfl"ache des Gesteins geradlinige Umrisse zeigend, doch ist er im Allgemeinen nicht h"aufig. Nickeleisen ist fein eingesprengt in ziemlich gro"ser Menge in dem Meteorstein enthalten und findet sich in sehr feinen K"ornern auch in den Kugeln, besonders den dunklen, wie man auf der geschliffenen Fl"ache sehen kann. Auf dieser erkennt man auch den Magnetkies, der in feinen Teilen zum Teil mit Nickeleisen verwachsen vorkommt und im frischen Bruch gar nicht erkannt werden kann; ebenso sieht man auch erst auf der geschliffenen Fl"ache die schwarzen K"orner, die teils einzeln, teils mit dem Nickeleisen verwachsen darin vorkommen, doch muss man Acht haben, sie hier nicht mit den feinen L"ochern, die sich auf der Schlifffl"ache finden und dadurch entstehen, dass Teilchen von Nickeleisen beim Schleifen aus der Fl"ache herausgerissen werden, wie auch mit den K"ornern von Nickeleisen und Magnetkies, welche alle bei gewisser Beleuchtung schwarz aussehen, zu verwechseln.

Dem Chondrit von Erxleben ist der von Kl.-Wenden (1843, Sept. 16) so "ahnlich, dass man beide nicht voneinander unterscheiden kann. Ebenso auch der von Abich beschriebene Chondrit von Stauropol im Kaukasus (1857, M"arz 25), der von Grewingk und Schmidt beschriebene Ch. von Pillistfer in Livland (1863, April 15) und einige andere.

2) Ensisheim (1492, Nov. 7). Eine feste, feink"ornige Masse, die teils schw"arzlichgrau und nur schimmernd, teils graulichwei"s und etwas st"arker gl"anzend ist. Beide Massen enthalten eingewachsene Kugeln, die aber stark mit der Grundmasse verwachsen und stets etwas dunkler und gl"anzender als die Grundmasse gef"arbt sind. Die graulichwei"se Masse durchzieht die schwarze wie Adern nach allen Richtungen und teilt sie dadurch in eine Menge eckiger St"ucke, w"ahrend sie selbst, da wo sie breiter wird, wieder eine Menge kleiner eckiger St"ucke der schw"arzlichgrauen Masse einschlie"st, so dass dadurch der ganze Meteorit ein breccienartiges Ansehen erh"alt. Die schwarze und die graulichwei"se Masse schneiden scharf aneinander ab. Die schwarze Masse ist indessen bei weitem vorherrschend, und manche kleineren Bruchst"ucke enthalten nur sie und nichts von der wei"sen.

Nickeleisen ist in der ganzen Masse eingesprengt, aber nicht so gleichm"a"sig und in solcher Menge wie bei den Ch. von Erxleben und Klein-Wenden, die K"orner sind aber meistenteils sehr fein und dann nur auf einer geschliffenen Fl"ache zu erkennen; sie werden nur hier und da etwas gr"osser. Magnetkies ist viel seltener, findet sich aber in einzelnen gr"o"seren K"ornern.

Der Stein ist von vielen gl"anzenden und schwarzen Abl"osungsfl"achen durchzogen.

Der Ch. von Chantonnay (1812, Aug. 5) ist dem von Ensisheim "ahnlich, die Masse hat hier ebenfalls stellenweise eine verschiedene Farbe, eine dunklere und hellere, und erstere ist noch schw"arzer als bei dem Ch. von Ensisheim, aber beide liegen in gro"sen Partien aneinander und schneiden nicht so scharf ab. Der Stein nimmt wie der von Ensisheim geschliffen eine gute Politur an.

3) Gr"uneberg (1841, M"arz 22). Lichte aschgraue feink"ornige und feste Grundmasse, worin eine gro"se Menge von Kugeln von Schrot- und Hirsekorngr"o"se mit ihr fest verwachsen liegen, so dass sie auf der Bruchfl"ache des Gesteins auch ihre Bruchfl"achen zeigen. Sie sind gr"o"stenteils von gr"unlichgelber, doch zuweilen auch von schw"arzlichgrauer Farbe und im Bruch uneben. Viel eingesprengtes Nickeleisen, das meistenteils in kleinen feinen Partien, doch zuweilen auch in 2 bis 3 Linien langen K"ornern darin enthalten ist; auch verh"altnism"a"sig viel Magnetkies, der meistenteils in einzelnen noch gr"o"seren Partien als das Nickeleisen, seltener in feinen Teilen vorkommt; beide "ofter miteinander verbunden; an dem kleineren St"ucke der Sammlung wird ein Schrotkorn-gro"ses Korn von Magnetkies von Nickeleisen fast ganz umschlossen. Die Menge, in der sich der Magnetkies findet, zeichnet diesen Meteoriten ganz besonders aus. Er hat Abl"osungsfl"achen, die aber nur zum Teil schwarz sind.

Dem Ch. von Gr"uneberg sehr "ahnlich sind eine gro"se Menge von Meteoriten, wenngleich nicht alle so viel Magnetkies enthalten, wie die von Cabarras-County, Mez"o-Madaras, Tabor, Toulouse, Barbotan, Tipperary, Seres, Krasnoi-Ugol, Wessely u. s. w.

4) Mauerkirchen (1768, Nov. 20). Lichte graulichwei"se zerreibliche Grundmasse mit sehr vielen eingewachsenen Kugeln von fast gleicher oder nur etwas mehr gelblicher Farbe als die Grundmasse. Die Kugeln sind untereinander meistens von gleicher lichter Farbe, nur zuweilen finden sich solche, die nach der Oberfl"ache etwas dunkler gef"arbt sind; ganz gleichm"a"sig gef"arbte dunklere Kugeln kommen nicht vor. Ungeachtet der nur zerreiblichen Grundmasse sieht man auf der Bruchfl"ache doch meistens nur den Bruch der Kugeln; dann fein eingesprengtes Nickeleisen, auch etwas ebenso beschaffenen Magnetkies.

Dem Ch. von Mauerkirchen sehr "ahnlich sind die von Iowa (nicht zu unterscheiden), Linum, Apt, Bachmut, Lissa, Politz, New-Concord, Slobodka u. s. w.
\begin{center}
Mikroskopische Untersuchung.
\end{center}
\paragraph{}
Wenn man von diesen Meteoriten m"oglichst d"unne Platten schleift, so werden oft mehrere der Gemengteil so durchsichtig, dass man unter dem Mikroskop ihre Struktur und Form erkennen kann. Diess gelingt zwar nur vollkommen bei den festeren, wie bei den Ch. von Erxleben, Klein-Wenden, Stauropol; bei andern, die weniger fest und mehr br"ocklig sind, wie bei den Ch. von Aigle, Mauerkirchen, Timochin, nur teilweise, indem nur einzelne ihrer Gemengteil wie die Kugeln durchsichtig oder zum Teil durchsichtig werden; dennoch sind auch diese belehrend. Ich will daher das Aussehen einiger dieser Platten beschreiben und die Resultate angeben, die diese Art der Untersuchung ergeben hat.

1) Erxleben. Eine kleine, d"unn geschliffene Pl"atte\footnote{\frakfamily{Sie ist Fig. 7 Taf. 3 in ihrer nat"urlichen Gr"o"se dargestellt; die Fig. 1, 2, 3 sind Abbildungen einzelner Stellen derselben bei 140-maliger Vergr"o"serung. Ich verdanke das Pl"attchen noch dem verstorbenen Dr. Oschatz, er diese Art von Pr"aparaten mit gro"ser Geschicklichkeit anfertigte. Indessen werden sie jetzt von seinem Nachfolger, dem Optiker Krieg, hier in Berlin schon von gleicher G"ute gemacht.}} lie"s schon bei schwacher Vergr"o"serung erkennen: eine Grundmasse, wasserhelle Kristalle, schwarze undurchsichtige K"orner, eine grau aussehende Kugel und andere graue Partien. Die durchsichtigen Kristalle wie \emph{a} und \emph{b} in Fig. 1 und 2 Taf. 3 sind nur hier und da sichtbar; wo sie an die Grundmasse angrenzen, sind sie gew"ohnlich nicht scharf und regelm"a"sig begrenzt; nur einmal auf der kleinen Beobachtungsplatte bildet ihr Umriss ein ziemlich regelm"a"siges Sechseck \emph{a} in Fig. 1; wo sie dagegen an die schwarzen K"orner angrenzen, da ist die Begrenzung gew"ohnlich ganz scharf und geradlinig. Sie sind gr"o"stenteils ungef"arbt, nur stellenweise sind sie, wie auch die Grundmasse, etwas olivengr"un gef"arbt, doch verl"auft sich die Farbe nach allen Seiten und scheint mehr von au"sen eingedrungen zu sein. Die Grundmasse ist im Ansehen von diesen Kristallen wenig verschieden, sie unterscheidet sich eigentlich nur dadurch, dass ihre k"ornigen Zusammensetzungsst"ucke weniger gro"s als die Kristalle sind, doch gehen sie in der Gr"o"se in diese "uber. Die Zusammensetzungsfl"achen sind "uberall unregelm"a"sig laufende krumme Fl"achen, schwarz oder grau, wie dies aber auch der Fall ist bei den Zusammensetzungsfl"achen der K"orner des Marmors von Carara, wenn man eine d"unn geschliffene Platte dieser Substanz unter dem Mikroskop betrachtet. Dies r"uhrt in diesem Fall teils von der schr"agen Lage der Zusammensetzungsfl"achen gegen die Schlifffl"ache her, teils aber von einer gro"sen Menge sehr Kleiner schwarzer K"orner, die mehr oder weniger zusammenliegen und bei starker Vergr"o"serung deutlich zu sehen sind.\footnote{\frakfamily{Siehe Fig. 4, die den Kristall \emph{b} in Fig. 1 bei 360-maliger Vergr"o"serung darstellt.}}

Der schwarze Gemengteil findet sich in kleineren und gr"o"seren K"ornern und Partien, die gr"o"seren mit unregelm"a"sigen, oft ganz zackigen Umrissen, die kleineren von mehr rundlicher Form. Letztere liegen in gro"ser Menge neben den gr"o"seren in der Grundmasse, zum Teil sind sie auch in den durchsichtigen Kristallen eingewachsen.\footnote{\frakfamily{Neben diesen finden sich auch zuweilen kleine H"ohlungen mit einer Fl"ussigkeit und einer Luftblase angef"ullt, wie in irdischen Kristallen. Vergl. Fig. 3, welche den Kristall \emph{a} Fig. 1 bei 360-maliger Vergr"o"serung darstellt.}} Dieser schwarze Gemengteil zeigt, wenn man die Platte bei durchgelassenem Lichte betrachtet, keine oder eine nur sehr undeutliche Verschiedenartigkeit der Teile; schlie"st man aber dieses ab und betrachtet man ihn nur bei zur"uckgeworfenem Lichte, am besten bei hellem Sonnenlichte und bei st"arkerer, etwa 140 maliger Vergr"o"serung, so sieht man, dass namentlich die gr"o"seren K"orner h"aufig Gemenge von zwei bis drei Substanzen sind, einmal von einer, die auch jetzt noch schwarz erscheint, einer nun fast bleigrau und metallisch gl"anzenden, die das Nickeleisen ist (n Fig. 1 und 2) und einer dritten, die schw"arzlichbraun matt, aber voller Spr"unge oder wenigstens Stellen ist, die das Licht sehr stark reflektieren, und nach Vergleichung mit der direkt gesehenen polierten Platte, Magnetkies ist (m, Fig. 1 u. 2). Im reflektierten Lichte ohne Sonnenschein ist er unter dem Mikroskop von der schwarzen Masse kaum zu unterscheiden. Nickeleisen findet sich h"aufiger und in gr"o"seren Partien als Magnetkies, der im Ganzen selten solche bildet, wie in Fig. 2. Wo das Nickeleisen mit der schwarzen Substanz gemengt ist, sitzt diese gew"ohnlich in einzelnen kleinen Partien an der Au"senseite, und man findet wohl kaum gr"o"sere Nickeleisenpartien ohne Verwachsung mit der schwarzen Masse, wenn auch gr"o"sere schwarze K"orner ohne damit verwachsene Eisen vorkommen. Regelm"a"sige Begrenzung zeigt Nickeleisen, Magnetkies und der schwarze K"orper nicht; wo sie aber an die durchsichtigen Kristalle angrenzen, zeigen alle, wie angef"uhrt, den regelm"a"sigen Eindruck der Kristalle, sind also alle sp"ater wie diese Kristallisiert.

Was dieser schwarze K"orper sei, und ob er "uberhaupt einerlei ist, kann auch mit Bestimmtheit noch nicht ausgesprochen werden; denn wenn auch die meisten dieser K"orner, und "uberall die kleineren, schwarz und v"ollig undurchsichtig sind, so scheinen doch einzelne gr"o"sere ein schwaches dunkelgr"unes Licht durchzulassen.\footnote{\frakfamily{Eigent"umlich ist der schwarze K"orper \emph{c} in Fig. 1 Taf. 3. Er zeigt nur an einer Stelle Nickeleisen, die ganze "ubrige Masse ist br"aunlichgr"un und ohne Metallglanz; dreht man aber die Platte um, so sieht man hier nur Nickeleisen. Das Korn zeigt also auf der einer Seite die schwarze, nicht metallische Masse und auf der andern Seite Nickeleisen, und die Grenze zwischen beiden muss also zuf"allig gerade der Schlifffl"ache parallel gehen.}} Wird dadurch eher Verschiedenheit angedeutet? Es w"are m"oglich, und dann k"onnten vielleicht die schwarzen undurchsichtigen K"orner Chromeisenerz sein, da in dem Ch. von Erxleben durch die Analyse etwas Chromoxyd nachgewiesen ist, das wahrscheinlich von dem Chromeisenerz herr"uhrt, und dieses stets auch in sehr feinen Teilen undurchsichtig ist. Wof"ur dann aber die schw"arzlichgr"un durchscheinenden Teile zu halten sind, ist schwerer zu entscheiden. Am meisten w"urde man vielleicht geneigt sein, sie f"ur Augit zu halten, da dieser durch die Analyse wahrscheinlich gemacht ist und auch ziemlich allgemein als Gemengteil der Chondrit angenommen wird, ohne ihn irgend bestimmt beobachtet zu haben; aber wo der Augit in den Meteorsteinen erkannt ist, wie in dem Eukrit von Juvenas, ist er in d"unngeschliffenen Platten immer stark durchscheinend und braun; die s"amtlichen schwarzen K"orner aber f"ur Chromeisenerz zu nehmen, daf"ur scheinen sie f"ur die geringe Menge von Chromoxyd, die die Analyse stets nachgewiesen hat, in zu gro"ser Menge vorhanden zu sein. Ich muss also die Deutung dieser schw"arzlichgr"unen K"orner noch dahingestellt sein lassen.

Es bleiben nun noch die grauen Partien, die wahrscheinlich Durchschnitte von den Kugeln sind, wiewohl nur eine dieser Partien in dem untersuchten Pl"attchen (die Partie \emph{a} in Fig. 7)\footnote{\frakfamily{Sie ist in Fig. 7 bei 140maliger Vergr"o"serung und zum Teil in Fig. 6 bei 360maliger Vergr"o"serung gezeichnet.}} rund und scharf begrenzt ist; die andern unbestimmt begrenzten (\emph{d} in Fig. 1) scheinen Durchschnitte von nebeneinander liegenden Kugeln zu sein. In der einzelnen runden, scharf begrenzten Partie sieht man graue, untereinander parallele Streifen, die in einer durchsichtigen ungef"arbten Masse enge nebeneinander liegen. Die grauen Streifen (Fig. 5) scheinen aus lauter grauen runden K"ornern mit unbestimmten Umrissen zu bestehen, die in Reihen nebeneinander liegen, und zwischen denen dann wieder einzelne gr"o"sere, schwarze, scharf begrenzte K"orner von Chromeisenerz und feinere von Nickeleisen enthalten sind. Bei starker (360-maliger) Vergr"o"serung sieht es aber aus, als w"aren diese grauen K"orner nur durch lauter neben- und "ubereinander liegende krumme Spr"unge entstanden. Bei den andern grauen Partien \emph{d} in Fig. 1 liegen die grauen K"orner unregelm"a"siger, was wahrscheinlich davon herr"uhrt, dass die Schnittfl"ache durch diese Kugeln nach anderen Richtungen als in Fig. 5 geht; sie sind aber ebenso mit den schwarzen K"ornern gemengt. Was es f"ur eine Bewandtnis mit diesen grauen Streifen hat, lasse ich unentschieden. Die sp"ateren Beobachtungen zeigen, dass die Kugeln aus fasrigen Zusammensetzungst"ucken bestehen, die auch hier angedeutet zu sein scheinen.

Eine d"unn geschliffene Platte von dem Ch. von Klein-Wenden war gr"o"ser, man konnte mehr sehen; sie verhielt sich aber sonst wie die von dem Ch. von Erxleben; scharfe und geradlinige Umrisse der durchsichtigen Kristalle, wo sie an die schwarzen Partien, sei es an das Nickeleisen oder den Magnetkies oder die schwarze Substanz angrenzen, fanden sich auch hier. Kugeln mit grauen Streifen wie Fig. 5 Taf. 3 waren gerade nicht zu sehen, nur solche, die wie \emph{d} in Fig. 1 dunkle Flecken mit schwarzen K"ornern von Chromeisenerz und Nickeleisen enthielten; au"serdem fanden sich aber andere, die im Allgemeinen durchsichtig und ungef"arbt, aber mit vielen schwarzen Spr"ungen durchsetzt waren, deren zusammenh"angende Schw"arze sich aber bei starker Vergr"o"serung wie bei Fig. 4 in kleine, voreinander getrennte, schwarze K"orner aufl"oste. Eine solche sehr gro"se Kugel hatte das Ansehen von Fig. 8, die schwarzen dicken Spr"unge waren untereinander ziemlich parallel; sie wiederholten sich dabei schnell und schlossen, durch Querrisse verbunden, die scharf begrenzten, durchsichtigen Teile ein.

Eine noch gr"o"sere Platte von dem Ch. von Stauropol zeigte au"ser den gew"ohnlichen Erscheinungen eine noch gr"o"sere Menge von Olivin-Kristallen und von Kugeln. Die ersteren hatten zum Teil eine regelm"a"sige Form wie Fig. 3 u. 4 Taf. 4; sie waren durchsichtig, ungef"arbt, mit schwarzen Spr"ungen durchsetzt und hatten auch schwarze K"orner und Partien eingeschlossen. Die Kugeln waren hier recht deutlich zweierlei Art, und diese zwei Arten schon ganz bestimmt mit blo"sen Augen oder mit der Lupe auf der d"unn geschliffenen Platte zu unterscheiden, da die einen stellenweise durchsichtig, die andern nur grau durchscheinend waren. Auf einem St"ucke mit nur angeschliffener Fl"ache konnte man diesen Unterschied nicht wahrnehmen. Die durchsichtigen Kugeln schlossen sich ganz den Olivin-Kristallen an, sie schienen nur runde Kristalle oder runde Zusammenh"aufungen von unausgebildeten oder mehr oder weniger ausgebildeten Kristallen zu sein. Die ersteren waren mit schwarzen, untereinander ungef"ahr parallelen Spr"ungen durchsetzt, "ahnlich wie bei Fig. 8 Taf. 3,\footnote{\frakfamily{Oder mehr wie bei Fig. 10 Taf. 4, einer Kugel aus dem Ch. von Timochin, bei welchem die Spr"unge in den Kugeln gew"ohnlich weitl"aufiger sind.}} die andern hatten unregelm"a"sigere Spr"unge, und bei den dritten waren die Kristalle durch die "ubrigen Gemengteil, wie Grundmasse, Nickeleisen, Magnetkies und die schwarzen K"orner verbunden (wie Fig. 5 Taf. 4, wo die dazwischen eingeschlossene, metallische Substanz zuf"allig fast nur Magnetkies war, der sonst im Ch. von Stauropol nicht sehr h"aufig vorhanden ist). Diese letzteren Kugeln waren auch viel gr"osser als die andern, daher Fig. 5 (wie auch Fig. 3 und 4) nur bei 9Omaliger Vergr"o"serung gezeichnet sind. Diese Ansicht best"atigte die Untersuchung im polarisierten Licht, da die ersten Kugeln "uberall eine gleichm"a"sige F"arbung, die zweiten und dritten an verschiedenen Zusammensetzungsst"ucken und Kristallen verschiedene, "uberall an den Grenzen scharf abschneidende F"arbung zeigten. Die anderen Kugeln, welche mit blo"sen Augen betrachtet, grau und nur durchscheinend erschienen, wie die Kugel in dem Ch. von Erxleben Fig. 5 Taf. 3, waren auch unter dem Mikroskop wie diese mit grauen parallelen Streifen gestreift, nur deutlicher ("ahnlich wie die Kugel in dem Ch. von Krasnoi-Ugol, Fig. 8 Taf. 4) und an den verschiedenen Stellen nach 2 Richtungen, die gegeneinander einen spitzen Winkel machten und aneinander scharf abschnitten. Andere Kugeln zeigten nur graue, mehr unbestimmt verlaufende Flecken mit schwarzen, scharf begrenzten K"ornern. Zuweilen waren diese so dunkel und undurchsichtig, dass die hellen R"aume dazwischen nur sehr klein waren. Eine solche ganz runde Kugel ist Fig. 6 Taf. 4 dargestellt, sie ist noch dadurch ausgezeichnet, dass sie mit einem durchsichtigen, von schwarzen Flecken stellenweise ganz freien Ringe umgeben ist. Dicht neben ihr befindet sich eine kleinere, mehr unregelm"a"sig begrenzte, mit lichtern grauen und geraden parallelen Streifen (Fig. 7).

Die Patte von dem Ch. von Krasnoi-Ugol war im Allgemeinen sehr undurchsichtig, zeigte indessen doch die Kugeln sehr deutlich. Diese waren wiederum von der doppelten Art, die grau gestreiften waren aber hier in der Mehrzahl vorhanden. Sie lagen dann teils ganz einzeln in der Masse, teils waren sie zu mehreren zusammengeh"auft, sich gegenseitig in der Ausbildung st"orend. Eine solche einzelne, recht runde Kugel, deren Streifen auch recht scharf begrenzt sind, ist die vorhin erw"ahnte, in Fig. 8 Taf. 4 dargestellte Kugel; ihre Streifen haben die angegebene Lage, und "au"serlich ist sie wieder mit einer ganz hellen H"ulle umgeben. Neben ihr liegt eine Kugel der ersten Art, die aber viel kleiner ist. Eine Zusammenh"aufung von Kugeln ist Fig. 9 dargestellt.

Auch die Platte von dem Ch. von Ensisheim ist sehr dunkel und schwarz. Hierin fand sich ein sehr gro"ser, ziemlich regelm"a"sig begrenzter Kristall, der mit von Zeit zu Zeit wiederkehrenden, ziemlich parallelen Spr"ungen nach zwei untereinander rechtwinkligen Richtungen durchsetzt ist, wie dies auch bei den irdischen Olivin-Kristallen "ofter der Fall ist, bei denen dann die Spr"unge parallel der Quer- und L"angsfl"ache gehen.

Die Platten von den Ch. von Timochin, Ausson, Aigle und Mauerkirchen nahmen schon keine vollst"andige Politur an; nur die Kugeln und das Nickeleisen erschienen gl"anzend, und nur die ersteren waren durchscheinend, die "ubrige Masse gar nicht. Auch war es sehr schwer, Platten von einiger Gr"o"se zu erhalten, namentlich bei denen, der wie der Ch. von Mauerkirchen nur wenig und sehr fein verteiltes Eisen enthalten. Die Kugeln waren aber "uberall von der doppelten Art; eine Kugel der ersten Art, die sich in dem Ch. von Timochin fand, und in ihrer Art recht ausgezeichnet war, ist in Fig. 10 dargestellt. Die Ch. von New-Concord und Dharamsala gaben wieder Platten mit vollst"andigerer Politur, zeigten aber weiter nichts Neues.

Aus der mikroskopischen Untersuchung der d"unn geschliffenen Platten ergibt sich also, dass die unter dem Mikroskop schwarz erscheinende Masse au"ser Nickeleisen, Magnetkies und Chromeisenerz wahrscheinlich noch eine andere schwarze Substanz enth"alt, deren Natur noch nicht gekannt ist, und ferner, dass die eingewachsenen Kugeln bestimmt zweierlei Art sind, teils solche, die nur runde, zerkl"uftete Kristalle und offenbar Olivin-Kristalle, "ahnlich denen in dem Pallas-Eisen oder Zusammenh"aufungen derselben sind, teils solche, die aus fasrigen Zusammensetzungsst"ucken bestehen, die, wie auch die Beobachtung dieser Kugeln auf der Bruchfl"ache der Meteoriten mit blo"sen Augen oder mit der Lupe gelehrt hat, immer exzentrisch fasrig, nie radial fasrig sind.
\begin{center}
Chemische Beschaffenheit.
\end{center}
\paragraph{}
Kleine Splitter sowohl von der Grundmasse als von den Kugeln in der Platinzange gehalten und vor dem L"otrohr erhitzt, ver"andern wohl die Farbe und werden schwarz, schmelzen aber nicht.\footnote{\frakfamily{Man hat wohl oft von einer Schmelzbarkeit der Masse der Chondrit gesprochen, Hausmann bei dem Ch. von Bremerv"orde, Dufrenoy bei dem von Château Renard, Damour bei dem von Montrejeau (Ausson); ich habe dies nie gefunden.}} Nur das sehr fein geriebene Pulver schmilzt an den "au"sersten d"unnen R"andern zu einer gr"unlichgrauen oder graulichgr"unen Schlacke.\footnote{\frakfamily{Das Pulver wird dazu bekanntlich befeuchtet, auf der Kohle zu einer d"unnen Platte ausgebreitet, mit dem L"otrohr erhitzt und die nun zusammenh"angende Platte mit der Platinzange vorsichtig gefasst, und in der L"otrohrflamme gegl"uht. Um das Pulver recht fein reiben zu k"onnen, wurde aus dem zuerst erhaltenen gr"oblich Pulver des Meteoriten das darin enthaltene Nickeleisen mit den Magneten ausgezogen.}}

Wenn man kleine St"uckchen dieser Meteoriten in Chlorwasserstoffs"aure einige Zeit liegen l"asst, so werden Nickeleisen und Magnetkies unter Entwickelung von Wasserstoff und Schwefelwasserstoff und r"otlichgelber F"arbung der S"aure aufgel"ost und die den Meteoriten bildenden Silicate zersetzt. Nach Verlauf von einigen Tagen ist die S"aure schleimig geworden, und es hat sich ein Absatz von Kiesels"aure am Boden des Gef"a"ses und auf den St"ucken gebildet und auf der Oberfl"ache der S"aure eine geringe Menge Schwefel abgeschieden.\footnote{\frakfamily{Ich habe diese Abscheidung bei mehreren dieser Meteoriten bestimmt wahrgenommen, was die Anwesenheit von Magnetkies in diesen Meteorsteinen beweist. Ebenso sah sie auch Harris bei dem Ch. von Montrejeau (Ann. d. Chem. u. Pharm. 1859 B. 109, S. 183).}} W"ascht man den Absatz von den St"ucken ab, so erscheinen dieselben sehr br"ocklig und por"os; sie sind wei"s und erdig geworden, und die Kugeln, die von der S"aure weniger angegriffen werden, ragen daraus hervor; sie haben noch ihre Form und ihre graue Farbe behalten, wenn sie vorher so gef"arbt waren, wie bei den Ch. von Erxleben und Aigle, und k"onnten bei geh"origer Vorsicht wohl ganz isoliert werden. Bei l"angerer Einwirkung der S"aure wird aber der ganze Stein so m"urbe, dass man ihn mit den Fingern g"anzlich zu einem feinen Pulver zerdr"ucken kann. Dasselbe erscheint nun ganz wei"s und enth"alt nur einzelne feine schwarze Teile, die man durch Feinreiben und Schl"ammen wohl etwas konzentrieren, aber von den wei"sen Teilen nicht vollst"andig trennen kann. Sie sind wahrscheinlich nur Chromeisenerz; denn, wenn man den R"uckstand vor dem L"otrohr mit Phosphorsalz schmilzt, so erh"alt man ein Glas, das, solange es hei"s ist, durchsichtig ist und ein Kieselskelett einschlie"st, beim Erkalten aber opalisiert und eine wenn auch nur schwache doch deutliche chromgr"une Farbe erh"alt.

Betrachtet man die mit kalter Chlorwasserstoffs"aure digerierten und abgewaschenen Bruchst"ucke unter der Lupe, so erkennt man auf der Oberfl"ache einzelne kleine, aber stark demantgl"anzende Kristalle. Ich habe diese bei vielen dieser Meteorit, ganz besonders aber bei St"ucken von Aigle gesehen, die Monate lang in Chlorwasserstoffs"aure gelegen hatten. Unter dem Mikroskop bei m"a"siger Vergr"o"serung kann man, wenn man sie im reflektierten Lichte betrachtet, auch die Gestalt einzelner Fl"achen, von denen gerade das Licht zur"uckgest"ahlt wird, erkennen, man sieht dann ganz scharf begrenzte Rhomben oder Deltoide, auch Rechtecke u. s. w., doch l"asst sich danach die Form der Kristalle mit Sicherheit nicht bestimmen. Kocht man die St"ucke mit Wasser oder auch mit kohlensaurem Natron, so ver"andern sich die Kristalle nicht im Geringsten; aber es gelang mir nicht, die Kristalle zu isolieren. Schabte man die Kristalle von der Oberfl"ache, auf der sie sitzen, ab, so waren sie in dem erhaltenen Pulver nicht wieder zu erkennen. Ich habe so leider "uber ihre Natur nichts weiter ausmachen k"onnen. Wenn man kleine St"uckchen des Ch. von Aigle mit Chlorwasserstoffs"aure kocht und dann sowie vorhin behandelt, so waren die Kristalle viel unvollkommener, es scheint also, dass sie nur in kalter Chlorwasserstoffs"aure in der angegebenen Vollkommenheit erhalten werden k"onnen.\footnote{\frakfamily{Ich habe diese Kristalle mehreren meiner Freunde gezeigt, die nie einen Zweifel dar"uber hatten, dass es Kristalle w"aren.}}

Die chemische Zusammensetzung dieser Meteorit ist ungeachtet ihres oft verschiedenen Ansehens, doch sehr "ahnlich, wie dies schon Berzelius gefunden hat, der sich, wie oben angef"uhrt, mit dieser Abteilung der Meteorit besonders besch"aftigt hat, und wie dies auch alle sp"ateren Analysen ergeben haben. Rammelsberg hat in seiner Mineralchemie (S. 924) diese Analysen neu berechnet und sehr gut zusammengestellt. Sie sind s"amtlich nach der zuerst von Berzelius eingeschlagenen Methode, wonach durch Digestion des feinen Pulvers mit hei"ser Chlorwasserstoffs"aure die davon zersetzbaren Gemengteil von den davon nicht zersetzbaren getrennt werden, angestellt. Eine vollkommene Trennung der Gemengteil l"asst sich, wie der Erfolg gezeigt hat, auf diese Weise nicht erreichen, da keiner der Gemengteil, das Chromeisenerz ausgenommen, in Chlorwasserstoffs"aure vollkommen unl"oslich, der eine nur mehr oder minder als der andere l"oslich ist, also eine st"arkere oder schw"achere Chlorwasserstoffs"aure oder eine l"angere oder k"urzere Digestion damit bei h"oherer oder geringerer Temperatur oft sehr bemerkenswerte Unterschiede in der Analyse eines Meteoriten hervorbringen k"onnen; dennoch gew"ahrt diese Methode Anhaltspunkte, die f"ur die Deutung der Resultate von Wichtigkeit sind, und ist daher, wenn auch unvollkommen, immer sch"atzbar. Ich will von den vorhandenen Analysen hier nach den Werken von Rammelsberg\footnote{\frakfamily{Mineralchemie S. 925 u. s. f.}} nur einige derselben, und zwar die folgenden 4 anf"uhren, die Meteoriten der vier verschiedenen Gruppen betreffen:
\begin{enumerate}
    \item Klein-Wenden nach Rammelsberg.
    \item Chantonnay nach Berzelius.
    \item Blansko nach Berzelius.
    \item Kakowa nach Harris.
\end{enumerate}
\clearpage
\begin{center}
\begin{tabular}{ |l|r|r|r|r| }
    \hline
    Allgemeine Zusammensetzung. & 1 & 2\footnote{\frakfamily{Das Verh"altnis der metallischen zu den nicht metallischen Gemengteilen ist bei dieser Analyse von Berzelius nicht angegeben.}} & 3 & 4\\
    \hline\hline
    Nickeleisen & 22,90 & -,- & 20,13 & 23,20\\\hline
    Schwefeleisen\footnote{\frakfamily{Das Schwefeleisen ist von Rammelsberg als Sulphuret Fe berechnet. Die Annahme als Magnetkies Fe$^{5}$Fe (Fe$^{7}$S$^{8}$) "andert die Rechnung nur wenig.}} & 5,61 & -,- & 2,97 & 13,14\\\hline
    Chromeisenerz\footnote{\frakfamily{Das Chromeisenerz ist als FeCr berechnet.}} & 1,04 & -,- & 0,63 & 0,56\\\hline
    Silicate & 70,45 & -,- & 76,27 & 63,21\\
    \hline
\end{tabular}
\end{center}
\vspace{\medskipamount}
\begin{center}
\begin{tabular}{ |l|r|r|r|r| }
    \hline
    Zusammensetzung des Nickeleisens. & 1 & 2 & 3 & 4\\
    \hline\hline
    Eisen & 88,98 & -,- & 90,9 & 92,24\\\hline
    Nickel & 10,35 & -,- & 9,1 & 7,76\\\hline
    Kobalt & 10,35 & -,- & -,- & -,-\\\hline
    Kupfer & 0,21 & -,- & -,- & -,-\\\hline
    Zinn & 0,35 & -,- & -,- & -,-\\\hline
    Phosphor & 0,11 & -,- & -,- & -,-\\\hline
     & 100 & -,- & 100 & 100\\
    \hline
\end{tabular}
\end{center}
\vspace{\medskipamount}
\begin{center}
\begin{tabular}{ |l|r|r|r|r| }
    \hline
    Zusammensetzung der Silicate. & 1 & 2 & 3 & 4\\
    \hline\hline
    Magnesia & 32,75 & 27,79 & 32,49 & 27,06\\\hline
    Kalk & 3,66 & 1,52 & 1,22 & 1,52\\\hline
    Eisenoxydul & 10,92 & 19,47 & 11,45 & 24,40\\\hline
    Manganoxydul & 0,08 & 0,76 & 0,59 & -,-\\\hline
    Nickeloxyd & -,- & 0,90 & -,- & 0,20\\\hline
    Kupferoxyd & 0,21 & -,- & -,- & -,-\\\hline
    Natron & 0,41 & 1,24 & 0,98 & 1,31\\\hline
    Kali & 0,53 & 1,24 & 0,24 & 0,21\\\hline
    Tonerde & 5,23 & 2,95 & 2,94 & 2,61\\\hline
    Kiesels"aure & 46,18 & 44,16 & 48,95 & 44,68\\\hline
     & 99,97 & 98,79 & 98,96 & 98,86\\
    \hline
\end{tabular}
\end{center}
\paragraph{}
Diese Silicate bestehen in 100 Teilen aus:
\begin{center}
\begin{tabular}{ |l|r|r|r|r| }
    \hline
     & 1 & 2 & 3 & 4\\
    \hline\hline
    A. Zersetzbaren S. & 42,23 & 51,12 & 46,13 & 56,7\\\hline
    B. Unzersetzbaren S. & 57,77 & 48,88 & 53,87 & 43,3\\
    \hline
\end{tabular}
\end{center}
\clearpage
\begin{center}
A. Zusammensetzung der zersetzbaren Silicate.
\end{center}
\begin{center}
\begin{tabular}{ |p{23mm}|p{12mm}|p{8mm}|p{13mm}|p{8mm}|p{11mm}|p{8mm}|p{10mm}|p{8mm}| }
    \hline
     & Kl.-Wenden & Sauer. & Chanton. & Sauer. & Blansko & Sauer. & Kakova & Sauer.\\
    \hline\hline
    Magnesia & 20,00 & 9,27 & 17,56 & 10,48 & 19,89 & 9,77 & 11,2 & 10,09\\\hline
    Kalk & 0,89 & 9,27 & -,- & -,- & -,- & -,- & 0,7 & 10,09\\\hline
    Eisenoxydul & 4,53 & 9,27 & 14,72 & 10,48 & 6,88 & 9,77 & 24,4 & 10,09\\\hline
    Manganoxydul & 0,08 & 9,27 & 0,42 & 10,48 & 0,25 & 9,77 & -,- & -,-\\\hline
    Nickeloxyd & -,- & -,- & 0,23 & 10,48 & -,- & -,- & 0,2 & 10,09\\\hline
    Natron & -,- & -,- & 0,50 & 10,48 & 0,48 & 9,77 & -,- & -,-\\\hline
    Kali & -,- & -,- & 0,50 & 10,48 & 0,24 & 9,77 & -,- & -,-\\\hline
    Tonerde & -,- & -,- & -,- & -,- & 0,15 & 9,77 & -,- & -,-\\\hline
    Kiesels"aure & 16,72 & 8,68 & 16,67 & 8,65 & 18,20 & 9,45 & 19,5 & 10,12\\\hline
     & 42,22 & & 50,10 & & 46,09 & & 56,0 & \\
    \hline
\end{tabular}
\end{center}
\vspace{\medskipamount}
\begin{center}
B. Zusammensetzung der unzersetzbaren Silicate.
\end{center}
\begin{center}
\begin{tabular}{ |p{23mm}|p{12mm}|p{9mm}|p{13mm}|p{8mm}|p{11mm}|p{8mm}|p{10mm}|p{8mm}| }
    \hline
     & Kl.-Wenden & Sauer. & Chanton. & Sauer. & Blansko & Sauer. & Kakova & Sauer.\\
    \hline\hline
    Magnesia & 12,75 & 9,94\footnote{\frakfamily{Ohne den Sauerstoff des Nickeloxyds und so auch bei den folgenden.}} & 10,23 & 7,19 & 12,60 & 7,90 & 15,86 & 8,25\\\hline
    Kalk & 2,77 & 9,94 & 1,52 & 7,19 & 1,22 & 7,90 & 0,81 & 8,25\\\hline
    Eisenoxydul & 6,39 & 9,94 & 4,75 & 7,19 & 4,57 & 7,90 & -,- & -,-\\\hline
    Manganoxydul & -,- & -,- & 0,34 & 7,19 & 0,34 & 7,90 & -,- & -,-\\\hline
    Nickeloxyd & 0,21 & 9,94 & 0,67 & 7,19 & -,- & -,- & -,- & -,-\\\hline
    Natron & 0,41 & 9,94 & 0,49 & 7,19 & 0,50\footnote{\frakfamily{Kalihaltig.}} & 7,90 & 1,92 & 8,25\\\hline
    Kali & 0,53 & 9,94 & 0,25 & 7,19 & -,- & -,- & 0,26 & 8,25\\\hline
    Tonerde & 5,23 & 9,94 & 2,95 & 7,19 & 2,79 & 7,90 & 2,46 & 8,25\\\hline
    Kiesels"aure & 29,46 & 15,29 & 27,49 & 14,27 & 30,75 & 15,96 & 21,74 & 11,28\\\hline
     & 57,75 & & 48,69 & & 52,77 & & 43,05 & \\
    \hline
\end{tabular}
\end{center}
\vspace{\medskipamount}
\paragraph{}
Hieraus ergibt sich:

Dass das Nickeleisen der Chondrit in seiner chemischen Zusammensetzung mit dem f"ur sich vorkommenden in dem Meteoreisen "ubereinstimmt, 

dass die Silicate als Basen vorzugsweise Magnesia und Eisenoxydul enthalten, und alle andern, wie Tonerde, Kalk, Kali, Natron sich nur in sehr geringer Menge finden,

dass die Menge der zersetzbaren und unzersetzbaren Gemengteil in allen Chondriten ungef"ahr gleich ist,

dass bei den zersetzbaren Silicaten die Basen fast allein aus Magnesia und Eisenoxydul in etwas gegeneinander verschiedenen Mengen bestehen, w"ahrend die nicht zersetzbaren dieselben Basen mit kleinen Mengen von Natron, Kali, Kalk und besonders von Tonerde enthalten.

Bei den zersetzbaren Silicaten ist der Sauerstoff der Basen als gleich mit dem Sauerstoff der Kiesels"aure anzunehmen. Wenn in der Tat der Sauerstoff der Basen ein klein wenig gr"osser als der der Kiesels"aure ist, so erkl"aren dies Berzelius und Rammelsberg gewiss sehr richtig dadurch, dass wohl immer noch etwas Nickeleisen in diesem Teile der Chondrit zur"uckgeblieben ist, der mit dem Magnete nicht vollst"andig hat ausgezogen werden k"onnen, und dass durch die S"aure auch schon eine gewisse Menge von dem doch immer nur schwer, nicht v"ollig unzersetzbaren Teile zersetzt ist, dessen Basen, wie die kleinen Mengen von Tonerde, Kalk und die Alkalien beweisen, nun hier auftreten, w"ahrend die analytische Methode zur Folge hat, dass immer etwas Kiesels"aure von dem zersetzbaren Teile bei dem unzersetzbaren bleibt.

Bei dem unzersetzbaren Gemengteil ist der Sauerstoff der Basen meistenteils ungef"ahr halb so gro"s als der der S"aure.

Berzelius schloss daraus, dass der zersetzbare Gemengteil Olivin sei. Das Verh"altnis des Eisenoxyduls zur Magnesia ist bei dem Olivin der verschiedene Chondrit etwas verschieden, doch ist dies auch bei dem terrestrischen Olivin der verschiedenen Fund"orter der Fall. Den unzersetzbaren Gemengteil hielt er wiederum f"ur ein Gemenge und zwar eines Leucitartigen Minerals, welches die Tonerde und die Alkalien, und eines Augit-artigen, welches die andern Basen, Magnesia, Kalk und Eisenoxydul enthielt,\footnote{\frakfamily{Pongendorffs Ann. 1836 B. 15, S. 221.}} f"ur welchen ersteren indessen Rammelsberg, wegen seiner Aufl"oslichkeit in S"auren ein feldspatartiges Mineral und zwar den Labrador annahm,\footnote{\frakfamily{Mineralchemie S. 931.}} der zwar auch, aber schwerer als der Leucit zersetzbar ist und dessen Sauerstoffverh"altnis sich nicht viel von dem des Leucits entfernt.\footnote{\frakfamily{Der Sauerstoff der Basen zu dem der Kiesels"aure ist bei ihm wie 1 : 1 1/2.}}

Dieser Annahme ist man nun sp"ater teils gefolgt, Teils hat man wieder neue Kombinationen gemacht. So halten es Chancel und Moitessier\footnote{\frakfamily{\emph{Comptes rendus}, 1859 t. 47, p. 267 et p. 479.}} bei der Analyse des Ch. von Montrejeau (Ausson) f"ur wahrscheinlich, dass darin nicht Labrador und Augit, sonders Oligoklas und Hornblende enthalten sei, Sartorius\footnote{\frakfamily{Ber. d. Wiener Akad. 1859, B. 34 (S. 3).}} berechnet in dem unzersetzbaren Gemengteil des Ch. von Kakova nach der Analyse von Harris 82,17 Magnesia-Wollastonit und 17,4 Anorthit, und Abich\footnote{\frakfamily{Bulletin de l'acad. d. sciences de St. Petersbourg 1860, t. 2, p. 404.}} nimmt an, dass die Silicate in dem Ch. von Stauropol Hyalosiderit, Olivin und Labrador seien und ersterer den zersetzbaren, letztere den unzersetzbaren Gemengteil ausmachen.

Zu allen diesen Annahmen gibt aber die mineralogische Untersuchung, wie das eben Angef"uhrte dartut, nicht das mindeste Anhalten. Labrador, Oligoklas und Anorthit mit ihren tafelartigen Kristallen und der so charakteristischen Streifung auf den Spaltungsfl"achen des Querbruchs zeigt sich nie, Augit und Hornblende m"ussten bei ihrer Unersetzbarkeit durch S"auren doch nach Aufl"osung des zersetzbaren Gemengteil in dem unzersetzbaren zu finden sein, in welchem sich wohl etwas Chromeisenerz, aber nichts anderes erkennen l"asst. Magnesia- Wollastonit ist noch nie beobachtet, m"usste aber wie das Wollastonit durch S"auren zersetzbar sein, und k"onnte sich nicht, ebenso wenig wie der Anorthit in dem unzersetzbaren Gemengteil finden, was schon W"ohler bemerkt hat. Olivin kann bei seiner leichten Zersetzbarkeit auch nicht in dem unzersetzbaren Gemengteil vorkommen und ihn au"serdem in einem Gemenge mit Hyalosiderit anzunehmen, ist nicht zu billigen, da beide isomorph sind und isomorphe K"orper in einem Gemenge nie beobachtet sind und auch nicht vorkommen k"onnen, da man nicht einsieht, warum sich nicht beide h"atten, verbinden und in diesem Falle einen Olivin von mittlerem Eisengehalt h"atten bilden sollen.\footnote{\frakfamily{Der Beweis, den Abich f"ur die Richtigkeit seiner Annahme aus dem Umstand entnimmt, dass die Summe der spezifischen Gewichte der angenommenen Gemengteil mit den spezifischen Gewichten des Steins stimmt, ist scheinbar, da dergleichen Berechnungen bei den Gebirgsarten aus vielen Gr"unden noch nie sichere Resultate gegeben haben.}}

Man sieht, zu welcher Willk"ur es f"uhrt, wenn man Mineralien aus der Analyse eines Mineralgemenges berechnet, die man nicht gesehen hat, und noch dazu das chemische Verhalten dieser Mineralien gar nicht ber"ucksichtigt. Die einzigen Gemengteil, die man mit Sicherheit in diesen Meteoriten annehmen kann, weil man sie sehen kann, sind au"ser dem Nickeleisen und Magnetkies nur Olivin und Chromeisenerz. Woraus die Grundmasse besteht, ob der schwarze Gemengteil nur Chromeisenerz oder ein Gemenge desselben mit einer anderen Substanz sei, wie die mikroskopische Untersuchung wahrscheinlich macht, und woraus ganz besonders die Kugeln bestehen, wissen wir nicht. Freilich sind diese schon analysiert worden, aber nur in einer fr"uhen Zeit von Howard,\footnote{\frakfamily{Philos. transactions 1802 und daraus in Gilberts Annalen B. 13, S. 311.}} dem wir "uberhaupt die erste Analyse eines Meteoriten verdanken. Derselbe trennte bei der Analyse des Ch. von Benares die Kugeln von der Grundmasse und untersuchte beide besonders, nachdem er aus den letzteren das Nickeleisen mit den Magneten ausgezogen hatte, und fand nun beide ungef"ahr gleich zusammengesetzt, n"amlich:
\begin{center}
\begin{tabular}{ l r r }
     & in den Kugeln & in der Grundmasse\\
    Magnesia & 15,00 & 18,00\\
    Nickeloxyd & 2,5 & 2,5\\
    Eisenoxyd & 34,00 & 34,00\\
    Kiesels"aure & 50,00 & 48,00\\
\end{tabular}
\end{center}
\paragraph{}
Sp"ater sind sie bei der Analyse wenig beachtet. Berzelius stellte bei der Analyse des Ch. von Blansko aus Mangel an Material nur einige Versuche mit ihnen an, aus denen er indessen glaubte denselben Schluss wie Howard ziehen zu k"onnen, dass sie mit der "ubrigen Masse des Meteoriten gleich zusammengesetzt w"aren.\footnote{\frakfamily{Pongendorffs Annalen 1834, B. 33, S. 21.}} Er fand dabei, dass ein Teil ihres Pulvers gelatiniere, ein anderer von der S"aure gar nicht ver"andert werde. In der neusten Zeit glaubt zwar Damour, dem wir auch eine sorgf"altige Analyse des an diesen Kugeln so reichen Meteoriten von Montrejeau (Ausson) verdanken,\footnote{\frakfamily{\emph{Comptes rendus} 1859 t. 49, p. 31.}} dass er eine Analyse der Kugeln gegeben habe, indem er der Meinung ist, dass daraus der ganze von S"auren nicht oder schwer zersetzbare Teil des Meteoriten bestehe, den er besonders untersucht hat; dies ist aber doch noch zu beweisen, da er die Kugeln f"ur die Analyse nicht besonders ausgesucht hat. Diesen unzersetzbaren Teil, den er allein als aus den Kugeln bestehend annimmt und den er "ahnlich wie bei den oben angef"uhrten Analysen zusammengesetzt fand, h"alt er aber nicht f"ur ein einfaches Mineral, sondern f"ur ein Gemenge von Augit und Feldspat, berechnet somit wieder Mineralien, die gar nicht in den Kugeln zu erkennen sind und noch dazu ein Gemenge, das nie beobachtet ist, was ihm auch Leymerie\footnote{\frakfamily{\emph{Comptes rendus} 1859, t. 49, p. 247.}} vorwirft, der aus der Analyse von Damour nur den Schluss ziehen zu k"onnen glaubt, dass die Kugeln ein Silicat von Magnesia und Eisenoxydul mit h"oherem Kiesels"auregehalt als im Olivin seien, wenn nicht dieser h"ohere Kiesels"auregehalt von anh"angender Grundmasse herzuleiten sei.

Aus der mikroskopischen Untersuchung der Kugeln geht hervor, dass dieselben zweierlei Art ist, die einen vielleicht nur Olivin, die andern mit fasriger Struktur davon verschieden, ein wahrscheinlich Tonerde-haltiges Doppelsilicat. Es bleiben demnach von wirklich bestimmten Gemengteilen in den Chondriten immer nur noch Nickeleisen, Magnetkies, Olivin und Chromeisenerz, von nicht bestimmten die fragliche Grundmasse, der schwarze Gemengteil neben dem Chromeisenerz und die Kugeln mit fasriger Struktur, wenn die "ubrigen Olivin sind.
\subsection{\frakfamily{Howardit.}}
\paragraph{}
Zu dieser Art geh"oren nur wenige Meteorit; von den im mineralogischen Museum befindlichen nur der von
\begin{enumerate}
    \item Loutolax in Finnland, gef. 1813, 13. Dez.\footnote{\frakfamily{Nicht Lontolax, wie gew"ohnlich geschrieben wird (s. Nordenski"old in der weiter unten zitierten Schrift, auch Berzelius in Poggendorffs Ann. 1834 B. 33 letzte Seite unter den Berichtigungen, die von Berzelius herr"uhren. Lautolax, wie hier steht, ist nur die schwedische Aussprache von Loutolax). Nach Buchner m"usste es hei"sen Luotolaks, welches finnische Wort „Felsenbucht” bedeutet. (Poggendorffs Ann. 1862 B. 116, S. 643.)}}
    \item Bialystok in Russland, gef. 1827, 5. Okt.
    \item M"assing in Bayern, gef. 1803, 13. Der.
    \item Nobleborough in Maine, Ver. St. 1823, 7. Aug.
    \item Mallygaum in Kandeish, Ostindien 1843, 26. Juli.
\end{enumerate}
\paragraph{}
Bei der geringen Zahl der Meteoriten dieser Art ziehe ich es vor, sie einzeln zu beschreiben, und fange mit dem Ch. von Loutolax an, da mit diesem schon die meisten Untersuchungen gemacht sind.

1) Loutolax, Wiborgs L"an in Finnland. Die Steine, deren nur wenige gesammelt sind, da sie meistenteils auf die Eisdecke eines Sees fielen, in welchen sie beim Tauen einsanken, sind von Nordenski"old\footnote{\frakfamily{Bidrag til n"armere K"annedom af Finlands mineralier, p. 99 und daraus in: Neues Journ. f. Chem. u. Phys., T. 1, S. 60.}} beschrieben und von Berzelius,\footnote{\frakfamily{Pongendorffs Annalen 1834, B. 53, S. 30 und Rammelsbergs Mineralchemie, S. 940.}} wenn auch nur unvollst"andig, untersucht. Das mineralogische Museum besitzt davon nur zwei kleine St"uckchen, zusammen 0,327 Lth. schwer, welche beide ein Geschenk von Berzelius sind.

Der M. von Loutolax hat eine porphyrartige Struktur und besteht aus einer graulichwei"sen, feink"ornigen, sehr zerreiblichen Grundmasse mit eingemengten kleinen K"ornern von gr"unlichgelber, wei"ser und schwarzer Farbe.

Die gr"unlichgelben K"orner sind die h"aufigsten; sie sind h"ochstens von der Gr"o"se eines Stecknadelknopfes, meistens kleiner und unregelm"a"sig begrenzt, haben aber ganz das Ansehen wie Olivin, wof"ur sie auch von Nordenski"old und Berzelius gehalten sind. Vor dem L"otrohr werden sie dunkler von Farbe und schw"arzlichgr"un, schmelzen aber nicht.

Die wei"sen K"orner sind seltener und auch nicht gr"osser und zerbrechen dabei, wenn man sie herausnimmt, in noch kleinere St"uckchen. Sie sind nur unregelm"a"sig begrenzt, scheinen aber doch spaltbar zu sein. Vor dem L"otrohr sind sie nach Nordenski"old unschmelzbar und l"osen sich nur langsam in Borax und Phosphorsalz auf; mit Phosphorsalz opalisiert die Kugel beim Erkalten; mit Kobaltsolution befeuchtet, werden sie blau. Nordenski"old ist der Meinung, dass sie Leucit sind, doch damit w"are die Spaltbarkeit im Widerspruch. Eher m"ochte ich vermuten, dass sie Anorthit w"aren; die K"orner waren aber zu klein und die mir zu Geboten stehende Masse zu gering, um weitere Versuche damit machen zu k"onnen. Die Entscheidung muss daher noch dahin gestellt bleiben.\footnote{\frakfamily{Berzelius hat diese K"orner gar nicht erkannt, er verwechselt sie mit der Grundmasse, und glaubt Nordenski"old habe diese f"ur Leucit gehalten, was doch nicht der Fall ist.}} An manchen Stellen war der Olivin mit diesen wei"sen K"ornern fein gemengt.

Von schwarzen K"ornern finden sich einzelne gr"o"sere und kleinere in der Masse zerstreut. Die ersteren lassen sich meistens leicht aus der Masse, in der sie liegen, herausnehmen und hinterlassen darin den Eindruck ihrer Masse, sind aber doch meistens mit kleinen wei"sen K"ornern gemengt und zerbrechen beim Herausnehmen auch leicht. Sie sind im Bruch matt und dicht, nicht magnetisch, geben ein lichtes braunes Pulver, l"osen sich vor dem L"otrohr in Phosphorsalz nur langsam auf und geben ein nur schwach gr"un gef"arbtes Glas, das beim Erkalten opalisiert. Wahrscheinlich sind sie, ungeachtet der nur schwachen gr"unen F"arbung und des Opalisierens, was von der Beimengung herr"uhren mag, Chromeisenerz. Als ein tafelf"ormiges solches Korn mit rauer Oberfl"ache herausgenommen wurde, fand sich die entstandene H"ohlung mit kleinen K"ornchen gediegen Eisen umgeben, von denen eins, welches herausfiel, die Gr"o"se eines Hirsekorns hatte. Feine K"orner, wenn auch f"ur das Auge nicht sichtbar, finden sich noch in der Masse zerstreut, denn aus dem Pulver des Meteoriten kann man mit den Magneten feine Teilchen ausziehen, die im M"orser sich zu silberwei"sen, stark metallisch gl"anzenden Bl"attchen breitdr"ucken lassen. Ob sie nickelhaltig sind, haben weder ich noch Nordenski"old untersucht, der ihrer ebenfalls erw"ahnt. Berzelius leugnet das Dasein dieses gediegenen Eisens und h"alt die K"orner, die er mit den Magneten auszog, f"ur Magneteisenerz, da sie sich in Chlorwasserstoffs"aure ohne Geruch nach Schwefelwasserstoff und ohne Gasentwickelung zu einer dunkelgelben Fl"ussigkeit aufl"osen. Indessen m"ochte doch der starke metallische Glanz, die fast silberwei"se Farbe und die Dehnbarkeit, die beim Zerreiben im M"orser zum Vorschein kommen, keinen Zweifel gestatten, dass die K"orner gediegenes Eisen sind. Auf jeden Fall ist dies Verhalten f"ur die Beurteilung ihrer Beschaffenheit geeigneter als das Verhalten gegen S"aure, da eine schwache Gasentwickelung bei geringer angewandter Menge leicht "ubersehen werden kann.

Au"ser diesem Eisen bemerkte ich auch noch sehr kleine, metallisch gl"anzende K"orner von br"aunlichgelber Farbe, die offenbar Magnetkies sind. Sie finden sich aber nur so sparsam, dass man gewiss wird St"ucke aufl"osen k"onnen, ohne den Geruch nach Schwefelwasserstoff zu bemerken. Ein solches Korn Magnetkies sah auch Partsch, der aber wiederum von dem gediegenen Eisen nichts angibt.

Die zerreibliche Grundmasse, in welcher die angegebenen Gemengteil liegen, schmilzt vor dem L"otrohr zu einem schwarzen Glase, das nur sehr schwach magnetisch ist und sich in Phosphorsalz nur sehr langsam mit Hinterlassung eines R"uckstandes von Kiesels"aure aufl"ost. Das entstandene Glas ist gr"unlichwei"s, solange es hei"s ist, wird beim Erkalten erst wasserhell und opalisiert zuletzt.

Das spezifische Gewicht des Steins gibt Rummler zu 3,07 an.

"au"serlich ist der Stein mit einer schwarzen gl"anzenden Rinde umgeben.

Bei einer leider nur unvollst"andig gebliebenen Analyse dieses interessanten Meteorsteins zersetzte Berzelius denselben, nachdem alles dem Magnete Folgsame ausgezogen war, mit K"onigswasser und erhielt nur 6,45 pC. unl"oslichen R"uckstand; 93,55 Teile wurden von der S"aure zersetzt. Diese bestanden aus:
\begin{center}
\begin{tabular}{ l r r }
     & & Sauerstoff\\
    Magnesia & 32,92 & 13,17\\
    Eisenoxydul & 28,61 & 6,35\\
    Manganoxydul & 0,79 & 0,18\\
    Tonerde & 0,26 & 0,12\\
    Kiesels"aure & 37,42 & 19,42\\  
    Zinnoxid, Kupferoxyd, Kali und Natron & Spur & \\  
     & 100,00 & \\
\end{tabular}
\end{center}
\paragraph{}
Hiernach h"atte dieser Teil die Zusammensetzung eines Olivins, der 1 Atom Eisenoxydul gegen 2 Atome Magnesia enth"alt. Indessen ist er doch nicht f"ur ein einfaches Mineral zu halten, da der Augenschein lehrt, dass er ein Gemenge ist.

Die unzersetzten 6.45 wurden mit Flusss"aure behandelt, wodurch ungef"ahr 1 pC. Chromeisenerz unaufgel"ost blieb; das "ubrige war ein Silicat von Tonerde, Eisenoxydul, Manganoxydul, Kalk und Magnesia, nach Berzelius in Verh"altnissen, wie sie etwa in dem unzersetzten Teil in dem Meteorstein von Blansko vorkommen.

2) Bialystok (Dorf Knasta) in Russland. Das mineralogische Museum besitzt noch jetzt, nachdem davon ein Teil an das kais. Mineralienkabinett in Wien abgegeben, ein 4,857 Loth schweres St"uck, das die kais. Akademie der Wiss. in Petersburg Hrn. A. v. Humboldt verehrte und das von diesem dem mineralogischen Museum in Berlin "ubergeben ist.\footnote{\frakfamily{Vgl. Reise nach dem Ural, Altai und die kaspischen Meere von G. Rose Th. I, S. 77.}}

Der H. von Bialystock hat die gr"o"ste "ahnlichkeit mit dem vorigen und enth"alt in der feink"ornigen wei"sen Grundmasse dieselben Gemengteil, n"amlich:

Olivin in einzelnen K"ornern von Hirsekorn-Gr"o"se bis zu K"ornern von anderthalb Linien Durchmesser, ohne regelm"a"sige Form, die einen unebenen Bruch, verschiedene gelblichgr"une, grasgr"une, holzbraune Farben und Fettglanz haben.

Anorthit (?) in wei"sen K"ornern von h"ochstens Stecknadelknopf-Gr"o"se, schneewei"s und von unebenem Bruch; zuweilen sieht man eine Spaltungsfl"ache, die glatt auch wohl gestreift ist; ebenso br"ocklig wie bei dem H. von Loutolax.

Au"ser diesen K"ornern von Olivin und Anorthit, die einzeln in der Grundmasse zerstreut sind, sieht man in derselben aber noch mehrere 3-4 Linien gro"se Partien liegen, die ein k"orniges Gemenge der angef"uhrten Gemengteil sind. Sie sind scharf begrenzt und unregelm"a"sig von Gestalt, nehmen aber zuweilen fast das Ansehen eines Kristalls an, und erscheinen wie rektangul"are Prismen von Olivin mit ziemlich geraden Fl"achen, die aber im Innern durchg"angig mit den kleinen wei"sen Kristallen gemengt sind. Diese scheinbaren Kristalle lassen sich vollst"andig aus der Grundmasse herausnehmen und hinterlassen darin ihrer Form entsprechende Eindr"ucke (in dem St"ucke des min. Museums ist ein solcher scheinbarer Kristall von 6 Linien Breite und H"ohe und von 4 Linien Tiefe sichtbar). Sie sind wegen des h"aufig beigemengten Anorthits gew"ohnlich von lichter Farbe; es kommen aber auch Partien vor, die eine dunklere, ohne Lupe betrachtet, fast schw"arzlichgraue Farbe haben, doch sind dies ebensolche Gemenge nur mit vorwaltendem Olivin.

Es wird hiernach wahrscheinlich, dass auch die Grundmasse wenigstens haupts"achlich nichts anderes als ein eben solches nur feink"ornigeres Gemenge von Olivin und Anorthit ist, wie es die grobk"ornigen Partien sind.\footnote{\frakfamily{Diess w"are denn auch bei dem Stein von Loutolax zu vermuten.}} Der H. von Bialystock verhielte sich also in dieser R"ucksicht wie der weiter unten zu erw"ahnende Eukrit von Stannern, bei dem sich auch an ein und demselben St"ucke fein- und grobk"ornige Ab"anderungen finden; nur mit dem Unterschiede, dass letzterer gr"o"stenteils grobk"ornig, und nur selten und an einzelnen Stellen feink"ornig ist, der H. von Bialystock dagegen gr"o"stenteils feink"ornig, und nur an einzelnen Stellen grobk"ornig ist.

Die feink"ornige Grundmasse ist vor dem L"otrohr zu einem schwarzen Glase schmelzbar, das in d"unnen Splittern nur br"aunlich gef"arbt, und durchsichtig ist, wie die Grundmasse von Loutolax.\footnote{\frakfamily{Diess ist kein Beweis gegen die Annahme von Anorthit und Olivin in derselben, die beide unschmelzbar sind, da ein k"unstliches Gemenge von Anorthit und Olivin auch vor dem L"otrohr schmelzbar ist.}}

Gediegen Eisen findet sich ebenfalls in dem H. von Bialystock; man sieht es in "au"serst kleinen K"ornern in den groben k"ornigen und besonders den dunkleren Partien liegen; besser erkennt man es noch, wenn man kleine St"ucke zu Pulver reibt, wo man dann die feinen Eisenk"orner mit dem Magnete ausziehen kann. Ebenso kann man auch einzelne feine K"orner von Magnetkies erkennen.

Spezifisches Gewicht nach Rummler 3,17. "au"serlich schwarze gl"anzende Rinde.

Fein gerieben, und mit Chlorwasserstoffs"aure in einem Reagenzglase gekocht, wird er zersetzt\footnote{\frakfamily{Ob vollst"andig, habe ich leider nicht untersucht.}} und gelatiniert. Mit Wasser gesch"uttelt, kann die abgeschiedene Kiesels"aure durch Abschl"ammen von kleinen schwarzen K"ornern, wenn auch nicht vollst"andig getrennt werden, die sehr wahrscheinlich Chromeisenerz sind, doch in zu geringer Menge in der kleinen Probe, die zu den Versuchen genommen, vorhanden waren, um weitere Versuche damit anstellen zu k"onnen.

3) M"assing (Dorf St. Nicolas) Landgericht Eggenfelde in Bayern; ein St"uck 1,433 Lth. schwer, von den dort einzeln gefallenen Steinen von 3 1/4 Pfund,\footnote{\frakfamily{Vergl. den Bericht daruber vom Prof. Imhof in Munchen in Gilberts Annalen von 1804. Bd. 18, S. 330.}} aus der Chladnischen Sammlung.

Gleicht dem H. von Bialystock vollkommen. In der graulichwei"sen feink"ornigen zerreiblichen Grundmasse liegen h"aufig K"orner von Olivin von verschiedener Gr"o"se und Farbe, gr"unlichgelb bis grasgr"un, von der Gr"o"se eines Hirsekorns bis zu der eines groben Schrotes, dann wieder gr"o"sere unvollkommen ausgebildete Kristalle von schmutzig gr"unlichgrauer Farbe, die kleine wei"se K"orner eingewachsen enthalten. Diese liegen auch sonst in der Masse, haben einen unebenen Bruch, und lie"sen von Spaltbarkeit nichts sehen.

Mit den Magneten lassen sich aus der gepulverten Masse Eisenk"orner ausziehen.

Spezifisches Gewicht nach Rummler 3,21, einer Kugel daraus (wahrscheinlich einer von den eingemengten unvollkommenen Kristallen) 3,26.

"au"sere schwarze gl"anzende Rinde.

Eine neuere chemische Analyse dieses Meteoriten ist nicht vorhanden, die "altere Untersuchung von Imhof hat folgende Bestandteile gegeben:
\begin{center}
\begin{tabular}{ l r }
    Magnesia & 23,25\\
    Eisenoxyd & 32,54\\
    Kiesels"aure & 31,0\\
    Nickel & 1,35\\
    Eisen & 1,8\\
    Verlust an Schwefel und Nickel & 10,06\\
     & 100\\  
\end{tabular}
\end{center}
\paragraph{}
Die H. von Nobleborough und Mallygaum sind nach den Proben in dem mineralogischen Museum den vorigen sehr "ahnlich, doch sind diese St"ucke zu klein, um zu einer besonderen Beschreibung dienen zu k"onnen.

Hiernach ist also der Howardit wahrscheinlich nur ein mehr oder weniger feink"orniges Gemenge von vorherrschendem Olivin mit Anorthit und einer geringen Menge von Chromeisenerz, Nickeleisen und Magnetkies.
\subsection{\frakfamily{Chassignit.}}
\paragraph{}
Zu dieser Art geh"ort unter den Meteoriten des Berliner Museums nur ein einziger, n"amlich der am 3. Oktober 1815 zu Chassigny bei Langres (Dep. der haute Marne) gefallene Meteorit, daher ich nach ihm die Art Chassignit zu nennen vorgeschlagen habe. Nachrichten "uber die Erscheinungen bei seinem Falle und eine Beschreibung seiner Masse haben Pistolet, Calmelet und Gillet de Laumont, eine chemische Analyse der Masse hat Vauquelin geliefert.\footnote{\frakfamily{Pistolet und Vauquelin in den Annales de Chemie et de Physique, 1816. t. 1, p. 49 und daraus in Gilberts Annalen von 1818, Bd. 58, S. 176; Calmelet und Gillet de Laumont in den Annales des mines, 1816. t. 1, p. 489.}} Er hatte bei seinem Falle in dem Boden ein Loch von einem viertel Meter Tiefe und einem halben Meter Breite eingeschlagen, und zersprang dabei in viele St"ucke, die weit herumgeworfen wurden, eins sogar bis zu einer Entfernung von 240 Schritt. Alle Steine, die man gesammelt hat, wogen beinahe 4 Kilogramm.\footnote{\frakfamily{Pistolet, Arzt in Langres, der 2 Tage nach dem Falle an Ort und Stelle war, sammelte noch 60 kleine Bruchst"ucke, das gr"o"ste, in dessen Besitz er gelangte, wog beinahe 1 Kilogrammen.}} Das Berliner Museum besitzt davon nur ein kleines 0,79 Loth schweres, jedoch deutliches St"uck, das aus der Chladnischen Sammlung stammt, und einige kleine Proben, die sp"ater erworben wurden.

Der Chassignit ist hiernach eine kleink"ornige fast gleichartige Masse von nur geringem Zusammenhalt, so dass sie sich schon zwischen den Fingern zerbr"ockeln l"asst. Sie ist von einer gr"unlichgelben Farbe, die sich etwas ins grau zieht, wenig gl"anzend von Fettglanz, nur an den Kanten durchscheinend und von der H"arte des Feldspats. Darin sind hie und da kleine br"aunlichschwarze K"orner eingesprengt, und noch viel sparsamer als diese, noch kleinere fast mikroskopische K"orner und Kristalle (Hexaeder, wie es scheint) von einer gelben metallisch gl"anzenden Substanz. Die Rinde ist d"unn, glatt und glanzlos; sie wirkt schon schwach auf die Magnetnadel; das Innere nicht. In der unmittelbaren N"ahe der Rinde ist der Stein braun gef"arbt, wie schon Calmelet bemerkt.\footnote{\frakfamily{Im Innern der Masse beobachtete Calmelet bei einem St"ucke „un indice de cristal plus complet,” dessen Form er f"ur ein niedriges schiefes rhombisches Prisma hielt. Gillet de Laumont, der das beschriebene St"uck von Calmelet eingeh"andigt bekam, hatte diesen Kristall aus der Masse, worin er steckte, herausgenommen und n"aher untersucht, und glaubte daran die Form des Augits erkannt zu haben; er gibt seine H"ohe zu 4 Millimeter an. Da die Beschreibung aber nur sehr unbestimmt ist, gar nicht angef"uhrt wird, dass sich der Kristall von der umgebenden Masse an Farbe unterschieden habe, ein einzelner Augitkristall in der "ubrigen ganz verschiedenen Masse sehr auffallend w"are, so scheint es mir, dass dem Ganzen nur eine T"auschung zum Grunde liegt, und der angebliche Kristall nur eine zuf"allig mehr regelm"a"sige Form eines der K"orner sei, die die ganze Masse zusammensetzen.}}

Vauquelin fand, dass die Masse mit Chlorwasserstoffs"aure gelatiniert, dass sie von Schwefels"aure mit Hinterlassung eines pulverf"ormigen R"uckstandes von Kiesels"aure zersetzt wird, ohne Entwickelung von Wasserstoffgas, wie bei den eisenhaltigen Meteoriten, und eine farblose Aufl"osung gibt.

Ich fand ihn ferner vor dem L"otrohr schwer schmelzbar zu einer schwarzen magnetischen Schlacke von derselben Beschaffenheit wie die Rinde.

Die Analyse von Vauquelin gab:  
\begin{center}
\begin{tabular}{ l r }
    Magnesia & 32,0\\
    Eisenoxyd & 31,0\\
    Kiesels"aure & 33,9\\
    Chromium & 2,0\\
     & 98,9\\
\end{tabular}
\end{center}
\paragraph{}
Nimmt man an, dass das Eisen als Oxydul in dem Steine enthalten sei, so verhalten sich die Sauerstoffmengen von
\begin{center}
Mg : Fe : Si = 12,57 : 6,19 : 17,60 
\end{center}
\begin{center}
oder von Mg + Fe : Si = 18,76 : 17,6.
\end{center}
d. i. beinahe wie 1 : 1. Das Silicat ist also Olivin, und zwar ein sehr eisenreicher Olivin, wodurch sich seine Schmelzbarkeit erkl"art.\footnote{\frakfamily{Vergl. die Zeitschrift d. d. geologischen Ges. 1861, Bd. 13, S. 526.}} Der gew"ohnliche Olivin, der 7 bis 12 pC. Eisenoxydul enth"alt, ist vor dem L"otrohr unschmelzbar, aber der Olivin des Kaiserstuhls, der sog. Hyalosiderit, der 28,49 pC. Eisenoxydul enth"alt, schmilzt wie der Olivin des Chassignit; ihm ist dieser vergleichbar. Auch das spezifische Gewicht stimmt damit; v. Schreibers und Rummler geben das spez. Gewicht des ganzen Steins zu 3,55 an.\footnote{\frakfamily{Partsch, Meteoriten, S. 147.}} Das spez. Gew. des gew"ohnlichen Olivins ist nach Mohs 3,441, das des eisenreichen ist genau noch nicht gekannt, aber notwendig etwas gr"osser.\footnote{\frakfamily{Die Angabe des spez. Gew. des Hyalosiderit von Walchner = 2,875 ist offenbar unrichtig und zu niedrig.}} Vauquelin hebt noch den Umstand hervor, dass in diesem Meteoriten kein Nickel vorhanden sei, was er der Abwesenheit des Eisens zuschreibt; dass sich aber auch in dem Olivin kein Nickel findet, ist in "ubereinstimmung mit den "ubrigen Erscheinungen, da in allen meteorischen Olivinen, die untersucht sind, kein Nickel gefunden ist.
\paragraph{}
Die kleinen sparsam eingemengten schwarzen K"orner h"alt Vauquelin f"ur metallisches Chrom. An dem kleinen St"ucke des min. Museums konnte ich bei keinem dieser schwarzen K"orner nur irgendetwas von regelm"a"siger Form wahrnehmen. Sie sind aber br"aunlichschwarz, von schwachem Halbmetallglanz, ganz undurchsichtig, geben zerrieben ein gelblichbraunes Pulver, sind schwach magnetisch, werden gegl"uht st"arker magnetisch, und l"osen sich vor dem L"otrohr in Phosphorsalz und Borax zu einem sch"onen gr"unen Glase auf. Diess sind alles Eigenschaften des Chromeisenerzes, daher man auch wohl unbedenklich die schwarzen K"orner f"ur dieses nehmen darf.

Schwieriger scheint die Bestimmung der gelben metallisch gl"anzenden K"orner. Sie l"osen sich in Salzs"aure nicht auf; dieser Umstand und die bei einigen derselben wahrscheinlich vorkommende Hexaederform sprechen f"ur Eisenkies; bei der geringen mir zu Geboten stehenden Menge habe ich aber keine weiteren Versuche machen k"onnen, und da Eisenkies in den Meteoriten bis jetzt noch nicht mit Sicherheit beobachtet ist, so w"aren allerdings noch weitere Versuche w"unschenswert, um "uber diesen Umstand mit Bestimmtheit zu entscheiden. Dass Vauquelin keinen Schwefel angibt, w"urde nichts beweisen, da bei der "au"serst geringen Menge, in der er nur vorhanden sein k"onnte, er ihm leicht entgangen sein kann.

Der Meteorit von Chassigny ist hiernach ein derber eisenreicher Olivin mit sparsam eingemengtem Chromeisenerz und einer noch geringeren Menge einer Substanz, die m"oglicher Weise Eisenkies sein k"onnte.\footnote{\frakfamily{Eine Analyse dieses Meteoriten, die Damour nach meiner ersten Beschreibung desselben in der d. d. geol. Ges. und nach der Lesung dieser Abhandlung in den \emph{Comptes rendus} (1862, t. 4, p. 591) bekannt gemacht hat, best"atigt dies Resultat vollkommen. Sie gibt an:
\begin{center}
\begin{tabular}{ l r r }
     & & Sauerstoff\\
    Magnesia & 37,76 & 12,48\\
    Eisenoxydul & 26,70 & 5,93\\
    Manganoxydul & 0,45 & 0,1\\
    Kali & 0,66 & \\
    Chromoxyd & 0,75 & \\
    Kiesels"aure & 35,20 & 18,33\\
    Chromeisenerz u. Augit & 3,77 & \\
\end{tabular}
\end{center}
Es folgt hieraus die genauere Formel
\begin{center}
(2Mg + Fe)$^{2}$ Si
\end{center}
welche die des Hyalosiderit vom Kaiserstuhl ist. Der in der Analyse angef"uhrte Augit ist nur angenommen, nicht bewiesen. Das spezifische Gewicht gibt Damour zu 3,57 an.}}
\subsection{\frakfamily{Chladnit.}}
\paragraph{}
Die vierte Art enth"alt wie die dritte nur einen Meteoriten, den, welcher zu Bishopville im n"ordlichen Teil des Sumter-Distrikts in S"ud-Carolina im M"arz 1843 gefallen ist. Es fiel dort nur ein Stein von ungef"ahr 13 Pfd., der bald nach seinem Falle in den Besitz des Prof. Shepard gelangte, dem wir auch die erste Beschreibung und Analyse dieses merkw"urdigen Steins verdanken.\footnote{\frakfamily{Silliman American Journal of Science and arts, sec. Ser. t. 2, p. 337.}} Nachher haben noch Sartorius\footnote{\frakfamily{Ann. d. Chem. u. Pharm. B. 79, S. 369.}} und Rammelsberg\footnote{\frakfamily{Monatsber. d. k. Akad. d. Wiss. zu Berlin 1861, S. 895.}} Untersuchungen damit angestellt. Die Berliner Sammlung besitzt davon ein gr"o"seres St"uck von 4,86 Loth und mehrere kleinere.

Der Stein ist von "au"serst geringem Zusammenhalt und zerf"allt in St"ucke bei dem geringsten Druck. Er ist im Allgemeinen von porphyrartigk"orniger Struktur. In einer hellgrauen, wei"s gefleckten, kleink"ornigen Grundmasse finden sich au"ser einigen andern Einmengungen von geringerer Bedeutung nur schneewei"se Kristalle von verschiedener Gr"o"se und in gro"ser Menge eingeschlossen, oft sich gegenseitig begrenzend und in der Ausbildung sich st"orend. Der gr"o"ste Kristall auf den St"ucken des Berliner Museums zeigt auf der Bruchfl"ache des Gesteins einen Durchschnitt von der Gestalt eines Rechtecks mit schwach abgestumpften Ecken, dessen L"ange einen halben Zoll betr"agt und dessen Breite etwas geringer ist; ein anderer zeigt den eines symmetrischen Sechsecks mit zwei parallelen l"angeren Seiten, das etwas "uber einen halben Zoll lang und etwas weniger als ein Viertel Zoll breit ist; doch sind bei beiden die Seiten der Durchschnitte nicht sehr geradlinig, die Umrisse aller "ubrigen Kristalle indessen noch undeutlicher. Die beiden ersteren Kristalle sind parallel einer Spaltungsfl"ache durchgebrochen, die aber nicht sehr vollkommen, sondern durch unebenen Bruch unterbrochen ist. Die Bruchfl"ache ist gl"anzend von Perlmutterglanz und wird von matten Streifen parallel den l"angeren Seiten der Durchschnitte durchzogen. Wodurch diese hervorgebracht werden, kann ich nicht angeben, man sieht aber dasselbe Aussehen auch bei den Durchschnitten anderer Kristalle, dagegen wieder andere ganz glatte Spaltungsfl"achen in einer Richtung zeigen. Endlich finden sich auch runde wei"se K"orner von verschiedener bis Erbsengro"se, die beim Herausnehmen runde, aber unebene H"ohlungen in dem Gestein hinterlassen, auch nur durchscheinend sind, aber Glasglanz haben. Nimmt man die Kristalle oder K"orner aus dem Gestein heraus, so zerfallen sie in kleine St"uckchen, da sie mit vielen geraden und krummen Kl"uften nach den verschiedensten Richtungen durchsetzt sind, und dann ist es nicht mehr m"oglich, die einzelnen Teilchen zu unterscheiden.

Es ist wahrscheinlich, dass die K"orner von den Kristallen verschieden sind und bei diesen sich vielleicht auch noch die mit glatten und gestreiften Spaltungsfl"achen unterscheiden. Shepard erw"ahnt auch der runden K"orner und trennt sie von den Kristallen; er erkl"art die ersteren f"ur Anorthit, ohne jedoch etwas anders zu ihrer Charakterisierung anzuf"uhren als einige L"otrohrversuche, die keine wesentlichen Verschiedenheiten von den mit den Kristallen angestellten gaben. Die Kristalle bezeichnet er als eine neue Mineralgattung, der er, wie schon S. 29 angef"uhrt, den Namen Chladnit gegeben hat und den ich mir erlaubt habe, wie ebenfalls angegeben, in den von Shepardit umzu"andern. Von der Form dieses Shepardit macht Shepard auch nur unbestimmte Angaben, wenngleich er Kristalle beobachtet hat, die nach seiner Angabe beinahe einen Zoll im Durchmesser haben. Im Allgemeinen haben sie nach ihm das Ansehen der gew"ohnlichen Formen des Feldspats oder Albits; die primitive Form w"are ein doppelt schiefes Prisma; durch Spaltung, die sich mit Leichtigkeit bewerkstelligen lie"se, w"aren Winkel von 120$^{\circ}$ und 60$^{\circ}$ zu erhalten.

Auch Sartorius spricht von der Form dieser Kristalle, und vergleicht sie mit der des Wollastonits, ohne weitere Winkel anzugeben; indessen hat er sie nur bei kleinen mikroskopischen Kristallen bemerkt und auch nicht angef"uhrt, wie diese vorkommen.

Die H"arte des Shepardit ist nach diesen Beobachtern die des Feldspats, das spezifische Gewicht nach Shepard 3,116, nach Sartorius 3,039.

Vor dem L"otrohr schmilzt der Shepardit nur an den Kanten zu einem wei"sen Email\footnote{\frakfamily{Nach Shepard schmilzt er ohne Schwierigkeit.}}; in Borax l"ost er sich in Pulverform leicht zu einem klaren Glase auf; in Phosphorsalz in St"ucken langsam, in Pulverform leicht zu einem Glase, das, solange es hei"s ist, wasserhell ist und ein Kieselskelett einschlie"st, beim Erkalten aber opalisiert.

Von hei"ser Chlorwasserstoffs"aure wird der Shepardit nur sehr schwer angegriffen. Diess ist auch mit den runden K"ornern der Fall, die Shepard f"ur Anorthit h"alt; fein gerieben und mit Chlorwasserstoffs"aure l"angere Zeit gekocht, gibt die verd"unnte und filtrierte Aufl"osung derselben mit Ammoniak erst nach einiger Zeit einige br"aunliche Flocken und das zur"uckbleibende Pulver knirscht nach wie vor mit dem Glasstab. Wenn ich K"orner derselben Art wie Shepard untersucht habe, so k"onnen diese kein Anorthit sein; denn dieser gibt gepulvert und mit Chlorwasserstoffs"aure gekocht sehr leicht eine Gallerte.\footnote{\frakfamily{Vergl. dar"uber die Anmerkung beim Anorthit des Eukrit.}}

Die chemische Zusammensetzung des Shepardit wird von Shepard (a), Sartorius (b) und Rammelsberg (c) folgenderma"sen angegeben:
\begin{center}
\begin{tabular}{ |l|r|r|r| }
    \hline
     & \emph{a} & \emph{b} & \emph{c}\\
    \hline\hline
    Magnesia & 28,25 & 27,12 & 34,80\\\hline
    Kalk & -,- & 1,82 & 0,66\\\hline
    Natron & 1,39 & -,- & 1,14\\\hline
    Kali & -,- & -,- & 0,70\\\hline
    Manganoxydul & -,- & -,- & 0,2\\\hline
    Tonerde & -,- & 1,48 & 2,72\\\hline
    Eisenoxyd & -,- & 1,70 & 1,25\\\hline
    Kiesels"aure & 70,41 & 67,14 & 57,52\\\hline
    Wasser & -,- & 0,67 & 0,80\\\hline
     & 100,95 & 99,93 & 99,79\\
    \hline
\end{tabular}
\end{center}
\paragraph{}
Nach Shepard w"are also der Shepardit, abgesehen von der geringen Menge Natron, nur eine Verbindung von Kiesels"aure mit Magnesia. Shepard allein hat keine Tonerde gefunden, was wahrscheinlich daher r"uhrt, dass er die reinsten Kristalle zur Analyse genommen hat, wie er auch die gr"o"ste Masse zu seiner Benutzung hatte und bei den analysierten Proben der beiden andern Chemiker, die nur kleinere Bruchst"ucke anwenden konnten, noch ein anderes Silicat beigemengt gewesen ist. Der Shepardit hat hiernach also dieselben Bestandteile wie der Olivin, nur in anderen Verh"altnissen; bei dem Shepardit ist der Sauerstoff der Magnesia (mit Einschluss des Natrons) zu dem der Kiesels"aure wie 1 : 3, w"ahrend er beim Olivin nur wie 1 : 1 ist. Der Shepardit ist also anderthalb kieselsaure Magnesia:
\begin{center}
Mg$^{2}$Si$^{3}$
\end{center}
und die berechnete Zusammensetzung:
\begin{center}
\begin{tabular}{ l r }
    Magnesia & 30,2\\
    Kiesels"aure & 69,8\\
\end{tabular}
\end{center}
w"ahrend der Olivin nur eine halb kieselsaure Magnesia ist:
\begin{center}
Mg$^{2}$Si
\end{center}
\paragraph{}
Sartorius von der Ansicht ausgehend, dass er ein Gemenge analysiert habe, nimmt an, dass die gefundene Tonerde von einem sogenannten Kalk-Labrador, d. i. von einem solchen Labrador herr"uhre, der nur Kalkerde als einatomige Basis enthielte. Er berechnet nach der Tonerde einen solchen Labrador, dessen Menge etwa 5 pC. betr"agt, zieht diesen von dem Ganzen ab und findet nun f"ur den R"uckstand, den seiner Meinung nach reinen Shepardit, dieselbe Formel wie Shepard. Dessen ungeachtet scheint mir eine solche Annahme nicht gerechtfertigt; denn einmal kennt man einen Labrador von solcher Zusammensetzung "uberhaupt nicht, und dann kann man Kristalle von der Form des Labradors nicht allein nicht in dem Chladnit erkennen, derselbe ist auch noch nie in den Meteoriten beobachtet worden. Eher k"onnte man als Beimengung Anorthit annehmen, da dieser doch in gewissen Meteorsteinen bestimmt beobachtet ist; aber auch diese Annahme ist nach dem oben Angef"uhrten nicht wahrscheinlich.

Rammelsberg, um zu sehen, ob seiner zu analysierenden Probe etwas Aufl"osbares beigemengt sei, digerierte das feine Pulver zuerst mit konzentrierter warmer Chlorwasserstoffs"aure und schloss nachher den R"uckstand mit kohlensaurem Natron auf. Da er aber durch die S"aure nur 7,55 pC. ausziehen konnte, die haupts"achlich aus Kiesels"aure und Magnesia ohne Tonerde bestanden, so war es wahrscheinlich, dass diese nur von einer anfangenden Zersetzung des Shepardit herr"uhrten, und er rechnete daher das durch Chlorwasserstoffs"aure Ausgezogene den Bestandteilen des R"uckstandes zu. Er "uberzeugte sich aber auf eine andere Weise, dass er ein Gemenge untersucht habe, indem er eine besondere Menge von dem Pulver der ausgesuchten wei"sen Kristalle schl"ammte, das Leichteste und Schwerste f"ur sich analysierte und in beiden eine nicht unbetr"achtliche Verschiedenheit in der Zusammensetzung fand. Bei der Schwierigkeit, die Beimengung zu bestimmen, unterl"asst es daher Rammelsberg ganz, f"ur die Zusammensetzung des Ganzen eine Formel aufzustellen. Wenn es demnach nun auch nicht zu bezweifeln ist, dass in dem Chladnit ein eigent"umliches Mineral wie der Shepardit vorkommt, so ist es doch auch ebenso wahrscheinlich, dass neben ihm sich noch ein anderes Tonerde-haltiges Mineral beigemengt findet, das man noch nicht kennt, wie auch von dem Shepardit selbst noch vieles, wie z. B. seine Form, fast ganz unbekannt ist.

Die "ubrigen nur in geringer Menge vorkommenden Gemengteil sind:

1. Nickeleisen, das teils in der Grundmasse, teils in und zwischen den wei"sen Kristallen eingemengt ist, gew"ohnlich nur sparsam und in kleinen K"ornern, nach Shepard doch zuweilen von der Gr"o"se einer Erbse. Ich selbst beobachtete in den St"ucken der Berliner Sammlung mehrere nicht viel kleinere K"orner. Das Eisen ist aber stets mit Eisenoxydul bedeckt, und dies hat auch die umgebende Masse braun gef"arbt, und zwar noch st"arker als bei den meisten "ubrigen Meteoriten. Bei Jen erw"ahnten gr"o"seren K"ornern kann man aber durch Schaben mit dem Messer den Metallglanz und die lichte stahlgraue Farbe des Innern erkennen.

2. Magnetkies in kleinen K"ornern selten.

3. Ein schwarzes Mineral, das nach Shepard Schwefelchrom ist.\footnote{\frakfamily{Shepard nannte es nach Hrn. von Schreibers: Schreibersit, welchen Namen dann Haidinger in den von Shepardit umge"andert hat und wof"ur nun nach dem Obigen nach vollst"andigerer Kenntnis dieses Gemengteil ein neuer Name zu machen w"are.}} Es findet sich in kleinen K"ornern oder kleinen G"angen, die den Chladnit durchsetzen, nach Shepard auch in kleinen prismatischen Kristallen, die auf den Seitenfl"achen gestreift und nach diesen Spuren von Spaltungsfl"achen zeigen. Sie sind unvollkommen metallisch gl"anzend, undurchsichtig, von der H"arte des Flussspats. Vor dem L"otrohr schmilzt dies Mineral schwer zu einem schwarzen Glase, das magnetisch ist; im Kolben erhitzt, bildet sich ein Sublimat von Schwefel. Mit Borax und Phosphorsalz bildet es gr"une Gl"aser; die Farbe ist im Allgemeinen intensiv, aber st"arker im Boraxglase als im Phosphorsalze und besonders bei der Schmelzung in der Inneren Flamme und nach Zusatz von Zinn.

4. Kleine gelbe Kristalle zwischen den K"ornern des Chladnit, vor dem L"otrohr unschmelzbar; sie werden wei"s und verhalten sich wie Shepardit und sind vielleicht nur durch Eisenoxydhydrat gef"arbter Shepardit.

Weiter habe ich nichts beobachtet. Shepard gibt ein blaues Mineral an, was er Jodolith nennt und vielleicht das ist, was ich Grundmasse genannt habe, ein honiggelbes, Apatit "ahnliches Mineral, Apatoid genannt, und Spuren von Schwefel. Au"serdem gibt er auch noch das ungef"ahre Verh"altnis an, in welchem die von ihm angef"uhrten Gemengteil in diesem Meteoriten enthalten sind; ich f"uhre es indessen nicht an, da diese ganze Angabe zu hypothetisch ist.

Die "au"sere glasige Rinde ist sehr merkw"urdig; ganz un"ahnlich der der "ubrigen Meteorsteine ist sie, wie sie auch Shepard beschreibt, gefleckt, schwarz, wei"s, bl"aulichgrau, die schwarzen Stellen gl"anzend und Obsidian-"ahnlich, die wei"sen und grauen meistenteils matt, die wei"sen doch auch gl"anzend und durchscheinend wie Porzellan und Email. Sie ist von einer Menge Risse und Spr"unge durchzogen.
\subsection{\frakfamily{Shalkit.}}
\paragraph{}
Diese Art enth"alt auch nur einen Meteoriten, den M. von Shalka in Bankoora, Ostindien, nach welchem Fundort ich den Namen der Art gebildet habe. Haidinger hat die Nachrichten dar"uber zusammengestellt und den Stein beschrieben.\footnote{\frakfamily{Sitzungsber. d. math.-naturw. Kl. d. k. Akad. d. Wiss. 1860. Bd. 41. S. 251.}} Er fiel den 30. November 1850 als ein einziger Stein und schlug in den nicht harten Boden gegen 4 Fu"s ein, wobei er in viele Tr"ummer zerschellte, und Bruchst"ucke bis 20 Fu"s weit umherflogen. Der Stein mochte einen Umfang von 3 1/2 Fu"s gehabt haben; die schwarze Rinde war stellenweise abgetrennt. Haidinger erhielt durch die k. Asiatische Gesellschaft in Bengalen mehrere St"ucke zusammen "uber 199 Centigr. schwer; ein gr"o"seres St"uck von 6 Pfund 3529 Grains wurde an das Britische Museum in London gesandt, von welchem Dr. Auerbach aus Moskau einige St"ucke erhielt, von denen er mir ein kleines haselnussgro"ssen St"uck "uberlie"s. Sp"aterhin erhielt ich selbst durch Hrn. Maskelyne aus dem Britischen Museum ein viel gr"o"seres St"uck von 4,79 Lth. Gewicht.

Nach diesen St"ucken ist die Masse ein kleink"orniges Gemenge von kleinen schneewei"sen, in einer Richtung deutlich spaltbaren K"ornern mit andern meistens etwas gr"o"seren gr"unlichgrauen, worin wieder noch gr"o"sere, oft ganz scharf begrenzt liegen, die eine schw"arzlichgr"une Farbe haben. Beide Farben geben aber stellenweise bei denselben K"ornern in eine lichtere gr"unlichgelbe "uber, in welchem Fall dann die K"orner eine gro"se "ahnlichkeit mit Olivin erhalten, wie auch schon der erste Beschreiber dieses Meteorsteins, Piddington in Calcutta, bemerkt hat.\footnote{\frakfamily{A. a. O. S. 257.}} Die gr"unen K"orner haben zuweilen einen Durchmesser von einem Zoll, sind dann aber sehr kl"uftig; sie haben einen kleinmuschligen Bruch, Glasglanz, der in den Fettglanz "ubergeht, sind an den Kanten mehr oder weniger durchscheinend, bei lichterer Farbe mehr, und unvollkommen nach einer Richtung spaltbar. Die Masse zum Teil sehr br"ocklig, das kleine St"uck, was auch im Allgemeinen feink"orniger ist, mehr als das gro"se.

Nach Haidinger besteht dieser Meteorstein ebenfalls aus feink"ornigen wei"slichen mit gr"ober k"ornigen dunkelaschgrauen Teilen, die bis 2 Linien in jeder Richtung gro"s sind und selbst wieder in kugligen Massen von mehrz"olligem Durchmesser in den wei"sen feink"ornigen eingeschlossen sind, oder von denselben wie Gangweise durchsetzt werden. Bei den gr"o"seren K"ornern sah er eine ziemlich deutliche Spaltungsfl"ache, die einen unvollkommene Winkel von 100$^{\circ}$ und 80$^{\circ}$ macht, ihre H"arte ist 6,5; indessen ist Haidinger der Meinung, dass diese ganze Masse nur eine einzige Spezies bilde, da ihre Teile zwar eine verschiedene Farbe h"atten, aber vollkommen ineinander "ubergingen.

In dieser Masse kommen nun noch K"orner von Chromeisenerz eingemengt vor, die zwar gew"ohnlich nur sehr klein und fein, doch in dem gr"o"seren St"ucke fast die Gr"o"se einer kleinen Erbse erreichen. Sie sind schwarz, stark gl"anzend von unvollkommenem Metallglanz, haben einen braunen Strich, und sind gew"ohnlich unregelm"a"sig begrenzt, doch auch in Oktaedern kristallisiert. K. v. Hauer beobachtete schon ein solches\footnote{\frakfamily{A. a. O. S. 257.}}; in dem gr"o"seren St"ucke sind deren mehrere enthalten; ihre Fl"achen sind so gl"anzend, dass man mit ziemlicher Genauigkeit ihre Winkel messen kann. K"orner und Kristalle kommen "ofter in den gr"unen K"ornern eingewachsen vor, sie sind schwer aus denselben vollst"andig herauszul"osen, da auch sie kl"uftig und br"ocklig sind.

Haidinger fand das spezifische Gewicht eines ziemlich reinen St"uckes (d.h. wo wohl weniger Chromeisenerz eingemengt war) 3,412, Piddington gibt 3,66 an.

Der Stein hat eine d"unne schw"arzlichbraune ganz matte Rinde, die aber doch uneben und fein adrig ist, und aus dem K"orner von Chromeisenerz herausragen.

Die von dem eingemengten Chromeisenerz m"oglichst befreite Masse ist von K. v. Hauer chemisch untersucht worden, sie besteht hiernach aus
\begin{center}
\begin{tabular}{ l r r }
     & & Sauerstoff\\
    Magnesia & 19,00 & 7,60\\
    Kalk & 1,53 & 0,43\\
    Eisenoxydul & 20,65 & 4,58\\
    Tonerde & Spur & \\
    Kiesels"aure & 57,66 & 30,50\\
     & 98,84\\
\end{tabular}
\end{center}
Haidinger, der die Masse f"ur gleichartig h"alt, betrachtet sie hiernach f"ur eine besondere Spezies, die mit dem Olivin und Shepardit in qualitativer Hinsicht "ubereinstimmend, in quantitativer Hinsicht zwischen beiden steht. Denn w"ahrend des Sauerstoffs der Basis zur S"aure sei
\begin{center}
beim Olivin = 1 : 1, und  
\end{center}
\begin{center}
beim Shepardit = 1 : 3, w"are er  
\end{center}
\begin{center}
bei dieser Spezies = 1 : 2,42.  
\end{center}
\paragraph{}
Dieses Verh"altnis n"ahere sich dem, welches Stromeyer bei dem olivin"ahnlichen Mineral aus der angeblich von Grimma (Steinbach) stammenden Eisenmasse gefunden habe, wo jenes Verh"altnis = 1 : 2,6 ist, und f"ur welches Mineral Rammelsberg die Formel RSi + 2R$^{2}$Si$^{3}$ aufgestellt habe.\footnote{\frakfamily{Diess gibt das Verh"altnis = 1 : 2,8; besser passt zu dem obigen die Formel RSi+R$^{2}$Si$^{3}$.}} Er schl"agt demnach f"ur dies Mineral den Namen Piddingtonit vor.

Dieser Meinung von der Gleichartigkeit der Masse m"ochte ich nicht beistimmen. Die Farbe der dunklen K"orner geht wohl in die gr"unlichgelbe "uber, aber nicht in die wei"se. Die K"orner mit gr"unlichgelber bis schw"arzlichgr"uner Farbe haben alle Eigenschaften des Olivins; ist dies aber der Fall, dann k"onnen die wei"sen K"orner nichts anderes als Shepardit sein, und der Shalkit w"are hiernach ein Gemenge von Olivin, Shepardit und Chromeisenerz, und von dem Chassignit nur durch die Anwesenheit des Shepardit verschieden.

Mit dieser Annahme stimmt auch das chemische Verhalten "uberein, denn sehr fein zerrieben, wird er von hei"ser Chlorwasserstoffs"aure zum Teil zersetzt, wie ein Gemenge von Olivin und Shepardit; das feine Pulver ist nur an den "au"sersten Kanten schmelzbar, noch schwerer als der Chassignit, und wird wie dieser schwarz.

Legt man die Hauersche Analyse zum Grunde, so w"urde der Shalkit auf 1 Atom Olivin 2 Atome Shepardit enthalten, also aus
\begin{center}
R$^{2}$Si + 2Mg$^{2}$Si$^{3}$
\end{center}
bestehen, denn in diesem w"urde der Sauerstoff von R : Si sein = 6 : 14, d. i. = 1 : 2,33.
\paragraph{}
Der Shalkit ist hiernach ein Gemenge von basisch und saurer kieselsaurer Magnesia oder von 2 Silicaten, die gleiche Basen in verschiedenen S"attigungsstufen enthalten, ein Gemenge wie es der Art unter den Gebirgsarten der Erde nicht vorkommt, wenngleich der Granit, wenn er Feldspat und Kaliglimmer enth"alt, etwas "ahnliches darstellt. Das Gemenge des Shalkit ist aber noch dadurch ausgezeichnet, dass die beiden Silicate in demselben in einem sehr einfachen Verh"altnis gegeneinander zu stehen scheinen, so dass sie sehr gut eine chemische Verbindung bilden k"onnten, "ahnlich dem Serpentin, der eine wasserhaltige Verbindung von basischer und neutraler Magnesia ist:
\begin{center}
Mg$^{2}$Si + MgSi + 2H
\end{center}
und es w"are daher wohl m"oglich, dass wie der Shalkit alle Gebirgesarten die Gemengteil in ebenso einfachen Verh"altnissen enthalten, wie die Mineralien die Bestandteile.\footnote{\frakfamily{Vergl. weiter unten die Anmerkung, S. 134.}}
\subsection{\frakfamily{Die Kohligen Meteorit.}}
\begin{enumerate}
    \item von Alais, Dep. du Gard, Frankreich, gef. 1806 den 15. M"arz.
    \item von Cold Bokkeveld am Cap der guten Hoffnung, gef. 1838 den 13. Oktober.
    \item von Kaba, SW. von Debreczin in Ungarn, gef. 1857 den 15. April.
    \item von Orgueil, Montauban, Frankreich, gef. 1864 den 14. Mai.
\end{enumerate}
\paragraph{}
Ich habe mit diesen keine besondere Untersuchungen angestellt, und verweise daher auf die vorhandenen Beschreibungen der drei ersten in Buchners Meteoriten, S. 19, 60 und 94 und des letzten in den \emph{Comptes rendus} 1864, t. 58.
\subsection{\frakfamily{Eukrit.}}
\paragraph{}
Es geh"oren zu dieser Abteilung nur wenige Meteoriten; von den Meteoriten des Berliner mineralogischen Museums nur der von:
\begin{enumerate}
    \item Juvenas, Dep. Ardèche, Frankreich, gef. 1821 den 15. Juni.
    \item Stannern bei Iglau, M"ahren, gef. 1808 den 22. Mai.
    \item Jonzac, Dep. Charente infér., Frankreich, gef. 1819 den 13. Juni.
    \item Petersburg, Lincoln Cty, Tennessee, V. St. N. A., gef. 1855 den 5. Aug.; au"serdem wahrscheinlich noch der vor Konstantinopel (gef. 1805 im Juni), der nach Partsch und Reichenbach eine sehr gro"se "ahnlichkeit mit namentlich den 3 erstgenannten dieser Abteilung hat. Bei der geringen Anzahl dieser Meteoriten ziehe ich es ebenfalls vor, dieselben einzeln zu beschreiben und fange mit dem E. von Juvenas an, der unter allen der ausgezeichnetste und am deutlichsten kristallinisch ist.
\end{enumerate}
\paragraph{}
1) Der E. von Juvenas ist meistenteils deutlich k"ornig und die Gemengteil "uber liniengro"s; was ihn indessen f"ur die Untersuchung besonders wichtigmacht, stellenweise auch voller H"ohlungen‚ an deren W"anden sich die Gemengteil regelm"a"sig begrenzt haben. Diess ist namentlich mit dem Augit der Fall, der "uberhaupt der vorwaltende Gemengteil ist. Derselbe erscheint hier in der Form, in welcher er sich gew"ohnlich in vulkanischen Gesteinen findet, als Kombination des rhombischen Prismas von ungef"ahr 88$^{\circ}$ mit der Quer- und L"angsfl"ache und dem schiefen Prisma von ungef"ahr 120$^{\circ}$.\footnote{\frakfamily{Vergl. die ausf"uhrliche Beschreibung in meiner ersten Arbeit "uber diesen Meteoriten in Poggendorffs Ann. von 1825 B. 4, S. 173.}} Er hat glatte und gl"anzende Fl"achen, so dass er ziemlich gut mit dem Reflexionsgoniometer gemessen werden kann. Die Neigung, die sich am besten bestimmen lie"s, war die Neigung der L"angsfl"ache zur Fl"ache des vertikalen Prismas, die ich von 136$^{\circ}$ 3’-5’ fand. Eine Spaltbarkeit war weder bei den aufgewachsenen noch bei den unregelm"a"sig begrenzten eingewachsenen Kristallen bei ihrer Kleinheit zu bemerken; sie sind im Bruch uneben, von Farbe schw"arzlichbraun, ungef"ahr wie der Nosean vom Laacher See, gl"anzend von Fettglanz und nur aus den "au"sersten Kanten durchscheinend. Vor dem L"otrohr schmilzt dieses Augit in der Platinzange zu einem schwarzen Glase, das von Magneten angezogen wird, was vor der Schmelzung nicht der Fall ist. In Borax wird er zu einem von Eisen schwach gef"arbten Glase aufgel"ost, in Phosphorsalz ebenso, nur schwerer und unter Ausscheidung von Kiesels"aure. Mit wenig Soda schmilzt er zu einer schwarzen Kugel, mit mehr Soda geht die Masse in die Kohle und das Eisenoxydul wird reduziert.

Der Anorthit ist in den H"ohlungen gew"ohnlich kleiner und viel undeutlicher Kristallisiert als das Augit, indessen sieht man doch das gew"ohnliche rhomboedrische Prisma mit der L"angsfl"ache \emph{M}, durch deren Vorherrschen die Kristalle gew"ohnlich ein tafelartiges Ansehen erhalten, sowie die schiefe Endfl"ache \emph{P}, parallel welcher die Kristalle immer sehr deutlich spaltbar erscheinen. Die Endfl"ache ist meistenteils glatt, nur zuweilen zeigt sie den der Kante mit der L"angsfl"ache parallel gehenden einspringenden Winkel, der den Zwillingskristall verk"undet. Winkel sind nicht zu messen, aber das ganze Ansehen der Kristalle zeigt den ein- und eingliedrigen Feldspat. Er ist schneewei"s und auf den Spaltungsfl"achen stark perlmutterartig gl"anzend. Vor dem L"otrohr nur sehr schwer an den Kanten schmelzbar. Gepulvert und mit hei"ser Chlorwasserstoffs"aure behandelt, gelatiniert er und gibt sich dadurch bestimmt als Anorthit zu erkennen.\footnote{\frakfamily{Dass der Anorthit, mit hei"ser Chlorwasserstoffs"aure digeriert, gelatiniert, ist bisher noch nicht angegeben, ist aber ein leichtes Mittel, ihn vom Labrador zu unterscheiden. In meiner ersten Beschreibung dieser Kristalle (a. a. O. S. 176) hatte ich sie gest"utzt auf die Angabe von Laugier, dass der Meteorit von Juvenas von S"auren nur schwierig angegriffen w"urde, und auf eine von mir ausgef"uhrte Untersuchung des alkalischen Bestandteils f"ur Labrador gehalten. Shepard hatte die Kristalle zuerst f"ur Anorthit ausgegeben (\emph{Report on American meteorites} p. 7. New Haven 1848), ohne weitere Beweise f"ur seine Behauptung zu liefern; Rammelsberg hat dies zuerst durch seine Analyse bewiesen (Pongendorffs Ann. von 1848 B. 73, S. 585).}}

Au"ser diesen wesentlichen Gemengteilen finden sich noch drei unwesentliche, kleine strohgelbe tafelartige Kristalle, etwas Magnetkies und in noch geringerer Menge etwas Nickeleisen. Die ersteren sind wenig gl"anzend und undurchsichtig, sie finden sich nicht gleichm"a"sig verbreitet in der Masse, im Allgemeinen sehr sparsam, am h"aufigsten in der Gegend der Drusenr"aume und in diesen selbst. Von einer bestimmten Kristallform ist indessen nichts zu erkennen. Sie haben wohl das Ansehen von Titanit und Rammelsberg h"alt sie auch daf"ur, weil er bei der Analyse etwas Titans"aure gefunden hat, aber ihr Verhalten vor dem L"otrohr ist damit nicht v"ollig in "ubereinstimmung. Sie schmelzen vor dem L"otrohr in der Platinzange an den Kanten zu einem schwarzen Glase, das magnetisch ist, und l"osen sich in Phosphorsalz, in geringer Menge zugesetzt, langsam aber v"ollig zu einem Glase auf, das hei"s etwas gr"unlich gef"arbt und kalt ganz wasserhell ist; in gr"o"serer Menge zugesetzt, bleibt Kiesels"aure ungel"ost zur"uck, und die Kugel opalisiert beim Erkalten, wird aber nicht bl"aulichwei"s in der innere Flamme, auch nicht wenn Zinn zugesetzt wird, was diese Kristalle vom Titanit unterscheidet, der auch (z. B. der gelbe von Arendal) an des R"andern nicht zu einem magnetischen Glase schmilzt. Dieser letztere Umstand w"are nicht entscheidend, er k"onnte von einem gr"o"seren Eisengehalt herr"uhren, aber die blaue Farbe m"usste doch zum Vorschein kommen. Es bleibt also "uber diese Substanz noch einige Unsicherheit, man sieht aus ihrem Verhalten nur so viel, dass sie ein Silicat ist und Eisenoxydul enth"alt.

Der Magnetkies findet sich in dem E. von Juvenas hier und da in kleinen K"ornern eingemengt, und in den H"ohlungen zuweilen kristallisiert. Ich habe in diesen zwei freilich nur sehr kleinen Kristallen gefunden, von denen der eine aber doch eine gro"se Menge von Fl"achen zeigt. Er ist eine Kombination eines Hexagondodekaeders mit der Basis, die wenig vorherrscht, einem stumpfen Hexagondodekaeder, dessen Fl"achen die Kombinationskanten die Grundform mit der Basis schwach abstumpfen, den Fl"achen des ersten sechsseitigen, und untergeordnet auch des zweiten sechsseitigen Prismas, und des ersten stumpferen Hexagondodekaeders als Abstumpfungen der Endkanten der Grundform. Die Fl"achen sind so glatt, dass ich die Winkel messen konnte. Ich fand den Winkel der Grundform in den Endkanten 126$^{\circ}$ 49’; der Winkel in den Seitenkanten betr"agt also 127$^{\circ}$ 6’.\footnote{\frakfamily{Vergl. die ausf"uhrlichere Beschreibung a. a. O. S. 180.}} So ausgebildete Kristalle sind bei dem tellurischen Magnetkies noch nicht vorgekommen, wiewohl die Kristalle bei diesem oft viel gr"o"ser sind. Sie sind gew"ohnlich nur Kombinationen eines regul"aren sechsseitigen Prismas mit der Basis, und wenn auch Fl"achen von Hexagondodekaedern vorkommen, so sind sie bisher nie so glatt und gl"anzend gefunden worden, dass man ihre Winkel mit Genauigkeit h"atte bestimmen k"onnen; die Winkel des Magnetkieses sind daher durch den meteorischen Magnetkies zuerst bestimmt. Sonst ist er wie der tellurische tombakbraun von Farbe und metallisch gl"anzend; sein Magnetismus jedoch nur sehr gering.\footnote{\frakfamily{In meiner fr"uheren Abhandlung gab ich an, dass er gar nicht magnetisch w"are; wenn man aber ein kleines St"uckchen auf eine Stahlplatte legt, so wird er doch von den Magneten etwas angezogen.}} Das Verhalten vor dem L"otrohr oder gegen S"auren ist ganz wie das des tellurischen, obgleich bei der geringen Menge, die nur genommen werden konnte, und weil sie nicht frei von ansitzendem Gestein war, nicht gesehen werden konnte, ob bei der Aufl"osung in S"auren etwas Schwefel zur"uckblieb. In der Aufl"osung wurde kein Nickel gefunden.

Nickeleisen findet sich in dem E. von Juvenas nur in sehr geringer Menge und nur in so feinen Teilen, dass man es auf der Bruchfl"ache des Gesteins nicht sehen und nur auf einer angeschliffenen Fl"ache durch seinen Metallglanz erkennen kann. Noch bestimmter kann man sich von seiner Gegenwart "uberzeugen, wenn man ein St"uckchen von dem E. von Juvenas pulvert, das Nickeleisen mit den Magneten auszieht und dann im M"orser breitdr"uckt.\footnote{\frakfamily{In der fr"uheren Beschreibung hatte ich das Nickeleisen bei seiner geringen Menge ganz "ubersehen.}}

Diese letztgenannten Gemengteil sind jedoch nur wie angegeben in sehr geringer Menge vorhanden. Das Nickeleisen ist ganz fein eingesprengt, der Magnetkies wohl in etwas gr"o"seren K"ornern oder unregelm"a"sigen Partien, aber doch auch nur sehr sparsam hier und da enthalten, die gelben Tafeln sind an den Stellen, wo sie sich finden, wohl ziemlich h"aufig, aber sie selbst sind doch nur klein und finden sich an anderen Stellen gar nicht, so dass alle diese letzteren Gemengteil auf das Ansehen des Meteoriten keinen Einfluss aus"uben. Dieses wird nur bedingt durch das Augit und Anorthit; dessen ungeachtet ist doch das Ansehen an verschiedenen Stellen sehr verschieden. Auch da wo die Gemengteil am gr"obsten sind, kann man das Gemenge nicht grobk"ornig nennen, es ist nur kleink"ornig, aber solche Stellen wechseln mit andern feink"ornigen, und gehen teils allm"ahlig darin "uber, teils schneiden sie von diesen scharf ab, wie dies auch bei dem Chondrit und Howardit der Fall ist. Zuweilen liegt in einer gr"o"seren feink"ornigen Partie eine kleinere gr"ober k"ornige, in die sie schnell "ubergeht, zuweilen sieht man in einer solchen eine kleine noch feiner k"ornige mit ziemlich scharf begrenzten Umrissen liegen; diese erscheinen dann gew"ohnlich von etwas dunklerer Farbe, ohne deswegen mehr Augit zu erhalten, zuweilen sind aber diese dunkleren Partien grobk"ornig, und die dunklere Farbe r"uhrt nun in der Tat von einer gr"o"seren Menge Augit her.

Das spezifische Gewicht des E. von Juvenas ist nach Rummler 3,11, nach Flangergues 3,09.\footnote{\frakfamily{Partsch, Meteoriten, S. 147.}}

Die schwarze Rinde ist wie die der Howardit stark gl"anzend und adrig, also ganz verschieden von der der Chondrit.

In einer d"unn geschliffenen Platte unter dem Mikroskop erscheint der Anorthit durchsichtig und ziemlich wasserhell, und mit nur wenigen Rissen durchsetzt, der Augit hellbraun durchscheinend, aber mit vielen gr"oberen braunen Spr"ungen und feineren Streifen durchzogen. Ein gr"o"serer Kristall zeigte gerade, untereinander parallele, sich in kleinen Entfernungen wiederholende dunkelbraune Spr"unge, die von andern "ahnlichen aber unregelm"a"sig verlaufenden durchsetzt wurden, und au"ser diesen noch feinere gerade, eng nebeneinander liegende, braune Streifen, die die geraden Spr"unge unter einem Winkel von etwa 80$^{\circ}$ schnitten. Auf eine "ahnliche Weise verhalten sich die Spr"unge und Streifen bei allen "ubrigen Kristallen. Nickeleisen in sehr kleinen K"ornern macht sich hin und wieder durch seinen Metallglanz kenntlich, Magnetkies war bei seiner unregelm"a"sigen Verteilung in dem Gestein auf der Platte gerade nicht vorhanden. Der Anorthit erscheint im Allgemeinen in kleineren Kristallen als das Augit, aber sie sind fast "uberall regelm"a"sig begrenzt, und namentlich sind die Querdurchschnitte durch seine tafelartigen Kristalle h"aufig zu sehen, dagegen das Augit bei gr"o"seren Individuen nie regelm"a"sige Umrisse hat. Man sieht so bestimmt, dass Anorthit der zuerst, Augit der sp"ater kristallisierte ist, wie auch der erstere schwerer, der letztere weniger schwer vor dem L"otrohr schmelzbar ist.

Vor dem L"otrohr schmilzt das Gestein fein zerrieben sehr leicht zu einem schwarzen stark magnetischen Glase, daher sich auch der Glanz der Rinde erkl"art. Das Gestein schmilzt leichter als die Gemengteil einzeln, wie dies auch oft bei gemengten tellurischen Gebirgsarten der Fall ist.

Fein gepulvert und mit hei"ser Chlorwasserstoffs"aure digeriert, gelatiniert der Anorthit, und die Gelatine schlie"st das Augit unver"andert ein, der daher auf diese Weise leicht und bestimmt von dem Anorthit zu trennen ist.

Auf diese Weise ist nun auch der E. von Juvenas von Rammelsberg analysiert worden.\footnote{\frakfamily{Poggendorffs Annalen von 1848, Bd. 73, S. 585 und Rammelsbergs Mineralchemie, S. 937.}} Er erhielt so:
\begin{center}
\begin{tabular}{ l r }
    A. mit S"auren Zersetzbares & 36,77\\
    B. mit S"auren Unzersetzbares & 63,23\\
\end{tabular}
\end{center}
und in diesen wie in dem Ganzen (C) die folgenden Bestandteile:
\begin{center}
\end{center}
\begin{center}
\begin{tabular}{ |p{23mm}|p{12mm}|p{8mm}|p{13mm}|p{8mm}|p{11mm}| }
    \hline
     & A. & Sauer. & B. & Sauer. & C.\\
    \hline\hline
    Kalk & 6,64 & 1,90 & 3,59 & 1,03 & 10,23\\\hline
    Magnesia & 0,13 & 0,05 & 6,31 & 2,52 & 6,44\\\hline
    Natron & 0,37 & 0,09 & 0,26 & 0,06 & 0,63\\\hline
    Kali & 0,12 & 0,02 & -,- & -,- & 0,12\\\hline
    Eisenoxydul & -,- & -,- & 19,48 & 4,32 & 19,48\\\hline
    Tonerde & 12,40 & 5,79 & 0,15 & 0,07 & 12,55\\\hline
    Eisenoxyd & 1,21 & 0,36 & -,- & -,- & 1,21\\\hline
    Kiesels"aure & 15,41 & 8,00 & 32,92 & 17,09 & 48,33\\\hline
    Phosphors"aure & 0,28 & -,- & -,- & -,- & 0,28\\\hline
    Titans"aure & -,- & -,- & 0,10 & -,- & 0,10\\\hline
    Eisen & 0,16 & -,- & -,- & -,- & 0,16\\\hline
    Schwefel & 0,09 & -,- & -,- & -,- & 0,09\\\hline
    Eisenoxydul & -,- & -,- & 0,92 & -,- & 0,92\\\hline
    Chromoxyd & -,- & -,- & 0,43 & -,- & 0,43\\\hline
     & 36,81 & & 64,16 & & 100,97\\
    \hline
\end{tabular}
\end{center}
\paragraph{}
Die Analyse ergab demnach die bekannten Bestandteile des Anorthits und Augits und au"ser diesen noch kleine Mengen anderer Bestandteile, in dem zersetzbaren Teile etwas Phosphors"aure und Eisenoxyd, und in dem unzersetzbaren etwas Titans"aure und Chromoxyd. Nickeleisen ist dagegen nicht aufgef"uhrt, bei seiner geringen Menge hat es Rammelsberg wie die fr"uheren Beobachter "ubersehen. Sieht man aber von diesen in geringer Menge enthaltenen Besonderen Bestandteilen ab, so verhalten sich bei dem zersetzbaren Gemengteil der Sauerstoff des Kalks, der Magnesia und der Alkalien zur Tonerde und zur Kiesels"aure, also von
\begin{center}
R : R : Si = 2,06 : 6,15 : 8,00 = 1,03 : 3,07 : 4
\end{center}
also fast wie 1 : 3 : 4, der Zusammensetzung des Anorthits gem"a"s.
\paragraph{}
In dem unzersetzbaren Gemengteil der Sauerstoff der s"amtlichen Basen zu dem der Kiesels"aure, also
\begin{center}
R : Si = 8,00 : 17,09 = 1 : 2,13
\end{center}
also fast wie 1 : 2, der Zusammensetzung des Augits gem"a"s.
\paragraph{}
Die Analyse ist demnach in vollkommener "ubereinstimmung mit dem Ergebnis der mineralogischen Untersuchung.

Rammelsberg sucht aber noch die "ubrigen Bestandteile, die die Analyse ergeben hat, zu erkl"aren, und eine vollst"andige quantitative Bestimmung der Gemengteil des E. von Juvenas zu geben. Von dem Schwefel, dessen Menge durch eine besondere Untersuchung bestimmt ist, wurde angenommen, dass er vom Magnetkiese, und vom Chromoxyd, dass es von Chromeisenerz herr"uhre, und es ist daher schon gleich so viel Eisen als zum ersteren, und Eisenoxydul als zum letzteren geh"ort, in der angegebenen Analyse aufgef"uhrt (der Magnetkies ist wie fr"uher als FeS, das Chromeisenerz als FeCr berechnet). Das Eisenoxyd wurde als von Magneteisenerz, die Titans"aure als von Titanit, und die Phosphors"aure als von Apatit herr"uhrend angenommen. Indem er nun bei seiner Berechnung von der Gesamtmischung ausgeht, bestimmt er zuerst nach der Tonerde und den Alkalien den Anorthit, nach dem Schwefel den Magnetkies, nach dem Eisenoxyd des Magneteisenerzes, nach der Phosphors"aure den Apatit, nach der Titans"aure den Titanit, und bekommt nun einen Rest, in welchem der Sauerstoff von R : Si wie 1 : 2,16 ist, also beinahe wie im Augit.

Die Annahme von Chromeisenerz ist wie die von Magnetkies gerechtfertigt, denn, wenn man auch das erstere noch nicht wie den letzteren beobachtet hat, so ist doch das Chromeisenerz ein so gew"ohnlicher Gemengteil der Meteorit, dass es nach dem gefundenen Chromoxyd in dem E. von Juvenas sehr wahrscheinlich ist. Dies ist aber nicht mit den "ubrigen Annahmen der Fall. Magneteisenerz ist weder in diesem noch "uberhaupt in einem Meteoriten mit Sicherheit beobachtet, daher auch wahrscheinlich das Eisenoxyd noch vom Magnetkiese herr"uhrt, und die Bestimmung des Schwefels, der bei der Zersetzung des E. von Juvenas als Schwefelwasserstoffgas entwichen ist, zu gering ausgefallen. Mit Titanit stimmt wie angef"uhrt das Verhalten der gelben tafelartigen Kristalle vor dem L"otrohr nicht v"ollig "uberein, auch ist die Titans"aure in dem unzersetzbaren Gemengteil gefunden, und nicht in dem zersetzbaren, wie doch der Fall sein m"usste, wenn Titanit in dem E. von Juvenas enthalten w"are, und Apatit, wiewohl sehr m"oglich, ist doch noch nie in den Meteoriten beobachtet. Ich f"uhre daher diese Berechnung der Gemengteil, als noch nicht hinreichend begr"undet, hier nicht auf.

2) Der Eukrit von Stannern ist dem von Juvenas sehr "ahnlich, doch nicht so deutlich kristallinisch, wie dieser. Er ist nach den St"ucken, die mir vorliegen, nicht so grobk"ornig, wie der E. von Juvenas; der Anorthit auch nicht so regelm"a"sig begrenzt und seine Spaltungsfl"achen eigentlich nie recht deutlich zu sehen.

Die d"unngeschliffene Platte ist ebenfalls bei weitem nicht so belehrend, wie bei dem E. von Juvenas. Anorthit und Augit sind nicht so scharf gesondert, der erstere ist nicht so regelm"a"sig begrenzt und von einer gro"sen Menge kleiner Risse durchsetzt, wodurch er in unregelm"a"sige St"ucke abgeteilt ist, und an Durchsichtigkeit verliert; das Augit ist dunkler und undurchsichtiger; sehr feine Teile von Nickeleisen sind hin und wieder auch hier zu sehen.

Ein Wechsel von groben und feinen k"ornigen Partien kommt ebenfalls vor. Auch hier gehen dieselben ineinander "uber, oder schneiden wenigstens nicht scharf aneinander ab, aber in den feinen k"ornigen Partien liegen wieder scharf begrenzt sehr feink"ornige graue Partien, die, mit der Lupe betrachtet, immer noch gemengt erscheinen, und wieder kleine dunkle Teile auch scharf begrenzt einschlie"sen, die wie Augit aussehen, nicht regelm"a"sig begrenzt und sehr undurchsichtig sind; zu Pulver zerrieben und unter dem Mikroskop betrachtet wird das Gemenge noch deutlicher.

Drusenr"aume enth"alt der E. von Stannern nicht, auch habe ich die gelben Bl"attchen des E. von Juvenas in ihm nicht beobachtet, dagegen findet sich Magnetkies wie in diesem. Sein spezifisches Gewicht gibt Rummler 3,01--3,17 an. Die "au"sere schwarze Rinde ist gl"anzend und adrig wie bei dem E. von Juvenas. Verhalten vor dem L"otrohr und mit S"auren dasselbe. Das Nickeleisen kann man ebenso wie bei dem E. von Juvenas auffinden, nur muss man, da es in feineren Teilen und sparsamer enthalten ist, ein gr"o"seres St"uck zerreiben.\footnote{\frakfamily{Man muss nat"urlich Sorge tragen, dass, wenn man mittelst Mei"sels und Hammers ein St"uck zum Zerreiben lostrennt, keine Eisenteile von dem Mei"sel an dem abgetrennten St"ucke sitzen bleiben, die dann zu Irrt"umern Veranlassung geben k"onnen.}}

Der E. von Stannern ist schon vor dem von Juvenas von Rammelsberg jedoch mit denselben Resultaten untersucht worden. Von in Chlorwasserstoffs"aure zersetzbaren Gemengteilen haben sich darin gefunden
\begin{center}
34,08 Teile,
\end{center}
\begin{center}
von unzersetzbaren 65,02 Teile
\end{center}
ein Verh"altnis, das fast ganz mit dem bei dem E. von Juvenas gefundenen "ubereinkommt.\footnote{\frakfamily{Diese "ubereinstimmung ist sehr bemerkenswert; sie l"asst vermuten, dass auch hier wie bei dem Shalkit die Gemengteil in einfachen Atomverh"altnissen miteinander verbunden sind. Vergleicht man in den Analysen des E. von Stannern und Juvenas von Rammelsberg die Sauerstoffmengen der Kiesels"aure in dem zersetzbaren und unzersetzbaren Gemengteil, also in dem darin enthaltenen Anorthit und Augit, so findet man, dass diese betragen in 100 Teilen\\
des E. von Stannern: 8,59 und 16,69\\
des E. von Juvenas: 8,00 und 17,09.\\
Sie verhalten sich also fast genau wie 1 : 2. Da nun der Anorthit 2, das Augit nur 1 Atom Kiesels"aure enth"alt, so w"urde daraus folgen, dass in dem Eukrit von Stannern und Juvenas 1 Atom Anorthit und 4 Atome Augit enthalten sind. Diese gro"se Menge des Augits in Vergleich zum Anorthit k"onnte bei Ansicht der St"ucke zu gro"s erscheinen; Joch muss man bedenken, dass das Augit ein viel h"oheres spezifisches Gewicht als der Anorthit hat. Es ist bei dem Augit des Eukrit nicht untersucht, kann aber bei seinen hohen Eisengehalten nicht viel unter 3,4 betragen, w"ahrend der Anorthit nur ein Gewicht von 2,76 hat; daher das Augit verh"altnism"a"sig weniger Raum einnimmt als der Anorthit. (S. 10. Eukrit im Nachtr.)}} Rammelsberg findet auch etwas Chromoxyd und Eisenoxyd, und nimmt danach Chromeisenerz und Magneteisenerz an, von denen nun das bei dem E. von Juvenas Gesagte gilt. Phosphors"aure und Titans"aure sind nicht darin gefunden, aber auch vielleicht nicht aufgesucht.

3) Das kleine St"uck von dem E. von Jonzac in dem mineralogischen Museum ist ein ziemlich grobk"orniges Gemenge von Augit und Anorthit. Sein spezifisches Gewicht betr"agt nach Rummler 3,07--3,08. Man hat von ihm nur eine "altere Analyse von Laugier, die auf die Trennung der Gemengteil keine R"ucksicht nimmt, aber im Ganzen ein "ahnliches Resultat wie die der E. von Juvenas und Stannern gegeben hat.

4) Der E. von Petersburg. Ein auf den ersten Anblick sehr fremdartig aussehender Eukrit. Er fiel etwa 3 Pfd. schwer zu Petersburg in Tennessee den 5. Aug. 1855 und ist haupts"achlich durch die Nachrichten, die Shepard\footnote{\frakfamily{Silliman American Journ. of Sc. and arts 1857 Sec. ser. v. 24, p. 134.}} dar"uber mitgeteilt hat, bekannt geworden. Das mineralogische Museum besitzt zwei nicht sehr gro"se St"ucke, die zu ihm von Shepard durch Dr. Bondi gelangt sind. Der gr"o"ste Teil des Steins besteht nach diesen aus einer graulichwei"sen, feink"ornigen, zerreiblichen Masse, die, mit der Lupe betrachtet, doch nur ein feines Gemenge vor kleinen braunen und schneewei"sen K"ornern ist, worin aber einzelne 1 bis 2 Linien gro"se, gr"unlichgelbe K"orner von Olivin, auch sehr kleine K"orner von Magnetkies liegen, und kleine Rostflecke anzeigen, dass auch Nickeleisen darin vorkommt, wie man denn auch aus dem Pulver mit dem Magnet eine f"ur einen Eukrit nicht unbetr"achtliche Menge von Nickeleisen ausziehen kann. In dieser so beschaffenen Masse liegen einzelne einen halben Zoll im Durchmesser haltende Partien eines gr"oberen Gemenges der braunen und wei"sen K"orner, die sich nun deutlich als ein Gemenge von Augit und Anorthit darstellen und mit andern, die bei den E. von Juvenas und Stannern vorkommen, die gr"o"ste "ahnlichkeit haben; daher denn auch die grauen Partien nur f"ur ein feink"orniges Gemenge von Augit und Anorthit zu halten sind. Auf der andern Seite finden sich auch in der grauen Masse 2 bis 3 Linien dicke eckige schwarze Partien, die an der umliegenden Masse scharf abschneiden, und einen ebenen matten Bruch haben und worin auch mit der Lupe ein Gemenge nicht zu erkennen ist, die dennoch aber nur ein inniges Gemenge sein m"ochten, da es wie das graue vor dem L"otrohr an den Kanten zu einem schwarzen Glase schmilzt, das vom Magnete schwach angezogen wird und sich in Phosphorsalz mit Hinterlassung der Kiesels"aure zu einem von Eisen schwach gr"un gef"arbten Glase aufl"ost.

Das spezifische Gewicht nach Prof. Smith in Louisville 3,28.

"au"serlich eine schwarze gl"anzende Rinde wie bei den E. von Juvenas und Stannern.

Auch die chemische Untersuchung, die Smith mit dem E. von Petersburg angestellt, hat dieselben Resultate geliefert, wie die der "ubrigen Eukrit sie ergab:
\begin{center}
\begin{tabular}{ l r }
    Kalk & 9,01\\
    Magnesia & 8,13\\
    Natron & 0,83\\
    Eisenoxydul & 20,41\\
    Tonerde & 11,05\\
    Kiesels"aure & 49,21\\
    Eisen & 0,5\\
    Mangan & 0,04\\
    Schwefel & 0,06\\
    Nickel & Spur\\
    Phosphor & Spur\\
     & 99,23\\
\end{tabular}
\end{center}
\paragraph{}
Shepard fand die Menge des eingemengten Nickeleisens bei einem Versuche 2,5 pC.

Der E. von Petersburg ist also ein mit den "ubrigen im Allgemeinen "ubereinstimmender Eukrit, der sich von diesen nur durch etwas eingemengten Olivin und eine noch etwas gr"o"sere Menge von Nickeleisen unterscheidet.

Shepard, der das ganze gefundene St"uck zu untersuchen Gelegenhatte, beschreibt den Stein als bestehend aus einer aschgrauen Masse, die er f"ur Anorthit h"alt, mit eingewachsenen, mehr oder weniger abgerundeten Kristallen von schneewei"sem Chladnit, gr"unem Augit, gelblichgr"unem Olivin und Chromeisenerz. Die graue Masse macht nach ihm etwa 3/5 des ganzen Steins aus, die Kristalle von Augit erreichen zuweilen eine Gr"o"se von 1/4 Zoll und von eben dieser Gr"o"se kommen auch Kristalle von Anorthit\footnote{\frakfamily{Wenn ich die Beschreibung (a. a. O. S. 136) richtig verstehe.}} vor. Au"ser diesen Einmengungen beobachtete Shepard noch ein hartes rotes, scheinbar in Dodekaedern Kristallisiertes Mineral, von dem er sagt, dass es die gr"o"ste "ahnlichkeit mit dem in dem Meteorit von Nobleborough (Maine) vorkommenden und von ihm f"ur Granat erkl"arten Minerale habe. Diese Beschreibung weicht wesentlich von der meinigen ab, ich habe zwar nur zwei kleine St"ucke zur Benutzung gehabt, doch w"are es w"unschenswert gewesen, dass die Anwesenheit von mehreren der von Shepard genannten Mineralien wie namentlich des Chladnit n"aher bewiesen w"are. Die von mir beschriebenen schwarzen Partien sind nicht erw"ahnt. Shepard gibt nun auch hier noch nach Smiths und seinen Analysen eine ungef"ahre Berechnung der Gewichtsmengen der Gemengteil, die ich auch hier nicht anf"uhre, da sie auf zu unsicher Annahmen beruht. Von Augit gibt er nur 1 pC. an.

\centerline{*\hspace{15mm}*\hspace{15mm}*\hspace{15mm}*\hspace{15mm}*}
\clearpage
\section{\frakfamily{Schlussbemerkungen.}}
\paragraph{}
Aus dem Bisherigen ergibt sich, dass die Meteoriten Gemenge verschiedener Mineralien sind, wie die tellurischen Gebirgsarten, und es scheint daher von Interesse, eine Vergleichung der kosmischen Mineralien und Gebirgsarten (Meteoritenarten) mit den tellurischen Mineralien und Gebirgsarten anzustellen.

Die in den Meteoriten vorkommenden Mineralien sind:

1. Meteoreisen d. i. gediegenes Eisen, welches etwas nickelhaltig ist; es ist nach drei untereinander rechtwinkligen Richtungen, parallel den Fl"achen des Hexaeders spaltbar, stahlgrau, metallisch gl"anzend und findet sich derb und eingesprengt, derb eine besondere Meteoritenart bildend, und im Pallasit, eingesprengt in mehreren Meteoritenarten, namentlich im Chondrit und Mesosiderit.

2. T"anit.

3. Schreibersit.

4. Rhabdit. Die drei Eisenverbindungen, die in dem Meteoreisen meistenteils regelm"a"sig eingewachsen vorkommen. Sie sind von gleicher Farbe und Glanz, wie das Nickeleisen und daher bei unversehrtem Zustande des letzteren nicht sichtbar, sind aber in verd"unnter Salpeter- und Chlorwasserstoffs"aure schwerer l"oslich, als das Meteoreisen und treten daher aus seiner Oberfl"ache hervor, wenn man dasselbe in solchen S"auren einige Zeit hat liegen lassen.

Der T"anit, ein nickelreicheres Eisen als das Meteoreisen, findet sich bei dem in oktaedrischen Bruchst"ucken vorkommenden Meteoreisen in d"unnen Bl"attchen zwischen den Oktaederfl"ache parallel gehenden Schalen, und durch sein Hervortreten auf den ge"atzten Schnittfl"achen werden vorzugsweise die Widmanst"attenschen Figuren hervorgebracht, wie bei dem Eisen von Toluca, Elbogen etc.

Der Schreibersit (Lamprit von v. Reichenbach), ein Phosphornickeleisen, findet sich in glatten K"ornern und unvollkommen ausgebildeten Kristallen in der Mitte der Schalen und diesen parallel bei einigen der vorigen Ab"anderungen des Meteoreisens, z. B. dem Eisen von Cosby, Arva, Lenarto etc. |

Der Rhabdit, ebenfalls ein Phosphornickeleisen, findet sich in kleinen, fast mikroskopischen, quadratischen Prismen, die nach drei Richtungen parallel den Kanten des Hexaeders in dem Meteoreisen regelm"a"sig eingewachsen sind, z. B. von Braunau, Seel"asgen, Misteca. Sie finden sich in dem Meteoreisen gew"ohnlich nur da, wo der Schreibersit fehlt, oder wenn sie mit diesem in einem und demselben Meteoreisen vorkommen, wie z. B. in dem von Arva, doch nur in den St"ucken, die keinen Schreibersit enthalten. Da nun der Rhabdit wie der Schreibersit aus Phosphornickeleisen besteht, so w"are es hiernach wohl m"oglich, dass beide nur verschiedene Zust"ande einer und derselben Verbindung sind.

5. Grafit in kleinen derben Partien in dem Meteoreisen eingemengt von verschiedener bis Walnussgro"se (M. von Toluca), zuweilen in Pseudomorphosen in der Form von kleinen Hexaedern (M. von Arva).

6. Troilit (FeS), durch die Analyse als Einfach-Schwefeleisen erkannt, findet sich in mehr oder weniger gro"sen, gew"ohnlich unregelm"a"sig begrenzten Partien in dem Meteoreisen eingewachsen wie der Grafit, doch h"aufiger als dieser (M. von Toluca, Seel"asgen, Lockport). Nur einmal, in dem Meteoreisen vom Cap der guten Hoffnung, habe ich ihn auf der Schnittfl"ache des Eisens in regelm"a"sig begrenzten Umrissen beobachtet, ohne seine Form weiter bestimmen zu k"onnen. In dem Eisen von Sarepta erscheint er mit d"unnschaligen Zusammensetzungsst"ucken in einer Richtung, wie "ofter der tellurische Magnetkies; ob er deshalb auch da, wo er mit unregelm"a"siger Begrenzung vorkommt, stets ein Individuum und kein Aggregat ist, kann nicht ausgemacht werden, da der Troilit nicht spaltbar ist und gew"ohnlich einen unebenen Bruch hat. Er ist teils ganz ohne Einmengungen von Nickeleisen wie in dem Eisen von Seel"asgen, teils enth"alt er diese in feinen Partien, auch enth"alt er zuweilen Schwefelnickel chemisch gebunden (M. von Sevier County). Ob indessen alles Schwefeleisen, welches in dem Meteoreisen vorkommt, dem Troilit und nicht zum Teil auch dem folgenden Magnetkies angeh"ort, ist noch auszumachen, sowie auch, wohin das Schwefeleisen geh"ort, welches in kleinen Partien sich in dem Pallasit und Mesosiderit findet.

7. Magnetkies findet sich in kleinen Kristallen in dem Eukrit von Juvenas und nach Shepard auch in dem Chondrit von Richmond. Er ist hier durch seine Form als Magnetkies kenntlich und es ist hiernach wahrscheinlich, dass auch das Schwefeleisen in den "ubrigen Eukriten und Chondriten, in denen es immer nur in kleinen Partien und eingesprengt vorkommt, Magnetkies sei; dies wird auch noch dadurch wahrscheinlich, dass viele der Chondrit beim Aufl"osen in Chlorwasserstoffs"aure etwas Schwefel absetzen. Indessen k"onnte auch in diesem neben dem Magnetkies, oder auch statt dieses Troilit vorkommen. Bei dieser Unbestimmtheit ist in dem Obigen vorl"aufig angenommen, dass alles Schwefeleisen in dem Meteoreisen Pallasit und Mesosiderit: Troilit und in den "ubrigen Meteoriten: Magnetkies sei.

8. Chromeisenerz; sehr verbreitet in den Meteoriten; wenngleich immer nur in sehr geringer Menge. Es findet sich zuweilen kristallisiert in Oktaedern, gew"ohnlicher in K"ornern und eingesprengt; am deutlichsten kristallisiert und in den gr"o"sten K"ornern, die zuweilen die Gr"o"se einer kleinen Erbse erreichen, in dem Shalkit, seltener und in weniger gro"sen K"ornern in dem Meteoreisen und in dem Pallasit, namentlich von Brahin, feiner eingesprengt in dem Chassignit, Chondrit, Howardit. In dem Meteoreisen ist es "ofter in Troilit ganz eingewachsen, wie bei dem M. von Carthago und Schwetz.

9. Quarz in kleinen fast mikroskopischen doch mit dem Reflexionsgoniometer messbaren Kristallen in dem Meteoreisen von Toluca als gro"se Seltenheit. Bei der Analyse des Meteoreisens ist Kiesels"aure "ofter gefunden.

10. Olivin (Mg,Fe)$^{2}$ Si; ein h"aufiger und wichtiger Gemengteil der Meteorit. Er kommt kristallisiert, in K"ornern und auch derb vor. In den ausgezeichnetsten Kristallen in dem Pallasit, Mesosiderit und Shalkit, in kleineren K"ornern im Chondrit und Howardit, derb und fast die ganze Masse des Steins ausmachend im Chassignit. In dem Pallasit sind die Kristalle porphyrartig eingewachsen, oft ganz rund, wenn sie sich nicht gegenseitig begrenzt und in der Ausbildung gest"ort haben, meistens aber noch einzelne und selbst viele sehr glatte und gl"anzende Fl"achen zeigend, die sich teils nicht ber"uhren, teils in Kanten schneiden.\footnote{\frakfamily{Die Querfl"ache \emph{M} fehlt stets, wie sie auch bei den eingewachsenen tellurischen Olivinen nicht vorkommt.}} Die Kristalle sind oft vollkommen durchsichtig, gelblichgr"un und stark gl"anzend von Glasglanz; sie enthalten h"aufig r"ohrenartige Einschl"usse, die untereinander und der Hauptaxe parallel sind. In dem Mesosiderit namentlich von Hainholz erhalten die K"orner zuweilen die Gr"o"se einer Walnuss, sind aber nicht durchsichtig, sondern nur an den Kanten durchscheinend, und gelblich- bis r"otlichbraun; im Shalkit haben sie oft eine ebenso bedeutende Gr"o"se, sind aber sehr kl"uftig, und durch dunkle schw"arzlichgr"une Farbe ausgezeichnet, die aber an einzelnen Teilen eines und desselben Kornes lichter wird und in die gelblichgr"une Farbe "ubergeht. Von dieser Farbe sind sie auch meistens in den "ubrigen Meteoriten. Der Olivin der meisten Meteorit ist vor dem L"otrohr unschmelzbar, er ver"andert sich gar nicht, oder wird nur dunkler von Farbe, der Olivin im Chassignit schmilzt aber vor dem L"otrohr, wenn auch nur schwer, in Folge seines gr"o"seren Eisengehaltes zu einem schwarzen Glase wie der Hyalosiderit, mit dem er auch in seine Eisengehalte "ubereinstimmt.

11. Shepardit Mg$^{2}$Si$^{3}$; ein Hauptgemengteil des Chladnit, in welchem er in eingewachsenen unvollkommen ausgebildeten Kristallen bis zu der Gr"o"se eines halben Zolles und dar"uber vorkommt; schneewei"s sehr br"ocklig, fast unaufl"oslich in Chlorwasserstoffs"aure. Findet sich au"serdem im Shalkit, doch in viel kleineren K"ornern, ist aber sonst mit Sicherheit nicht beobachtet.

12. Augit findet sich im Eukrit und Mesosiderit; in dem ersteren von schw"arzlichbrauner Farbe, kleink"ornig, und in den Drusenr"aumen bei dem E. von Juvenas in sehr deutlichen Kristallen, wie er gew"ohnlich in den Doleriten und neueren vulkanischen Gesteinen vorkommt, Kristallisiert. Von Basen enth"alt er vorzugsweise Eisenoxydul, n"achstdem Magnesia und Kalkerde; seine Formel: (Fe, Mg, Ca) Si. In dem Mesosiderit ist er grobk"orniger, namentlich in dem von der Sierra de Chaco und schw"arzlichgr"un. Wenn man ihn auch in dem Chondrit aufgef"uhrt hat, so ist dies eine blo"se Annahme, die man bei der Berechnung der Analyse gemacht hat, und die sich auf eine wirkliche Beobachtung desselben nicht gr"undet.

13. Anorthit CSi + AlSi ist haupts"achlich im Eukrit enthalten, der fast nur ein kleink"orniges Gemenge von ihm mit Augit ist. Er ist schneewei"s, in den k"ornigen Zusammensetzungsst"ucken deutlich spaltbar und oft zwillingsartig verwachsen. In den Drusen des Eukrit von Juvenas ist er auskristallisiert, doch viel undeutlicher und kleiner als der mit ihm zusammen vorkommende Augit. Mit Sicherheit ist er in anderen Meteoriten nicht nachgewiesen, doch ist er wahrscheinlich auch im Howardit und Chladnit enthalten.

Diess sind die in den Meteoriten des Berliner Museums beobachteten und bestimmten Mineralien; zu den beobachteten aber noch nicht bestimmten Mineralien geh"oren:

1. Die Kugeln in dem Chondrit; sie sind im Innern undeutlich fasrig und von gelblichgrauer bis graulichschwarzer Farbe; nicht selten findet man Kugeln von beiden Farben in einem und demselben Meteoriten, und zuweilen beide Farben bei einer und derselben Kugel, und dann teils solche mit einem lichtern Kern bei dunkler H"ulle, teils solche mit dunklem Kern bei lichter H"ulle. Sie sind in ihrer Zusammensetzung noch nicht vollst"andig untersucht, scheinen aber haupts"achlich ein Magnesia-Silicat zu sein, unterscheiden sich aber von dem Olivin dadurch, dass sie, wenn auch nicht vollkommen unl"oslich, doch sehr schwer l"oslich sind.

2. Die schwarze Substanz, die sich in dem Chondrit nach den Beobachtungen unter dem Mikroskop findet.

3. Die gelben tafelartigen Kristalle, die in dem Eukrit von Juvenas vorkommen, mit dem Titanit "ahnlichkeit haben, von demselben aber wahrscheinlich doch verschieden sind.

4. Die wei"sen K"orner im Howardit, sowie auch die, welche neben dem Shepardit im Chladnit vorkommen.

5. Das nach Shepard Schwefel und Chrom haltige Mineral im Chladnit u. s. w.

Au"serdem sind bei den Analysen der Meteorsteine Bestandteile gefunden, die auf noch nicht beobachtete Mineralien schlie"sen lassen. Dahin geh"ort die von Berzelius gefundene geringe Menge von Zinns"aure, die "ofter mit dem Chromeisenerz bei der Aufl"osung des Meteoreisens zur"uckbleibt und wahrscheinlich von einer geringen Menge Zinnstein herr"uhrt, wenn auch die Zinns"aure, die in der Aufl"osung des Meteoreisens enthalten ist, von dem Metall herr"uhren mag, das mit dem Nickeleisen verbunden gewesen ist. Ferner die Tonerde und die Alkalien, die in dem Chondrit, Howardit und Chladnit, so wie auch die geringen Mengen von Phosphors"aure und Titans"aure, die in dem Eukrit gefunden sind.

Zu den Mineralien, die in den Meteoriten h"aufig aufgef"uhrt werden, wiewohl ich sie bei denen des Berliner Museums gar nicht, oder nicht mit Sicherheit beobachtet habe, geh"oren Magneteisenerz, Eisenkies, Labrador, Leucit, Schwefel u. s. w. Auffallend ist besonders die Abwesenheit des Magneteisenerzes in den Meteoriten,\footnote{\frakfamily{Vergl. dar"uber was bei dem Howardit (S. 109) und Eukrit (S. 133) gesagt ist. Beim Meteoreisen findet es sich wohl, ist aber hier nur sekund"arer Bildung (S. 42).}} es scheint in diesen "uberall durch das Chromeisenerz vertreten zu sein. Ebenso wie das Eisenoxydul fehlt auch das Eisenoxyd, und dieses nicht blo"s als selbstst"andiges Mineral, es scheint auch selbst als Bestandteil anderer Mineralien zu fehlen. Unter den metallischen Substanzen wird auch noch das Blei genannt, welches sich in dem Eisen von Tarapaca findet, da aber dieses Eisen noch zu das problematischen Meteoriten geh"oren m"ochte,\footnote{\frakfamily{Diese Meinung "au"serte zuerst gegen mich Baron v. Reichenbach, als er in dem Berliner Museum das fr"uher auch f"ur meteorisch gehaltene Eisen von Gr. Kamsdorf sah, mit welchem das Eisen von Tarapaca die gr"o"ste "ahnlichkeit haben sollte. Als ich sp"ater dieses Eisen selbst sah, fiel mir ebenfalls seine gro"se "ahnlichkeit mit dem Kamsdorfer Eisen auf, und meine Weitern Untersuchungen best"arkten noch meine Zweifel an dem meteorischen Ursprung desselben, denn ge"atzt gibt es keine Widmanst"attensche Figuren, und verh"alt sich "uberhaupt nicht wie achtes Meteoreisen, und ebenso verschieden verh"alt sich das damit vorkommende Silikat, indem es vor dem L"otrohr leicht schmelzbar ist, was auch bei den "achten Meteoriten nicht vorkommt.}} so habe ich auch das Blei als Gemengteil der Meteorit noch nicht aufgef"uhrt.

Die bekannten Mineralien der Meteoriten kommen daher nur zum Teil mit den tellurischen Mineralien "uberein. Meteoreisen, ja selbst nur reines Eisen, kommt als urspr"ungliche Bildung unter den tellurischen Mineralien nicht vor, denn das wenige tellurische Eisen was angef"uhrt wird, scheint immer nur eine sekund"are Bildung und durch einen lokalen Reduktionsprozess entstanden zu sein, und ist dann nie nickelhaltig. Ebenso ist T"anit, Schreibersit, Rhabdit, Troilit und Shepardit noch nicht beobachtet, dagegen Grafit, Magnetkies, Chromeisenerz, Olivin, Augit und Anorthit unter den tellurischen Mineralien h"aufig vorkommen, und von Quarz es nur auffallend ist, dass er, fast das h"aufigste Mineral auf der Erde, unter den Meteoriten so "au"serst selten ist, bei den Steinmeteoriten gar nicht, und nur im Meteoreisen beobachtet ist. Wenn demnach elementare Stoffe unter den Meteoriten durchaus nicht gefunden sind, die nicht auch auf der Erde bekannt sind, so ist dies bei den Verbindungen, die sie untereinander bilden, nicht so vollst"andig der Fall.

Die in den Meteoriten vorkommenden Mineralien kommen nun entweder f"ur sich allein vor oder in einem Gemenge miteinander, worin sie mit den tellurischen Gebirgsarten ganz "ubereinstimmen. Aber es ist merkw"urdig, dass sie fast nur auf zweierlei Weise vorkommen, und dass die davon abweichende Weise gewisserma"sen nur eine Ausnahme bildet. Man teilt den Meteorit ein in Eisen- und Steinmeteorit; der bei weitem gr"o"ste Teil der ersteren jedoch besteht nur aus Meteoreisen, und ebenso der gr"o"ste Teil der Steinmeteoriten aus Chondrit. Das Meteoreisen ist zum gr"o"sten Teil ein einfaches Mineral, worin der T"anit, Schreibersit und Rhabdit meistenteils regelm"a"sig eingewachsen sind, und nur in sehr geringer Menge vorkommen. Der Chondrit ist ein so feink"orniges Gemenge, dass es bis jetzt noch nicht gelungen ist, die Gemengteil vollst"andig zu erkennen. In der mehr oder weniger dunklen graulichwei"sen feink"ornigen Masse erkennt man au"ser den nach unbekannten Kugeln und dem eingesprengten Meteoreisen und Magnetkies nur K"orner von Olivin und Chromeisenerz, das "ubrige nicht. Die bei weitem am h"aufigsten vorkommende Art der Steinmeteoriten ist also in ihrer Beschaffenheit noch unvollkommen gekannt.

Zu den Eisenmeteoriten geh"oren noch 2 Arten, der Pallasit und der Mesosiderit. Ersterer ist ein Gemenge von Meteoreisen mit Olivin, letzterer von Meteoreisen mit Olivin und Augit. Der erstere hat eine porphyrartige Struktur, das Meteoreisen bildet die Grundmasse, worin ein Olivinkristalle eingewachsen sind; bei dem letzteren ist die Struktur schon k"ornig zu nennen, das Eisen tritt an Masse gegen die andern Gemengteil noch mehr zur"uck als bei dem Pallasit.

Die "ubrigen Steinmeteorit sind auch mehr oder weniger gemengt.

Der so selten vorkommende Chassignit ist fast nur ein k"orniger eisenreicher Olivin, worin etwas Chromeisenerz eingemengt ist.

Der Shalkit ein k"orniges Gemenge von vorwaltendem Olivin mit Shepardit und etwas Chromeisenerz.

Der Chladnit besteht vorzugsweise aus Shepardit, die andern Gemengteil sind nicht mit Sicherheit bekannt.

Der Howardit ein feink"orniges Gemenge von Olivin wahrscheinlich mit Anorthit.

Der Eukrit endlich ein solches Gemenge aus Augit und Anorthit.

Fast alle diesen Meteorit sind durch den Olivin, den sie enthalten, ausgezeichnet; der Chassignit besteht fast nur daraus, der Shalkit enth"alt ihn in vorwaltender Menge, in dem Chondrit und Howardit ist er ein wesentlicher Gemengteil, und der Eukrit enth"alt ihn zwar als solchen nicht, aber in dem E. von Petersburg ist er doch als unwesentlicher Gemengteil vorgekommen. Nur der Chladnit enth"alt ihn, so viel man wei"s, nicht, doch findet sich statt seiner das Trisilicat der Magnesia, der Shepardit. Neben dem Olivin ist das Chromeisenerz ein fast in allen Meteoriten, wenn auch nur in geringer Menge vorkommender Gemengteil.

Vergleicht man die Meteoriten, die kosmischen Gebirgsarten, mit den tellurischen, so ergibt sich, dass sie g"anzlich von diesen verschieden sind bis auf den Eukrit, der doch unter den tellurischen Gebirgsarten auch nur selten vorkommt und erst in der neuern Zeit aufgefunden ist. Haughton in Dublin beschrieb zuerst ein solches Gemenge von Carlingfors in Irland, wo es in G"angen im Steinkohlengebirge vorkommt,\footnote{\frakfamily{Vergl. Roth: Gesteinsanalysen, S. 52.}} gab aber das den Anorthit begleitende Mineral f"ur Hornblende aus, was nach den St"ucken, die ich Hrn. Skott verdanke, Augit ist; doch sind ja auch Gesteine, die Gemenge von Anorthit und Hornblende sind, in der neuern Zeit bekannt geworden, wie z. B. das bisher f"ur Diorit gehaltene Gestein vom Kontschekowskoj Kamen im Ural nach Potyka ein solches ist,\footnote{\frakfamily{A. a. O. S. 52.}} und solche m"ogen auch in Irland vorkommen. Die Anorthitgesteine und namentlich der Eukrit sind gewiss aber noch viel h"aufiger, als man glaubt. Durch das Gelatinieren des Anorthits mit Chlorwasserstoffs"aure war ein leichtes Mittel gegeben, denselben vom Labrador zu unterscheiden; ich fand so, dass viele f"ur Hypersthenfels gehaltene Gesteine nichts anderes als Eukrit sind, wie z. B. der sogenannte Hypersthenfels von der Insel Skye, der hier so verbreitet ist, und von dem das Berliner Museum viele und sch"one St"ucke besitzt, die die Hrn. v. Dechen und v. Oeynhausen dort gesammelt und dem Berliner Museum "ubergeben haben. Die tellurischen Eukrit unterscheiden sich zwar dadurch von den meteorischen, dass sie grobk"orniger sind und das Augit in ihnen nicht braun, sondern gr"un ist, doch sind dies unwesentliche Unterschiede. Einen wichtigen Unterschied w"urde das Vorkommen des gediegenen Eisens in den meteorischen Eukriten machen, doch ist dieses ja in den meisten, wie in dem von Stannern und Juvenas, nur in solcher geringen Menge vorhanden, dass es bisher in ihnen ganz "ubersehen ist, und also einen wesentlichen Gemengteil nicht abgeben kann.

Ein Unterschied zwischen dem tellurischen und meteorischen Eukrit findet aber immer statt; der erstere verh"alt sich zu diesem wie die "alteren vulkanischen Gebirgsarten zu den neuern, wie die Gebirgsarten der Granit- und Gr"unsteingruppe zu denen der Trachyt- und Basaltgruppe. Wie jene dieselben Gemengteil haben wie diese, und beide sich oft nur in unwesentlichen Eigenschaften voneinander unterscheiden, so ist dies auch hier der Fall. Der tellurische Eukrit geh"ort zu den Gebirgsarten der Gr"unsteingruppe, der meteorische zu denen der Basaltgruppe.

Indessen w"are es wohl m"oglich, dass auch unter den Gebirgsarten der Basaltgruppe, namentlich unter dem, was man Dolerit genannt hat, ein Eukrit vork"ame, der dann in seinen Eigenschaften noch mehr mit dem meteorischen "ubereinstimmen w"urde. Unter den in Island vorkommenden Laven ist schon mehrfach Anorthit als Gemengteil beobachtet; die von Genth analysierte Tjorsa-Lava kommt in der Zusammensetzung dem Eukrit von Stannern sehr nahe; es ist wohl m"oglich, dass dies ein Eukrit ist, wie ihn Roth auch schon vorl"aufig dazu gestellt hat.\footnote{\frakfamily{A. a. O. S. 52.}}

Mit diesen Gebirgsarten der Basaltgruppe sind "uberhaupt auch nur die Meteoriten zu vergleichen. Sie kommen mit diesen "uberein durch die meistenteils k"ornige Struktur, durch den g"anzlichen Mangel freier und die verh"altnism"a"sig geringe Menge der gebundenen Kiesels"aure\footnote{\frakfamily{Ich sehe hierbei von den kleinen Quarzkristallen ab, die ich in dem Meteoreisen von Toluca und nur in diesem gefunden habe.}} und durch den Reichtum an Olivin. Dies sind aber ziemlich alle Vergleichungspunkte, die den Meteorit darbieten. Letztere unterscheiden sich wesentlich durch das metallische, stets nickelhaltige Eisen und die "ubrigen, unter den tellurischen Mineralien nicht beobachteten Verbindungen, die sie enthalten, durch die geringe Menge von Silicaten mit Tonerde und Alkali und ferner durch die g"anzliche Abwesenheit des Magneteisenerzes, das in den neuern vulkanischen Gebirgsarten der Erde so verbreitet ist. Bei den Meteoriten ist, wie schon bemerkt, das Magneteisenerz durch das Chromeisenerz vertreten. Dieses kommt in den tellurischen Gebirgsarten auch vor, hat hier aber ein ganz anderes geognostisches Vorkommen, indem es hier nicht an den Olivin gebunden ist; indessen kommt es doch auch hier mit einem Magnesiasilicat vor, wenn auch einem wasserhaltigen, dem Serpentin, der nun freilich h"aufig eine Metamorphose von Olivin ist, aber doch nicht gew"ohnlich in den F"allen, wo der Olivin in den Gebirgsarten der Basaltgruppe vorkommt.\footnote{\frakfamily{Ein tellurisches Mineralgemenge ist noch mit den Meteoriten zu vergleichen; das sind de Olivinkugeln, die teils eingeschlossen in dem Basalt, teils in dem Basalttuff vorkommen. Sie sind ein k"orniges Gemenge von Olivin und Augit; enteilten sie noch Nickeleisen, so k"amen sie mit dem Mesosiderit "uberein. Wenn die Olivinkugeln durch diese Beziehung ein gewisses Interesse haben, so erregen sie noch ein anderes durch ihre r"atselhafte Bildung. Wenn sie in dem Basalt liegen, so sind sie darin keine Bildungen, die beim Erstarren des Basaltes entstanden sind, wie die einzelnen Olivinkristalle, die neben ihnen in dem Basalt vorkommen, denn sie haben nicht die Struktur solcher Bildungen, sie k"onnen daher nur Einschl"usse sein; wenn sie aber solche sind, so muss man fragen, woher sie kommen, und diese Frage ist schwer zu beantworten.}}

Aber nicht nur in der Art der Gemengteil zeigen die Meteorit Unterschiede von den tellurischen Gebirgsarten, sie finden sich auch in der Struktur. Die Porphyrstruktur, die bei dem Pallasit vorkommt, ist doch darin von der Porphyrstruktur des roten Porphyrs und anderer tellurischen Gebirgsarten verschieden, dass hier die Grundmasse nie ein einfaches Mineral, sondern ein k"orniges bis dichtes Gemenge verschiedener Mineralien oder eine amorphe Masse ist. Ebenso sind die tellurischen Gebirgsarten von kugliger Struktur dadurch verschieden, dass ihre eingeschlossenen Kugeln, wenn sie fasrig sind, stets radial fasrig sind, wie dies nie bei den Kugeln des Chondrits der Fall ist, wo man ihre Struktur erkennen kann.

Auch die k"ornige Struktur der Meteoriten ist dadurch ausgezeichnet, dass bei ihnen ein so schneller Wechsel in Korou und Farbe des Gesteins vorkommt. Bei dem Eukrit von Stannern finden sich ganz feink"ornige, ja dicht zu nennende Ab"anderungen neben kleink"ornigen und schneiden scharf an diesen ab; und etwas "ahnliches kommt auch bei den andern Eukriten sowie auch bei den Howarditen vor. In dem Chondrit von Gr"uneberg ist die eine H"alfte des St"uckes grau, die andere wei"s, und auch hier schneiden beide Ab"anderungen scharf ab. In dem Chondrit von Chantonnay und Siena treffen auf diese Weise sich graue und schwarze Teile, und bei dem Chondrit von Ensisheim hat es fast das Ansehen, als durchsetze die eine die andere in G"angen. Bei den tellurischen Gebirgsarten grenzen auch wohl feine und grobk"ornige Ab"anderungen aneinander, doch findet dieses im Ganzen nur selten statt, und der "ubergang ist nie so scharf. Diess Zusammenkommen von so verschieden aussehenden Variet"aten dicht nebeneinander gibt den Meteoriten ein breccienartiges Ansehen, und als solche Breccien hat man sie ja auch oft schon beschrieben. Wenn aber auch bei ihnen so verschiedenartig aussehende Ab"anderungen scharf aneinander abschneiden, es sind sie doch in ihren wesentlichen Eigenschaften nicht voneinander verschieden die Meteoriten daher nie Breccien im Sinne unserer Gebirgsarten, und dieser Ausdruck, wenn er f"ur sie gebraucht wird, ist immer nur ein uneigentlicher.

Drusig werden die Meteoriten selten, aber es kommt doch in ausgezeichneter Weise vor bei dem Eukrit von Juvenas.\footnote{\frakfamily{Besonders sch"on sollen die Drusen zu sehen sein bei dem gro"sen, 42 Kilogramm schweren Bruchst"uck dieses Meteoriten, welches sich in dem Muse d'histoire naturelle in Paris befindet und das gr"o"ste ist, welches existiert.}} Shepard beschreibt noch Drusen bei dem Chondrit von Richmond; andere drusige Meteoriten als diese sind aber nicht bekannt. ---

Ungeachtet aller der genannten Verschiedenheiten haben die Steinmeteorit doch eine nicht zu leugnende "ahnlichkeit mit den neuern vulkanischen Gebirgsarten, und bei dem hohen Interesse, das die Meteoriten als au"sertellurische K"orper gew"ahren, ist diese "ubereinstimmung, wie sie auch immer sei, von gro"ser Wichtigkeit.
\clearpage
\section{\frakfamily{Nachtrag.}}
\paragraph{}
W"ahrend des Druckes der letzten Bogen erhielt ich noch einige neue Meteoriten, "uber welche ich hier noch Einiges hinzuf"uge.

1. Mesosiderit von Atacama in Chile. Er wurde auf einem Bergpasse 50 engl. Meilen von Copiapo gefunden, von Hrn. Brower erworben, der ihn nach New York brachte und dem Union College in Schenectady "ubergab, wo er vom Professor Joy analysiert und beschrieben wurde.\footnote{\frakfamily{Silliman, Amer. Journ. of Sc. and arts 1864, v. 37, p. 243.}} Durch Hrn. Prof. Chandler in Schenectady erhielt ich eine kleine Probe und ein Modell des urspr"unglichen Meteoriten. Er hat hiernach eine ellipsoidische Gestalt mit ziemlich ebener Oberfl"ache und eine L"ange, Breite und H"ohe von etwa 4, 3 1/2 und 3 Zoll. Sein Gewicht betrug nach Joy 1784 Grammen, sein spez. Gew. 4,35. Nach der "ubersandten Probe hat er auf den ersten Anblick so gro"se "ahnlichkeit mit dem Mesosiderit von der Sierra de Chaco in Atacama (vergl. oben S. 81), dass man geneigt sein k"onnte, ihn mit diesem f"ur identisch zu halten. Nach der Analyse von Joy enth"alt er 57,657 metallische Teile und 42,419 Silicate, die wieder bestehen in 100 Teilen aus:
\begin{center}
\begin{tabular}{ l r }
    Eisen & 83,76\\
    Nickel & 9,18\\
    Kobalt & 1,45\\
    Mangan & 0,65\\
    Kupfer & 0,07\\
    Phosphor & 0,20\\
    Schwefel & 4,67\footnote{\frakfamily{Mit 8,17 Eisen zu 12,84 FeS verbunden.}}\\
     & 99,98\\
\end{tabular}
\end{center}
\begin{center}
\begin{tabular}{ l r }
    Eisenoxydul & 24,47\\
    Manganoxydul & 2,29\\
    Magnesia & 10,05\\
    Kalk & 3,63\\
    Nickel- u. Kobaltoxyd & 0,17\\
    Tonerde & 8,86\\
    Chromoxyd & 1,12\\
    Kiesels"aure & 48,61\\
    Zinns"aure & 0,44\\
     & 99,64\\
\end{tabular}
\end{center}
\paragraph{}
Joy berechnet hiernach in den Silicaten:
\begin{center}
\begin{tabular}{ l r }
    Chromeisenerz & 1,64\\
    Olivin & 27,43\\
    Labrador & 70,13\\
     & 99,20\\
\end{tabular}
\end{center}
\paragraph{}
Hr. Joy hat aber hierbei das Augit "ubersehen, der in dem kleinen "ubersandten St"ucke ganz deutlich, und von derselben Beschaffenheit wie bei dem Mesosiderit von der Sierra de Chaco enthalten ist; ich konnte einen kleinen Kristall herausnehmen, und mich durch Messung der Winkel der Spaltungsfl"achen "uberzeugen, dass er Augit sei. Olivin ist au"serdem ganz deutlich sichtbar, Chromeisenerz nicht, doch nach dem bei der Analyse gefundenen Chromoxyd sehr wahrscheinlich. Hr. Joy berechnet nun noch aus der Analyse in dem Meteorit eine bedeutende Menge Labrador; allerdings sieht man in der "ubersandten Probe kleine Kristalle, die eine deutlich gestreifte Spaltungsfl"ache zeigen und vielleicht Labrador oder ein anderer ein- und eingliedriger Feldspat wie Anorthit sein k"onnten, und der gefundene Tonerdegehalt, der f"ur Augit allein zu gro"s erscheint, spricht auch f"ur die Annahme eines solchen Feldspats, wenn er auch nicht in solcher Menge wie ihn Joy annimmt in dem Meteoriten enthalten sein k"onnte. In dem viel gr"o"seren St"ucke des Mesosiderits von der Sierra de Chaco in dem Berliner Museum habe ich aber diese Kristalle nicht beobachten k"onnen; und es w"are demnach wohl m"oglich, dass dadurch eine Verschiedenheit zwischen den Meteoriten von der Sierra de Chaco und Copiapo angedeutet w"are; ich habe daher einstweilen noch in der folgenden "ubersicht den Meteoriten von Copiapo als Besonderen Meteoriten hinter dem von der Sierra de Chaco aufgef"uhrt.

2. Meteoreisen von dem Indianischen Territorium Dakota in Nord-Amerika. Dr. Jackson in Boston erhielt 1863 von dieser Eisenmasse, die sich in einem "uber 90 engl. Meilen von jeder Stra"se oder Wohnung entferntem Grunde gefunden hatte, und deren Gewicht auf 100 Pfund gesch"atzt wurde, ein 10 Pfund 10 Unzen schweres St"uck. Er fand sein spez. Gew. 7,952, die H"arte wie die des weichsten schmiedbaren Eisens, und seine chemische Zusammensetzung bestehend aus:
\begin{center}
\begin{tabular}{ l r }
    Eisen & 91,735\\
    Nickel & 6,532\\
    Zinn & 0,063\\
    Phosphor & 0,010\\
     & 98,340\\
\end{tabular}
\end{center}
Ein zweiter Versuch bestimmte den Nickelgehalt zu 7,080.\footnote{\frakfamily{Sillimann, Amer. Journ. of sc. and arts, 1863, v. 36. p. 259.}}
\paragraph{}
Ich erhielt von Hrn. Prof. Shepard eine von dem St"ucke des Dr. Jackson abgeschnittene, 4 Linien dicke und 3,319 Loth schwere Platte, die an den R"andern zum Teil noch mit der nat"urlichen Oberfl"ache begrenzt und hier mit einer nur ganz d"unnen Rinde von Eisenoxydhydrat bedeckt ist. Das Eisen zeigt ge"atzt die "atzunglinien, die "uber die ganze Fl"ache fortlaufen, wie bei dem Eisen von Braunau, und ebenso die kleinen eingewachsenen Rhabditkristalle, geh"ort also zu der bis jetzt nur selten vorgekommenen Abteilung des Meteoreisens, das nur aus einem Individuum ohne schalige Zusammensetzung besteht. Beim Zerrei"sen des Eisens w"urden also die hexaedrischen Spaltungsfl"achen sichtbar werden. --- Au"ser dem Rhabdit findet sich in der Platte Schreibersit in mehreren ungew"ohnlich gro"sen Partien, die an einer Stelle zusammengeh"auft sind, eingewachsen.

3. Meteoreisen von Tucson (s. oben S. 73). Ich erhielt von diesem Eisen von Hrn. Shepard eine d"unne 1,635 Loth schwere Platte. Die polierte Fl"ache ist voller kleiner runder H"ohlungen. Ge"atzt zeigt sie grobk"ornige Zusammensetzungsst"ucke, von denen einige bei einer gewissen Beleuchtung eine lichte graue, andere eine dunklere graue Farbe haben; bei anderer Beleuchtung verhalten sie sich umgekehrt. Die Zusammensetzungsst"ucke haben eine sehr d"unne Einfassung von T"anit, und viele der kleinen H"ohlungen haben nun gl"anzende W"ande erhalten, die auf der "ubrigen matten Fl"ache hervorleuchten. Die Zusammensetzungsfl"achen zeigen feine linienartige gerade Furchen, die eine von den "atzunglinien etwas verschiedene Beschaffenheit haben. Da nun bei diesem Eisen auch oktaedrische Spaltbarkeit\footnote{\frakfamily{Buchners Meteoriten, S. 183.}} (soll wohl hei"sen oktaedrische Zusammensetzung) angegeben wird, so habe ich dieses Eisen in der folgenden "ubersicht auf das Eisen von Seel"asgen und Tucuman folgen lassen.

4. Meteoreisen von Wayne County, Ohio Ver. St. 1859. Eine kleine ge"atzte Platte mit deutlichen Widmanst. Fig. von Hrn. Shepard erhalten. Das Eisen folgt in der "ubersicht nach dem von Nebraska.

5. Das Meteoreisen von Cranbourne. Das kleine St"uck der Sammlung (s. oben S. 73) l"asst ge"atzt noch die Widmanst"atten. Fig. erkennen, die auch sonst angegeben werden; ich habe es deshalb auf das Eisen von Denton folgen lassen.

6. Pallasit von Rogue River Mountains, Oregon, N. Amerika 1859, wovon ich kleine St"uckchen von Hrn. Shepard und

7. Mesosiderit (?) von Niakornak in Gr"onland, von Hrn. Greg erhalten habe, wurden der erstere hinter dem P. von Brahin, der letztere hinter dem M. von Hainholz eingeordnet.

8. Den mikroskopischen Zeichnungen auf den Taf. 3 und 4 habe ich noch eine neue in Fig. 12 Taf. 3 hinzugef"ugt, die eine Stelle auf einer d"unnen Platte des Chondrits von Chantonnay in derselben Vergr"o"serung wie die Fig. 3 und 4 darstellt. Sie zeigt eine Gruppe durchsichtiger ungef"arbter (Olivin) Kristalle, die von der schwarzen Substanz fast ganz umgeben, und daher "uberall, wo sie an diese angrenzen, regelm"a"sig begrenzt sind. Au"serdem sieht man in diesen Kristallen kleine sehr feine, schwarze haarf"ormige Kristalle, in verschiedenen H"ohen liegen, die wahrscheinlich "ahnliche r"ohrenf"ormigen Einschl"usse, wie die in den Olivinkristallen des Pallas-Eisens sind.

9. Eukrit. Berechnet man aus den durch die Analyse gefundenen Gewichtsmengen des Anorthits und Augits und den spezifischen Gewichten (2,76 und 3,35) die Volumina der Gemengteil, so ergibt sich das Volumenverh"altnis von Anorthit und Augit in dem E. von Stannern wie 12,35 : 19,40 und in dem E. von Juvenas: \emph{a}) wenn man alles durch die S"aure Zersetzte f"ur Anorthit und das Unzersetzte f"ur Augit nimmt, wie 13,32 : 18,87 und \emph{b}) wenn man bei der Berechnung die Phosphors"aure, das Schwefeleisen und Chromeisenerz in der Analyse von Rammelsberg abzieht, wie 13,14 : 18,72; oder in dem E. von Stannern beinahe wie 2 : 3, in dem von Juvenas wie 5 : 7.

Berechnet man das spezifische Gewicht des Gemenges aus den Gemengteilen, so erh"alt man f"ur den E. von Stannern: 3,14; f"ur den E. von Juvenas: \emph{a}) 3,10, \emph{b}) 3,13, was mit den gefundenen Werten nahe "ubereinkommt. (Vergl. oben S. 134 die zweite Anmerkung.)

\centerline{*\hspace{15mm}*\hspace{15mm}*\hspace{15mm}*\hspace{15mm}*}
\clearpage
\subsection{\frakfamily{"Ubersicht der Meteoriten in dem mineralogischen Museum von Berlin.}}
\begin{center}
1. Eisenmeteorit.
\end{center}
\begin{center}
1. Meteoreisen.
\end{center}
\begin{center}
a. Aus einem Individuum bestehend, ohne schalige Zusammensetzung.
\end{center}
\begin{center}
\begin{footnotesize}
\begin{tabular}{ |p{7mm}|p{9mm}|p{45mm}|p{23mm}|p{20mm}| }
    \hline
    Fortl. Zahl & Jahr der Auffundg. & Name und Fundort & Gewicht des Hauptst"ucks & Gewicht aller St"ucke\\
    \hline\hline
    1 & 1847 & gefallen 14. Juli. Braunau (Hauptmannsdorf), B"ohmen & 2 Pfd. 21,30 Lth. & 3 Pfd. 6,82 Lth.\\\hline
    2 & 1838 & Claiborne, County Alabama, Ver. St. N. A. & 9,30 Lth. & 9,52 Lth.\\\hline
    3 & 1860 & Saltillo (Coahuila), Mexico & 0,96 Lth. & 1,32 Lth.\\\hline
    4 & 1863 & Dakota Territorium, Ver. St. N. A. & 3,32 Lth. & 3,32 Lth.\\
    \hline
\end{tabular}
\end{footnotesize}
\end{center}
\begin{center}
b. Aus vielen grobk"ornigen Individuen bestehend.
\end{center}
\begin{center}
\begin{footnotesize}
\begin{tabular}{ |p{7mm}|p{9mm}|p{45mm}|p{20mm}|p{23mm}| }
    \hline
    Fortl. Zahl & Jahr der Auffundg. & Name und Fundort & Gewicht des Hauptst"ucks & Gewicht aller St"ucke\\
    \hline\hline
    5 & 1847 & Seel"asgen, Schwiebus, Frankfurt, Preussen & 3 Pfd. 8,55 Lth. & 8 Pfd. 16,83 Lth.\\\hline
    6 & 1856 & Nelson County, Kentucky, V. St. & 16,31 Lth. & 21,77 Lth.\\\hline
    7 & 1853 & Union County, Georgia, V. St. & 2,38 Lth. & 3,26 Lth.\\\hline
    8 & 1788 & Tucuman (Otumpa), Argent. Rep., S. A. & 7,89 Lth. & 8,50 Lth.\\\hline
    9 & 1850 & Tucson, Sonora, Mexico & 1,63 Lth. & 1,76 Lth.\\
    \hline
\end{tabular}
\end{footnotesize}
\end{center}
\clearpage
\begin{center}
c. Aus einem Individuum bestehend, mit schaliger Zusammensetzung parallel den Fl"achen des Oktaeders.
\end{center}
\begin{center}
\begin{footnotesize}
\begin{tabular}{ |p{7mm}|p{9mm}|p{45mm}|p{24mm}|p{24mm}| }
    \hline
    Fortl. Zahl & Jahr der Auffundg. & Name und Fundort & Gewicht des Hauptst"ucks & Gewicht aller St"ucke\\
    \hline\hline
    10 & 1829 & Bohumilitz, Prachin, B"ohmen & 2 Pfd. 19,78 Lth. & 2 Pfd. 22,36 Lth.\\\hline
    11 & 1856 & Brazos, Texas, V. St. & 0,65 Lth. & 0,65 Lth.\\\hline
    12 & 1856 & Denton County, Texas, V. St. & 0,68 Lth. & 0,68 Lth.\\\hline
    13 & 1861 & Cranbourne, Melbourne, Australien & 0,28 Lth. & 0,28 Lth.\\\hline
    14 & 1840 & Cosby Creek, Coke County, Tennessee, V. St. & 3,20 Lth. & 14,34 Lth.\\\hline
    15 & 1844 & Arva (Szlanicza), Ungarn & 18,88 Lth. & 2 Pfd. 2,12 Lth.\\\hline
    16 & 1854 & Sarepta, Saratow, Russland & 4 Pfd. 2,07 Lth. & 5 Pfd. 0,93 Lth.\\\hline
    17 & 1840 & Sevier County, Tennessee, V. St. & 11,88 Lth. & 11,88 Lth\\\hline
    18 & 1816 & Bemdegó, Bahia, Brasilien & 1,20 Lth. & 1,20 Lth.\\\hline
    19 & 1850 & Schwetz, Marienwerder, Preussen & 10 Pfd. 1,21 Lth. & 17 Pfd. 19,65 Lth.\\\hline
    20 & 1850 & Ruffs Mountain, Newberry, V. St. & 8,03 Lth. & 8,03 Lth.\\\hline
    21 & 1851 & Seneca River, New-York, 9 V. St. & 1,03 Lth. & 1,03 Lth.\\\hline
    22 & 1784 & Toluca-Thal, Mexico & 10 Pfd. 12,80 Lth. & 18 Pfd. 29,14 Lth.\\\hline
    23 & 1834 & Misteca, Oaxaca, Mexico & 2 Pfd. 13,70 Lth. & 2 Pfd. 13,70 Lth.\\\hline
    24 & 1784 & Sierra blanca, Huajaquilla, Mexico & 8,49 Lth. & 8,91 Lth.\\\hline
    25 & 1856 & Tula (Netschaevo), Russland & 22,94 Lth. & 22,94 Lth.\\\hline
    26 & 1861 & Robertson County, Tennessee, V. St. & 10,34 Lth. & 10,34 Lth.\\\hline
    27 & 1844 & Carthago, Smith County, Tennessee, V. St. & 1 Pfd. 16,44 Lth. & 1 Pfd. 18,42 Lth.\\\hline
    28 & 1819 & Burlington, Otsega County, New-York, V. St. & 6,27 Lth. & 7,17 Lth.\\\hline
    29 & 1856 & Marshall County, Kentucky, V. St. & 4,39 Lth. & 4,39 Lth.\\\hline
    30 & 1823 & St. Rosa, Tunja, Columbien (Karsten) & 0,03 Lth. & 0,03 Lth.\\\hline
    31 & 1856 & Orange River, S"ud-Afrika & 1,74 Lth. & 1,74 Lth.\\\hline
    32 & 1814 & Texas (Red River) V. St. & 6,38 Lth. & 6,38 Lth.\\\hline
    33 & 1815 & Lenarto, Scharosch, Ungarn & 14,98 Lth. & 26,78 Lth.\\\hline
    34 & 1804 & Durango, Mexico & 1 Pfd. 2,67 Lth. & 1 Pfd. 22,42 Lth.\\\hline
    35 & 1854 & Werchne-Udinsk, West-Sibirien & 1 Pfd. 4,17 Lth. & 1 Pfd. 4,17 Lth.\\\hline
    36 & 1811 & Elbogen, B"ohmen & 9,92 Lth. & 11,95 Lth.\\
    \hline
\end{tabular}
\end{footnotesize}
\end{center}
\begin{center}
\begin{footnotesize}
\begin{tabular}{ |p{7mm}|p{9mm}|p{48mm}|p{20mm}|p{23mm}| }
    \hline
    Fortl. Zahl & Jahr der Auffundg. & Name und Fundort & Gewicht des Hauptst"ucks & Gewicht aller St"ucke\\
    \hline\hline
    37 & 1856 & Nebraska Territory, V. St. & 0,78 Lth. & 0,78 Lth.\\\hline
    38 & 1859 & Wayne County, Ohio, V. St. & 0,07 Lth. & 0,07 Lth.\\\hline
    39 & 1854 & Madoc, Ober-Canada & 1,74 Lth. & 1,74 Lth.\\\hline
    40 & 1835 & Black Mountain, Buncumbe Cty, Nord-Carolina, V. St. & 1,57 Lth. & 1,57 Lth.\\\hline
    41 & 1828 & Caille, Grasse, Var, Frankreich & 5,69 Lth. & 6,23 Lth.\\\hline
    42 & 1751 & gefallen 26. Mai. Agram (Hraschina), Kroatien & 0,64 Lth. & 1,69 Lth.\\
    43 & 1839 & Ashville, Buncumbe County, Nord-Carolina, V. St. & 0,38 Lth. & 0,82 Lth.\\\hline
    44 & 1830 & Guildford, Nord-Carolina, V. St. & 0,06 Lth. & 0,06 Lth.\\\hline
    45 & 1853 & L"owenfuss, Gro"s-Namaqualand, S. Afrika & 3,61 Lth. & 3,61 Lth.\\\hline
    46 & 1845 & Lockport (Cambria), New-York, V. St. & 2,86 Lth. & 2,86 Lth.\\\hline
    47 & 1856 & Jewell Hill, Madison County, Nord-Carolina, V. St. & 6,06 Lth. & 6,06 Lth.\\\hline
    48 & 1856 & Oldham County, (Lagrange), Kentucky, V. St. & 1 Pfd. 5,55 Lth. & 1 Pfd. 29,85 Lth.\\\hline
    49 & 1854 & Putnam County, Georgia, V. St. & 1,48 Lth. & 1,48 Lth.\\\hline
    50 & 1854 & Tazewell, Claiborne County, Tennessee, V. St. & 1 Pfd. 6,70 Lth. & 1 Pfd. 13,51 Lth.\\
    \hline
\end{tabular}
\end{footnotesize}
\end{center}
\begin{center}
d. Aus gro"sk"ornigen, schalig zusammengesetzten Individuen bestehend.
\end{center}
\begin{center}
\begin{footnotesize}
\begin{tabular}{ |p{7mm}|p{9mm}|p{45mm}|p{24mm}|p{24mm}| }
    \hline
    Fortl. Zahl & Jahr der Auffundg. & Name und Fundort & Gewicht des Hauptst"ucks & Gewicht aller St"ucke\\
    \hline\hline
    51 & 1792 & Zacatecas, Mexico & 2 Pfd. 13,18 Lth. & 2 Pfd. 25,15 Lth.\\
    \hline
\end{tabular}
\end{footnotesize}
\end{center}
\clearpage
\begin{center}
e. Aus feink"ornigen Individuen bestehend.
\end{center}
\begin{center}
\begin{footnotesize}
\begin{tabular}{ |p{7mm}|p{9mm}|p{45mm}|p{24mm}|p{24mm}| }
    \hline
    Fortl. Zahl & Jahr der Auffundg. & Name und Fundort & Gewicht des Hauptst"ucks & Gewicht aller St"ucke\\
    \hline\hline
    52 & 1823 & St. Rosa (Tocavita), Tunja, Columbien (Boussingault) & 28,20 Lth. & 2 Pfd. 1,22 Lth.\\\hline
    53 & 1823 & Rasgatà, Zipaquira bei Bogota, Columbien & 4,79 Lth. & 4,79 Lth.\\\hline
    54 & 1849 & Chesterville, S"ud-Carolina, V. St. & 14,93 Lth. & 25,66 Lth.\\\hline
    55 & 1788 & Tucuman (Otumpa), Argent. Rep., S. Amerika & 3,11 Lth. & 3,11 Lth.\\\hline
    56 & 1763 & Senegal, Land Siratik, Bambuk, S. Afrika & 4,46 Lth. & 4,46 Lth.\\\hline
    57 & 1851 & Salt River, Kentucky, V. St. & 1,12 Lth. & 1,12 Lth.\\\hline
    58 & 1801 & Cap d. g. Hoffnung (zw. Sonntags- u. Boschemannssl.) & 1 Pfd. 8,50 Lth. & 1 Pfd. 13,05 Lth.\\\hline
    59 & 1845 & Babb's Mill, Greenville, Green Cty, Tennessee, V. St. & 2,56 Lth. & 2,90 Lth.\\\hline
    60 & 1845 & De Kalb County, Tennessee, V. St. & 0,18 Lth. & 0,18 Lth.\\
    \hline
\end{tabular}
\end{footnotesize}
\end{center}
\clearpage
\begin{center}
2. Pallasit.
\end{center}
\begin{center}
\begin{footnotesize}
\begin{tabular}{ |p{7mm}|p{9mm}|p{55mm}|p{24mm}|p{24mm}| }
    \hline
    Fortl. Zahl & Jahr der Auffundg. & Name und Fundort & Gewicht des Hauptst"ucks & Gewicht aller St"ucke\\
    \hline\hline
    61 & 1776 & Krasnojarsk, Jeniseisk, Sibirien (Pallas-Eisen) & 1 Pfd. 23,21 Lth. & 5 Pfd. 23,06 Lth.\\\hline
    62 & 1822 & Brahin, Minsk, Russland & 15,28 Lth. & 27,26 Lth.\\\hline
    63 & 1863 & Rogue River Mountains, Oregon N. Amerika &  & \\\hline
    64 & 1827 & Atacama, Chile, S"ud-Amerika & 23,31 Lth. & 1 Pfd. 17,52 Lth.\\\hline
    65 & 1751 & Steinbach, zw. Eibenstock u. Joh. Georgenstadt, Sachsen & 1,38 Lth. & 3,01 Lth.\\\hline
    66 & 1861 & Rittersgr"un, Schwarzenberg, Sachsen & 29,27 Lth. & 1 Pfd. 5,65 Lth.\\\hline
    67 & 1861 & Breitenbach, B"ohmen, nahe bei Joh. Georgenstadt & 6,64 Lth. & 8,77 Lth.\\\hline
    68 & 1814 & Bitburg, Trier, Preussen & 0,62 Lth. & 0,62 Lth.\\
    \hline
\end{tabular}
\end{footnotesize}
\end{center}
\begin{center}
3. Mesosiderit.
\end{center}
\begin{center}
\begin{footnotesize}
\begin{tabular}{ |p{7mm}|p{9mm}|p{55mm}|p{24mm}|p{24mm}| }
    \hline
    Fortl. Zahl & Jahr der Auffundg. & Name und Fundort & Gewicht des Hauptst"ucks & Gewicht aller St"ucke\\
    \hline\hline
    69 & 1862 & Sierra de Chaco, Atacama, Chile & 23,90 Lth. & 25,3 Lth.\\\hline
    70 & 1864 & Atacama, 50 engl. Meilen von Copiapo & 0,70 Lth. & 0,70 Lth.\\\hline
    71 & 1856 & Hainholz, Minden, Preussen & 20,18 Lth. & 1 Pfd. 1,75 Lth.\\\hline
    72 &  & Niakornak, Gr"onland & 0,39 Lth. & 0,39 Lth.\\
    \hline
\end{tabular}
\end{footnotesize}
\end{center}
\clearpage
\begin{center}
2. Steinmeteorit.
\end{center}
\begin{center}
1. Chondrit.
\end{center}
\begin{center}
\begin{footnotesize}
\begin{tabular}{ |p{7mm}|p{7mm}|p{13mm}|p{48mm}|p{22mm}|p{22mm}| }
    \hline
    Fortl. Zahl & Jahr der Auffundg. & Datum des Falles & Name und Fundort & Gewicht des Hauptst"ucks & Gewicht aller St"ucke\\
    \hline\hline
    1 & 1843 & 16. Sept. & Klein Wenden, Nordhausen, Erfurt, Preuss. & 4 Pfd. 23,10 Lth. & 5 Pfd. 1,85 Lth.\\\hline
    2 & 1812 & 15. April & Erxleben, Magdeburg, Preussen & 3,39 Lth. & 11,88 Lth.\\\hline
    3 & 1863 & 8. Aug. & Pillistser, Livland & 1,10 Lth. & 1,23 Lth.\\\hline
    4 & 1862 & 7. Okt. & Klein Menow, Alt-Strelitz, Meklenburg & 0,74 Lth. & 1,12 Lth.\\\hline
    5 & 1857 & 24. M"arz & Stauropol, n"ordliche Seite des Kaukasus & 0,97 Lth. & 0,97 Lth.\\\hline
    6 & 1861 & 12. Mai & Butsura (Qutahar), Bengalen, Ostindien & 5,47 Lth. & 5,47 Lth.\\\hline
    7 & 1492 & 7. Nov. & Ensisheim, Elsass, Frankreich & 25,63 Lth. & 1 Pfd. 23,66 Lth.\\\hline
    8 & 1812 & 5. Aug. & Chantonnay, Vendée, Frankreich & 13,06 Lth. & 17,01 Lth.\\\hline
    9 & 1753 & 3. Juli & Tabor (Plan, Strkow), B"ohmen & 2,81 Lth. & 4,67 Lth.\\\hline
    10 & 1768 & 13. Sept. & Lucé en Marne, Sarthe, Frankreich & 1,36 Lth. & 1,36 Lth.\\\hline
    11 & 1790 & 24. Juli & Barbotan, Landes, Frankreich & 13,36 Lth. & 18,14 Lth.\\\hline
    12 & 1805 & 25. M"arz & Doroninsk, Irkutzk, Sibirien & 3,14 Lth. & 3,14 Lth.\\\hline
    13 & 1813 & 10. Sept. & Limerick (Adair, Scagh u. s. w.), Irland & 0,24 Lth. & 0,24 Lth.\\\hline
    14 & 1810 & Aug. & Tipperary (Mooresfort), Irland & 2,32 Lth. & 2,32 Lth.\\\hline
    15 & 1766 & Juli & Albareto, Modena, Italien & 0,07 Lth. & 0,07 Lth.\\\hline
    16 & 1812 & 10. April & Toulouse, Haute Garonne, Frankreich & 1,75 Lth. & 1,75 Lth.\\\hline
    17 & 1818 & Juni & Seres, Macedonien, T"urkei & 1,98 Lth. & 2,93 Lth.\\\hline
    18 & 1829 & 9. Sept. & Krasnoi-Ugol, R"asan, Russland & 4,30 Lth. & 4,30 Lth.\\\hline
    19 & 1822 & 30. Nov. & Allahabad (Futtehpore), Ostindien & 0,37 Lth. & 0,37 Lth.\\\hline
    20 & 1831 & 9. Sept. & Wessely, Hradisch, M"ahren & 0,21 Lth. & 0,21 Lth.\\\hline
    21 & 1841 & 22. M"arz & Gr"uneberg (Heinrichsau), Liegnitz, Preuss. & 1 Pfd. 12,83 Lth. & 1 Pfd. 17,20 Lth.\\\hline
    22 & 1849 & 31. Okt. & Cabarras County, Nord-Carolina, V. St. & 6,27 Lth. & 8,04 Lth.\\\hline
    23 & 1852 & 4. Sept. & Mez"o-Madaras, Marosch, Siebenb"urgen & 5 Pfd. 11,33 Lth. & 5 Pfd. 15,85 Lth.\\\hline
    24 & 1838 & 18. April & Akburpur, Sahurunpur, Ostindien & 0,58 Lth. & 0,58 Lth.\\
    \hline
\end{tabular}
\end{footnotesize}
\end{center}
\begin{center}
\begin{footnotesize}
\begin{tabular}{ |p{7mm}|p{7mm}|p{13mm}|p{48mm}|p{22mm}|p{22mm}| }
    \hline
    Fortl. Zahl & Jahr der Auffundg. & Datum des Falles & Name und Fundort & Gewicht des Hauptst"ucks & Gewicht aller St"ucke\\
    \hline\hline
    25 & 1824 & 15. Jan. & Renazzo, Ferrara, Italien & 0,14 Lth. & 0,18 Lth.\\\hline
    26 & 1753 & 7. Sept. & Luponnas, Ain, Frankreich & 0,10 Lth. & 0,10 Lth.\\\hline
    27 & 1859 & 26. M"arz & Harrison County, Kentucky, V. St. & 1,18 Lth. & 1,18 Lth.\\\hline
    28 & 1785 & 19. Febr. & Eichst"adt (Wittmes), Baiern & 0,96 Lth. & 0,96 Lth.\\\hline
    29 & 1798 & 13. Dez. & Benares, Bengalen, Ostindien & 0,36 Lth. & 0,61 Lth.\\\hline
    30 & 1857 & 27. Dez. & Pegu (Quenggouk, NNO. von Bassein) & 0,86 Lth. & 1,06 Lth.\\\hline
    31 & 1825 & 10. Febr. & Nanjemoy, Maryland, V. St. & 1,99 Lth. & 1,99 Lth.\\\hline
    32 & 1807 & 13. M"arz & Timochin, Juchnow, Smolensk, Russland & 13,14 Lth. & 24,04 Lth.\\\hline
    33 & 1852 & 23. Jan. & Nellore, Yatoor, Madras, Ostindien & 5,56 Lth. & 5,66 Lth.\\\hline
    34 & 1856 & 12. Nov. & Trenzano, SW. von Brescia, Lombardei & 0,42 Lth. & 0,42 Lth.\\\hline
    35 & 1853 & 6. M"arz & Segowlee, Sarun, Ostindien & 0,37 Lth. & 0,37 Lth.\\\hline
    36 & 1807 & 14. Dez. & Weston, Connecticut, V. St. & 1,05 Lth. & 1,27 Lth.\\\hline
    37 & 1808 & 19. April & Parma (Casignano, Borgo San Donino) & 0,92 Lth. & 0,92 Lth.\\\hline
    38 & 1820 & 12. Juli & Lixna, D"unaburg, Witebsk, Russland & 2,06 Lth. & 4,91 Lth.\\\hline
    39 & 1833 & 25. Nov. & Blansko, Br"unn, M"ahren & 1,62 Lth. & 1,62 Lth.\\\hline
    40 & 1828 & 4. Juni & Richmond, Virginia, V. St. & 0,96 Lth. & 1,50 Lth.\\\hline
    41 & 1822 & 13. Sept. & La Baffe, Epinal, Vosges, Frankreich & 0,64 Lth. & 0,64 Lth.\\\hline
    42 & 1857 & 10. Okt. & Ohaba, O. von Karlsburg, Siebenb"urgen & 0,06 Lth. & 0,06 Lth.\\\hline
    43 & 1838 &  & Gouv. Pultava, Russland & 0,48 Lth. & 0,48 Lth.\\\hline
    44 & 1836 & 11. Nov. & Macao, Rio Grande del Norte, Brasilien & 2,24 Lth. & 2,24 Lth.\\\hline
    45 & 1851 & 17. April & G"uterslohe, Minden, Preussen & 1 Pfd. 21,34 Lth. & 1 Pfd. 23,44 Lth.\\\hline
    46 & 1860 & 2. Febr. & Alessandria, Piemont & 0,08 Lth. & 0,08 Lth.\\\hline
    47 & 1794 & 16. Juni & Siena, Toscana, Italien & 3,13 Lth. & 3,62 Lth.\\\hline
    48 & 1857 & 28. Febr. & Parnallee, S. von Madura, S. Hindostan & 24,50 Lth. & 24,97 Lth.\\\hline
    49 & 1855 & 13. Mai & Bremerv"orde, Stade, Hannover & 16,94 Lth. & 16,94 Lth.\\\hline
    50 & 1858 & 9. Dez. & Ausson, Haute Garonne, Frankreich & 28,85 Lth. & 1 Pfd. 2,88 Lth.\\
    \hline
\end{tabular}
\end{footnotesize}
\end{center}
\begin{center}
\begin{footnotesize}
\begin{tabular}{ |p{7mm}|p{7mm}|p{13mm}|p{48mm}|p{22mm}|p{22mm}| }
    \hline
    Fortl. Zahl & Jahr der Auffundg. & Datum des Falles & Name und Fundort & Gewicht des Hauptst"ucks & Gewicht aller St"ucke\\
    \hline\hline
    51 & 1815 & 18. Febr. & Durala, Petialah, Delhi, Ostindien & 1,84 Lth. & 1,84 Lth.\\\hline
    52 & 1768 & 20. Nov. & Mauernkirchen, Oestreich ob der Enns & 9,93 Lth. & 13,79 Lth.\\\hline
    53 & 1863 & 7. Dez. & Tirlemont (Tourinnes-la-Grosse), Belgien & 15,20 Lth. & 15,20 Lth.\\\hline
    54 & 1863 & 2. Juni & Buschhof, Kurland & 4,52 Lth. & 4,52 Lth.\\\hline
    55 & 1847 & 25. Febr. & Linn County, Iowa, V. St. & 17,72 Lth. & 18,71 Lth.\\\hline
    56 & 1854 & 5. Sept. & Linum, Fehrbellin, Potsdam, Preussen & 3 Pfd. 13,83 Lth. & 3 Pfd. 13,83 Lth.\\\hline
    57 & 1803 & 3. Okt. & Apt (Saurette), Vancluse, Frankreich & 0,96 Lth. & 0,96 Lth.\\\hline
    58 & 1814 & 15. Febr. & Bachmut, Jekaterinoslaw, Russland & 3,84 Lth. & 3,84 Lth.\\\hline
    59 & 1860 & 1. Mai & New-Concord, Muskinjum Cty, Ohio, V. St. & 26 Pfd. 24,30 Lth. & 27 Pfd. 4,39 Lth.\\\hline
    60 & 1825 & 14. Sept. & Hanaruru, Owahu, Sandwich-Inseln & 3,86 Lth. & 3,86 Lth.\\\hline
    61 & 1834 & 12. Juni & Charwallas, Ostindien & 0,05 Lth. & 0,05 Lth.\\\hline
    62 & 1858 & 19. Mai & Kakowa, NW. von Orawitza, Tem. Banat & 0,58 Lth. & 0,58 Lth.\\\hline
    63 & 1787 & 13. Okt. & Charkow (Bobrik), Russland & 0,15 Lth. & 0,25 Lth.\\\hline
    64 & 1795 & 13. Dez. & Wold Cottage, Yorkshire, England & 0,18 Lth. & 0,18 Lth.\\\hline
    65 & 1798 & 8. od. 12. Mz. & Salès, Villefranche, Rhone, Frankreich & 0,96 Lth. & 0,96 Lth.\\\hline
    66 & 1715 & 11. April & Schellin, Garz, Stargard, Pommern & 0,34 Lth. & 0,34 Lth.\\\hline
    67 & 1803 & 26. April & Aigle, Orne, Normandie, Frankreich & 1 Pfd. 1,83 Lth. & 4 Pfd. 1,12 Lth.\\\hline
    68 & 1860 & 14. Juli & Dhurmsala, Pundjab, Ostindien & 12,02 Lth. & 12,02 Lth.\\\hline
    69 & 1805 & Nov. & Asco, Corsica & 0,40 Lth. & 0,40 Lth.\\\hline
    70 & 1808 & 3. Sept. & Lissa, Jung-Bunzlau, B"ohmen & 1 Pfd. 7,35 Lth. & 1 Pfd. 15,27 Lth.\\\hline
    71 & 1829 & 14. Aug. & Deal (Long Branch), New-Yersey, V. St. & 0,02 Lth. & 0,02 Lth.\\\hline
    72 & 1860 & 28. M"arz & Kheragur, Agra, Ostindien & 0,28 Lth. & 0,28 Lth.\\\hline
    73 & 1810 & 23. Nov. & Charsonville, Orleans, Loire, Frankreich & 2,18 Lth. & 2,18 Lth.\\\hline
    74 & 1811 & 12. M"arz & Kuleschowka, Poltowa, Russland & 0,12 Lth. & 0,23 Lth.\\\hline
    75 & 1811 & 8. Juli & Berlanguillas, Burgos, Spanien & 1,89 Lth. & 2,31 Lth.\\
    \hline
\end{tabular}
\end{footnotesize}
\end{center}
\begin{center}
\begin{footnotesize}
\begin{tabular}{ |p{7mm}|p{7mm}|p{13mm}|p{48mm}|p{22mm}|p{22mm}| }
    \hline
    Fortl. Zahl & Jahr der Auffundg. & Datum des Falles & Name und Fundort & Gewicht des Hauptst"ucks & Gewicht aller St"ucke\\
    \hline\hline
    76 & 1814 & 5. Sept. & Agen, Lot und Garonne, Frankreich & 1,09 Lth. & 1,09 Lth.\\\hline
    77 & 1818 & 10. April & Zaborzika, Volhynien, Russland & 2,65 Lth. & 3,25 Lth.\\\hline
    78 & 1818 & 10. Aug. & Slobodka, Juchnow, Smolensk, Russland & 7,45 Lth. & 7,45 Lth.\\\hline
    79 & 1819 & 13. Okt. & Politz, K"ostritz, Gera, F"urstenth. Reuss & 1 Pfd. 11,49 Lth. & 1 Pfd. 13,43 Lth.\\\hline
    80 & 1829 & 8. Mai & Forsyth, Monroe County, Georgia, V. St. & 1,11 Lth. & 1,22 Lth.\\\hline
    81 & 1852 & gefunden & Mainz, Grossherzogtum Hessen & 0,12 Lth. & 0,17 Lth.\\\hline
    82 & 1831 & 18. Juli & Vouillé, Poitiers, Vienne, Frankreich & 0,25 Lth. & 0,25 Lth.\\\hline
    83 & 1833 & 27. Dez. & Okniny, Kremenetz, Volhynien, Russland & 3,92 Lth. & 3,92 Lth.\\\hline
    84 & 1839 & 13. Febr. & Little Piney, Loiret, Missouri, V. St. & 0,82 Lth. & 0,87 Lth.\\\hline
    85 & 1841 & 12. Juni & Château-Renard, Loiret, Frankreich & 15,81 Lth. & 26,93 Lth.\\\hline
    86 & 1827 & 16. Febr. & Mhow, Ghazeepore, Allahabad, Ostindien & 0,09 Lth. & 0,09 Lth.\\\hline
    87 & 1855 & 13. Mai & Insel Oesel, Russland & 1,30 Lth. & 1,30 Lth.\\\hline
    88 & 1848 & 20. Mai & Castine, Maine, V. St. & 0,04 Lth. & 0,04 Lth.\\\hline
    89 & 1808 &  & Morodabad, Delhi, Ostindien & 0,12 Lth. & 0,12 Lth.\\\hline
    90 & 1804 & 5. April & Glasgow (Dorf High Possil), Schottland & 0,05 Lth. & 0,05 Lth.\\\hline
    91 & 1842 & 26. April & Milena, Warasdin, Kroatien & 0,58 Lth. & 0,58 Lth.\\\hline
    92 & 1838 & 6. Juni & Chandakapoor, Beraar, Ostindien & 0,30 Lth. & 0,30 Lth.\\\hline
    93 & 1804 &  & Darmstadt & 0,08 Lth. & 0,08 Lth.\\
    \hline
\end{tabular}
\end{footnotesize}
\end{center}
\clearpage
\begin{center}
2. Howardit.
\end{center}
\begin{center}
\begin{footnotesize}
\begin{tabular}{ |p{7mm}|p{7mm}|p{13mm}|p{48mm}|p{22mm}|p{22mm}| }
    \hline
    Fortl. Zahl & Jahr der Auffundg. & Datum des Falles & Name und Fundort & Gewicht des Hauptst"ucks & Gewicht aller St"ucke\\
    \hline\hline
    94 & 1813 & 13. Dez. & Loutolax, Wiborg, Finland & 0,24 Lth. & 0,33 Lth.\\\hline
    95 & 1827 & 5. Okt. & Bialystok (Dorf Knasta), Russland & 4,86 Lth. & 5,09 Lth.\\\hline
    96 & 1803 & 13. Dez. & M"assing (Dorf St. Nicolas), Baiern & 1,43 Lth. & 1,43 Lth.\\\hline
    97 & 1823 & 7. Aug. & Nobleborough, Maine, V. St. N. A. & 0,04 Lth. & 0,04 Lth.\\\hline
    98 & 1843 & 26. Juli & Mallygaum, Kandeish, Ostindien & 0,03 Lth. & 0,03 Lth.\\
    \hline
\end{tabular}
\end{footnotesize}
\end{center}
\begin{center}
3. Chassignit.
\end{center}
\begin{center}
\begin{footnotesize}
\begin{tabular}{ |p{7mm}|p{7mm}|p{13mm}|p{48mm}|p{22mm}|p{22mm}| }
    \hline
    Fortl. Zahl & Jahr der Auffundg. & Datum des Falles & Name und Fundort & Gewicht des Hauptst"ucks & Gewicht aller St"ucke\\
    \hline\hline
    99 & 1815 & 3. Okt. & Chassigny, Langres, Haute Marne, Frankr. & 0,79 Lth. & 0,99 Lth.\\
    \hline
\end{tabular}
\end{footnotesize}
\end{center}
\begin{center}
d. Shalkit.
\end{center}
\begin{center}
\begin{footnotesize}
\begin{tabular}{ |p{7mm}|p{7mm}|p{13mm}|p{48mm}|p{22mm}|p{22mm}| }
    \hline
    Fortl. Zahl & Jahr der Auffundg. & Datum des Falles & Name und Fundort & Gewicht des Hauptst"ucks & Gewicht aller St"ucke\\
    \hline\hline
    100 & 1850 & 30. Nov. & Shalka in Bancoora, Ostindien & 4,79 Lth. & 5,07 Lth.\\
    \hline
\end{tabular}
\end{footnotesize}
\end{center}
\clearpage
\begin{center}
5. Chladnit.
\end{center}
\begin{center}
\begin{footnotesize}
\begin{tabular}{ |p{7mm}|p{7mm}|p{13mm}|p{48mm}|p{22mm}|p{22mm}| }
    \hline
    Fortl. Zahl & Jahr der Auffundg. & Datum des Falles & Name und Fundort & Gewicht des Hauptst"ucks & Gewicht aller St"ucke\\
    \hline\hline
    101 & 1843 & 25. M"arz & Bishopville, S"ud-Carolina, V. St. & 10,73 Lth. & 13,90 Lth.\\
    \hline
\end{tabular}
\end{footnotesize}
\end{center}
\begin{center}
6. Kohligen Meteorit.
\end{center}
\begin{center}
\begin{footnotesize}
\begin{tabular}{ |p{7mm}|p{7mm}|p{13mm}|p{48mm}|p{22mm}|p{22mm}| }
    \hline
    Fortl. Zahl & Jahr der Auffundg. & Datum des Falles & Name und Fundort & Gewicht des Hauptst"ucks & Gewicht aller St"ucke\\
    \hline\hline
    102 & 1806 & 15. M"arz & Alais, Gard, Frankreich & 1,49 Lth. & 1,49 Lth.\\\hline
    103 & 1838 & 13. Okt. & Cold Bokkeveld, Cap d. g. Hoffn., S. Africa & 0,55 Lth. & 1,29 Lth.\\\hline
    104 & 1857 & 15. April & Kaba, SW. v. Debreczin, Ungarn & 0,05 Lth. & 0,05 Lth.\\\hline
    105 & 1864 & 14. Mai & Orgueil (Montauban), Frankreich & 7,875 Lth. & 7,875 Lth.\\
    \hline
\end{tabular}
\end{footnotesize}
\end{center}
\begin{center}
7. Eukrit.
\end{center}
\begin{center}
\begin{footnotesize}
\begin{tabular}{ |p{7mm}|p{7mm}|p{13mm}|p{48mm}|p{22mm}|p{22mm}| }
    \hline
    Fortl. Zahl & Jahr der Auffundg. & Datum des Falles & Name und Fundort & Gewicht des Hauptst"ucks & Gewicht aller St"ucke\\
    \hline\hline
    106 & 1821 & 15. Juni & Juvenas, Ardèche, Frankreich & 1 Pfd. 4,28 Lth. & 2 Pfd. 1,42 Lth.\\\hline
    107 & 1808 & 22. Mai & Stannern, Iglau, M"ahren & 26,99 Lth. & 6 Pfd. 27,65 Lth.\\\hline
    108 & 1819 & 13. Juni & Jonzac, Charente, Frankreich & 0,13 Lth. & 0,13 Lth.\\\hline
    109 & 1855 & 5. Aug. & Petersburg, Lincoln Cty, Tennessee, V. St. & 3,47 Lth. & 4,52 Lth.\\
    \hline
\end{tabular}
\end{footnotesize}
\end{center}
\begin{center}
Das Pfund = 30 Loth = 1/2 Kilogramm, 1 Gramm also = 0,6 Loth.
\end{center}
\clearpage
\section{\frakfamily{Erkl"arung der Kupfertafeln.}}
\begin{center}
(Die St"arke der Vergr"o"serung ist bei den mikroskopischen Zeichnungen selbst angegeben.)
\end{center}
\subsection{\frakfamily{Tafel 1}}
\paragraph{}
Figur 1 --- Die Ecke \emph{o} desselben vergr"o"sert dargestellt, an welcher m"oglichst genau alle "atzunglinien, die an demselben mit der Lupe beobachtet werden konnten, angegeben sind. Man sieht an dieser Zeichnung, dass die "atzunglinien wohl meistenteils parallel den Durchschnittslinien mit den Leucitoaeder (\emph{a} : \emph{a} : $\mathfrak{\frac{1}{2}}$ \emph{a}), wie z. B. die Linien \emph{lm} und \emph{mn}, doch auch zuweilen parallel den Durchschnittslinien mit dem Triakisoktaeder (\emph{a} : $\mathfrak{\frac{1}{2}}$ \emph{a} : $\mathfrak{\frac{1}{2}}$ \emph{a}), wie die Linien \emph{uv} und \emph{vw} gehen.

Figur 2 --- Eine Hexaederfl"ache des Meteoreisens, in welcher die Richtungen angegeben sind nach welchen die Fl"ache nach der "atzung gestreift erscheint.

Figur 3 --- Ein St"uck Eisen von Braunau des Berl. Museums etwas vergr"o"sert gezeichnet, an welchem zwei Spaltungsfl"achen \emph{A} und \emph{C}, und eine Schnittfl"ache \emph{D} sich befinden, welche letztere ungef"ahr einer Dodekaederfl"ache parallel geht. Die R"uckseite ist von nat"urlicher Oberfl"ache begrenzt. Die Fl"achen \emph{ACD} sind ge"atzt, und dadurch die kleinen Rhabditkristalle sichtbar gemacht, deren ganz bestimmte Lage in Bezug auf die Hexaederfl"achen teils durch kleine Striche teils durch Punkte, je nachdem die Kristalle parallel einer Hexaederfl"ache liegen oder rechtwinklig auf ihr stehen, angegeben ist, die aber sonst hier nur willk"urlich eingezeichnet sind, und daher die Menge und Gruppierung derselben wie sie in der Wirklichkeit stattfindet nicht angeben.

Figur 4 --- Ein kleines St"uck des Eisens von Braunau aus dem Berliner Museum in nat"urlicher Gr"o"se, an welchem 3 Fl"achen \emph{A}, \emph{B}, \emph{C} m"oglichst genau parallel den Spaltungsfl"achen angeschliffen und sodann ge"atzt sind.

Figur 5 --- Eine Stelle von einem Hausenblasenabdruck der Fl"ache \emph{C}, Fig. 4, unter dem Mikroskop, an welcher man die "atzunglinien und Rhabditkristalle sieht.

Figur 6 --- Eine Stelle von einem Hausenblasenabdruck einer "ahnlichen geatzten Fl"ache des Eisens von Seel"asgen, wie sie Fig. 4, Taf. 2 dargestellt ist. Die Rhabditkristalle sind untereinander ungef"ahr rechtwinklig, der Schnitt des Eisens muss an dieser Stelle also ungef"ahr parallel einer Spaltungsfl"ache gehen. Die Rhabditkristalle der einen Lage sind s"amtlich zuf"allig gr"osser als die der andern. An den Durchschnitten der auf den Spaltungsfl"achen rechtwinklig stehenden Kristalle, sieht man, dass die Querschnitte rechtwinklig, aber ihre Seiten an dieser Stelle noch immer, wenn auch zuweilen, den Hexaederfl"achen parallel sind, die auf der Schnittfl"ache rechtwinklig stehen. In der Mitte ein Einschluss.

Figur 7 --- Die Stelle Fig. 6 mit ihrer Umgebung in nat"urlicher Gr"o"se.

Figur 8 --- Eine Stelle eines Hausenblasenabdrucks von einer anderen ge"atzten Schnittfl"ache des Eisens von Seel"asgen als die, worauf sich Fig. 6 bezieht.

Figur 9 --- Eine Stelle eines Hausenblasenabdrucks der ge"atzten Schnittfl"ache \emph{D} Fig. 3 des Eisens von Braunau. Man sieht die L"angsschnitte der Rhabditkristalle, die der Kante von \emph{D} und \emph{C} parallel liegen; sie sind hier an den Enden aber anders begrenzt als die Rhabditkristalle bei dem Eisen von Seel"asgen Fig. 6.

Figur 10 --- Eine der wasserhellen Stellen der Olivinplatte Fig. 11 vergr"o"sert; man sieht die geradlinigen, untereinander parallelen, haarf"ormigen Einschl"usse des Olivins.

Figur 11 --- Eine kleine Platte in nat"urlicher Gr"o"se, die aus einem Olivinkristall aus dem Pallas-Eisen geschnitten und mit vielen braun gef"arbten Kl"uften durchsetzt ist, so dass nur wenig Stellen wasserhell geblieben sind.

Figur 12 --- Die haarf"ormigen Einschl"usse von Fig. 10 einer andern kleinen Olivinplatte von der Schnittfl"ache des Kristalls schief durchschnitten, noch mehr vergr"o"sert.

\subsection{\frakfamily{Tafel 2}}
\paragraph{}
Figur 1 --- Eine ge"atzte Schnittfl"ache des Meteoreisens von Zacatecas in nat"urlicher Gr"o"se.

Figur 2 und Figur 3 --- Die in diesem Eisen vorkommenden Schreibersitkristalle, wie sie an den Stellen \emph{a} und \emph{b} von einem Hausenblasenabdruck der Schnittfl"ache Fig. 1 unter dem Mikroskop erscheinen.

Figur 4 --- Ge"atzte Schnittfl"ache des Meteoreisens von Seel"asgen.

Figur 5 --- Durchschnitt von einer Troilitpartie in dem Meteoreisen von Seel"asgen; sie ist zun"achst von einer H"ulle von T"anit (?) und dann von den Zusammensetzungsst"ucken des Meteoreisens umgeben.

Figur 6 --- Eine ge"atzte Platte des Meteoreisens von Bohumilitz in nat"urlicher Gr"o"se.

Figur 7 --- Hausenblasenabdruck der mit \emph{a}, \emph{b}, \emph{c}, \emph{d}, \emph{e}, \emph{f}, \emph{g} bezeichneten Stellen von Fig. 6 bei geringer Vergr"o"serung; \emph{c}, \emph{d}, \emph{e}, \emph{f}, \emph{g} sind die Schalen des Meteoreisens mit ihren "atzelinien und eingeschlossenen Rhabditkristallen, welche beide aber nur unvollst"andig wiedergegeben sind; \emph{a} und \emph{b} die von ihnen eingeschlossenen R"aume mit den T"anitbl"attern, \emph{h} ein anderer Einschluss.

Figur 8 --- Der Einschluss \emph{h} von Fig. 7 noch mehr vergr"o"sert. Er ist zu vergleichen mit dem Einschluss in dem Meteoreisen von Seel"asgen Fig. 6 Taf. 1.

Figur 9 --- Schnittfl"ache des Chondrits von Siena; Teile von graulichschwarzer Farbe schneiden an anderen von lichte graulichwei"ser Farbe scharf ab.

\subsection{\frakfamily{Tafel 3}}
\paragraph{}
Figur 1, Figur 2, Figur 7 --- 1 und 2 sind Stellen einer d"unn geschliffenen Platte des Chondrits von Erxleben unter dem Mikroskop; Fig. 7 zeigt diese Platte in nat"urlicher Gr"o"se. \emph{a} und \emph{b} Fig. 1 und 2 sind ziemlich regelm"a"sig begrenzte Kristalle, wahrscheinlich Olivin, \emph{d} in Fig. 1 wahrscheinlich Durchschnitte von Kugeln, \emph{n} und die "ubrigen vertikal gestrichelten Parthien Nickeleisen, \emph{m} und die horizontal gestrichelten Parthien Magnetkies, die schwarzen Parthien Chromeisenerz und die noch zu bestimmende schwarze Substanz; \emph{c} in Fig. 1 die schwarze Substanz, die nur in der Mitte etwas Nickeleisen enth"alt, welches auf der R"uckseite die ganze Fl"ache von \emph{c} einnimmt. Wo die durchsichtigen (Olivin-)Kristalle an Nickeleisen, Magnetkies oder die schwarze Substanz angrenzen, sind sie meistenteils regelm"a"sig begrenzt, was man besonders bei \emph{a} und \emph{b} in Fig. 2 sehen kann.

Figur 3 --- Der Kristall (Olivin) \emph{a} von Fig. 1 st"arker vergr"o"sert; man erkennt nun darin mehrere H"ohlungen, die eine Fl"ussigkeit mit einer Luftblase einschlie"sen, wie dies auch bei tellurischen Kristallen vorkommt.

Figur 4 --- Der Kristall \emph{b} von Fig. 1 st"arker vergr"o"sert; man sieht nun, dass die dunkle F"arbund der R"ander dadurch entsteht, dass sich auf ihnen eine Menge schwarzer K"orner befinden.

Figur 5 --- Die kleine Kugel bei \emph{a} von Fig. 7, wie Fig. 1 und 2 vergr"o"sert; man sieht die grauen Streifen in ihr, die ebenfalls eine gro"se Menge von kleinen schwarzen K"ornern enthalten.

Figur 6 --- Ein Teil dieser Kugel noch st"arker und wie 3 und 4 vergr"o"sert; die schwarzen K"orner in den grauen Streifen sind noch deutlicher.

Figur 8 --- Durchschnitt einer Kugel in einer d"unngeschliffenen Platte des Chondrits von Kl. Wenden; die Kugel ist von anderer Art als die Fig. 5, viel durchsichtiger und mit einer Menge schwarzer untereinander ziemlich paralleler Querspr"unge durchsetzt, wahrscheinlich ein unvollkommen ausgebildeter Olivinkristall. Etwas Nickeleisen ist in ihr eingewachsen.

Figur 9 --- Meteoreisen vom Cap der guten Hoffnung, mit zwei fast unter rechtem Winkel zusammensto"senden Schnittfl"achen, die ge"atzt sind, in nat"urlicher Gr"o"se. Man sieht hellere und dunklere breite Streifen sich gerader Richtung "uber die Fl"achen hinziehen. Bei \emph{B}, \emph{D}, \emph{G} finden sich Rostflecken; in dem hellen Streifen bei \emph{g} ist ein regelm"a"sig begrenzter Troilitkristall, bei \emph{A} sind Schreibersit oder Rhabditkristalle eingewachsen; kleinere dergleichen Kristalle finden sich auf der "ubrigen Fl"ache hier und da und gl"anzen bei bestimmter Beleuchtung.

Figur 10 --- Die Fl"ache \emph{AC} von Fig. 9 in anderer Lage, bei welcher nun die dunklen Streifen hell und die hellen dunkel erscheinen.

Figur 11 --- Stelle an der Grenze der Streifen \emph{g} und \emph{h} von dem Hausenblasenabdruck der Fl"ache \emph{AC} Fig. 9. Die Fl"ache erscheint mit seinen K"ornern bedeckt, die in dem Streifen \emph{h} rund, in dem Streifen \emph{g} mit andern gemengt sind, die in die L"ange gezogen erscheinen.

Figur 12 --- Stelle von einer d"unngeschliffenen Platte des Chondrits von Chantonnay in ebenso starker Vergr"o"serung, wie die Fig. 3 und 4. Sie zeigt eine Gruppe durchsichtiger (Olivin-)Kristalle, die fast ganz von der schwarzen Substanz umgeben, und daher fast ganz regelm"a"sig begrenzt sind, und feine untereinander parallele haarf"ormige Einschl"usse haben, wie die Olivinkristall in dem Pallas-Eisen.

\subsection{\frakfamily{Tafel 4}}
\paragraph{}
Figur 1 --- Olivinkristall aus dem Pallas-Eisen; er ist fast vollst"andig bis auf die Zusammensetzungsfl"ache \emph{d}, in welcher er sich mit einem andern Kristalle begrenzt hatte, ist aber fast vollkommen abgerundet bis auf einzelne Fl"achen, die sich nicht in Kanten schneiden, und daher auch runde Umrisse haben. Die vorhandenen Fl"achen liegen merkw"urdiger Weise alle in einer Zone, in der Zone der L"angsprismen, die auf diese Weise recht vollst"andig ausgebildet ist.

Figur 2 --- R"ohrenartige Einschl"usse von verschiedenem Ansehen, wie sie in dem Olivin des Pallas-Eisen vorkommen, noch mehr vergr"o"sert als in Fig. 10 Tafel 1, wo sie in ihrer nat"urlichen Lage gezeichnet sind.

Figur 3 und Figur 4 --- Olivinkristall aus dem Chondrit von Stauropol.

Figur 5 --- Kuglige Zusammenh"aufung von Olivinkristallen in demselben Chondrit.

Figur 6 und Figur 7 --- Kugeln aus diesem Chondrit.

Figur 8--- Eine Kugel in dem Chondrit von Krasnoi-Ugol, neben welcher sich rechts eine kleinere Kugel, die wahrscheinlich ein kugliger Olivinkristall ist, befindet.

Figur 9 --- Zusammenh"aufung von Kugeln in demselben Chondrit.

Figur 10 --- Kugliger Olivin aus dem Chondrit von Timochin.
\section{\frakfamily{Abbildungen.}}
\clearpage
\setlength\intextsep{0pt}
\pagestyle{fancy}
\fancyhf{}
\rhead{\frakfamily{Tafel 1.}}
\cfoot{\frakfamily{\thepage}}
\begin{figure}[p]
\includegraphics[scale=0.5,keepaspectratio]{Figures/Table1/Fig1.png}\tiny\frakfamily 1
\includegraphics[scale=0.5,keepaspectratio]{Figures/Table1/Fig2.png}\tiny\frakfamily 2
\includegraphics[scale=0.5,keepaspectratio]{Figures/Table1/Fig3.png}\tiny\frakfamily 3
\includegraphics[scale=0.5,keepaspectratio]{Figures/Table1/Fig5.png}\tiny\frakfamily 5
\tiny\frakfamily   6
\includegraphics[scale=0.5,keepaspectratio]{Figures/Table1/Fig6.png}
\includegraphics[scale=0.5,keepaspectratio]{Figures/Table1/Fig8.png}\tiny\frakfamily 8
\includegraphics[scale=0.5,keepaspectratio]{Figures/Table1/Fig9.png}\tiny\frakfamily 9
\includegraphics[scale=0.5,keepaspectratio]{Figures/Table1/Fig10.png}\tiny\frakfamily 10
\includegraphics[scale=0.5,keepaspectratio]{Figures/Table1/Fig4.png}\tiny\frakfamily 4
\includegraphics[scale=0.5,keepaspectratio]{Figures/Table1/Fig7.png}\tiny\frakfamily 7
\includegraphics[scale=0.5,keepaspectratio]{Figures/Table1/Fig11.png}\tiny\frakfamily 11
\includegraphics[scale=0.5,keepaspectratio]{Figures/Table1/Fig12.png}\tiny\frakfamily 12
\end{figure}
\clearpage
\rhead{\frakfamily{Tafel 2.}}
\cfoot{\frakfamily{\thepage}}
\begin{figure}[p]
\tiny\frakfamily 1
\includegraphics[scale=0.5,keepaspectratio]{Figures/Table2/Fig1.png}
\includegraphics[scale=0.5,keepaspectratio]{Figures/Table2/Fig4.png}\tiny\frakfamily 4
\includegraphics[scale=0.5,keepaspectratio]{Figures/Table2/Fig2.png}\tiny\frakfamily 2
\includegraphics[scale=0.5,keepaspectratio]{Figures/Table2/Fig3.png}\tiny\frakfamily 3
\tiny\frakfamily 7
\includegraphics[scale=0.5,keepaspectratio]{Figures/Table2/Fig7.png}
\includegraphics[scale=0.5,keepaspectratio]{Figures/Table2/Fig6.png}\tiny\frakfamily 6
\includegraphics[scale=0.5,keepaspectratio]{Figures/Table2/Fig5.png}\tiny\frakfamily 5
\includegraphics[scale=0.5,keepaspectratio]{Figures/Table2/Fig8.png}\tiny\frakfamily 8
\includegraphics[scale=0.5,keepaspectratio]{Figures/Table2/Fig9.png}\tiny\frakfamily 9
\end{figure}
\clearpage
\rhead{\frakfamily{Tafel 3.}}
\cfoot{\frakfamily{\thepage}}
\begin{figure}[p]
\includegraphics[scale=0.5,keepaspectratio]{Figures/Table3/Fig1.png}\tiny\frakfamily 1
\includegraphics[scale=0.5,keepaspectratio]{Figures/Table3/Fig2.png}\tiny\frakfamily 2
\includegraphics[scale=0.5,keepaspectratio]{Figures/Table3/Fig3.png}\tiny\frakfamily 3
\includegraphics[scale=0.5,keepaspectratio]{Figures/Table3/Fig7.png}\tiny\frakfamily 7
\includegraphics[scale=0.5,keepaspectratio]{Figures/Table3/Fig4.png}\tiny\frakfamily 4
\includegraphics[scale=0.5,keepaspectratio]{Figures/Table3/Fig5.png}\tiny\frakfamily 5
\includegraphics[scale=0.5,keepaspectratio]{Figures/Table3/Fig6.png}\tiny\frakfamily 6
\includegraphics[scale=0.5,keepaspectratio]{Figures/Table3/Fig8.png}\tiny\frakfamily 8
\includegraphics[scale=0.5,keepaspectratio]{Figures/Table3/Fig9.png}\tiny\frakfamily 9
\includegraphics[scale=0.5,keepaspectratio]{Figures/Table3/Fig10.png}\tiny\frakfamily 10
\includegraphics[scale=0.5,keepaspectratio]{Figures/Table3/Fig11.png}\tiny\frakfamily 11
\includegraphics[scale=0.5,keepaspectratio]{Figures/Table3/Fig12.png}\tiny\frakfamily 12
\end{figure}
\clearpage
\rhead{\frakfamily{Tafel 4.}}
\cfoot{\frakfamily{\thepage}}
\begin{figure}[p]
\includegraphics[scale=0.5,keepaspectratio]{Figures/Table4/Fig1.png}\tiny\frakfamily 1
\includegraphics[scale=0.5,keepaspectratio]{Figures/Table4/Fig2.png}\tiny\frakfamily 2
\includegraphics[scale=0.5,keepaspectratio]{Figures/Table4/Fig3.png}\tiny\frakfamily 3
\includegraphics[scale=0.5,keepaspectratio]{Figures/Table4/Fig4.png}\tiny\frakfamily 4
\includegraphics[scale=0.5,keepaspectratio]{Figures/Table4/Fig5.png}\tiny\frakfamily 5
\includegraphics[scale=0.5,keepaspectratio]{Figures/Table4/Fig6.png}\tiny\frakfamily 6
\includegraphics[scale=0.5,keepaspectratio]{Figures/Table4/Fig7.png}\tiny\frakfamily 7
\includegraphics[scale=0.5,keepaspectratio]{Figures/Table4/Fig8.png}\tiny\frakfamily 8
\includegraphics[scale=0.5,keepaspectratio]{Figures/Table4/Fig9.png}\tiny\frakfamily 9
\includegraphics[scale=0.5,keepaspectratio]{Figures/Table4/Fig10.png}\tiny\frakfamily 10
\end{figure}
\clearpage
\end{document}
