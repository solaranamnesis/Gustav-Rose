\documentclass[a4paper, 11pt, oneside]{article}
\usepackage[utf8]{inputenc}
\usepackage[T1]{fontenc}
\usepackage[ngerman]{babel}
\usepackage{fbb} %Derived from Cardo, provides a Bembo-like font family in otf and pfb format plus LaTeX font support files
\usepackage{booktabs}
\setlength{\emergencystretch}{15pt}
\usepackage{fancyhdr}
\usepackage{graphicx}
\graphicspath{ {./} }
\begin{document}
\begin{titlepage} % Suppresses headers and footers on the title page
	\centering % Centre everything on the title page
	%\scshape % Use small caps for all text on the title page

	%------------------------------------------------
	%	Title
	%------------------------------------------------
	
	\rule{\textwidth}{1.6pt}\vspace*{-\baselineskip}\vspace*{2pt} % Thick horizontal rule
	\rule{\textwidth}{0.4pt} % Thin horizontal rule
	
	\vspace{1.5\baselineskip} % Whitespace above the title
	
	{\scshape\LARGE Beschreibung und Einteilung der Meteoriten\\auf Grund der Sammlung im Mineralogischen Museum zu Berlin.}
	
	\vspace{1\baselineskip} % Whitespace above the title

	\rule{\textwidth}{0.4pt}\vspace*{-\baselineskip}\vspace{3.2pt} % Thin horizontal rule
	\rule{\textwidth}{1.6pt} % Thick horizontal rule
	
	\vspace{1\baselineskip} % Whitespace after the title block
	
	%------------------------------------------------
	%	Subtitle
	%------------------------------------------------
	
	{\scshape Von Gustav Rose.} % Subtitle or further description
	
	\vspace*{1\baselineskip} % Whitespace under the subtitle
	
    {\scshape\small Aus den Abhandlungen der Königliche Akademie der Wissenschaften zu Berlin 1863.\\Mit vier Kupfertafeln.} % Subtitle or further description
    
	%------------------------------------------------
	%	Editor(s)
	%------------------------------------------------
    \vspace*{\fill}

	\vspace{1\baselineskip}

	{\small\scshape Berlin. 1864.}
	
	{\small\scshape{In Kommission bei F. Dümmlers Verlags-Buchhandlung Harrwitz und Gossmann.}}
	
	\vspace{0.5\baselineskip} % Whitespace after the title block

    \scshape Internet Archive Online Edition  % Publication year
	
	{\scshape\small Namensnennung Nicht-kommerziell Weitergabe unter gleichen Bedingungen 4.0 International} % Publisher
\end{titlepage}
\setlength{\parskip}{1mm plus1mm minus1mm}
\clearpage
\tableofcontents
\clearpage
\section{Geschichte der Meteoritensammlung in dem mineralogischen Museum von Berlin auf Grund deren die neue Einteilung der Meteoriten gemacht ist.}
\paragraph{}
Die Meteoritensammlung macht einen besonderen Teil des mineralogischen Museums der Berliner Universität aus. Als bei der Gründung der Universität im Jahre 1810 auch das mineralogische Museum durch Übernahme der Mineraliensammlung der früheren General-Bergbau-Direktion gegründet wurde, waren die wenigen Meteoriten, die sich in derselben befanden, noch nicht getrennt und mit den übrigen Mineralien vereinigt. Wie viele Meteoriten sich schon damals in ihr befanden, lässt sich nicht angeben, da darüber die Nachweisungen fehlen, indessen enthielt sie doch schon manche kostbare Stücke, wie ein großes prachtvolles Exemplar von dem Pallas-Eisen, das in einer Sammlung Russischer Mineralien enthalten war, die Kaiser Alexander I dem Könige Friedrich Wilhelm III im Jahre 1803 zum Geschenk gemacht hatte, so wie ein großes Stück von dem Durango-Eisen, welches Al. von Humboldt aus Mexiko mitgebracht und dem damaligen Direktor Dietrich Karsten für die Sammlung übergeben hatte. Weiß, der nach Karstens Tode Direktor des mineralogischen Museums wurde, hatte ein großes Interesse für die Meteoriten und ließ keine Gelegenheit vorübergehen, die sich ihm zur Erwerbung von Meteoriten darbot, doch fand sich dieselbe im Anfang, wo das Interesse für die Meteoriten überhaupt noch nicht so lebhaft war wie jetzt, nicht häufig. Die erste größere Bereicherung erhielt das Museum erst durch den Ankauf der Mineraliensammlung von Klaproth nach dessen im Jahre 1817 erfolgten Tode, indem sich darin nach Weglassung aller Arten, die sich später als unecht erwiesen haben, Steinmeteorit von 12 Fundörtern und Eisenmeteorit von 5 Fundörtern befanden, und mehrere derselben in mehreren Exemplaren vertreten waren. In dem im Jahre 1826 vollendeten Kataloge des Museums sind Meteorit von 31 Fundörtern aufgeführt, und zwar Steinmeteorit von 21 und Eisenmeteorit von 9 Fundörtern. Aber schon im nächsten Jahre vermehrte sich die Sammlung um mehr als das Doppelte durch das Vermächtnis Chladni's, wodurch die ganze berühmte Meteoritensammlung dieses um die Meteoritenkunde so verdienten Gelehrten dem Berliner Museum zufiel.\footnote{Chladni starb den 3. April 1827 auf einer Reise in Breslau. Er stand in den freundschaftlichsten Beziehungen zu dem damaligen Direktor des Berliner Museums, wie überhaupt zu den Berliner Gelehrten, und dies Verhältnis hatte ihn bei dem Wunsche seine Sammlung gemeinnützig zu machen, den er in seine Testamente ausdrücklich ausgesprochen hatte, wohl besonders bewogen, seine Sammlung dem Berliner Museum zu vermachen.} Sie bestand in Steinmeteoriten von 31 und Eisenmeteoriten von 10 verschiedenen Fundörtern, unter denen 18 neue Meteorit sich befanden.

Durch den Ankauf der Sammlung des Medizinal-Raths Bergemann im Jahre 1837 erhielt das Museum einen Zuwachs an Steinmeteoriten von 9 und von Eisenmeteoriten von 4 Fundörtern, doch waren darunter nur 2 neue Fundörter. Die späteren Erwerbungen geschahen nun nur durch Kauf, Tausch oder Schenkung einzelner Meteorit, wobei vor Allem das große Verdienst hervorzuheben ist, welches sich Al. von Humboldt durch die Schenkung so vieler ausgezeichneter Meteorit um das Museum erworben hat. Bei dem Tode des Professor Weiß im Jahre 1856 belief sich die Zahl der verschiedenen Meteoriten auf 90; sie ist seit dieser Zeit auf 176 gestiegen.\footnote{Wobei die mir zweifelhaften Eisenmeteoriten von Scriba, Hemalga, Newstead, Livingstone und Melrose, die in den Katalogen von der Wiener, Göttinger und Londoner Sammlung aufgeführt werden, nicht gerechnet sind.} Einen großen Zuwachs erhielt sie noch in der neuesten Zeit\footnote{Noch vor der Lesung des dritten Teils dieser Abhandlung.} durch den Ankauf einer ganzen Meteoritensammlung vom Prof. Shepard in New Haven in den Vereinigten Staaten, zu welchem die Akademie auf das liberalste die Mittel bewilligte.\footnote{Vergl. die Monatsberichte der Akademie von 1862, S. 644.} Die Sammlung stammte zum Teil aus der großen Meteoritensammlung des Prof. Lawrence Smith in Louisville, V. St., und enthielt neben vielem Neuen einzelne Stücke von bedeutender Größe wie ein fast vollständiges Exemplar von dem M. von New Concord von 26 Pfund und 24,3 Loth und eine große Platte von dem Toluca-Eisen mit vielen Einschlüssen, die über einen Fuß lang und einen halben Fuß breit ist.

Von vielen Seiten aufgefordert, ein Verzeichnis der in dem Berliner mineralogischen Museum befindlichen Meteoriten bekannt zu machen, schien es mir zweckmäßig in diesem Verzeichnis nicht, wie man gewöhnlich zu tun pflegt, die Eisen- und Steinmeteorit in der zufälligen Ordnung ihrer Fall- oder Fundzeit aufzuführen, sondern das Gleichartige zusammenstellend, sie nach ihrer mineralogischen Beschaffenheit zu ordnen. Ich fand zu einer solchen Anordnung umso mehr Veranlassung, als ich beabsichtigte, mit einem solchen Verzeichnis eine neue Aufstellung der Meteoriten des Berliner Museums vorzunehmen. Ich habe deshalb sämtliche Meteoriten des Museums genau untersucht und dazu sämtliche Stein- und Eisenmeteorit anschleifen lassen, und letztere geätzt, da man nur auf diese Weise bei den ersteren einen Überblick über die Gemengteil erhalten, bei den letzteren die Struktur erkennen kann, eine Arbeit, die lange aufhielt. Außerdem hatte ich von einem großen Teil der ersteren dünne Platten schleifen lassen, und von den geätzten Flächen der letzteren Hausenblasenabdrücke gemacht, um sie unter dem Mikroskop zu beobachten, und bin nun dadurch zu den Resultaten gelangt, die ich mir erlaube, hiermit der Akademie vorzulegen.

Systematische Anordnungen der Meteoriten sind schon von Partsch,\footnote{Die Meteoriten oder die vom Himmel gefallenen Steine und Eisenmassen im k. k. Hof-Mineralien-Kabinette zu Wien 1843, S. 162.} Shepard\footnote{\emph{Report on American meteorites} (from the Amer. Journ. of Science and arts, 2. Ser). New Haven 1848 p. 16.} und in der neusten Zeit von Reichenbach\footnote{Anordnung und Einteilung der Meteoriten in Pongendorffs Annalen 1859, B. 107, S. 155.} versucht worden. Partsch teilt die Meteoriten ein zuerst in Stein- und Eisenmeteorit, und letztere in normale und anomale; eine Einteilung und Benennung, die Reichenbach sehr tadelt, da man bei Meteoriten von normal und anomal nicht reden könne. Vergleicht man aber die Reihung selbst, so ist diese sehr naturgemäß. Zu den anomalen rechnet er nur 4, die von Alais, Capland, Chassigny und Simonod, von denen die drei ersteren allerdings auch von besonderer Art sind. Den von Simonod kenne ich nicht, er wird von Reichenbach als Meteorit ganz verworfen.\footnote{A. a. O. S. 163.} Die normalen werden in 2 Abteilungen geteilt, in solche die kein metallisches Eisen enthalten, wie a) die von Juvenas, Stannern, Konstantinopel, Jonzac, die nach der damaligen Annahme aus Augit und Labrador bestehen und b) die von Bialystock, Loutolax, Nobleborough und Mässing, die außerdem noch Olivin enthalten und ein breccienartiges Ansehen haben, worauf dann die große Schaar derer folgt, die Eisen eingesprengt enthalten. Hätte Partsch die ganze erstere Abteilung noch zu den anomalen gerechnet, so wäre die Einteilung noch passender gewesen, und hätte er die Meteoriten der ersteren Abteilung ungewöhnliche, die der letzteren gewöhnliche genannt, so würden diese Ausdrücke Reichenbach vielleicht weniger Gelegenheit zum Anstoß gegeben haben. Aber die Einteilung ist doch immer noch nicht näher gerechtfertigt und zu unbestimmt.

Das Meteoreisen teilt Partsch in ästiges und derbiges; das erstere erhält durch eingemengten Olivin eine schwammige Gestalt, das letztere hat eine unbestimmte Form und geringe Beimengungen. Zu den ersteren gehören die Meteoriten von Atacama, Krasnojarsk (das Pallas-Eisen) und von Brahin, zu den letzteren alle übrigen, die nun noch weiter, je nachdem sie durch Ätzung mehr oder weniger deutliche oder auch gar keine Widmanstättenschen Figuren geben, eingeteilt werden.

Die Einteilung von Shepard bezieht sich zwar hauptsächlich nur auf den amerikanischen Meteoriten, nimmt aber doch auch auf einige ausländische Rücksicht. Er teilt den Meteoriten ebenfalls zuerst ein in Eisen- und Steinmeteorit und beide dann weiter wie folgt:

I. Klasse: Eisenmeteorit.

1. Ordnung: Dehnbare und gleichartige.

1. Sekt.: Reine (Scriba, Walker County).

2. Sekt.: Legierte.

a) Feinkörnige (Green County, Texas, Dickson County, Burlington).

b) Grobkörnige (De Kalb, Ashville, Guildford, Carthago).

2. Ordnung: Dehnbare und ungleichartige.

1. Sekt.: Blasig-olivinige (Krasnojarsk).

2. Sekt.: Blasig-pyritische (Cambria).

3. Sekt.: Pyritisch-graphitische (Cocke County).

3. Ordnung: Spröde.

1. Sekt.: Reine (Redford County, Randolph County).

2. Sekt.: Legierte (Otsego County).

II. Klasse: Steinmeteorit.

1. Ordnung: Trachytische.

1. Sekt.: Olivinige.

a) Grobkörnige (Weston, Richmond).

b) Feinkörnige (Nobleboro, Little Piney).

2. Sekt.: Augitisch (Juvenas).

3. Sekt.: Chadnitisch (Bishopville).

4. Sekt.: Kohligen (Cold Bokkeveld).

2. Ordnung: Trappartige.

1. Sekt.: Gleichartige (Chantonnay).

2. Sekt.: Porphyrartige (Renazzo).

3. Ordnung: Bimmsteinartige (Waterville).\footnote{Nicht meteorisch.}

Reichenbach gibt wohl den, meiner Meinung nach einzig richtigem Weg zur Einteilung der Meteoriten an, befolgt ihn aber selbst nicht. Er sagt, es wäre am natürlichsten, der Meteorit nach der Verschiedenheit der Mineralspezies, die sie enthalten, einzuteilen, da wir diese aber noch zu wenig kennen, so müssen wir es so machen wie die Botaniker bei den natürlichen Systemen der Pflanzen und die Meteoriten nach der allgemeinen Ähnlichkeit reihen. Er teilt demnach dieselben in 9 Sippen und jede wieder in verschiedene Gruppen, indem er mit den eisenfreien Steinmeteoriten von Langres (Chassigny), Bishopville, Jonzac, Juvenas, Stannern, Konstantinopel anfängt, durch die eisenhaltigen zu den Eisenmeteoriten fortgeht, die noch Olivin enthalten, und mit den aus fast reinem Eisen bestehenden Meteoriten schließt. Es kann nicht fehlen, dass ein so scharfblickender Kenner der Meteoriten, wie Baron Reichenbach, nicht eine Menge neuer interessanter Zusammenstellungen macht und Beziehungen zwischen Meteorsteinen hervorhebt, die früher nicht beachtet waren; sein System ist aber doch nur, wie er selbst das Verfahren der Botaniker nennt, ein „geistreiches Tatonnement“, es ist ihm derselbe Vorwurf zu machen, den er dem Partschen Systeme macht, es fehlt ihm ein Einteilungsprinzip; wir scheinen doch hinreichend in der Kenntnis der Meteoriten vorgerückt, um vollständig die strengen Grundsätze in Anwendung bringen zu können, die uns bei der Einteilung der Gebirgsarten, mit denen die Meteoriten doch vollständig zu vergleichen sind, leiten. Wie man dort aus einem jeden selbstständigen Gemenge eine besondere Gebirgsart macht, so muss man es auch hier tun, und wenn man allerdings auch noch nicht vollständig alle Gemengteil der Meteoriten genau kennt, so weiß man davon doch so viel, um das Zusammengehörige zusammenstellen zu können.

Ich behalte zuerst die alte Einteilung in Eisen- und Steinmeteorit bei, je nachdem der Meteorit nur oder vorzugsweise aus Eisen, und zwar Nickeleisen, oder vorzugsweise aus einem Gemenge von Silicaten bestehen, in denen das Nickeleisen nur untergeordnet oder gar nicht enthalten ist.

I. Die Eisenmeteorit machen 3 Arten aus, Meteoritenarten kann man sie nennen, wie man in der Petrographie Gebirgsarten oder Felsarten sagt.

Die 1. Art besteht aus Nickeleisen, das nur in geringer Menge mit einigen Eisenverbindungen gemengt ist; ich nenne sie Meteoreisen.

Die 2. Art besteht aus demselben Meteoreisen, worin Kristalle von Olivin porphyrartig eingewachsen sind. Der von Pallas am Jenisei gefundene Eisenmeteorit war der erste der Art, den man kennen lernte; er ist bekannt unter dem Namen Pallas-Eisen und bildet noch immer einen Hauptrepräsentanten dieser Art; ich schlage daher vor, die ganze Art Pallasit zu nennen.

Die 3. Art ist ein körniges Gemenge von Meteoreisen und Magnetkies mit Olivin und Augit. Ich nenne sie Mesosiderit von $\mu\varepsilon\sigma$o$\varsigma$ in der Mitte stehend und $\sigma\iota\delta\eta\varrho$o$\varsigma$ Eisen, da sie aus einer ziemlich gleichen Menge von metallischen Eisenverbindungen und Silicaten besteht und so gewissermaßen in der Mitte zwischen den Eisen- und Steinmeteoriten steht.

II. Die Steinmeteorit sind in 7 Arten zu teilen, für die ich die folgenden Namen vorschlage:


1. Chondrit (von $\chi$o$\nu\delta\varrho$o$\varsigma$, die kleine Kugel). Sie ist die erste und hauptsächlichste Art, die den größten Teil der Steinmeteorit enthält. Sie ist durch kleine Kugeln ausgezeichnet, die aus einem noch nicht bestimmten Magnesia-Silicat bestehen und in einem feinkörnigen Gemenge eingemengt sind, das aus Olivin, Chromeisenerz, einer schwarzen noch zu bestimmenden Substanz, sowie aus Nickeleisen und Magnetkies besteht.

2. Howardit zu Ehren Howards benannt, dem wir die erste Analyse eines Meteorsteins verdanken; ein feinkörniges Gemenge von Olivin mit einem weißen Silicat, möglicher Weise Anorthit, und mit einer geringen Menge von Chromeisenerz und Nickeleisen. Sie enthält die Meteorsteine von Loutolax, Bialystock und Mässing u. s. w.

3. Chassignit, von Chassigny, dem Fallorte des einzigen bekannten Meteoriten dieser Art; ein kleinkörniger eisenreicher Olivin mit sparsam eingemengten kleinen Körnern von Chromeisenerz.

4. Chladnit nach Chladni benannt; ein Gemenge von Shepardit (Mg$_{2}$Si$_{3}$) mit einem noch näher zu bestimmenden Tonerde-haltigen Silicate mit geringen Mengen von Nickeleisen, Magnetkies und einigen anderen noch zu bestimmenden Substanzen. Hierher gehört auch nur ein Meteorit, der von Bishopville.\footnote{Mit Chladnit hat zwar Shepard, der diesen Meteorit zuerst untersucht und beschrieben hat, schon das in ihm vorkommende Magnesia-Silicat bezeichnet, doch schien es mir zweckmäßiger, nach Chladni, der sich um die Meteoritenkunde so viele Verdienste erworben hat, einen Meteoriten, als ein Mineral zu benennen, wenn sich dieses auch bis jetzt nur in einem Meteoriten gefunden hat. Ich möchte dann weiter vorschlagen, den bisherigen Chladnit: Shepardit zu nennen, der zwar auch schon einer in diesen Meteoriten sparsam vorkommenden Substanz gegeben ist, die nach Shepard Schwefelchrom ist, die indessen doch in ihren Eigenschaften noch erst sehr wenig gekannt ist; Vorschläge, mit denen Hr. Shepard selbst sich einverstanden gegen mich erklärt hat.}

5. Shalkit, der Meteorstein von Shalka, ein körniges Gemenge von vorwaltendem Olivin mit Shepardit und Chromeisenerz.

6. Die kohligen Meteorit, wie von Bokkeveld und Alais, die ich nicht genauer untersucht habe und für die ich daher noch einen eigenen Nannen aus Setze.

7. Eukrit von $\varepsilon\nu\kappa\varrho\iota\tau$o$\varsigma$ wohl bestimmbar, da die mineralogische Beschaffenheit dieser Art bis auf einige Nebendinge ganz klar ist, und ihre wesentlichen Gemengteil vollkommen bestimmbar sind. Ein hauptsächlich aus Augit und Anorthit bestehendes körniges Gemenge mit einer geringen Menge Magnetkies, meistens noch geringerer Menge Nickeleisen, zuweilen mit kleinen, näher zu bestimmenden tafelartigen Kristallen (Juvenas) und mit etwas Olivin (Petersburg, V. St.). Es gehören hierhin die Meteoriten von Juvenas, Stannern, Jonzac und Petersburg.
\clearpage
\section{Eisenmeteorit.}
\paragraph{}
Die ersten bestimmten Angaben über die Natur des Meteoreisens haben wir von Howard im Jahre 1802 erhalten, der bei Gelegenheit der chemischen Analyse des 1798 bei Benares in Bengalen gefallenen Meteorsteins die merkwürdige Entdeckung machte, dass das in demselben eingesprengte Eisen Nickel enthalte.\footnote{Philosophical transactions von 1802 und daraus in Gilberts Annalen.} Er fand darin 35 pC. und einen ähnlichen Gehalt in andern gediegenen Eisenmassen, die für meteorisch angesehen wurden, in dem Eisen von Otumpa in Brasilien, von Krasnojarsk in Sibirien (dem Pallas-Eisen), dem Eisen aus Böhmen und vom Senegal.

Klaproth bestätigte später den Nickelgehalt bei der Untersuchung des Meteoreisens von Agram und Ellbogen, wenngleich er die Menge darin weit geringer fand (nur 3,5 und 2,5 pC.), und man war nun seit der Zeit gewohnt, den Nickelgehalt als ein charakteristisches Kennzeichen des Meteoreisens zu betrachten, und bei zufällig auf der Oberfläche der Erde gefundenen Eisenmassen ihren meteorischen Ursprung erst dann anzunehmen, wenn die chemische Untersuchung einen Gehalt an Nickel nachgewiesen hatte, eine Annahme, wozu man auch jetzt noch berechtigt ist, da man noch nie ein Meteoreisen ohne Nickelgehalt gefunden hat. Tellurisches gediegenes Eisen enthält keinen Nickel, ist überhaupt eine außerordentliche Seltenheit und wohl zufällig nur durch einen Reduktionsprozess entstanden.

Später fand Stromeyer in dem Meteoreisen neben dem Nickel etwas Kobalt, das ja auch in den tellurischen Mineralien so häufig das Nickel zu begleiten pflegt und darauf auch etwas Kupfer, und Laugier in dem Meteoreisen von Krasnojarsk und Brahin etwas Chrom, 0,5 pC., das aber wie das in den Meteorsteinen vorkommende, wo es Laugier schon früher gefunden hatte, von eingemengtem Chromeisenerz herrührt.

Am meisten bereichert wurde unsere Kenntnis von der chemischen Beschaffenheit des Meteoreisens durch die genauen Analysen derselben von Berzelius, die er zuerst auf Veranlassung des Grafen Caspar Sternberg mit dem in Böhmen aufgefundenen Eisen von Bohumilitz\footnote{Pongendorffs Annalen von 1833 B. 27, S. 118.} und dann bei seiner großen Arbeit über die Meteoriten überhaupt, die durch die Übersendung der 1833 bei Blansko in Mähren gefallenen Meteorsteine durch Baron von Reichenbach veranlasst wurde, mit dem Meteoreisen von Krasnojarsk (dem Pallas-Eisen) und dem von Elbogen\footnote{A. a. O. von 1834 B. 33, S. 123.} angestellt hatte. Er schied durch Behandlung des Meteoreisens mit verdünnter Salpetersäure einen darin löslichen und einen anderen darin unlöslichen Teil, und fand auf diese Weise bei dem Eisen von Bohumilitz (a), Krasnojarsk (b), und Elbogen (c):

 | \emph{a} | \emph{b} | \emph{c}  
Eisen | 93,775 | 88,042 | 88,231  
Nickel | 3,812 | 10,732 | 7,517  
Kobalt | 0,213 | 0,455 | 0,762  
Magnesium | - | 0,050 | 0,279  
Mangan | - | 0,132 | Spur  
Zinn und Kupfer | - | 0,066 | Spur  
Kohle | - | 0,043 | -  
Schwefel | - | Spur | Spur  
Unlösliches | 2,200 | 0,480 | 2,211  
 | 100,000 | 100,000 | 100,000  

Der in verdünnter Salpetersäure unlösliche Rückstand bestand aus metallischen Körnern und Schüppchen, die schwer zu Boden liegen, und aus einer feinen verteilten schwarzen kohleähnlichen Masse, die sich leicht in der Flüssigkeit aufschlämmen lässt. Die ersteren waren merkwürdiger Weise Phosphormetalle und bestanden aus:

Eisen | 65,987 | 48,67 | 68,11  
Nickel | 15,008 | 18,33 | 17,72  
Magnesium | – | 9,66 | 17,72  
Phosphor | 14,023 | 18,47 | 14,17  
Kiesel | 2,037 | - | -  
Kohle | 1,422 | - | -  
98,477 | 95,13 | 100,00  

Die letztere, die beim Erhitzen rauchte und sodann verglimmt, wurde nur bei dem Pallas-Eisen quantitativ untersucht und bestand hier aus:

Eisen 57,18
Nickel 34,00
Magnesium 4,52
Zinn und Kupfer 3,75
Kohle 0,55

Eine Spur von Phosphor, die Berzelius fand, glaubt er umschlossenen Teilen der Phosphorverbindung zu schreiben zu müssen. Bei dem Rückstand aus dem Bohumilitz-Eisen wurden auch noch etwas Kiesel und Chromeisenerz gefunden. Dieser feinere Teil des Rückstands ist daher von dem schwereren wesentlich verschieden zusammengesetzt.

Durch Berzelius wurden also in dem Meteoreisen 6 neue Stoffe aufgefunden: Phosphor, Zinn, Mangan, Magnesium, Kiesel und Kohle, von denen der Phosphor ganz besonders bemerkenswert ist, da solche Phosphormetalle, wie sie in dem Meteoreisen hiernach enthalten sind, unter den tellurischen Mineralien nicht bekannt sind.

Nach Berzelius wurden nun noch Analysen von anderen Meteoreisenmassen und von andern Chemikern nach denselben oder ähnlichen Methoden gemacht, die aber ganz ähnliche Resultate gegeben haben.\footnote{Vergl. die Aufzählung derselben in Rammelsbergs Mineralchemie, S. 902.} In allen wurde ein in verdünnter Säure unlöslicher, hauptsächlich aus Phosphornickeleisen bestehender Rückstand erhalten, derselbe war wie bei Berzelius stets nur in sehr geringer und in sehr veränderlicher Menge enthalten, und außerdem waren die Verhältnisse von Phosphor gegen Eisen und Nickel so verschieden, dass sich eine gemeinschaftliche Formel für die chemische Zusammensetzung dieser Verbindung nicht aufstellen lässt.\footnote{Es geht dies aus den Berechnungen von Rammelsberg hervor, wonach bei den verschiedenen Analysen auf 1 Atom Phosphor 2, 3 1/2, 5, 6, 8, 14, 15, 18, 30 Atome Metall kommen. (A. a. O. S. 948) Es scheint aber nicht, dass man die beiden Arten des Rückstandes, die Berzelius wohl unterschieden, getrennt hat.} Diesem unlöslichen Rückstand hat Haidinger bei Gelegenheit der von Patera ausgeführten Analyse des Meteoreisens von Arva, worin derselbe in verhältnismäßig großer Menge enthalten ist, den Namen Schreibersit gegeben\footnote{Österreich. Blätter für Lit. 1847, N. 175, S. 644 und N. Jahrbuch für Min. von 1848, S. 698.} zu Ehren des früheren Direktors des kaiserlichen Mineralienkabinetts in Wien, der sich um die Meteoritenkunde durch die Herausgabe seiner Beiträge zur Geschichte und Kenntnis meteorischer Stein- und Metallmassen verdient gemacht hat.

Aber schon viel früher, als Berzelius durch seine chemischen Untersuchungen die gemengte Beschaffenheit des Meteoreisens dartat, hatte sie von Widmanstätten in Wien auf eine andere Weise bewiesen. Derselbe zeigte nämlich schon 1808, dass wenn man an dem Meteoreisen angeschliffene und polierte Flächen mit einer schwachen Säure ätzt, gewisse Figuren hervortreten, die man seitdem die Widmanstättenschen Figuren genannt hat. Die Fläche, die vor dem Ätzen ganz gleichartig aussieht, oder nur bei höchster Politur und nach dem Anhauchen schwache Andeutungen der Figuren gibt, erscheint nun überall mit schmalen, glanzlosen, unter einander parallelen Streifen bedeckt, die nach verschiedenen Richtungen gehend, sich unter verschiedenen schiefen Winkeln durchneiden, von dünnen, hervortretenden, metallisch glänzenden Leisten eingefasst werden und dunklere matte Felder einschließen, was alles eine sehr komplizierte Struktur des Meteoreisens anzeigt. Besonders schön fielen diese Figuren auf dem großen Stücke aus, das 1812 von der Elbogener Eisenmasse abgeschnitten und nach Wien gebracht war. Da die schmalen einfassenden Leisten, die wenig oder gar nicht von der verdünnten Säure angegriffen werden, bei der Ätzung aus der übrigen Masse etwas hervortreten, so kam Widmanstätten auf die Idee, die geätzten Eisenmassen wie einen Schriftsatz in der Buchdruckerpresse abdrucken zu lassen, was auch vollkommen gelang. Er konnte dadurch vollkommen naturgetreue Abbildungen liefern, wie sie die Kunst nicht darzustellen vermag. V. Schreibers beschrieb in dem eben genannten Werke\footnote{Beiträge etc. S. 70, vergl. auch Partsch Meteoriten, S. 100.} die einzelnen Teile des Meteoreisens, die Streifen, Einfassungsleisten und Zwischenfelder, und gab auch einen Abdruck von der geätzten Fläche der großen Elbogener Masse des Wiener Mineralien-Kabinetts; die tieferen Stellen sind darin weiß und nur die höheren Stellen, die Leisten, schwarz, zum Teil auch die Zwischenfelder, die oft wieder gestreift erscheinen. Auch von andern Eisenmassen ließ er Abdrücke machen, die später herausgegeben werden sollten, wozu es aber nicht gekommen ist, und die dann nur an einzelne Personen verteilt wurden. Nach v. Schreibers wurden dergleichen Abdrücke nun auch von andern hergestellt, von Partsch in seine Werke die Meteoriten, von Haidinger in den Sitzungsberichten der Wiener Akademie, von mir selbst in Pongendorffs Annalen u. s. w. Außerordentlich schön sind die Abdrücke in den neusten Abhandlungen von Haidinger, die die Figuren des Meteoreisens von Sarepta und Arva darstellen.\footnote{Sitzungsberichte der kaiserl. Akad. d. Wiss. vom 24. Juli 1862.}

Indessen geben nicht alle Eisenmeteorit Widmanstättenschen Figuren, und namentlich ist dies der Fall bei der im Jahre 1847 bei Braunau gefallenen Eisenmasse, bei welcher Haidinger\footnote{Berichte der Versammlungen der Freunde der Naturwissenschaften in Wien, 1847 und daraus in Pongendorffs Ann. 1847 B. 72, S. 580.} die merkwürdige Entdeckung machte, dass sie in ihrer ganzen Masse nach denselben drei untereinander rechtwinkligen Richtungen parallel den Flächen des Hexaeders spaltbar sei. Es waren zwei Massen gefallen, die beide in die Hände des Prälaten vom Kloster zu Braunau Hrn. Rotter gelangten, der die größere, 42 Pfd. 6 Lth. schwere Masse zerschneiden ließ und einzelne Stücke davon den verschiedenen Museen als Geschenk übersandte.\footnote{Auch das Berliner Museum erhielt auf diese Weise ein ausgezeichnet schönes Stück, 2 Pfund 21,3 Loth schwer.} An dem Stücke, welches das k. Mineralien-Kabinett in Wien erhielt, machte Haidinger die obige Beobachtung. Es war wie die übrigen zum Teil durchschnitten und die weitere Trennung durch Zerreißung hervorgebracht, so dass also stellenweise der natürliche Bruch sichtbar war. Da die Spaltungsflächen auf der ganzen Bruchfläche und demnach auch wahrscheinlich durch das ganze Stück in gleicher Richtung fortgehen, so ist das ganze Stück und so auch die ganze Masse, von der es abgeschnitten, ein Stück eines Individuums, eines Kristalls, dessen äußere Form nicht mehr wahrgenommen werden kann, weil er beim Durchzuge durch die Luft zerplatzt und die einzelnen Stücke an der Oberfläche abgeschmolzen sind, dessen innere Struktur in den Stücken aber erhalten ist. Haidinger ließ das erhaltene Stück anschleifen und ätzen; es entstanden nun keine Widmanstättenschen Figuren, aber andere gerade und untereinander parallele Linien nach mehreren Richtungen wurden sichtbar, die nachher von Neumann\footnote{Naturwissenschaftliche Abhandl. gesammelt von Haidinger 1849 B. 3, Abt. 2, S. 45.} ihrer Richtung nach sorgfältig beschrieben und gedeutet wurden, worauf ich später zurückkommen werde.

Sehr wichtige und interessante Untersuchungen über die Struktur des Meteoreisens hat nun in der letzten Zeit der Baron von Reichenbach,\footnote{Pongendorffs Ann. 1861 B. 114, S. 99, 250, 264,477.} gemacht. Er unterscheidet bei den Eisenmeteoriten, die Widmanstättenschen Figuren geben, vier Gemengteil, die durch die Ätzung einer angeschliffenen Fläche sichtbar werden und die er mit dem Namen Balkeneisen oder Kamazit, Bandeisen oder Tänit, Fülleisen oder Plessit und Glanzeisen oder Lamprit bezeichnet. Das Balkeneisen bildet auf der geätzten Fläche die unter einander parallelen Streifen, die sich unter schiefen Winkeln (von 30, 60 und 120 Graden) durchschneiden und nimmt somit den größten Raum ein; es wird durch Ätzung grau und glanzlos und zeigt sich nun mit einer Menge unter einander paralleler Linien nach Art des Braunauer Eisens bedeckt, die Reichenbach Schraffirungslinien nennt und für Andeutungen von Spaltungsflächen hält; in vielen Fällen erscheint es aber selbst wieder körnig, wie namentlich in dem Eisen von Ruffs mountain. Das Bandeisen fasst die Streifen des Balkeneisens ein und bedeckt sie in papierdünnen Blättern zu beiden Seiten; es wird von der verdünnten Säure schwach rötlichgelb gefärbt, sonst wenig oder gar nicht angegriffen, und ragt daher auf der geätzten Fläche über dem Balkeneisen leistenartig etwas hervor. Das Fülleisen erfüllt die drei oder vierseitigen Felder, die von dem Balkeneisen eingeschlossen werden; es wird von der Ätzung wie das Balkeneisen angegriffen und erhält dabei eine noch dunkle graue Farbe, wie dieses. Es ist in manchen Abänderungen wie in dem Eisen von Ruffs mountain gar nicht vorhanden, füllt auch häufig die Felder nicht allein aus, sondern enthält oft noch eine große Menge Blättchen von Bandeisen, die in untereinander paralleler Richtung enge nebeneinander und bei vierseitigen Feldern gewöhnlich zwei parallelen Seiten, oft aber auch zum Teil den beiden andern parallel liegen, in welchem letzteren Fall die Blätter in einer Diagonale des Vierecks aneinandergrenzen. Reichenbach nennt diese die Zwischenfelder ausfüllenden Blätter des Bandeisens Kämme. Das Glanzeisen liegt in einzelnen länglichen Körnern und Streifen in der Mitte des Balkeneisens; es wird durch die verdünnte Säure gar nicht angegriffen und behält den vollen Glanz und die lichte, stahlgraue fast zinnweiße Farbe, die es durch die Politur der Fläche erhalten hat. Es findet sich nicht in allen Eisenmeteoriten, sehr ausgezeichnet in dem von Lenarto und Arva.

Reichenbach prüfte diese 4 Eisenarten noch weiter nach einer Methode, die schon Widmanstätten angewandt hatte, durch das Anlaufen in der Hitze. Er zeigte, dass das Balkeneisen zuerst anläuft, dann das Fülleisen und zuletzt das Band- und Glanzeisen. Da nun auch das erstere von der Säure am leichtesten, die letzteren am schwersten angegriffen werden, so sieht man, dass die Wirkungen der Hitze und der Säure gleichen Schritt halten, wie denn auch beide Erscheinungen auf stärkerer oder schwächerer Verwandtschaft zum Sauerstoff beruhen. Bei einer Hitze, bei welcher das Balkeneisen schon dunkelblau geworden ist, erscheint das Fülleisenbläulichrot und das Bandeisen goldgelb. Stahl läuft aber bekanntlich bei 230° C. gelb, bei 263° purpurrot, bei 290° blau an. Die Hitze also, die das Balkeneisen schon blau macht, färbt erst das Fülleisen purpurrot und das Bandeisen goldgelb.

Eine vollständige Trennung sämtlicher Gemengteil für die chemische Untersuchung konnte Reichenbach nicht bewerkstelligen, doch glückte es ihm wenigstens einigermaßen für einen derselben, für das Bandeisen. Manche dieser Eisenmeteoriten, wie namentlich der von Cosby Creek, die vor ihrer Auffindung vielleicht lange Zeit in der feuchten Erde gelegen haben, sind nämlich an der Oberfläche sehr stark oxydiert und zerteilen sich hier parallel den Blättern des Bandeisens in Platten, welche Zerteilung durch leises Hämmern noch vollständiger bewirkt werden kann. Die oxydierten Platten des Balkeneisens sind aber hier mit papierdünnen Blättern des Bandeisens bedeckt, die sich nun mit Leichtigkeit von dem Balkeneisen ablösen und so in hinreichender Menge zur Analyse gewinnen lassen.\footnote{Diese Blättchen von Bandeisen können so zuweilen von bedeutender Größe erhalten werden; so beschreibt Reichenbach ein Stück von dem Cosby-Eisen in seiner Sammlung das mit einem Blatte Bandeisen bedeckt ist, das eine Länge von 3 Zoll bei einer Breite von 2 Zoll hat.} Reichenbach verfuhr so mit dem Eisen von Cosby; das gesammelte Bandeisen untersuchte er zuerst in Rücksicht des spezifischen Gewichtes, er fand dasselbe 7,428, etwas größer als das spezifische Gewicht der ganzen Masse, das 7,260 beträgt, es wurde sodann von seinem Sohne Reinold v. Reichenbach analysiert, der zugleich auch eine Analyse der ganzen Masse machte. Er fand\footnote{Vergl. Pongendorffs Ann. 1861 B. 114, S. 258.} in dem Bandeisen (a) und in der ganzen Masse nach 2 Analysen (b) und (c):

a | b | c  
Eisen | 85,714 | 90,125 | 89,324  
Nickel | 13,215 | 9,786 | 10,123  
Kobalt | 0,550 | 9,786 | 0,422  
Schwefel | 0,226 | Spur | Spur  
Phosphor | 0,295 | 0,089 | 0,131  
 | 100 | 100 | 100  

Die Analyse gab also in dem Bandeisen einen etwas größer Nickelgehalt, auch etwas mehr Schwefel und Phosphor und dafür weniger Eisen als in der ganzen Masse, welches Verhältnis gegen die übrigen Gemengteil sich noch etwas größer stellen würde, wenn man bei der Analyse der ganzen Masse das Bandeisen hätte entfernen können. Reichenbach glaubte indessen durch diese Analyse noch keine völlige Aufklärung über die chemische Beschaffenheit des Bandeisens erhalten zu haben, da er bei der Besichtigung desselben unter dem Mikroskop fand, dass noch eine Menge anders gearteter Körperchen in demselben eingelagert waren.

In den Eisenmeteoriten, die keine Widmanstättenschen Figuren geben, hat nach Reichenbach die eine oder die andere dieser Eisenarten überhandgenommen und die andern kommen dann nur ganz unregelmäßig und untergeordnet und zum Teil auch gar nicht darin vor. So besteht das Eisen von Braunau fast nur aus Balkeneisen, und das Eisen vom Cap der guten Hoffnung und von Rasgatà ist Reichenbach geneigt, als fast ganz aus Fülleisen bestehend anzunehmen.

Die meisten Abänderungen des Meteoreisens enthalten aber nun noch einen andern Gemengteil, feine nadelförmige Kristalle oder Nadeln, wie sie Reichenbach kurzweg nennt. Wöhler\footnote{Annalen der Chem. u. Pharm. B. 81, S. 254.} beobachtete sie zuerst bei dem Meteoreisen von einem unbekannten Fundort\footnote{Reichenbach hält dies Meteoreisen für das von Santa Rosa in Columbien, da es aber Widmanstättenschen Figuren gibt, stimmt es wenigstens nicht mit dem überein, welches Boussingault von dort mitgebracht und an v. Humboldt geschenkt hat.} sowohl auf dessen polierter und geätzter Fläche, als auch in dem Rückstande bei seiner Behandlung mit verdünnter Salpetersäure, wo sie unter dem Mikroskop kenntlich wurden. Reichenbach zeigte,\footnote{Vergl. Poggendorffs 1862, B. 115, S. 148.} dass sie in den meisten Eisenmeteoriten enthalten sind und bei der Ätzung einer polierten Fläche derselben zum Vorschein kommen, wobei sie einen ausgezeichneten Parallelismus durch die ganze Masse zeigen. Ihre Länge überschreitet selten 2 Linien. Reichenbach hält sie für eine vollkommenere Ausbildung des Bandeisens.\footnote{Zu diesen Einmengungen würden auch noch die kleinen Eisenkügelchen zu zählen sein, die ich zuerst und dann ausführlich Reichenbach beschrieben (Pongendorffs Ann. 1861 B. 113, S. 187 und B. 115, S. 152), und die auch auf der geschliffenen Fläche in ihren Durchschnitten sichtbar werden sollen. Die Annahme von solchen Kügelchen beruht aber, wie ich mich jetzt überzeugt habe, auf einem Irrtum. Die angeblichen runden Kugeln sind nichts anderes als Stellen, die beim Ätzen durch eine ansitzende Luftblase vor dem Angriff der Säure geschützt waren. Die Luftblase bildete sich durch die Art, wie ich das Meteoreisen ätzte und die darin bestand, dass ich dasselbe mit der polierten wohl gereinigten Fläche in die verdünnte Säure tauchte und darin eine halbe bis eine ganze Minute hielt, wobei dann öfter eine Luftblase an der Fläche sitzen blieb, die den Angriff der Säure verhinderte. Wenn man die Fläche vorher mit Wasser nass macht oder die Fläche nicht mit einem Male unter die Oberfläche der Säure bringt, sondern erst mit einer Seite und sie dann mehr und mehr neigt, bis sie ganz in Wasser eingetaucht ist, so bleiben keine Luftblasen hängen. Daher kommt es, dass, wie Reichenbach erwähnt, er die Eisenkügelchen nur bei den Stücken des Berliner Museums und nicht in den Stücken seiner eigenen Sammlung gesehen hatte.}

Außer den genannten, vorzugsweise aus metallischem Eisen bestehenden Einmengungen kommen in den Eisenmeteoriten noch andere, teils gröbere teils feinere, mehr oder weniger häufig vor. Zu den ersteren gehören Schwefeleisen, Grafit und besonders Olivin.

Das Schwefeleisen ist nach den Untersuchungen von Smith und Rammelsberg kein Magnetkies, wie man bisher angenommen hatte, sondern einfach Schwefeleisen, FeS, Troilit, wie es Haidinger zu nennen vorgeschlagen hat\footnote{Zur Erinnerung an den Berichterstatter des Meteoritenfalls von Albareto bei Modena 1766, Domenico Troilit, der schon lange vor Chladni die Tatsächlichkeit der Meteoritenfälle bewies, freilich ohne seiner Meinung Geltung verschaffen zu können. Vergl. Sitzungsbericht d. k. Akad. der Wiss. März 1863.}; also eine Verbindung, die unter den Mineralien der Erde bisher noch nicht bekannt ist. Es besteht nach Rammelsberg\footnote{Monatsber. d. k. Pr. Akad. d. Wiss. 1864, S. 29.} in zwei Abänderungen aus dem Eisen von Seeläsgen (a) und Sevier County (b) und nach der Berechnung nach der Formel (c) aus:

 | a | b | c  
Eisen | 63,41 | 62,22 | 63,64  
Nickel | - | 1,76 | -  
Mangan | 0,64 | - | -  
Schwefel | 35,91 | 36,01 | 36,36  
 | 99,96 | 99,99 | 100,0  

Es enthält also bald Nickel (Schwefelnickel) bald ist es, obgleich mitten in dem nickelhaltigen Meteoreisen vorkommend, davon frei, wie der Olivin in dem Pallas-Eisen (s. weiter unten), doch scheint das erstere häufiger zu sein, da auch Smith in dem Troilit aus dem Meteoreisen von Tazewell etwas Nickel angibt. 

Der Troilit ist bis jetzt in dem Meteoreisen nur derb vorgekommen, in mehr oder weniger großen Körnern und unregelmäßigen Massen, zuweilen in der Form von Zylindern von mehr als 1 Zoll Größe und mehreren Linien Dicke, wie Reichenbach beobachtet hat. Er ist im Bruch uneben, zeigt aber öfter dünnschalige Zusammensetzungsstücke, wie dies öfter beim Magnetkies, z. B. von Bodenmais vorkommt; tombakbraun, metallisch glänzend, spezifisches Gewicht 4,787 (Seeläsgen), 4,817 (Sevier County). Diess hohe spezifische Gewicht unterscheidet ihn von dem Magnetkiese, dessen Gewicht nicht über 4,623 hinauszieht. Ebenso unterscheidet er sich durch sein Verhalten gegen Chlorwasserstoffsäure, indem er sich darin ohne einen Rückstand von Schwefel auflöst. Nickeleisen, Chromeisenerz und Grafit kommen öfter in ihm eingemengt vor, doch ist er zuweilen auch davon ganz frei, wie die Analyse des Troilit von Seeläsgen durch Rammelsberg beweist.

Wenn es so erwiesen ist, dass einfach Schwefeleisen in dem Meteoreisen vorkommt, so bleibt es doch noch auszumachen übrig, ob alles Schwefeleisen in demselben von derselben Art sei oder ob neben diesen nicht auch Magnetkies vorkommt. In den Steinmeteoriten ist, wie bekannt, das Vorkommen dieses letzter nicht zweifelhaft, da, wenn darüber auch noch keine chemischen Untersuchungen angestellt sind, das Schwefeleisen in dem Meteorstein von Juvenas kristallisiert vorkommt, und an der Kristallform als Magnetkies erkannt werden kann. Ist es daher möglich, dass dieser auch in den Eisenmeteoriten vorkommt, so ist dies doch noch nicht erwiesen, wie auf der anderen Seite auch das Vorkommen von einfach Schwefeleisen in den Steinmeteoriten nicht bewiesen ist; ich werde daher bis auf Weiteres das Schwefeleisen der Eisenmeteorit, auch wo es noch nicht untersucht ist, als Troilit und das der Steinmeteorit als Magnetkies aufführen. Der Grafit findet sich in kleinen abgerundeten, im Innern aus dicht zusammengehäuften Schüppchen bestehenden Parthien bis zu der Größe einer Haselnuss oder Walnuss, zuweilen aber auch, wie Haidinger bei dem Eisen von Arva beobachtete\footnote{Pongendorffs Ann. 1846 B. 67, S. 437.} in Pseudomorphosen.\footnote{Haidinger glaubte darin die Form einer Kombination des Hexaeders mit dem Pentagondodekaeder zu erkennen und nimmt daher an, dass die Pseudomorphosen aus Eisenkies entstanden wären, eine Ansicht, die ich jedoch nicht teilen möchte, da Eisenkies mit Sicherheit in den Meteoriten bis jetzt nicht beobachtet ist, und die Pseudomorphosen selbst, die Hr. Haidinger die Güte hatte, mir zur Ansicht zu schicken, mir mehr die Form eines Hexaeders mit zu geschärften als mit schief abgestumpften Kanten zu haben schienen. Man kann nun aber fragen, woraus die Pseudomorphosen dann entstanden wären? Am nächsten liegt hier nun wohl die Annahme, dass dies der Diamant gewesen sei; wenn aber auch diese Annahme durch die Form der Pseudomorphose und die Möglichkeit der Bildung gerechtfertigt wird, so findet sie doch darin eine große Schwierigkeit, dass eben Diamanten in den Meteoriten bisher noch nicht beobachtet sind.} Der Olivin findet sich in einzelnen abgerundeten Körnern in manchen Abänderungen in großer Menge, wie namentlich in dem Meteoreisen, das Pallas 1776 am Jenisei im östlichen Sibirien gefunden hatte. Biot\footnote{Bulletin des sciences, par la soc. philomatique. 1820 p. 89.} schloss aus dem optischen Verhalten der Olivin-Körner, dass dieselben wirkliche Kristalle wären, und ich beobachtete,\footnote{Pongendorffs Ann. 1825 B. 4, S. 186.} dass die Körner, wie wohl meistenteils, ganz rund, wo sie frei im Eisen liegen, doch schon einzelne sehr glänzende Flächen und in seltenen Fällen sogar in großer Menge enthalten. Außerdem Pallas-Eisen enthalten auch noch das Meteoreisen von Brahin (Gouv. Minsk) und von der Wüste Atacama und andere solche Kristalle.

Wo aber diese Einschlüsse vorkommen, sind sie stets, wie Reichenbach hervorhob, von einer Hülle von Balkeneisen umgeben, und wenn sie sich in einem Meteoriten, der Widmanstättenschen Figuren gibt, finden, so fangen diese immer erst in einer gewissen Entfernung, die 1 bis 3 Linien beträgt, an, sich regelmäßig zu entwickeln. Dies zeigt sich besonders schön bei den Olivin-Einschlüssen. Sind sie in großer Menge vorhanden, wie in Pallas-Eisen und in dem Eisen von Brahin und Atacama, so dass sie oft nur wenig Raum zwischen sich lassen, so wird dieser von dem Balken-, Band- und Fülleisen meistenteils ganz ausgefüllt, und zwar so, dass zuerst an dem Olivin sich eine dünne Lage von Balkeneisen anlegt, dann eine viel dünnere Lage von dem Bandeisen folgt und zuletzt das Fülleisen den inneren Raum einnimmt, wie man dies auf einer durch ein solches Meteoreisen gelegten Schnittfläche, die man geätzt hat, sehr gut sehen kann. Sind die Räume zwischen den Olivinkristallen größer, so bilden sich in dem Fülleisen die Widmanstättenschen Figuren; bei den genannten Meteoriten sieht man jedoch diese nur selten, aber bei dem Eisen von Steinbach und Rittersgrün, wo die Olivine kleiner sind und die Eisenmasse zwischen ihnen größer ist, sind auch die Widmanstättenschen Figuren größer, umso mehr als auch nun die Einfassung des Olivins durch das Balkeneisen schmaler ist.

Die Widmanstättenschen Figuren waren früher in den Olivin-haltigen Eisenmeteoriten ganz übersehen, bis sie Partsch in dem Eisen von Steinbach entdeckte,\footnote{Die Meteoriten, S. 91.} der dadurch auf den gleichen Ursprung von vielen Stücken Meteoreisen, die in den verschiedenen Sammlungen mit der Angabe von verschiedenen Fundörtern aufgeführt waren, schloss. Sie wurden nachher auch von Reichenbach beschrieben.

Zu den feiner Einmengungen, die sich in den Eisenmeteoriten finden, gehören eine Menge kleiner mikroskopischer, meist farbloser, doch auch farbiger glänzender Steinchen von mehr als Quarzhärte, die Wöhler außer dem Schreibersit als Rückstand bei der Auflösung des Eisens von Rasgatà erhielt, jedoch nicht weiter untersuchte und einige kleine Quarzkristalle, die ich in dem Eisen von Toluca beobachtete\footnote{Pongendorffs Ann. 1861 B. 113, S. 184.} und die noch so groß und glänzend waren, dass ich ihre Winkel mit Genauigkeit bestimmen konnte. Sie steckten zwar nur in der äußern oxydierten Rinde, doch so, dass man nicht daran zweifeln konnte, dass sie sich in dem Meteoreisen gebildet hatten und nicht erst später hineingekommen waren.

Alle diese Eisenmeteorit, die man zufällig auf der Oberfläche der Erde findet, sind mit solcher Rinde von Eisenoxydhydrat umgeben, die sich erst durch Oxydation gebildet hat. Diese Oxydation geht aber in der Regel nicht weit, und die so entstandene Rinde schützt die Eisenmeteorit vor ihrer Zerstörung und bewirkt, dass sie sich Jahrtausende weiter unversehrt erhalten. Sie ist die Ursache, dass die Eisenmeteorit, die nur so selten fallen, dass man nur die Fallzeit von dreien kennt, doch häufig gefunden werden, so dass man jetzt in den Sammlungen mehr als halb so viel Eisen- wie Steinmeteoriten hat. Die Eisenmeteoriten, welche man hat fallen sehen und unmittelbar nach ihrem Falle gesammelt hat, wie die von Agram und Braunau, haben keine solche Rinde von Brauneisenerz; die rundlichen Erhabenheit und Vertiefungen, die sich überall an der Oberfläche derselben finden, sind, wie Reichenbach gezeigt hat, mit einem dünnen Überzuge von Magneteisenerz bedeckt, der sich bei dem Durchzuge durch die Luft an der Oberfläche durch Schmelzung und Oxydation bildet,\footnote{Pongendorffs Ann. 1858 B. 103, S. 637. Reichenbach spricht hier von Eisenoxydul, was wohl nur heißen soll Oxydoxydul oder Magneteisenerz.} und so äußerst dünn ist, weil das gebildete Magneteisenerz beim Schmelzen abtropft und nur das wenigste durch Adhäsion haften bleibt. Unter diesem Überzuge findet sich dann eine 1 bis 1 Linien dicke Lage, in welcher das Eisen ganz körnig geworden ist, wie auch schon Reichenbach bei dem Eisen von Braunau beobachtet hat\footnote{A. a. O. 1862 B. 115, S. 135.} und auch in den Abdrücken dieses Eisens von Haidinger zu sehen ist,\footnote{Sitzungsberichte der math.-naturw. Klasse d. k. Akad. der Wissenschaft. 1855 B. 15, S. 354, Fig. 5.} was beweist, dass das Eisen vor der Oxydation seinen Aggregatzustand ändert. Dass sich auf der Oberfläche dieser Eisenmassen beim Liegen in und auf der feuchten Erde nicht bloß Eisenoxydhydrat, sondern auch Magneteisenerz bildet, hat Krantz gezeigt,\footnote{Pongendorffs Ann. 1857, B. 101, S. 152.} der auf der Oberfläche des Eisens von Toluca oktaedrische Kristalle dieser Substanz beobachtet hat.
\subsection{Meteoreisen.}
\paragraph{}
Das Meteoreisen von den verschiedenen Orten, wo man es zufällig gefunden oder hat fallen sehen, ist von verschiedener Art. Diese Massen sind:

a) nur Stücke eines Individuums oder eines Kristalls ohne schalige Zusammensetzung,

b) Aggregate grobkörniger Individuen, ebenfalls ohne schalige Zusammensetzung,

c) Individuen mit schaligen Zusammensetzungsstücken parallel den Flächen des Oktaeders (Meteoreisen, das durch Ätzung Widmanstättenschen Figuren gibt),

d) Aggregate mit großkörnigen, schalig zusammengesetzten Individuen,

e) Aggregate mit feinkörnigen Zusammensetzungsstücken.

a) Meteoreisenmassen, welche Stücke eines Kristalls ohne schalige Zusammensetzung sind.

Hierher gehört vor Allen:

1. das Eisen von Braunau in Böhmen (gefallen am 14. Juni 1847), das schon oben S. 34 erwähnt ist und an welchem Haidinger die Beobachtung machte, dass es Spaltungsflächen ganz gleicher Art nach drei untereinander rechtwinkligen Richtungen, also nach den Flächen des Hexaeders hat. Das Meteoreisen hat also dieselbe Form, wie das künstlich dargestellte reine Eisen, bei welchem sich auch oft Massen mit körnigen Zusammensetzungsstücken finden, die die Spaltungsflächen des Hexaeders deutlich wahrnehmen lassen.\footnote{Z. B. bei dem Eisen, das lange Zeit als Rostbalken gedient hat.}

Neumann zeigte,\footnote{Naturwissenschaftliche Abhandlungen, herausgegeben von Haidinger, 1849, B. 3, Abt. 2, S. 45.} dass, wenn man eine polierte Schnittfläche dieses Meteoreisens mit verdünnter Salpetersäure ätzt, sich auf derselben eine Menge gerader und unter einander paralleler Linien oder linienförmiger Furchen bilden, die meistenteils eng neben einander liegen und nach mehreren sich unter verschiedenen Winkeln durchschneidenden Richtungen gehen, von denen aber die Linien einer oder zweier Richtungen stets vorwalten, die Linien der andern Richtungen nur untergeordnet und mehr stellenweise vorkommen. Er bestimmte mit großer Sorgfalt die Lage dieser Linien und zeigte, dass sie auf einer Hexaederfläche nach sechs Richtungen gehen, nach den zwei Diagonalen der Hexaederfläche, \emph{ac} und \emph{bd} (Taf. 1 Fig. 2) und nach 4 andere Richtungen, die den Linien \emph{af}, \emph{ag}, \emph{df}, und \emph{de}, die aus 2 benachbarten Winkeln nach den Mitten der gegenüberliegenden Seiten gezogen werden können, parallel gehen. Die beiden Diagonalen \emph{ac} und \emph{db} schneiden sich also unter rechten Winkeln und ebenso je 2 der übrigen Linien, die aus verschiedenen Winkeln der Hexaederfläche auslaufen, wie \emph{ag} und \emph{df} oder \emph{af} und \emph{de}, dagegen die beiden Linien, die aus einem Winkel auslaufen, wie \emph{de} und \emph{df}, gegen einander Winkel von 36° 52’ und jede dieser Linien mit der benachbarten Seite der Hexaederfläche einen Winkel von 26° 34’ macht.\footnote{Neumann gibt wohl nur aus Versehen den ersten Winkel zu 35° 14’ an, und die Winkel, welche die Linien \emph{af} und \emph{ag} oder \emph{de} und \emph{df} mit den sie durchschneidenden Diagonalen machen, zu 72° 23’ und zu 107° 37’ statt zu 70° 34’ und zu 108° 26’.} Man sieht diese Linien in Fig. 1 (Taf. 1), die eine Zeichnung in vergrößertem Maßstabe von der Hexaederecke \emph{o} an einem Stücke des Braunauer Eisens Fig. 4 ist, an welchem die drei Flächen \emph{A}, \emph{B}, \emph{C} möglichst genau parallel den 3 Spaltungsflächen des Eisens geschliffen und darauf geätzt sind. Die Atzungslinien sind auf Fig. 1 möglichst getreu nach der Natur gezeichnet. Man sieht hier, wie die Linien, die parallel der Richtung \emph{ag} (Fig. 2) gehen, vorherrschen und gruppenweise wiederkehren, während andere nur stellenweise vorkommen und die erster bald durchschneiden, bald nicht. Noch besser sieht man sie in Fig. 5, die eine kleine Stelle in der Gegend von \emph{r} auf der Fläche \emph{C} (Fig. 1) 140-mal vergrößert darstellt.\footnote{Diese Fig. ist auf die Weise gezeichnet, dass von der Fläche \emph{C} Fig. 4 zuerst ein Hausenblasenabdruck gemacht, von der Stelle \emph{r} auf ihr, sodann ein unter dem Mikroskop vergrößertes photographisches Bild angefertigt und dasselbe darauf abgezeichnet wurde. Die Richtung der Linien ist daher genau die der Natur, und die Winkel würden bis auf die Sekunde genau sein, wenn die geschliffene Fläche hätte der Hexaederfläche genau parallel geschliffen werden können. Es ist ein großer Übelstand, der die Untersuchung sehr erschwert, dass man das Meteoreisen nicht spalten kann, sondern alle Hexaederflächen, die man haben will, erst anschleifen lassen muss, was immer mühsam ist, und mit großer Genauigkeit doch nicht ausgeführt werden kann. Indessen sind in diesem Fall die Abweichungen von den Winkeln, die man auf der Zeichnung messen kann, nicht sehr abweichend von den berechneten, woraus hervorgeht, dass die geschliffene Fläche in ihrer Lage wenigstens nicht sehr von der Spaltungsfläche abweicht. Das photographische Bild zu dieser Figur hat Hr. Dr. Hrn. Vogel freundlichst dargestellt.} Die Linien derselben Richtung sind in Fig.5 mit denselben Buchstaben bezeichnet wie in Fig. 2; es fehlen also in Fig. 5 nur die Linien nach den Richtungen \emph{ac} und \emph{af}. Die Linien erscheinen hier oft unterbrochen, die einen dicker und die andern dünner, und die einen erscheinen bald von den andern durchsetzt, bald durchsetzen sie die andern.

Es ist schwer zu sagen, wofür man diese Linien halten soll. Sie charakterisieren keineswegs das Meteoreisen allein, sie finden sich vollkommen ebenso bei dem künstlich dargestellten Eisen, wie Prestel gezeigt hat\footnote{Sitzungsberichte der math. naturw. Klasse der k. k. Akad. der Wiss. B. 15, S. 355.} und, wie ich mich selbst überzeugt habe, unter andern bei einem schönen Spaltungsstück von solchem künstlich dargestellten Eisen, das ich noch Mitscherlich verdanke, bei welchem eine Hexaederkante 1 1/2 Zoll lang ist; die Linien sind feiner als bei dem Meteoreisen, sonst von derselben Art. Da sie auf der Hexaederfläche parallel gehen den Durchschnittslinien von einem Hexaeder mit 4 andern, die in Zwillingsstellung mit dem erster stehen, so dass mit den 4 Eckenaxen des erster immer eine Eckenaxe der 4 andere Individuen parallel ist, um welche diese um 180 gedreht erscheinen, so hielt sie Neumann bei dem Braunauer Eisen auch für die Durchschnittslinien von 5 auf diese Weise regelmäßig verwachsenen Individuen. Andere halten diese Linien für Anzeigen von versteckten Spaltungsflächen, und allerdings würden die Durchschnitte des Hexaeders mit dem Ikositetraeder (\emph{a}:\emph{a}:1/2 \emph{a}) oder dem Triakisoktaeder (1/2 \emph{a}:1/2 \emph{a}:\emph{a}) ganz dieselben Linien geben. Ich möchte sie am liebsten mit den Eindrücken vergleichen, die bei allen Kristallen durch die Ätzung entstehen und die namentlich Leydolt, der sie Vertiefungsgestalten nennt, beim Quarz und Aragonit so schön dargestellt und beschrieben hat. Sie haben bei diesen zwar keine Linienform, sondern erscheinen wie vertiefte Ecken, sind aber doch häufig linienartig aneinandergereiht. Man hat diese Eindrücke erst bei so wenigen Körpern genau untersucht; sie werden sich gewiss in den verschiedenen Fällen sehr verschieden verhalten.

Ich habe an dem Stücke Fig. 4 versucht, die Fortsetzung der Linien einer Fläche auf den benachbarten zu verfolgen, um zu entscheiden, ob die Linien den Durchschnitten des Hexaeders mit dem Triakisoktaeder oder dem Leucitoaeder parallel gehen, aber ich fand, dass bald das eine, bald das andre der Fall war, wie aus Fig. 1 zu ersehen ist. Die meisten dieser Linien, wie \emph{lm} und \emph{mn}, \emph{rs} und \emph{st}, \emph{ps} und \emph{sq}, gehen allerdings parallel den Durchschnitten mit dem Leucitoaeder, doch andere wie \emph{uv} und \emph{vw} parallel den Durchschnitten mit dem Triakisoktaeder. Es ist freilich oft schwer die zusammengehörigen Linien zu erkennen, doch glaube ich mich nicht zu irren, wenn ich annehme, dass die Durchschnittslinien nach beiden Formen vorkommen.

Neben den Ätzungslinien sieht man auf der geätzten Schnitt- oder Spaltungsfläche bei einiger Aufmerksamkeit überall zerstreut noch kleine nadelförmige, metallisch glänzende Kristalle, wie sie Wöhler und v. Reichenbach auch bei anderen Meteoreisen beobachtet haben (vgl. oben S. 38), hervorragen. Beide Beobachter sahen schon, dass sie häufig untereinander parallel, und die Ursache des Schillerns der geätzten Flächen in bestimmten Richtungen sind. Bei dem Braunauer Eisen kann man nun auf das bestimmte sehen, dass sie eine untereinander und in Bezug auf das Eisen, worin sie eingemengt sind, ganz bestimmte Lage haben. Ich konnte dies recht gut beobachten bei einem kleinen Stücke, das durch zwei natürliche Spaltungsflächen und durch eine Schnittfläche begrenzt ist, die ungefähr die Richtung einer Dodekaederfläche hat. Es ist in Taf. 1. Fig. 3 in etwas vergrößertem Maßstabe dargestellt; die zwei natürlichen Spaltungsflächen sind mit \emph{A} und \emph{C}, die angeschliffene Fläche mit \emph{D} bezeichnet; auf der Hinterseite ist das Stück durch die natürliche Oberfläche begrenzt. Man kann hier deutlich sehen, dass kleine prismatische Kristalle auf jeder Spaltungsfläche mit ihren Hauptaxen in zwei den Kanten der Spaltungsflächen parallelen Richtungen, die Kristalle also in dem ganzen Stücke nach den dreierlei Hexaederkante d. i. nach drei untereinander rechtwinkligen Richtungen liegen. Auf der Schlifffläche, der Dodekaederfläche, sieht man nur Kristalle, die parallel der Hexaederkante liegen, als deren Abstumpfungsfläche die Dodekaederfläche erscheint und die auch hier wie die Kristalle auf den Spaltungsflächen in ihrer wahren Länge erscheinen. Dreht man die Fläche \emph{D} um die ihr parallele Hexaederkante, so reflektieren eine große Menge der kleinen Kristalle zu gleicher Zeit das Licht, so wie man in die Richtung der zu der Kante gehörenden Hexaederflächen kommt, also der Flächen \emph{C} und der dritten Hexaederfläche \emph{B}, von der bei dem Stücke auf \emph{A} immer noch Spuren zu sehen sind; die Seitenflächen dieser kleinen Kristalle sind also selbst wie die Hexaederflächen rechtwinklig gegeneinander geneigt, die Kristalle also quadratische Prismen, die nicht nur mit ihren Hauptaxen parallel einer der 3 Hexaederkante, sondern auch mit ihren Seitenflächen parallel den 2 Hexaederflächen dieser Kante liegen. Indessen schillert das Stück noch in anderen Richtungen dazwischen, so dass sich hieraus nur ergibt, dass die Kristalle entweder noch mehrere Seitenflächen haben oder wenn sie nur quadratische Prismen mit ihren Seitenflächen nicht überall untereinander parallel liegen.

In Fig. 3 sind diese kleinen Kristalle auf den Flächen \emph{A}, \emph{C}, \emph{D} durch kleine Striche und Punkte bezeichnet, je nachdem man sie im Längs- oder Querschnitt sieht; doch soll durch sie nur ihre Lage im Allgemeinen angegeben werden, sie sind willkürlich hineingezeichnet, in der Natur sind sie sehr ungleich verteilt und liegen bald einzeln, bald gruppenweise beisammen. Diess letztere sieht man besonders bei einem Stücke des Braunauer Eisens der Berliner Sammlung, an welchem eine Fläche, ungefähr in der Richtung einer Oktaederfläche, angeschliffen ist; man sieht hier nur die Querschnitte der Kristalle; durch das gruppenweise Beisammenliegen erscheint aber die Fläche wie gesprenkelt. Bei dem Stücke Fig. 3 sind auf der Fläche \emph{D}, die der Kombinationskante mit \emph{C} parallel liegenden Kristalle in großer Menge vorhanden.

Noch deutlicher als mit bloßen Augen erscheinen die kleinen eingewachsenen Kristalle, wenn man von der geätzten Fläche einen Hausenblasenabdruck macht und diesen unter dem Mikroskop betrachtet. In Fig. 5 (Taf. 1) sind diese nadelförmigen Kristalle nur klein, und es finden sich zufällig hier meistenteils auch nur solche, die aus der Fläche \emph{C} senkrecht stehen, also nur ihre Querschnitte zeigen. Viel grösser als in dem Eisen von Braunau sind sie in dem von Seeläsgen, von dem Fig. 4 Taf. 2 die Zeichnung einer geätzten Schnittfläche in natürlicher Größe ist. Die Fig. 6-8 Taf. 1 sind Zeichnungen von Stellen der Fläche einer anderen ähnlichen Platte dieses Eisens in 140-maliger Vergrößerung, die wie Fig. 5 Taf. 1 dargestellt sind.\footnote{Das Eisenkorn, welches die Fig. 6 Taf. 1 gezeichnete Stelle, sowie den später zu erwähnenden dreieckigen Einschluss enthält, ist Fig. 7 in natürlicher Größe gezeichnet.} Man sieht in denselben Quer- und Längsschnitten der Prismen, und letztere in zwei untereinander rechtwinkligen Richtungen, woraus sich ergibt, dass die Schnittfläche parallel einer Hexaederfläche geht. Die Prismen erscheinen hier auf das bestimmte als quadratische, aber man sieht zugleich, dass, wenn auch ihre Hauptaxen einer der Hexaederkante, doch ihre Seitenflächen nicht immer den Flächen derselben parallel gehen, und dies scheint überall der Fall zu sein, wie man dies auch beim Drehen der Fläche \emph{D} Fig. 3 beobachten kann, wie eben gezeigt ist. In Fig. 6 sind zufällig die einen der horizontal liegenden Kristalle außerordentlich groß und scheinen hier an den Enden mit der geraden Endfläche begrenzt zu sein, während sie in Fig. 9 Taf. 1, welche eine Stelle von der Fläche \emph{D} Fig. 3 darstellt, doch eine andere Endkristallisation zu haben scheinen. In Fig. 5 und 8 Taf. 1 erscheinen die Kristalle nicht mit ganz parallelen Kanten, was hier offenbar davon herrührt, dass die Schnittfläche nicht genau parallel einer Hexaederfläche geht.

Es scheint mir zweckmäßig, diese kleinen eingewachsenen Kristalle mit einem besonderen Namen zu bezeichnen, ich werde sie daher in dem Folgenden mit dem Namen Rhabdit, von $\rho\alpha\beta\delta$o$\varsigma$ [rhabdos] der Stab, benennen.

Außer diesen feinen Rhabdit-Kristallen finden sich in dem Braunauer Eisen noch andere etwas größere unregelmäßig, zum Teil auch regelmäßig begrenzte Einmengungen, die von der verdünnten Salpetersäure auch nicht angegriffen werden und beim Ätzen des Eisens ihren metallischen Glanz und ihre stahlgraue Farbe behalten. Bei der Auflösung des Eisens in Salpetersäure oder Chlorwasserstoffsäure bleiben diese Einmengungen zurück, und man erkennt in dem Rückstande Kristalle mit regelmäßigen Formen. Fischer sah darin längliche, rechtwinklige Tafeln\footnote{Pongendorffs Ann. B. 73. S. 592.}; ich sah unter dem Mikroskop diese und andere Formen, doch war die angewandte Menge zu gering, um genügende Beobachtungen zu machen.

Von größer Einmengungen finden sich kleine runde oder längliche Parthien von Troilit, auf dessen geätzter Fläche kleine glänzende Punkte erscheinen, der also wohl Nickeleisen in feinen Teilen beigemengt enthält. In der Nähe dieses Troilit werden auf der geätzten Fläche die Ätzunglinien wohl feiner, gehen aber in völlig unveränderter Richtung bis zu ihm fort. Das erstere ist wohl nur eine Folge davon, dass der Troilit viel leichter auflöslich ist als das Meteoreisen, und die Salpetersäure in der Nähe des Troilit durch seine Auflösung noch mehr verdünnt wird, so dass sie auf das Meteoreisen in der Nähe des Troilit nur eine schwächere Einwirkung ausüben kann.

Die beiden oben S. 34 und S. 46 erwähnten Stücke des Braunauer Eisens in dem Berliner Museum haben beide zum Teil noch ihre natürliche Oberfläche mit ihrer rundlichen Erhabenheit und Vertiefungen und ihrer dünnen Decke von Magneteisenerz, und ebenso kann man unter dieser die 1 bis 1 1/2, Linien dicke Lage erkennen, in welcher das Eisen ganz körnig geworden ist.\footnote{Vergl. oben S. 42.}

2. Claiborne, County Alabama V. St., gefunden 1838. Eine 3 Zoll lange und 1 Zoll breite geschnittene Platte von Hrn. v. Reichenbach in Tausch erhalten. Sie gleicht der Platte des Braunauer Eisens, die parallel der Oktaederfläche geschnitten ist und zeigt daher außer den Ätzungslinien die Querschnitte des eingemengten Rhabdit. Außerdem kommen noch einige größere graue, metallisch glänzende Einmengungen vor, die teils eine Körnerteils eine Nadelform haben. Eine Bruchfläche befindet sich daran nicht.

3. Saltillo (Santa Rosa), Neu-Mexico, gefunden 1860. Zwei kleine Platten vom Prof, Shepard durch Tausch erhalten. Sie zeigen die Ätzelinien und Rhabdit-Kristalle sehr deutlich.

5. Meteoreisenmassen, welche Aggregate grobkörniger Individuen ohne schalige Zusammensetzung sind,

Zu diesen gehört besonders:

4. das Meteoreisen von Seeläsgen bei Schwiebus im Reg.-Bezirk Frankfurt in Preußen, gefunden 1847. Dasselbe besteht aus einer Menge größerer und kleinerer anscheinend unregelmäßig begrenzter und unregelmäßig verbundener Zusammensetzungsstücke, die auf den geätzten Schnittflächen die grüßte Ähnlichkeit mit den Schnittflächen des Braunauer Eisens zeigen. Fig. 4 Taf. II ist die schon oben S. 47 erwähnte Zeichnung einer geätzten Schnittfläche dieses Eisens aus der Berliner Sammlung. Man sieht auf den einzelnen Zusammensetzungsstücken die Ätzungslinien sehr schön, meistenteils die, weiche den Linien \emph{df}, \emph{ef} und \emph{db} in Fig. 2 Taf. I entsprechen, doch auch die andern angegebenen. Die Linien einer Richtung herrschen gewöhnlich vor, aber diese vorherrschenden Linien und somit alle übrigen liegen in den meisten Zusammensetzungstücken untereinander verschieden und nur in einigen fast oder ganz gleich. Man sieht ferner auch ohne Lupe die kleinen Kristalle des Rhabdit sowohl in ihren Längs- als Querschnitten als kleine Striche oder Punkte und erstere auf den Zusammensetzungsstücken, bei denen die Schnittfläche ungefähr parallel einer ihrer Spaltungsflachen geht, in zwei ungefähr aufeinander rechtwinkligen Richtungen. Noch besser sieht man sie auf dem Hausenblasenabdruck unter dem Mikroskop, wie sie in den schon oben beschriebenen Fig. 6—5 Tal. I dargestellt sind, Masche Stellen erscheinen auch gesprenkelt, kurz man sieht alle Erscheinungen, die das Eisen von Braunau dargeboten hat.\footnote{Die Ätzelinien und Rhabdit-Kristalle sind in der Zeichnung Fig. 4 nur zum Teil angegeben, es war nicht möglich, alle Einzelheiten rollständig wiederzugeben.} Bei einer bestimmten Beleuchtung erscheinen die einen Zusammensetzungsstücke glänzend und von fast dunkel bleigrauer Farbe, während die andern matt und von lichterer stahlgrauer Farbe sind. Diese bekommen dann in anderer Richtung mehr Glanz, wenngleich der Unterschied der Farbe bleibt. Es erhält auf diese Weise die geätzte Fläche das Ansehen von Damast, womit man dieselbe schon öfter verglichen hat. In der Zeichnung Fig, 4 Taf. II ist dies dadurch darzustellen versucht, dass den letzteren Zusammensetzungsstucken ein etwas grauer Ton gegeben ist, Die glänzenden Stücke haben zuweilen eine übereinstimmende Streifung, aber keineswegs immer, was man oft bei zwei dicht aneinander Grenzenden sehen kann.\footnote{Bei den mit \emph{n} und \emph{l} bezeichneten Individuen ist die Streifung und die Lage der eingeschlossenen Kristalle ganz verschieden, während sie Joch bei den Individuen \emph{n} und \emph{m} ganz gleich ist; und doch haben aller gleiche Glanz.} Worauf also diese gleiche Spiegelung beruht, kann ich nicht angeben.

Troilit ist häufig in dem Eisen von Seeläsgen eingemengt und findet sich zuweilen in größeren Partien, die teils eine zylinderförmige, teils kuglige, teils unförmliche Gestalt, doch immer eine ziemlich ebene Oberfläche haben. Sie sind mit einer etwa eine halbe bis eine ganze Linie dicken Schicht von einer in verdünnter Salpetersäure unlöslichen Substanz, wahrscheinlich einem nickelreicheren Nickeleisen als das Meteoreisen ist, umgeben, die wohl beim Ätzen etwas bräunlichgelb anläuft und dadurch wohl einige Ähnlichkeit mit Eisenkies hat, Joch nicht damit zu verwechseln ist, wie dies öfter geschehen ist.\footnote{Eine solche Troilitpartie mit ihrer Umgebung von dem schwerlöslichen Nickeleisen und den Zusammensetzungsstücken des Meteoreisens ist Fig. 5 Taf. II dargestellt.} Graphit kommt zuweilen in kleinen Parthien in dem Troilit eingemengt vor; aber kein Nickeleisen, und Nickel ist überhaupt nicht einmal chemisch verbunden in dem Troilit des Eisens von Seeläsgen enthalten, wie oben angegeben (vergl. S. 39).

5. Nelson County, Kentucky, Ver. St. N. A., gefunden 1856, ähnlich dem vorigen.

6. Union Cty, Georgia, Ver. St. N. A., gefunden 1853, ebenso.

7. Tucuman (Otumpa), Argentinische Rep. 5, A., gefunden 1788. Mit Angaben dieses Fundorts besitzt das mineralogische Museum 4 Stücke, die demselben auf verschiedene Weise zugekommen und von verschiedenem Ansehen sind. Das Hauptstück befand sich in einer Sammlung von Mineralien, die von dem preußischen Reisenden Sello, der auf Kosten der Regierung Brasilien und die südlich angrenzenden Freistaaten bereiste, aber auf der Reise starb, geschickt waren; es hatte den beiliegenden Zettel: Meteoreisen aus der Provinz Gr. Chaco, Geschenk des Canonego Dr. Bartholo Muños zu Buenos Aires; es hat geätzt ein ähnliches Ansehen wie das Meteoreisen von Seeläsgen, und nach ihm ist die Stelle in dem Verzeichnis bestimmt.

Die beiden folgenden Stücke, 0,56 und 0,05 Loth schwer, stammen aus der Chladnischen Sammlung und haben den Zettel: Bezirk St. Jago del Estero, Prov. Chaco Gualambo in 5. Amerika; sie sind klein, besonders das eine, können aber dem Ansehen nach wohl mit dem erster vereinigt werden.

Das vierte Stück, 3,11 Loth, war auch in einer der Selloschen Sendungen enthalten und hat auf dem Zettel keine andere Angabe als: Meteoreisen aus Tucuman; es ist ein flaches Stück mit feinkörnigem Bruch, gehört also zur 5. Abteilung. Diese Beschaffenheit. scheint mit der des Wiener Stückes aus Tucuman übereinzustimmen, da Partsch (Meteoriten, S. 129) von diesem anführt, dass es dem Eisen vom Senegal, welches zu dieser Varietät gehört, ähnlichsehe. Es muss daher noch unentschieden bleiben, ob die beiden Selloschen Stücke von einer und derselben Eisenmasse stammen oder ob unter Gran Chaco und Tucuman zwei ganz verschiedene Fundorte gemeint sind. Nimmt man an, dass die Stücke von einer Eisenmasse abstammen, und zwar von der großen Masse, die Rubin de Celis besucht, und deren Gewicht er auf 300 Ztr geschätzt hatte, so würde daraus folgen, dass auch die Eisenmeteorit an einer Stelle feinkörnig und an einer andern großkörnig sein können, was bei den Steinmeteoriten zwar häufig vorkommt, bei den Eisenmeteoriten aber noch nicht beobachtet ist.

c. Meteoreisenmassen, welche Stücke eines Kristalls mit schaliger Zusammensetzung parallel den Flächen des Oktaeders sind, d. h. Eisenmassen, die Widmanstättensche Figuren geben.

Meteoreisenmassen dieser Art sind die gewöhnlichsten, wenngleich die Erscheinung nicht überall gleich regelmäßig und deutlich ist. Sie besteht darin, dass das Eisen in der Form des Oktaeders aus lauter übereinander liegenden Schalen parallel den Flächen des Oktaeders zusammengesetzt erscheint, zwischen denen sich dünne Blättchen von dem in verdünnter Salpetersäure unlöslichen Nickeleisen, welches Reichenbach Tänit genannt hat, befinden, Sie beweist, dass die Kristallbildung ruckweise vor sich gegangen ist; sie hat von Zeit zu Zeit aufgehört, während welcher Zeit sich dann der Tänit abgelagert hat, natürlich in so geringer Menge, dass er die Anziehung und somit die weitere regelmäßige Ablagerung des Meteoreisens nicht verhindert hat. Es ist also eine Bildung, wie sie sowohl bei aufgewachsenen als auch eingewachsenen Kristallen häufig vorkommt, bei aufgewachsenen z. B. beim sogenannten Cap-Quarz aus Devonshire, wo eine geringe Menge von Eisenoxyd die Ursache der Schalenbildung ist; bei eingewachsenen z. B. beim Leucit in den Laven vom Vesuv oder bei dem Magneteisenerz in dem Schwedischen Eisenglanz von Norberg in Westmanland, bei welchem letztere die Schalen wie bei dem Meteoreisen parallel den Flächen des Oktaeders gehen. Hat das Meteoreisen lange Zeit in feuchter Erde gelegen, so oxydiert es sich hier an der Oberfläche und ändert sich in Eisenoxydhydrat um; die Oxydation folgt den Schalen, und es lösen sich oft ganz deutlich oktaedrische Teile ab, wie man dies sehr schön bei dem Eisen von Cosby und von Arva sehen kann (vergl. S. 37). Legt man nun Schnittflächen durch solche Massen, poliert und ätzt man dieselben, so ragen die Tänitblättchen mit glänzenden Kanten aus dem matten Grunde der Schnittfläche hervor, und es bilden sich die Widmanstättenschen Figuren, Auf diesen geätzten Schnittflächen kann man die Stärke und gegenseitige Stellung dieser Schalen am bester erkennen, wenn erstere auch von der Lage des Schnitts gegen die Schalen abhängig ist. Man sieht dann, dass sie in der Regel nur eine halbe Linie, zuweilen aber auch 1 bis 2 Linien dick sind, wie z. B. bei dem Eisen von Bohumilitz. Wo sie aber diese Dicke erreichen, haben sie nicht so ebene Flächen, und die Tänitblättchen zwischen den Schalen sind in dem Maaßen unebener. Zuweilen sind sie aber überaus geradflächig, wie z. B. bei dem Eisen von Elbogen, Agram, Texas, Tazewell u. s. w., so dass man: bei den Abdrücken von den geätzten Schnittflächen derselben mit ziemlicher Genauigkeit die Winkel, die die Schalen untereinander machen, messen und danach die Lage des Schnitts in der Eisenmasse bestimmen kann.\footnote{Bei dem Abdruck z. B. von der geätzten Schnitte der Elbogener Masse, der sich in die oben angeführten Werke von v. Schreibers befindet (vergl. oben S. 34), machen \emph{M} die drei schmalsten und geradlinigsten Streifen Winkel von 64°, 60 1/2° und 55 1/2° (in der beistehenden Zeichnung mit \emph{a}, \emph{ß}, \emph{y} bebe zeichnet), die von den Winkeln eines gleichseitigen Dreiecks nicht viel abweichen, daher der Schnitt der Eisenmasse beinahe \emph{E} parallel einer Oktaederfläche geführt ist. Die Schalen, die der vierten Oktaederfläche parallel geben, schneiden daher die Schnittfläche unter einem sehr spitzen Winkel, ihre Durchschnitte sind ß breiter und unregelmäßiger, und ihre genaue Richtung ist nun auch schwerer zu messen. Sie haben eine solche Richtung, dass ihre Durchschnitte mit der Schnittfläche einer Linie parallel gehen, die von \emph{a} aus der gegenüberliegenden Seite so trifft, dass sie mit diesem Winkel von ungefähr 80° und 100° bildet und der spitze Winkel \emph{d} dem Winkel \emph{ß} gegenüber liegt.}

Durch Einwirkung der verdünnten Salpetersäure zeigen sich auf den Querschnitten der Schalen die Ätzungslinien mehr oder weniger deutlich und mehr oder weniger eng nebeneinander liegend. Dadurch dass gewisse Richtungen bei diesen Linien vorherrschen und diese verschieden liegen in den verschiedenen Schalen, erhalten auch die Schnittflächen dieser Meteoriten das damastähnliche Ansehen, wie die der vorigen Abteilung. Zuweilen haben aber nebeneinander und auch auf angrenzenden Schnittflächen ganz gleich gelegene Schalen ganz verschieden liegende Ätzungslinien, wie ich dies z. B. ganz bestimmt bei dem Eisen von Schwetz beobachtet habe und zuweilen erscheinen selbst dieselben Schalen mit denselben Ätzungslinien auf der einen Hälfte glänzend und auf der andern matt; der Glanz richtet sich oft gar nicht nach den Schalen, die geätzte Fläche ist streifenweise glänzend und streifenweise matt, wie dies bei dem Eisen von Bohumilitz zu beobachten ist, und zuweilen erscheint jede Schale körnig und die körnigen Zusammensetzungsstücke verschieden gestreift, wie bei Ruffs mountain, wo diese Zusammensetzung auch Reichenbach hervorhebt. Es sind dies alles Verhältnisse, die noch der Erklärung bedürfen, die aber doch zum Teil von derselben Art sind wie bei dem Eisen von Seeläsgen, wo die Zusammensetzungsstücke, wie es scheint, unregelmäßig nebeneinander liegen, daher es doch wohl sein kann, dass ungeachtet der unregelmäßigen Form der einzelnen Zusammensetzungstücke ihre Lage gegeneinander doch eine gewisse Regelmäßigkeit haben kann.

Wie die Ätzungslinien, so sieht man auch die Rhabdit-Kristalle teils in ihren zwei aufeinander rechtwinkligen Längsschnitten, höchstens liniengroß, teils in ihren Querschnitten als Punkte; recht deutlich z. B. bei dem Eisen von Misteca, und es findet hier auch dasselbe statt, was bei den Ätzungslinien erwähnt ist, dass ihre Stellung in zwei benachbarten Schalen nicht immer gleich, ja oft ganz entgegengesetzt ist.

Außer dem Tänit und Rhabdit enthalten die Schalen noch eine andere bei der Ätzung glänzend bleibende Substanz von stahlgrauer Farbe eingeschlossen. Sie findet sich meistens in plattenförmigen Stücken stets: in der Mitte der Schalen und diesen mit ihren breiten Flächen parallel, und erscheint in manchem Meteoreisen recht häufig, wie z. B. in dem Eisen von Arva, Sarepta, Cosby und Lenarto, und wie dies in den schönen, schon oben S. 33 erwähnten Abdrücken, die Haidinger von den geätzten Schnittflächen der beiden ersten Eisenmassen bekannt gemacht hat, zu sehen ist. Auf der Schnittfläche erscheinen die Körner und Plättchen oft voller kleiner Vertiefungen, was dadurch entsteht, dass sie spröder als die umgebende Masse sind, und daher beim Schleifen des Meteoreisens einzelne Teile von ihnen leicht ‚herausgerissen werden. Reichenbach hat diese Körner und Plättchen wegen ihres starken Glanzes, den sie auch nach der Ätzung behalten, Lamprit genannt; sie sind aber offenbar dasselbe, was Haidinger schon früher bei dem Arva-Eisen Schreibersit genannt hat,\footnote{Vergl. darüber auch Haidinger in den Sitzungsber. d. math. naturw. Kl. d. k. Akad. d. Wiss. von 1862 B. 46.} daher der ältere Name größere Ansprüche hat, beibehalten zu werden.

Der Schreibersit und die Rhabdit-Kristalle finden sich jedoch nicht in jedem dieser Eisenmeteoriten. Während der Schreibersit in den genannten Meteoriten in verhältnismäßig großer Menge vorkommt; findet er sich in den Meteoriten von Schwetz, Misteca, Bohumilitz u. s. w. gar nicht, dagegen in diesen letzteren wiederum der Rhabdit: in großer Menge erscheint, Beide Substanzen scheinen sich’ demnach einander förmlich auszuschließen, In dem Arva-Eisen kommen zwar beide auf eine ausgezeichnete Weise vor, aber sie finden sich doch auch hier nicht beide in einem und: demselben Stücke, denn die einen enthalten Schreibersit, die andern nicht, eine Ungleichheit, die schon Partsch, Reichenbach und Haidinger bei diesem Eisen angegeben haben. Die, welche keinen Schreibersit haben, enthalten dafür den Rhabdit. Es wäre demnach wohl möglich, dass beide nur verschiedene Arten des Vorkommens einer und derselben Substanz wären, und was man von der chemischen Zusammensetzung dieser Massen kennt, ist dem nicht entgegen. Beide müssen wenigstens Phosphor enthalten, denn dies ergibt sich daraus, dass derselbe sowohl von Reinhold von Reichenbach in der Schreibersit führenden Abänderung des Arva-Eisens,\footnote{Vergl. Pongendorffs Ann. 1863 B. 119, S. 172. Es ist hier zwar nicht besonders angegeben, dass das untersuchte Stück eine Schreibersit führende Abänderung des Arva-Eisens ist, doch kann ich dies insofern bezeugen, als das von Reichenbach, dem Sohne, analysierte Stück aus dem Berliner Museum stammt, Reichenbach, der Vater, nämlich, dem bei seiner letzten Anwesenheit in Berlin die vielen eingeschlossenen Schreibersit-Körner in mehreren Stücken des Arva-Eisens des Berliner Museums auffielen, während eine große Masse des Arva-Eisens in seiner Sammlung diese gar nicht enthielten, hat sich eins dieser Berliner Stücke in Austausch gegen ein anderes aus seiner Sammlung aus, um es von seinem Sohne analysieren zu lassen, und so möglicher Weise die Zusammensetzung des darin eingeschlossenen Schreibersits zu erfahren. So entstand die oben angeführte Abhandlung in Pongendorffs Ann., die nun zwar nicht die genaue Zusammensetzung des Schreibersits, doch aber bestimmt ausmachte, dass der Phosphorgehalt dieses Meteoreisens von ihm herrühre.} als auch von Berzelius und Rammelsberg in dem Rhabdit führenden Meteoreisen von Bohumilitz gefunden ist. Da nun diese Eisenmassen von den in verdünnter Salpetersäure schwer löslichen Substanzen, so viel man weiß, nur noch den Tänit enthalten, derselbe aber nach Reinhold von Reichenbach keinen oder nur eine unbedeutende Menge von Phosphor enthält (vergl. oben S. 37), so muss er sowohl in dem Schreibersit als auch in dem Rhabdit enthalten sein, Sollte durch fortgesetzte Untersuchungen es sich bestätigen, dass Schreibersit und Rhabdit dieselbe Substanz sind‚ so muss natürlich der Name Rhabdit fortfallen.

Von den nach verschiedenen Richtungen gehenden Schalen des Meteoreisens werden öfter eckige Räume eingeschlossen (Zwischenfelder, wie sie v. Schreibers nennt), die mit einem Meteoreisen erfüllt sind, das mit ganz dünnen, untereinander parallelen Blättchen einer in verdünnter Salpetersäure unlöslichen Substanz durchsetzt wird. Die Blättchen haben in demselben Raum bald eine bald mehrere Richtungen und gehen bald den Schalen, die sie einschließen, parallel, bald nicht. Es kann fraglich sein, ob sie aus Tänit oder Schreibersit bestehen, doch möchte das erstere wahrscheinlicher sein, was auch die Meinung von Reichenbachs ist (vergl. oben S. 36). Solche Räume sind dieser Abteilung besonders eigen; ich habe zwei solcher nicht weit voneinander liegender Räume, die bei dem Eisen von Bohumilitz vorkommen, mit \emph{a} und \emph{b} bezeichnet, in Fig. 6 Taf. II in natürlicher Größe und in Fig. 7 25-mal vergrößert dargestellt. In der letzteren Fig. sieht man, dass der Raum \emph{a} eigentlich aus 3 Räumen besteht, deren jeder seine besonderen Blättchen hat, die in jedem nach mehreren Richtungen gehen, die sich gegenseitig durchschneiden. Auch der Raum 2 ist, soweit er gezeichnet ist, aus zweien zusammengesetzt, die aber beide nur Blättchen in einer Richtung enthalten. Die Räume \emph{a} und \emph{b} sind von den Schalen des Meteoreisens \emph{c}, \emph{d}, \emph{e}, \emph{f} und \emph{g} umgeben, die die Ätzelinien und die Rhabdit-Kristalle zeigen, die aber nur unvollständig in der Fig. wiedergegeben sind.

In der Ecke von \emph{a} Fig.7 befindet sich noch eine Bildung h, die eine besondere Beschaffenheit hat; sie ist in Fig. 8 besonders und 140-mal vergrößert dargestellt. Sie zeigt auf der Schnittfläche ungefähr parallellaufende gegliederte Streifen, zwischen denen sich kleine Körper befinden, die wie Kristalle aussehen. Dergleichen Bildungen finden sich in dem Eisen von Bohumilitz häufig, oft von noch bedeutenderer Größe, und kommen mit ähnlichen überein, die sich auch in dem Eisen von Seeläsgen finden, wo nur die kleinen Kristallähnlichen Körper fehlen. Eine solche ist in Fig. 7 und 6 Taf. I in natürlicher und 140-maliger Vergrößerung dargestellt.

In dem Folgenden ist das Meteoreisen dieser Abteilung in der Ordnung aufgeführt, dass zuerst die Abänderungen mit den dicken Schalen und dann die mit den dünner folgen.

8. Bohumilitz, Prachimer Kreis in Böhmen, gefunden 1829. Eine große dicke Platte, an 3 Seiten mit natürlicher Oberfläche begrenzt und eine kleine dünne Platte. Fig. 6 Taf. II. Die Schalen sind 1 bis 2 Linien dick, ziemlich geradflächig, die Ätzelinien auf denselben sehr deutlich, der eingemengte Rhabdit im Allgemeinen nicht groß, in einigen ziemlich häufig, in andern weniger. Die mit Tänit erfüllten Räume zwischen den Schalen des Meteoreisens (\emph{a} und \emph{b} in Fig. 7 Taf. II) nicht selten, gewöhnlich aber nur klein. Die große Platte enthält mehrere Parthien von Grafit eingemengt, die mit einer bei der Ätzung glänzend gebliebenen, stahlgrauen Rinde umgeben sind.

9. Brazos, Texas, V. St. 1856, kleines Stück, ähnlich dem vorigen.

10. Denton County, Texas, V. St. 1856. Kleines Stück ebenso.

11. Arva (Szlanicza) Ungarn. 1844. Sechs meistens ziemlich große Stücke, teils Platten, teils Stücke, die noch größere Teile der oxydierten natürlichen Oberfläche zeigen. Die Stücke sind interessant durch ihre ungleiche Struktur (vgl. oben S. 55). Vier derselben enthalten in der Mitte der Schalen, die bei allen von gleicher und von derselben Stärke, wie bei dem Eisen von Bohumilitz sind, eine große Menge von Körnern und Platten von Schreibersit, die nach dem Ätzen der Masse sehr glänzend hervortreten, während zwei andere deren gar nicht, dagegen den Rhabdit und diesen von einer Größe der Kristalle enthalten, wie ich sie kaum bei einem anderen Meteoreisen. gesehen habe; sie sind zuweilen über eine Linie lang und die quadratische Gestalt ihrer Durchschnitte ist mit der Lupe oft recht deutlich zu sehen. Die Ätzelinien sind bei beiden Abänderungen auf: den geätzten Flächen vorhanden, aber in beiden meistenteils nur schwach. Das Meteoreisen ist nach dem Ätzen nur matt, wenngleich auch hier bei einer bestimmten Beleuchtung bei gewissen Schalen glänzender als bei andern. An einer Platte der zweiten Abänderung findet sich eine platt-zylinderförmige, durch die Dicke der Platte hindurchgehende Masse von Troilit, die feinen Körnchen von Nickeleisen (?) eingemengt enthält, und mit einer Hülle einer auch nach dem Ätzen metallisch glänzend bleibenden, spiesgelben Substanz umgeben ist.

12. Cosby Creek, Coke County, East-Tennessee, V. St. 1840. Mehrere auf der Oberfläche oxydierte oktaedrische Bruchstücke von Hrn. Prof. Troost als Geschenk erhalten, zugleich mit einzelnen kleinen Stücken Grafit und Troilit aus demselben. Die Stücke gleichen der ersten Abänderung des Arva-Eisens außerordentlich, die Menge des eingemengten Schreibersits ist an einem Stücke noch etwas größer. An einem Stücke Grafit ist etwas Troilit eingemengt.

Verschieden von diesem Meteoreisen ist ein anderes, welches Hr. Ehrenberg ohne nähere Angabe des Fundorts als aus Tennessee von Hrn. C. T. Adae in Cincinnati Ver. St. erhalten\footnote{Vergl. Monatsber. der k. Pr. Akad. d. Wiss. von 1861, S. 517.} und dem mineralogischen Museum übergeben hatte. Den Erscheinungen der Ätzung nach würde sich dieses vielmehr den grobkörnigen Abänderungen, wie der von Seeläsgen anreihen lassen. Dies erkannte auch Hr. Shepard, als er das Stück im hiesigen Museum sah, wusste es aber doch keinem anderen bekannten Meteoreisen der Ver. Staaten zu vergleichen, daher ich dies Stück nur als Anhang hier aufführe.

13. Sarepta, Gouv. Saratow, Russland 1854. Drei Stücke, eine große Platte, die 8 Zoll lang, 3 1/2 Zoll breit und einen Zoll dick ist, ein kleines Stück und eine 2 Zoll lange und 1/4 Zoll dicke Platte. Die erste Platte ist am Rande rund herum mit natürlicher Oberfläche begrenzt und lässt hier auf der einen Seite die runde Wölbung der gefundenen Masse sehr gut erkennen.\footnote{Vergl. die Abbildung und Beschreibung derselben in den Sitzungsberichten der k. Akad. der Wiss. von 1862 B. 46. Das in der Wiener Sammlung befindliche Stück dieses Eisens, wovon Haidinger seiner Abhandlung einen so vortrefflich geratenen Abdruck hinzugefügt hat, ist die Hälfte einer ähnlichen Platte, wie die des Berliner Museums.} Die natürliche Oberfläche hat stellenweise ganz das Ansehen des Braunauer Eisens, und ist hier nur mit einer dünnen Rinde von Eisenoxyd-oxydul bedeckt, daher die Masse nicht lange Zeit in dem Boden, wenigstens nicht in einem feuchten gelegen haben kann. — Das zweite Stück hat keine, die kleine Platte an den Rändern nur zum Teil natürliche Oberfläche. Die drei Stücke zeigen aber vollkommen denselben Unterschied wie das Arva-Eisen. Während die beiden ersten Stücke der ersten Abänderung des Arva-Eisens gleichen, enthält die dritte Platte gar keinen Schreibersit, dagegen den Rhabdit in großer Menge.\footnote{Die Stücke sind indessen unzweifelhaft von derselben Masse. Dieselbe wurde in der Kalmückensteppe in der Nähe der Herrnhuterkolonie Sarepta an der Wolga gefunden, und von Hrn. Glitsch nach Moskau gesandt, wo Hr. Dr. Auerbach sie abformen und sodann zerschneiden ließ. Mehrere Stücke wurden darauf zum Verkaufe nach Herrnhut gesandt; von hier, und zwar durch Hrn. Möschler habe ich die beiden ersten Stücke, die dritte Platte von Hrn. Auerbach selbst erhalten.} Die Schalen sind bei diesem Stücke noch dicker als bei den beiden andern, und die eine Schnittfläche derselben hat ganz das Ansehen einer Schnittfläche des Seeläsgen-Eisens, und scheint hier ganz aus unregelmäßigen Zusammensetzungsstücken zu bestehen. Einzelne nach der Ätzung glänzend bleibende Körnchen, die hier eingewachsen sind, haben eine lichte speisgelbe Farbe, wie die Substanz, die bei dem Eisen von Bohumilitz den Grafit umgibt. Die Ätzelinien sind überall sehr deutlich Bei der großen Platte findet sich eine über einen halben Zoll große Troilitpartie, die aus dünnschaligen Zusammensetzungsstücken ohne merkliche Umhüllung durch eine andere Substanz besteht.

14. Sevier County, Tennessee V, St., 1840. Eine 3 1/2 Zoll lange schmale Platte. Sie enthält Schreibersit in ungefähr gleicher Menge wie die ersten Abänderungen: der beiden vorigen Meteoriten, doch sind die Schalen des Meteoreisens etwas schmäler. Die Platte wurde von Hrn. v. Reichenbach in Tausch erhalten. Shepard und Haidinger führen dieses Eisen nicht besonders auf, sondern vereinigen es mit dem Cosby-Eisen. Wesentliche Unterschiede in dem Ansehen der beiden Meteoriten finde ich nicht, doch habe ich die von Reichenbach angenommene Trennung einstweilen beibehalten.

15. Rio Bemdegó, Capitanie Bahia, Brasilien 1816. Kleines Stück, Geschenk von Al. v. Humboldt, der es von A. F. Mornay erhalten hatte. Die Schalen zeigen Ätzunglinien und Rhabdit, keinen Schreibersit.

16. Schwetz, Reg.-Bezirk Marienwerder, Preußen 1850. Das Berliner Museum, durch den Baurat Knoblauch, auf die bei dem Durchstich eines Sandhügels für die Ostbahn gefundene Eisenmasse, und deren wahrscheinlich meteorischen Ursprung aufmerksam gemacht, erhielt durch den Geh. Rath Wernich fast die ganze der etwas über 43 Pfund schweren Eisenmasse, Sie war, als sie vom Museum erhalten wurde, schon in 3 Stücke zerschlagen, wobei der Sprung natürlichen Klüften folgte. Das größte jetzt noch im Museum befindliche Stück, das noch 10 Pfund 1,21 Loth wiegt, ist zum Teil von der natürlichen Oberfläche, zum Teil von der Kluftfläche begrenzt, die an einer Seite unter spitzem Winkel zusammenstoßen. Die erstere ist sehr oxydiert, die letztere matt und schwarz, sonst aber wohl erhalten; sie ist hakig und zeigt schon die schalige Zusammensetzung der Masse sehr deutlich. Dieselbe gibt beim Ätzen sehr schöne Widmanst. Figuren, wie in dem Abdruck , den ich meiner Beschreibung dieses Eisens in Pongendorffs Ann.\footnote{B. 83, S. 594.} hinzugefügt habe, zu sehen ist. Die Querschnitte der Schalen sind darin etwas krumm, was wohl eine Folge, der beim Zerschlagen der Masse angewandten Gewalt ist. In den geätzten Schalen des Meteoreisens sieht man die Ätzelinien sowie auch den Rhabdit sehr deutlich. Troilit ist nur in kleinen Partien eingemengt. An einem Stücke der Sammlung findet sich mitten im Eisen ein kleines Korn von Chromeisenerz. Schreibersit ist nicht zu sehen.

17. Ruffs Mountain, Newberry, Süd-Carolina, Ver. St. 1850. Eine dicke Platte, an einer Seite mit natürlicher Oberfläche. Sehr merkwürdig durch die körnige Beschaffenheit der Eisenschalen, wie schon oben S. 35 bemerkt. Die körnigen Zusammensetzungsstücke einer und derselben Schale sind bei einer bestimmten Beleuchtung teils glänzend, teils matt; aber es sind nicht einzelne Zusammensetzungsstücke, die auf diese Weise miteinander abwechseln, sondern meistenteils immer wieder körnige Partien, eine schwer zu erklärende Erscheinung. Schreibersit in nicht großer Menge eingemengt.

18. Seneca River, New York, Ver. St. 1851. Kleines Stück. Die Meteoreisenschalen ebenfalls körnig, in den dickeren Schreibersit. Mit dünnen Tänitblättchen ausgefüllte Zwischenfelder häufig.

19. Toluca-Thal, Mexico 1784. Von den vielen Eisenmassen, die in dem fast 3 Meilen langen Raume des Toluca-Thales von einigen Unzen bis zu mehreren Ctrn. schwer, gefunden sind, und wahrscheinlich alle von einem Falle herrühren,\footnote{Vgl. Burkart in Leonhard u. Bronns n. Jahrb. f. Mineral. etc. von 1856, S. 297.} besitzt das Berliner Museum mehrere schöne Stücke; eine große Platte, einen Fuß lang, 7 Zoll breit und 10 Pfund 12,8 Loth wiegend, von Hrn. Shepard erhalten; eine kleinere Platte 5 1/2, Zoll lang und beinahe 4 Zoll breit, von Hrn. G. A. Stein erhalten, der die 230 Pfund schwere Masse, von der sie abgeschnitten, selbst aus Mexiko mitgebracht halte; ein überall mit Rinde umgebenes, größeres Stück von der Form einer plattgedrückten Kugel, von Hrn. Krantz erhalten, vor ihrer Teilung in 2 platte Stücke 3 Pfund 1,814 Loth schwer; ein kleineres, mehr zylinderförmiges Stück 24,93 Loth schwer, von dem k. Pr. Minister-Residenten in Washington Hrn. von Gerolt zum Geschenk erhalten, und mehrere andere kleinere Stücke, darunter eins aus der Chladnischen Sammlung, bei welchem freilich nur als Fundort Mexico angegeben war, das über den durch Ätzung erhaltenen Figuren nach hierhergehört. Alle diese Stücke verhalten sich bei der Ätzung sehr ähnlich; sie geben alle sehr schöne, überall gleiche Widmanst. Figuren und beweisen dadurch, dass sie alle einem Falle angehören. Die Schalen sind mehr als liniendick, sehr gerade und deutlich gestreift. Der Rhabdit ist darin sehr fein, in den meisten Stücken der Sammlung sieht man größtenteils nur die Querschnitte der kleinen Kristalle, was ein geflecktes Ansehen der Schalen hervorbringt. Bei gewisser Beleuchtung sind dieselben 1heils glänzend teils matt, wie besonders bei dem Stücke des Dr. Krantz, und zwar sind auch hier nach allen Richtungen laufende Schalen teils glänzend, teils matt, Schreibersit findet sich nicht, dagegen noch Partien von Troilit und Grafit. Die große Platte enthält ersteren in großer Menge und in Partien, die über 2 Zoll lang sind. Gewöhnlich kommen Troilit und Grafit miteinander verbunden vor, doch finden sie sich auch voneinander getrennt, und ebenso kommen beide mit einer Hülle einer speisgelb gefärbten, metallischen Substanz umgeben vor, finden sich aber auch ohne dieselbe. Das Steinsche Stück enthält den Troilit nur in kleineren Partien; Grafit habe ich darin nicht bemerkt. Dass in diesem Eisen zuweilen auch Quarz in sehr kleinen Kristallen vorkommt, zeigt das Stück im Besitz des Dr. Nagel in Berlin (vgl. oben S. 42), wo er allerdings nur in der oxydierten Rinde beobachtet ist, und endlich haben Wöhler und v. Reichenbach in einem von Stein erhaltenen, ursprünglich 19 Pfund wiegenden Stücke etwas Olivin beobachtet, der indessen in den Stücken der Berliner Sammlung nicht zu sehen ist. Das Krantzsche und Geroltsche Stück, welche beide rundum mit natürlicher Oberfläche begrenzt sind, zeigen auf ihr dieselben runden Vertiefungen wie das Braunauer Eisen, die Oberfläche ist indessen meistenteils schon in Eisenoxydhydrat umgeändert, und die durch Schmelzung entstandene Rinde von Magneteisenerz nur stellenweise erhalten. Dass an der Oberfläche des Toluca-Eisens Krantz kleine Oktaeder von Magneteisenerz beobachtet hat, ist oben angegeben (S. 43).

Das Toluca- Eisen ist vielfach chemisch untersucht. Die Analysen, die in Wöhlers Laboratorium nach denselben Methoden mit Teilen von sehr verschiedenen Stücken angestellt sind,\footnote{Vgl. Sitzungsb. der math.-naturw. Kl. d. k. Akad. d. Wiss. von 1856 B. 20, S. 217 etc.} haben kein völlig übereinstimmendes Resultat gegeben; die Analysen von Uricoechea und Pugh z. B. Abweichungen im Nickelgehalt von 5,02 bis 9,05 pC., doch bemerkt darüber schon Wöhler, dass bei der gemengten Beschaffenheit des Meteoreisens, wovon immer nur ein sehr kleiner Teil untersucht werden könnte, dies kein Beweis wäre, dass die analysierten Stücke von verschiedenen Meteoriten berührten. Die Übereinstimmung der Widmanst, Figuren beweist in diesem Fall besser den gleichen Ursprung aller dieser Massen als. die chemische Analyse. Sind aber die vielen in dem Toluca-Thal gefundenen Eisenmassen eines und desselben Ursprungs, so gehören sie einem der bedeutendsten Meteoritenfälle an, von denen man Kunde hat. Bemerkenswert ist, dass Pugh nur in der oxydierten Rinde, die er auch untersucht hat, Kieselsäure, und zwar 7,47 pC. gefunden hat, In dem Rückstande von Teilen der vom Dr. Stein mitgebrachten Stücke fanden Uricoechea und Pugh auch Körner von Olivin und einigen anderen nicht bestimmten rubinroten und himmelblauen Mineralien. Dergleichen fand auch Böking, ohne sie näher zu bestimmen, und außerdem noch etwas Grafit.

20. Misteca, im Staate Oajaca, Mexico, 1834. Ein parallelepipedisches Stück von 2 Pfd. 13,7 Lth., Abschnitt von einem, dem Geh. Bergrath Burkart in Bonn gehörigen Stücke.\footnote{Vgl. Burkart a. a. O. S. 305.} Es ist von zwei Schnittflächen, einer Bruchfläche und im Übrigen von natürlicher Oberfläche begrenzt. Die Schliffflächen zeigen sehr schöne Widmanst, Figuren, die denen des Toluca-Eisens sehr ähnlich, doch dadurch ausgezeichnet sind, dass auf ihnen die eingewachsenen Rhabdit-Kristalle viel größer und zahlreicher sind. Die Ätzunglinien sind sehr schwach und nur an manchen Stellen sichtbar. Troilit immer in sehr kleinen Partien, doch findet sich auf der einen Schnittfläche eine kleine, krumm zylinderförmige Höhlung, worin Troilit gesessen zu haben scheint. Eine größere runde Rinne auf der Oberfläche war vielleicht früher auch damit gefüllt. Die Bruchfläche ist schwach angelaufen, doch sind darauf die Schalen des Meteoreisens noch gut zu sehen. Die natürliche Oberfläche ist an einer Stelle durch Hammerschläge verletzt, an andern zeigt sie die dünne Rinde von Magneteisenerz noch deutlich.

21. Sierra blanca, unweit Villa nueva de Huajuquilla in Mexiko, 1784; ein Stück aus der Sammlung des Medizinal-Raths Bergemann,\footnote{Vergl. Burkart a. a. O. S. 278.} 15,28 Loth schwer. Die Widmanst. Figuren sind denen des Toluca-Eisens sehr ähnlich, die Ätzunglinien deutlicher, der Rhabdit undeutlicher, kein Schreibersit.

22. Tula (Netschaevo) Russland 1856. Ein drei Zoll langes, ziemlich quadratisches Prisma mit lauter Schnittflächen bis auf eine der Endflächen, die natürliche Oberfläche ist; von Hrn. Dr. Auerbach in Moskau erhalten. Die Widmanst. Fig. sehr schön, die Schalen so dick wie im Toluca-Eisen, die Ätzunglinien fein, Rhabdit nicht recht sichtbar. Auf einer der Seiten des Prismas sieht man den Durchschnitt einer Einmengung, die sich von einem Ende, fast 5 Linien breit, bis fast zum andern fortzieht, wo sie sich auskeilt, Sie besteht aus einem Gemenge von Nickeleisen mit einem harten, schwarzen, matten und undurchsichtigen Silicate. Haidinger\footnote{Sitzungsberichte der math.-naturw. Kl. d. k. Akad. d. Wiss. von 1860.} hat von dem Stücke der Wiener Sammlung, das mehrere dergleichen Einmengungen enthält, einen Abdruck gegeben, und Auerbach\footnote{Poggendorffs Ann. von 1863, B. 118, S. 363.} hat neuerdings das Silicat analysiert. Er fand darin:

Eisenoxydul 31,64  
Magnesia 16,63  
Kalk 0,77  
Natron 1,22  
Kali 0,20  
Tonerde 9,93  
Kieselsäure 37,76  

hält es aber selbst noch für ein Gemenge, da es sich von Chlorwasserstoffsäure nur zum Teil zersetzen lässt.

Was es mit dieser Einmengung für eine Bewandtnis hat, ist schwer zu sagen. Dergleichen Einmengungen sind bei andern Eisenmeteoriten noch nicht vorgekommen. Da die ursprünglich an 600 russische Pfunde schwere Masse, um sie zu zerkleinern, in ein Schmiedefeuer gebracht worden ist, so könnte man glauben, dass die Einmengung sich erst durch die Behandlung im Feuer gebildet hat, doch ist das Gemenge des Silicats und des Eisens in demselben so fein, die chemische Zusammensetzung des Silicats durch den großen Magnesiagehalt so verschieden von den gewöhnlichen Eisenschlacken, die Widmanst. Fig. in dem Eisen sind so regelmäßig, dass diese Annahme doch ihre Schwierigkeit hat.

23. Robertson County, Tennessee, V. St. 1861. Kleines Stück, die Widmanst. Fig. ähnlich denen des Tula-Eisens.

24. Carthago, Smith County, Tennessee, V. St. 1844. Ein großes, 1 Pfd. 16,44 Lth. schweres Stück, in Form eines dreiseitigen Prismas, von Hrn. von Reichenbach erhalten. Zwei Seitenflächen sind Schnittflächen, die dritte eine natürliche Kluftfläche. Die Widmanst. Fig. sind sehr schön, ähnlich denen des Toluca-Eisens. Troilit ist in kleinen Partien eingemengt. Er hat stets eine kleine Einfassung von Meteoreisen, jenseits welcher erst die Widmanst. Fig. anfangen. In der einen Partie von Troilit ist ein kleines Korn von Chromeisenerz eingemengt (s. oben S. 39).

25. Burlington, Otsego County, New York, Ver. St. 1819. Mehrere kleine Stücke, Widmanst. Fig. wie Toluca.

26. Marshall County, Kentucky, V. St. 1856, schöne Widmanst. Fig.; darin ziemlich viel Schreibersit.

27. St. Rosa bei Tunja, Columbien, ein ganz kleines Stück, 0,025 Loth schwer, das von Prof. Karsten aus Columbien mitgebracht ist. So klein es ist, so konnte ich doch noch eine Fläche anschleifen lassen und ätzen, und daran ganz deutliche Widmanst. Fig. erkennen.

28. Orange Fluss, Süd-Afrika, 1856, dünne Platte, ähnlich dem von Marshall Cty, auch mit etwas Schreibersit.

29. Texas (Red River), Ver. St. 1814. Eine 4 Zoll lange, dünne Platte. Geschenk von Hrn. B. Silliman. Schöne Widmanst. Figuren. Die Schalen etwas schmäler als bei Toluca. Ein wohlgeratener Abdruck von einer geätzten Platte des Texas-Eisens befindet sich in Sillimans Journal.

30. Lenarto, Scharosch, Ungarn, 1815. Dicke der Schalen wie bei Texas; sehr feine Ätzlinien, in einigen Stücken ziemlich viel Schreibersit, in andern weniger.

31. Durango, Mexico, 1804. Ein größeres Stück, 1 Pfd. 2,67 Lth. schwer, und mehrere kleinere, Geschenk von Al. v. Humboldt. Die Widmanst. Fig. noch schmäler als bei dem Texas-Eisen. Troilit in kleinen Partien eingemengt.

32. Werchne-Udinsk, West-Sibirien, 1854. Eine dicke Platte, 1 Pfd. 4,17 Lth. schwer, an den Rändern bis auf eine Schnittfläche mit natürlicher Oberfläche begrenzt. Letztere ist uneben und mit nur sehr dünner Rinde von Magneteisenerz bedeckt, das in kleinen Vertiefungen, überall aber undeutlich kristallisiert ist. Die Hauptflächen zeigen geätzt sehr schöne Widmanst. Fig. Troilit in kleinen Partien hier und da eingeschlossen.

33. Elbogen, Böhmen, 1811. Ein 9,92 Lth. schweres Stück, eine dünne Platte und ein aus der Masse geschmiedetes Federmesser; sämtliche Stücke aus der Chladnischen Sammlung. Die Widmanst. Fig. desselben sind aus dem schönen Abdruck in die Werke des Hrn. von Schreibers bekannt.

34. Nebraska Territory, Ver. St. 1856.

35. Madoc, Ober-Canada 1854.

36. Black mountain, Buncumbe Cty, Nord-Carolina, V. St. 1835.

37. Caille, Grasse, Var, Frankreich, 1828. Eine drei Zoll lange, 5,69 Lth. schwere Platte, Die Widmanst. Fig. sehr geradlinig und schön.

38. Agram (Hraschina), Kroatien, gefallen den 26. Mai 1751.

39. Ashville, Buncumbe County, Nord-Carolina, Ver. St. 1839. Schöne Widmanst. Fig., aber sehr schnell rostend.

40. Guildford, Nord-Carolina, Ver. St. 1830.

41. Löwenfluss, Groß Namaqualand, Süd-Afrika 1853.

42. Lockport (Cambria), New-York, V. St. 1845. Eine fast zwei Zoll lange Platte. Feine Widmanst. Fig. mit vielen rundlichen, haselnussgroßsen Partien von Troilit, die oft zusammengehäuft, aber stets mit einer dünnen Hülle einer lichten speisgelben Legierung umgeben sind.

43. Jewell Hill, Madison County, Nord-Carolina, Ver. St. 1856. Eine ähnliche, nur dickere Platte: ebenfalls sehr feine Widmanst. Fig.

44. Oldham County (Lagrange), Kentucky, Ver. St. 1856. Zwei große, dicke Platten, an den Rändern zum Teil mit natürlicher Oberfläche begrenzt, und eine kleinere Platte. Die Widmanst. Fig. sehr fein.

45. Putnam County, Georgia, V. St. 1854. Die Widmanst. Fig. sehr gerade, fein und zierlich. In dem kleinen Stücke findet sich ein Einschluss von Troilit, in welche kleinen Partien von Nickeleisen eingemengt sind.

46. Tazewell, Claiborne County, Tennessee, Ver. St. 1854. Die Widmanst. Fig. sehr geradlinig und in ihnen dünne Streifen von Schreibersit, die sich aber merkwürdiger Weise nur in den Schalen einer Richtung finden, was das Meteoreisen von Tazewell vor allen übrigen Abänderungen auszeichnet. Auch hier kleine Einmengungen von Troilit, der mit einer speisgelben metallischen Einfassung umgeben ist.

d. Meteoreisenmassen, die Aggregate großkörniger, schalig zusammengesetzter Individuen sind.

47. Zacatecas, Mexico, 1792. Ein Meteoreisen von sehr eigentümlicher Struktur, die nur in einem größeren Stücke erkannt werden kann. Das Berliner Museum besitzt davon neben mehreren kleinen Stücken eine ungefähr rechtwinklige zolldicke Platte, die ungefähr 3 Zoll breit und 3 1/2 Zoll lang, von einem größeren Stücke abgeschnitten ist, das der Geh. Bergrath Burkart aus Mexico selbst mitgebracht hat.\footnote{Vergl. N. Jahrbuch für Min. etc. von Leonhard und Bronn von 1856, S. 288.} Die Seiten der Platte werden zum Teil durch natürliche Oberfläche gebildet. Auf der geätzten Hauptfläche, die in Fig. 1 Taf. II dargestellt ist, sieht man, dass dies Meteoreisen aus grobkörnigen Zusammensetzungsstücken besteht, die über zollgroß und unregelmäßig begrenzt sind, und selbst wieder aus Zusammensetzungsstücken bestehen, die parallel den Flächen des Oktaeders liegen, wie bei dem Meteoreisen, das Widmanst. Fig. gibt. Diese schaligen Zusammensetzungsstücke sind auch nicht sehr regelmäßig begrenzt, ihre Richtung ist aber doch sehr gerade, was man aus dem eingemengten Schreibersit sehen kann, der in der Mitte derselben enthalten ist, und auf der Schnittfläche oft fast zusammenhängende, wenn auch meistenteils sehr dünne Streifen bildet. Unter dem Mikroskop sieht man jedoch, dass diese Streifen stets aus einzelnen Stücken bestehen, die in einer oder mehreren Reihen nebeneinander liegen. Die einzelnen Stücke sind zum Teil regelmäßig begrenzt und liegen in paralleler Stellung nebeneinander, sind also unvollkommen ausgebildete Kristalle. Fig. 2 stellt die Stelle bei \emph{a} (Fig. 1), Fig. 3 bei \emph{b} (Fig. 1) bei 90-maliger Vergrößerung in einen der größten Zusammensetzungsstücke der Platte dar. In Fig. 1 sind die Schalen, woraus die Zusammensetzungsstücke bestehen, nur unvollständig wiedergegeben, da sie sehr fein sind, der Schreibersit ist aber möglichst vollständig dargestellt. Die Schalen zeigen feine, doch deutliche Ätzlinien; eingewachsene Kristalle von Rhabdit kommen nicht vor, wenn auch der eingemengte Schreibersit in manchen Zusammensetzungsstücken so häufig und fein ist, dass man ihn leicht damit verwechseln kann. Auch hier sieht man bei einer bestimmten Beleuchtung einen Teil der schaligen Zusammensetzungsstücke glänzen, einen andern nicht, der dann bei anderer Beleuchtung glänzt, während es nun bei dem ersten nicht der Fall ist. Troilit in kleinen unregelmäßigen Partien ist häufig eingemengt und in den Figuren durch schwarze Farbe bezeichnet; er hat häufig eine dünne, metallisch glänzende Einfassung, die, soweit man es erkennen kann, von dem Schreibersit nickt verschieden erscheint; auch kleine Partien von Grafit kommen darin vor. Auf der natürlichen Oberfläche sieht man zwei runde, rinnenförmige Eindrücke, der eine davon 3 Zoll lang und 3/8 Zoll breit, die Baron Reichenbach, der sie hier gesehen, für Höhlungen hält, die durch ausgewitterten Troilit entstanden sind.

e. Meteoreisenmassen, welche Aggregate feinkörniger Zusammensetzungstücke sind.

Diese Eisenmassen erscheinen im Bruche fein- bis kleinkörnig, zeigen geschliffen und geätzt keine Widmanst. Fig., doch treten dann nadel- oder tafelförmige Kristalle von Rhabdit oder Schreibersit hervor, die gewöhnlich gar keine regelmäßige Lage haben. Troilit ist in einzelnen größeren oder kleineren Partien eingemengt.

48. St. Rosa (Tocavita) bei Tunga, 20 spanische Meilen NO. von Bogota in Columbien 1823.

49. Rasgatà, bei der Saline Zipaquira bei Bogota, Columbien 1823. Ich führe beide Eisenmassen zusammen auf, weil beide nach den Stücken des Berliner Museums die größte Ähnlichkeit miteinander haben. Das St. Rosa-Eisen wurde mir 1824 bei meinen Aufenthalten in Paris von Al. von Humboldt, der sie selbst von Boussingault erhalten, für das Berliner Museum übergeben. Es waren ursprünglich 2 vollständige kleine Eisenmassen von platt-kugelförmiger Gestalt, wenn auch mit vielen flachen Vertiefungen auf der Oberfläche. Von beiden wurden später einige Stücke abgeschnitten; sie wiegen jetzt noch 35,26 und 29,93 Loth; ein kleines von dem einen abgeschnittenes Stück ist 3,04 Loth schwer. Die Oberfläche derselben ist sehr oxydiert, und das Eisen hier in Eisenoxydhydrat umgeändert. Das Eisen selbst ist außerordentlich hart, im Bruche feinkörnig und nimmt geschliffen, eine sehr gute Politur an. Geätzt wird es im Allgemeinen matt, des eine Stück etwas fleckig; man sieht nur mit der Lupe kleine runde oder vielmehr noch in die Länge gezogene Erhabenheit, die auf ihrer Höhe noch kleinere runde, längliche, oft ganz linienartige, glänzend gebliebene Teile zeigen.

Von dem Eisen von Rasgatà besitzt das Museum eine 4,79 Loth schwere Platte, die an den Seiten von 2 Schnittflächen, im Übrigen von der natürlichen Oberfläche begrenzt ist. Das Berliner Museum hatte sie vom Direktor Partsch aus dem Wiener Kabinette erhalten, wo es von einem Stücke abgeschnitten war, das aus der Meteoritensammlung von Heuland stammt, der es selbst von Mariano de Rivero erhalten hatte.\footnote{Partsch Meteoriten, S. 127.} Die geätzte Schnittfläche gleicht denen der vorigen Stücke; die kleinen glänzend gebliebenen Teile sind vielleicht noch häufiger und vorzugsweise linienartig, zum Teil auch untereinander parallel. Sie sind meistenteils sehr fein, liegen aber wie bei den vorigen Stücken auf schon etwas erhabenen Teilen der Grundmasse.

Die Stücke stammen demnach sämtlich von Boussingault und Mariano de Rivero, weiche beide zusammen an den Fundörtern derselben waren und darüber die erste Nachricht gegeben haben.\footnote{Ann. de Chimie 1824, t. 25, p. 438.} Sie sahen in St. Rosa bei Tunja, 20 spanische (?) Meilen NO. von Bogota, eine große Eisenmasse, deren Gewicht sie auf 750 Kilogramme schätzten, bei einer Schmiede, der sich ihrer als Amboss bediente. Dieselbe hatte sich auf einem Hügel Tocavita, 1/4 Meile von St. Rosa, mit anderen kleineren Stücken in der Nähe gefunden. Andere Eisenmassen sahen sie in dem Dorfe Rasgatà in der Nähe der Saline Zipaquira bei Bogota, darunter Massen von 41 und 22 Kilogramme.

Das von Prof. Karsten mitgebrachte und oben (S. 64) erwähnte Eisen von Santa Rosa ist ein Stück der großen Eisenmasse, die bei der Schmiede liegt. Prof. Karsten ist zwar nicht selbst in Santa Rosa gewesen, sondern hatte die Stücke von einem Bewohner Santa Rosas erhalten, den er in Bogota kennen gelernt hatte. Da demselben das Vorhandensein der Eisenmasse bei der Schmiede wohl bekannt war, so hatte Prof. Karsten ihm eine Metallsäge mitgegeben, um mittelst derselben ein Stückchen von der Eisenmasse abzulösen und ihm dasselbe bei seiner Rückkehr nach Bogota zu bringen, was er nun auch getan hatte, wenn auch das Stück bei der Härte der großen runden Eisenmasse, der er nicht recht beikommen konnte, nur sehr klein ausgefallen ist. Prof. Karsten hatte aber keinen Zweifel an der Echtheit des Stückes; da es nun aber ganz anderer Art ist, als die Stücke von St. Rosa, die Boussingault an Humboldt gegeben, so müsste man annehmen, dass die verschiedenen auf dem Hügel Tocavita bei St. Rosa gefundenen Eisenmassen eine so große Verschiedenheit der Struktur gezeigt haben. Bei der Ähnlichkeit der Boussingaultschen Stücke mit denen von Rasgatà könnte man an eine Verwechselung der Fundorte denken, die indessen anzunehmen ich keine Berechtigung habe. Vielleicht könnte noch Hr. Boussingault selbst darüber Auskunft geben. Auch würde es wünschenswert sein, noch andere Beschreibungen der Stücke von St. Rosa und Rasgatà zu erhalten. — Dass Wöhler bei der Auflösung des Eisens von Rasgatà eine Menge kleiner mikroskopischer harter Steinchen als Rückstand neben dem Schreibersit erhielt, ist schon oben angeführt (vgl. oben S. 41).

50. Chesterville, Süd-Carolina, Ver. St. 1849. Drei Stücke, eins davon in zwei Stücke zerbrochen, ursprünglich 14,93 Loth, ein zweites 8,92 Loth schwer, alle tafelartig geschnitten, eins noch zum Teil mit natürlicher Rinde, die schwarz, dünn und uneben ist. Das zerbrochene Stück zeigt einen feinkörnigen Bruch mit stahlgrauer, ins Bleigraue sich ziehender Farbe. In der feinkörnigen Masse liegen etwas größere, glänzende Körner. Die geätzte Schnittfläche ist matt, mit der Lupe sichte man darauf lauter kleine rundliche Erhabenheit, und dazwischen einzelne unregelmäßig gestaltete, oder runde, auch ganz geradlinige, glänzende Teile. Bei dem durchgebrochenen Stücke eins der Bruch gerade durch ein darin befindliches Stück Troilit von Haselnussgroßsen, so dick wie die Platte selbst; derselbe ist braun, feinkörnig und wird von einer dünnen Hülle glänzender Körner umgeben, die etwas größer sind als die übrigen glänzenden Teile der Masse.

51. Tucuman (Otumpa), Argent. Rep. in Süd-Amerika, 1788. Das vierte der oben S. 51 erwähnten Stücke mit feinkörnigem Bruch.

52. Senegal im Lande Siratik und Bambuk, Afrika, 1763. Ein 4,462 Loth schweres Stück, zum Teil mit zwei Schnittflächen und einer Bruchfläche, zum Teil mit der natürlichen Oberfläche begrenzt. Letztere ist uneben und schwarz, die Bruchfläche ist etwas gröber körnig als bei Chesterville; die geätzte Schnittfläche zeigt auch die feine rundliche Erhabenheit von Chesterville, außerdem aber andere dünne, geradlinige, oft 2 Linien lange, die unregelmäßig durcheinanderlaufen, ohne sich zu schneiden, aber nur schwach glänzen, etwa wie bei Chesterville die erhabenen Teile in der nächsten Umgebung der stark glänzenden.

53. Salt River, Kentucky, V. St. 1851. Eine dünne, quadratzollgroße, 1,197 Loth schwere Platte. Graue matte Grundmasse, worin häufige lichtere längliche Teile schon etwas regelmäßig nach den Seiten eines ungefähr gleichseitigen Dreiecks liegen, in deren Mitte sich glänzende Teile befinden, die auch meistenteils von einer länglichen Form sind. Die eine Hälfte der geätzten Schnittfläche ist bei einer bestimmten Beleuchtung von lichter grauer Farbe, die andere dunkler; bei anderer Beleuchtung umgekehrt, die erste Hälfte dunkel und die andere licht. Das Eisen ist dem vom Senegal zu vergleichen, unterscheidet sich aber durch die regelmäßige Lage der lichtern Teile, die sehr rätselhaft ist.\footnote{Um die bei diesen wie den vorigen Meteoriten angegebenen Erscheinungen zu sehen, ist es nötig, dass die Schnittfläche gut poliert und dann nur schwach geätzt wird; bei starker Ätzung bildet sich nur eine körnige, graue Fläche, an der nichts weiter zu erkennen ist.} Es wäre wichtig, den Bruch zu sehen, der bei der dünnen Platte nicht zu beobachten ist.

54. Cap der guten Hoffnung (zwischen Sonntags- und Boschmanns-Fluss), S. A. 1801. Eine dicke Platte (Fig. 9 und 10 Taf. III) von Hrn. Prof. Breda in Harlem erhalten, und ein kleines Stück aus der Chladnischen Sammlung. Die Platte ist ein Abschnitt der großen, in dem Museum von Harlem aufbewahrten, angeblich 171 Pfd. schweren Masse.\footnote{Vergl. Partsch Meteoriten, S. 132.} Sie ist an einer Seite mit einer Schnittfläche, an den übrigen mit natürlicher Oberfläche begrenzt, die nur eine sehr dünne braune Rinde hat. Um den Bruch zu sehen, wurde, parallel der Fläche \emph{AF}, an der gegenüberliegenden Seite der Platte eine Scheibe abgeschnitten und zum Teil abgebrochen, was nicht ohne Schwierigkeiten ausgeführt werden konnte, da das Eisen sehr weich und sehr dehnbar ist, und darin einen großen Gegensatz mit dem Eisen von St. Rosa (S. 67) bildet. Es musste von zwei Seiten tief eingeschnitten werden, ehe sich die dünne Scheibe abbrechen ließ, so dass nur ein 3 Linien dicker Streifen mit Bruch entstand, an dem man jedoch deutlich seine Beschaffenheit wahrnehmen kann. Er ist ganz feinkörnig und lichte stahlgrau. Das übrig gebliebene, große Stück war wohl an einigen Stellen der Oberfläche angerostet, zeigte aber sonst keine merkliche Verschiedenheit. Als die Schnittflächen \emph{AF} und \emph{AC} der Platte geschliffen und poliert wurden, rosteten sie in sehr kurzer Zeit an den 3 Stellen \emph{B}, \emph{D}, \emph{G}, wie in der Fig. angegeben ist, und als sie darauf schwach geätzt wurden, zeigte sich der Rost auf denselben Stellen sehr bald wieder, ohne aber später merklich weiter fortzuschreiten; die übrigen Teile der Flächen wurden aber dabei merkwürdig verändert. Man sieht nun verschiedene, abwechselnd lichte und dunkelstahlgraue, mehr oder weniger breite Streifen in paralleler Richtung und mit scharfer Grenze über dieselben fortlaufen, die ihren Farbenton umtauschen, je nachdem das Licht in der einen oder die andere Richtung auf das Stück fällt. Hält man dasselbe so, dass die Kante \emph{AB} dem Beobachter zugekehrt und ungefähr horizontal ist, so erscheinen die Streifen so, wie sie in Fig. 9 angegeben sind: \emph{a}, \emph{c}, \emph{e}, \emph{g} sind hell, mit Ausnahme einiger feiner, dunkler Streifen in \emph{e} und \emph{g}; \emph{a}, \emph{d}, \emph{f}, \emph{h} dunkel, mit Ausnahme des hellen, feinen Streifen in \emph{d}. Hält man dagegen die Fläche \emph{AC} so, dass die Streifen dem Beobachter parallel gehen, wie in Fig. 10, so sind die Streifen, die früher hell waren, dunkel; dabei erscheint diese Fläche, wie auch schon in der früheren Lage, gefleckt und auf diesen Flecken teilweise mit feinen Streifen, die die breiten, schräg durchsetzen, gestreift. Hält man die Fläche so, dass sie etwas nach hinten geneigt ist, so ist die Stelle bei \emph{B}, wie in der Zeichnung angegeben, hell; ist die Fläche \emph{AC} ganz horizontal, so wird die Ecke bei \emph{B} dunkel, ist sie nach vorn geneigt, wieder hell.

Unebenheiten, die diese Veränderungen in dem Ton der Streifen bedingen könnten, sind mit den bloßen Augen und auch kaum mit den Mikroskopen wahrzunehmen. Ich habe von der Fläche AC einen Hausenblasenabdruck gemacht; unter dem Mikroskop erschien der lichte Streifen \emph{h} Fig. 10 ganz gekörnt, wie in Fig. 11; in dem dunklen \emph{g} waren diese Körner ebenfalls, aber mit andern gemengt, die in die Länge gezogen waren. Die die breiten Streifen durchsetzenden feinen Streifen entstehen dadurch, dass die Körner hier gedrängter liegen. Wie dadurch aber der Wechsel von hell und dunkel in den Streifen bewirkt wird, ist nicht einzusehen. Es müssten dazu noch weitere Untersuchungen angestellt, und namentlich noch Schnitte parallel und rechtwinklig auf den breiten Streifen gemacht, auch nachgesehen werden, ob die Lagen, die durch die Streifen auf den Schnittflächen des Stückes Fig. 9 angezeigt werden, durch die ganze Masse des großen Stückes in dem Museum von Harlem, wovon das beschriebene abgeschnitten ist, hindurchgehen.

Die chemische Beschaffenheit dieses so eigentümlichen Eisens ist schon mehrfach untersucht, zuletzt noch durch Uricoechea und Böking\footnote{Ann. d. Chem. und Pharm. B. 91, S. 252 und B. 95, S. 246.} in Wöhlers Laboratorium, wodurch der große Nickelgehalt desselben bis 15 pC., dem schon Holger und Wehrle gefunden hatten, bestätigt wurde. Als Wöhler im Mai d. J. in dem Berliner Museum das geätzte Stück sah, veranlasste ihn das eigentümliche Ansehen desselben, noch eine neue Analyse zu machen, mit Stücken, die ich ihm mitteilen konnte. Er fand darin:\footnote{Nach einer brieflichen Mitteilung, die ich mit seiner Erlaubnis hier bekannt mache.}

Nickel 16,215  
Kobalt 0,727  
Phosphor 0,148  

außerdem noch Spuren von Kupfer und Chrom, welches letztere auch schon Stromeyer darin nachgewiesen hatte. Doch ist die Untersuchung noch nicht abgeschlossen. Das schnelle Rosten an manchen Stellen setzt hier doch eine besondere Beschaffenheit voraus. — Noch ist zu bemerken, dass man in dem Streifen \emph{g} der Fläche \emph{AF}, was man so selten zu beobachten Gelegenheit hat, den sechsseitigen Durchschnitt eines Kristalls von Troilit, so wie in dem untern Streifen \emph{h} etwas Schreibersit von der in der Zeichnung angegebenen Gestalt sieht.

55. Babbs Mill, Greenville, Green County, Tennessee, Ver. St. 1845. Zwei kleine flache Stücke; bei dem größeren zwei Schnittflächen, die übrige Begrenzung natürliche Oberfläche, die in Eisenoxydhydrat umgeändert ist. Die große Schnittfläche ist geätzt, matt, die eine Hälfte dunkelgrau, die andere viel heller. Beide Schattierungen verlaufen ineinander, wie dies auch bei den Flecken in dem Eisen vom Cap vorkommt; ebenso derselbe Wechsel des Farbentons bei dem Wechsel in der Lage des Stücks; aber die geradlinigen Streifen sind in dem kleinen Stücke nicht sichtbar; ein Bruch auch nicht wahrnehmbar. In dem kleineren Stücke sieht man glänzende Einmengungen in Gestalt von gebogenen Linien, die bei dem größeren nicht sichtbar sind.

Wie in dem Äußern Ansehen, so gleicht dies Eisen auch dem vom Cap in der chemischen Beschaffenheit. Es hat ebenfalls einen sehr hohen Nickelgehalt, der nach der von Clark in Wöhlers Laboratorium angestellten Analyse sogar noch etwas höher ist als bei dem Cap-Eisen. Er fand nämlich Eisen 80,590, Nickel 17,104, Kobalt 2,037, schwerlösliche Phosphormetalle 0,124; außerdem noch Spuren von Mangan, Silicium und Magnesium.

56. De Kalb County, Tennessee, V. St. 1845. Zwei kleine Stückchen, wonach dies Eisen dem der beiden vorigen Fundörter sehr ähnlich zu sein scheint.

Anhang.

57. Tucson, Sonora, Mexico, 1850. Kleines Stückchen.

58. Cranburne, Melbourne, Australien, 1861. Drei kleine Stückchen.
\subsection{Pallasit.}
\paragraph{}
Gemenge von Meteoreisen mit Olivin. Das Meteoreisen bildet hier eine Grundmasse, in welcher Olivin-Kristalle porphyrartig eingewachsen sind.\footnote{Ästig und schwammig, wie es gewöhnlich beschrieben wird, erscheint es nur da, wo die Olivin-Kristalle herausgefallen sind, was bei der Trennung kleinerer Stücke von Größeren mit dem Hammer häufig der Fall ist.} Es gehören hierher die Eisenmeteorit von Krasnojarsk (das Pallas-Eisen), von Brahin, Atacama, Steinbach, Rittersgrün, Breitenbach\footnote{Den Eisenmeteorit von Breitenbach stelle ich vorläufig hierher, da das darin enthaltene Meteoreisen dieselben Widmanst. Fig. gibt, wie das in den Pallasiten von Steinbach und Rittersgrün, ohne aber die neben dem Meteoreisen vorkommenden Gemengteil schon untersucht zu haben.} und Bitburg.

Die Olivin-Kristalle sind bei dem Pallas-Eisen am schönsten ausgebildet. Sie sind hier 2 bis 4 Linien groß und zuweilen noch grösser, und liegen entweder ganz frei in dem Eisen oder zu mehreren nebeneinander, sich gegenseitig in der Ausbildung störend. Im ersteren Falle sind sie rund und nähern sich der Kugelgestalt mehr oder weniger; oft sind sie in die Länge gezogen und an einem Ende konisch zulaufend, wie man an ihren Durchschnitten auf den Schnittflächen des Eisens, oder an den bei der Lostrennung einzelner Stücke von der größeren Masse mit dem Hammer herausgefallenen Kristallen sehen kann. Ihre Oberfläche ist aber glatt und stark glänzend. Sie sind gelblichgrün und vollkommen durchsichtig, so dass man auf der geschliffenen Fläche des Pallasit auch bei den Durchschnitten größerer Kristalle die hintere Seite deutlich erkennen kann; indessen sind sie doch häufig mit Sprüngen durchsetzt, und auf diesen und in der Nähe derselben braun gefärbt, und dann mehr oder weniger undurchsichtig. Ungeachtet ihrer Abrundung zeigen sie aber noch einzelne Flächen, die sich gewöhnlich nicht in Kanten schneiden und runde Umrisse haben, aber an den Winkeln, die sie miteinander bilden, zu bestimmen sind. Am häufigsten fand ich die Flächen des Längsprisma \emph{k} (80° 54’) und die Zone der Längsprismen ausgebildet, wie z. B. in Fig. 1 Taf. IV, wo diese Zone rund um den Kristall zu verfolgen ist und sich auf der einen Seite die Flächen \emph{P}, \emph{h}, \emph{k}, \emph{i}, \emph{T}, \emph{i}, \emph{k} (Haüy), auf der entgegengesetzten Seite die Flächen \emph{h}, \emph{T} befinden; er ist in der den Flächen \emph{P} und \emph{T} parallelen Axe besonders ausgedehnt.\footnote{Bei \emph{d} befindet sich keine Kristall- sondern eine Zusammensetzungsfläche, in welcher der Kristall an einen andern angrenzte.} An andern Kristallen fand ich aber auch die Flächen anderer Zonen; ja ich fand sogar einen Kristall, den ich schon bei einer früheren Gelegenheit beschrieben,\footnote{Pongendorffs Ann. 1825 B. 4, S. 185. Er ist indessen dort ganz vollständig gezeichnet, obgleich die Flächen nur auf einer Seite ausgebildet sind; über \emph{T} die Flächen \emph{i}, \emph{k}, \emph{P} (letztere sehr groß), unter \emph{T}: \emph{i'}; neben \emph{T}: \emph{r}, \emph{s}, \emph{n}, über diesen \emph{l}, \emph{f}, \emph{e}, \emph{d}, unter ihnen \emph{l'}, \emph{f'}, \emph{e'}.} bei welchem mehrere Reihen von Flächen übereinander vorkommen. Wo sich mehrere Flächen finden oder dieselben schon einige Größe haben, schneiden sie sich auch öfter in scharfen Kanten. So findet sich bei einem Stücke der Sammlung ein in dem Eisen festsitzender Kristall, an welchem 2 Flächen sichtbar sind, die eine 3 Linien lange Kante bilden. Die Flächen sind größtenteils überaus glatt und glänzend, so dass sich die Kristalle zu den schärfsten Messungen eignen, die man bei dem Olivin anstellen kann. Nur die gerade Endfläche \emph{P} ist bei dem oben erwähnten, sehr ausgebildeten Kristalle parallel der Kante mit dem Längsprisma gefurcht, und Flächen mit solchen Furchen, die also auch wahrscheinlich die Flächen \emph{P} sind, habe ich auf ansitzenden Kristallen noch öfter bemerkt. Seiten sind die Kristalle um und um ausgebildet, gewöhnlich liegen zwei oder mehrere Kristalle nebeneinander. Sie begrenzen sich dann mit Zusammensetzungsflächen, die sich von den Kristallflächen gleich dadurch unterscheiden, dass sie immer etwas uneben und bei weitem nicht so glänzend wie jene sind.\footnote{Sie haben öfter eine fünfseitige Gestalt, wir in dem Stücke, welches Chladni beschreibt, wo drei solcher fünfseitigen Flächen zusammenstoßen, „so dass der Kristall einem Pentagondodekaeder sehr ähnlich ist," wie Chladni sagt, diese Zusammensetzungsflächen mit Kristallflächen verwechselnd. (S. dessen Feuermeteore, S. 322.)} Zuweilen sind die Kristalle nur durch eine schmale Lage von Eisen oder auch Troilit voneinander getrennt.

Betrachtet man die Kristalle mit einer Lupe, so sieht man häufig in ihnen ganz feine, haarförmige Einschlüsse, die ganz geradlinig und untereinander parallel, mehr oder weniger lang in verschiedenen Höhen des Kristalls liegen, und öfter Farben spielen. Besser erkennt man diese Einschlüsse noch, wenn man die Kristalle in dünn geschliffenen Platten unter dem Mikroskop betrachtet, wo sie bei 140-maliger Vergrößerung wie in Fig. 10 Taf. I erscheinen.\footnote{Diese Figur ist die Vergrößerung der kleinen, rechts liegenden hellen Stelle in der Platte Fig. 11, die aus einem sehr klüftigen Olivin-Kristall des Pallas-Eisens geschliffen, und in natürlicher Größe dargestellt ist.} Sie machen im Allgemeinen den Eindruck von Röhren, haben aber untereinander eine etwas verschiedene Beschaffenheit und erscheinen bei 360-maliger Vergrößerung, wie in der Fig. 2 Taf. IV dargestellt ist. Am häufigsten erscheinen sie, wie in \emph{a} Fig. 2, als zwei nebeneinander liegende, gerade Linien, dann sieht man in der Mitte dieser eine stärkere und schwärzere b; dann erscheinen die beiden Linien von \emph{a} in zwei schwächere geteilt \emph{c}, so dass man vier Linien sieht. In Innern sind sie teils ungefärbt, teils lichte grau oder dunkelschwarz. Zuweilen sind die Röhren unterbrochen und fangen in einiger Entfernung wieder an, \emph{b} Fig. 2, oder es ist nur die Färbung in der Röhre unterbrochen, wie bei \emph{e}. Eine ungewöhnlich starke Röhre \emph{f} erschien der ganzen Länge nach dunkel und nur an den Enden eine kleine Strecke etwas lichter und an dem einen Ende zuletzt ganz licht. Gewöhnlich erscheinen die Röhren scharf abgeschnitten, zuweilen aber hatten sie eine Endigung wie in \emph{b} unten angegeben. Fig. 12 Taf. I stellen schiefe Durchschnitte dieser Röhren in einer aus einen solchen Olivin-Kristalle geschliffenen Platte dar.

Es ist schwer zu sagen, wofür man diese Einschlüsse halten soll. Wenn ich sie Röhren genannt habe, so soll damit nur der Eindruck bezeichnet werden, den sie auf mich gemacht haben. Sie sind aber alle parallel, wenn sie auch nur in geringer Menge und vereinzelt in dem Kristalle liegen, und müssen also, da sie sich untereinander nicht berühren, eine ganz bestimmte Lage in dem Kristalle haben, worin sie liegen. Welche diese aber ist, war schwer auszumachen, da man gewöhnlich nur so wenige Flächen bei den Kristallen sieht, doch konnte ich bei einigen Kristallen nicht zweifeln, dass sie eine gegen die Endfläche \emph{P} rechtwinklige, also eine der Hauptaxe parallele Lage haben. Bei einem Kristall z. B., an welchem sich zwei Flächen \emph{k} und dazwischen die Fläche \emph{T} befindet, kann man bei hellem Lampenlichte deutlich sehen, dass die Fläche \emph{T} und die Röhren zu gleicher Zeit das Licht reflektieren, und letztere zugleich rechtwinklig gegen die Axe der Zone \emph{kT} liegen.

Der Olivin in dem Eisen von Brahin gleicht in Farbe, Durchsichtigkeit und Größe der Körner sehr dem Eisen von Krasnojarsk; einzelne herausgefallene Körner zeigten auch einzelne glatte Flächen wie bei diesem; auf der geschliffenen Fläche waren die Durchschnitte der Kristalle noch eckiger, so dass die Flächen sich vielleicht in noch viel größerer Anzahl finden als bei dem Sibirischen Eisen. Sie sind meistenteils sehr klüftig, zeigen aber auch die eingewachsenen röhrenartigen Einschlüsse sehr ausgezeichnet und in großer Zahl.

Die Olivin-Körner in dem Eisen von Atacama sind meistenteils noch grösser als die in dem Pallas-Eisen, oft 5/8 Zoll groß, aber sie sind noch viel klüftiger, vielleicht auch schon mehr oder weniger zersetzt, daher sie auf der geschliffenen Fläche keine Politur annehmen.

Der Olivin in dem Eisen von Rittersgrün findet sich in kleineren Körnern, die untereinander noch mehr zusammengehäuft und zu körnigen Partien verbunden sind, als dies bei dem Sibirischen Eisen der Fall ist; wo sie aber an das Eisen angrenzen, sind sie kristallisiert, und noch deutlicher als bei diesem. Bei einem Korne konnte ich die Flächen in der Zone der Längsprismen mit Sicherheit bestimmen, bei andern zeigten sich einzelne Flächen, die aber nicht bestimmt werden konnten; gewiss würde man bei bessern Stücken, als mir bis jetzt zu Geboten stehen, recht ausgebildete Kristalle finden können. Der gemessene Kristall ist aber hinreichend, um zu beweisen, dass die Kristalle die Winkel des Olivins haben. Andere Körner waren an der gegen das Eisen angrenzenden Seite rund, aber dabei stets etwas drusig; so glatte Körner wie in dem Pallas-Eisen habe ich hier nicht bemerkt. Die Farbe dieses Olivins ist in den Körnern grüner als bei dem Pallas-Eisen, doch erscheinen sie häufiger durch anfangende Zersetzung braun.

Der Olivin in dem Steinbach-Eisen gleicht dem von Rittersgrün vollkommen; an dem kleinen Stücke des Berliner Museums fand ich ein Korn, das ganz von Eisen umschlossen und völlig rund war, ein anderes, welches mehrere Flächen zeigte, die aber nicht bestimmt werden konnten.

Der Olivin in dem Eisen von Bitburg gleicht, nach dem kleinen Stücke des Berliner Museums zu urteilen, den beiden vorigen; Kristallflächen habe ich nicht beobachtet.

Das spezifische Gewicht des Olivins aus dem Pallas-Eisen wird von Stromeyer zu 3,332 angegeben in völliger Übereinstimmung mit dem der meisten der in den Basalten vorkommenden Olivine, das von dem Olivin von Steinbach nach demselben Chemiker zu 3,276.\footnote{Jahrb. d. Chem. u. Phys. 1825 B. 14, S. 275. Stromeyer nennt zwar das Eisen, worin dieser Olivin vorkommt, von Grimma, doch hat schon Chladni gezeigt, dass darunter das Eisen von Steinbach zu verstehen sei. Vergl. Chladni Feuermeteore, S. 326 und Partsch Meteoriten, S. 91.}

Vor dem Lötrohr und gegen Säuren verhält sich dieser Olivin wie der der Basalte. Vor dem Lötrohr schmilzt er nicht und verändert sich nicht.

In Rücksicht der chemischen Zusammensetzung ist der aus dem Pallas-Eisen außer Stromeyer und Walmstedt von Berzelius (1), der von Atacama von Schmid (2) und der von Steinbach (Grimma) von Stromeyer (3) analysiert:

 1|2|3  
Magnesia|47,35|43,16|25,83  
Eisenoxydul|11,72|17,21|9,12  
Manganoxydul|0,43|1,81|0,31  
Kieselsäure|40,86|36,92|61,88  
Zinnsäure|0,17|-|-  
Chromoxyd|-|-|0,33  
Glühverlust|-|-|0,45  

100,53|99,10|97,92

Die beiden ersteren Abänderungen haben also vollkommen die Zusammensetzung des terrestrischen Olivins, die erste enthält etwas weniger Eisenoxydul als die zweite und ist:

(7Mg + Fe)$^{2}$Si,

die zweite:

(4Mg + Fe)$^{2}$Si.\footnote{Vergl. Rammelsberg Mineralchemie, S. 438 und S. 503.}

Beide sind dadurch ausgezeichnet, dass sie, obgleich mitten in einem nickelhaltigen Eisen vorkommend, kein Nickeloxyd enthalten, welches doch Stromeyer, wenn auch nur in geringer Menge, in allen terrestrischen Olivinen nachgewiesen hat, was aber nur beweist, wie auch schon Stromeyer anführt, dass das Nickeloxyd so leicht reduzierbar ist und weniger Verwandtschaft zur Kieselsäure als das reine Metall zum Eisen hat. Merkwürdig ist ferner die, wenn auch nur geringe Menge von Zinnsäure in dem Olivin des Pallas-Eisens, die aber Berzelius neben etwas Kupferoxyd auch in dem Olivin von Boscovich bei Aufsig in Böhmen und in einem aus dem Dep. Puy de Dome gefunden hat. Sie ersetzt eine geringe Menge der Kieselsäure.

Der Olivin von Steinbach (Grimma) hat dagegen nach Stromeyer eine ganz andere Zusammensetzung als der terrestrische Olivin. Da nun aber seine Äußern Charaktere mit denen des übrigen Olivins stimmen, wenn auch die Winkel der Kristalle noch nicht bestimmt sind, das spezifische Gewicht auch nicht merklich verschieden ist, und so auch bei andern ächten Olivinen vorkommt, so muss hier offenbar ein Irrtum stattgefunden haben, wenn ich gleich nicht angeben kann, wodurch derselbe veranlasst ist.\footnote{Stromeyer hat noch einen andern Olivin, angeblich aus dem Eisen von Olumba (soll wohl heißen Otumpa) aus der Prov. Chaco Gualamba, analysiert (Schweigger, Journ f. Chem. u. Phys. 1825 B. 44, S. 275) Da dieses Meteoreisen aber keinen Olivin enthält, so möchte auch hier ein Irrtum stattgefunden haben, und da dieser analysierte Olivin in der Zusammensetzung ganz mit dem Olivin aus dem Pallas-Eisen nach Stromeyers Analyse stimmt, so wäre es möglich, dass eine Verwechselung mit diesem die Ursache davon ist.}

Das Eisen, welches die Grundmasse bildet, worin die Olivin-Kristalle eingeschlossen sind, findet sich bei den verschiedenen Pallasiten in größerer oder geringerer Menge. Das erstere ist der Fall bei den Pallasiten von Steinbach und Rittersgrün, und bei diesen kann daher seine Struktur am besten erkannt werden. Es zeigt auf den Schnittflächen, geätzt sehr schöne Widmanstättensche Figuren, deren Streifen sich auf den Schnittflächen bei den einzelnen Stücken des Berliner Museums stets überall parallel bleiben, also beweisen, dass das ganze Eisen jedes dieser Stücke zu einem Individuum besteht. Ob dies auch bei größeren Stücken der Fall ist, werden diese lehren; möglich, dass diese aus mehreren Stücken bestehen. Dennoch sind immer die einzelnen Körner oder die körnigen Partien des Olivins von einer hier nur sehr dünnen Einfassung von dem Meteoreisen umgeben, jenseits welcher erst die Widmanstättenschen Figuren anfangen. Das Eisen von Bitburg gleicht dem vorigen, doch sind die Widmanst. Figuren noch feiner. Bei den Pallasiten von Krasnojarsk, Brahin und Atacama ist der Olivin grösser, der Raum zwischen ihm geringer. Die Einfassung des Olivins von dem Meteoreisen ist im Verhältnis der Größe der Körner dicker, über dieser sieht man die dünne Lage des Tänit, wie dies Reichenbach ausführlich beschrieben (vgl. oben S. 41), worauf nun ein etwas dunkler gefärbtes Meteoreisen erscheint, das Reichenbach zu seinem Fülleisen (Plessit) rechnet. Wo diese Räume etwas grösser als gewöhnlich sind, erscheinen durch Ätzung darin noch Widmanst. Figuren. Wenn sich mehrere solcher Räume mit diesen Figuren auf einer geschliffenen Fläche finden, wie dies besonders bei dem Eisen von Atacama vorkommt, so sieht man selten, dass diese an den verschiedenen Stellen eine gleiche Lage haben, was beweist, dass größere Massen dieser Pallasit aus mehreren Eisenindividuen, wie das Meteoreisen von Seeläsgen, bestehen. Diess wird noch dadurch bestätigt, dass die in den Olivin-Kristallen des Pallas-Eisens vorkommenden röhrenartigen Einschlüsse, die in einem Kristall alle untereinander parallel sind, wenn man verschiedene Kristalle miteinander vergleicht, keine parallele Lage haben, was doch wahrscheinlich der Fall wäre, wenn das Eisen, worin sie liegen, ein Individuum wäre.

Die Einfassung des Olivins durch das Meteoreisen ist recht merkwürdig. Sie scheint zu beweisen, dass, nachdem der Olivin sich in dem flüssigen Eisen ausgeschieden hat, das den Olivin zunächst Umgebende zuerst fest wurde und denselben mit einer dünnen Hülle umgab, auf welche sich sogleich etwas Tänit legte und nun der innere Raum mit schaligen Lagen von Meteoreisen und Tänit, die die Widmanstättenschen Figuren bilden, oder nur mit dem sogenannten Plessit von Reichenbach ausgefüllt wurde. Aber dieser Plessit scheint selbst nichts anderes zu sein als ein schaliges Meteoreisen, in welchem nur die Schalen recht dünn und die Tänit-Lagen verhältnismäßig dick sind, so dass sie im Ganzen die Erscheinung darstellen, die Reichenbach unter dem Namen der Kämme beschrieben hat, und deren oben S. 36 Erwähnung getan ist. Denn wenn man von der geätzten Fläche des Pallas-Eisens einen Hausenblasenabdruck macht und den Abdruck des Fülleisens unter dem Mikroskop betrachtet, so sieht man das feine Gemenge deutlich und die sich durchschneidenden Lagen, ganz ähnlich denen, die die Widmanstättenschen Figuren bilden.

Außer dem Olivin und dem Meteoreisen, die die wesentlichen Gemengteil des Pallasit bilden, finden sich in demselben einige unwesentliche, die nur in mehr oder weniger geringen Menge darin vorhanden sind. Zu diesen gehört:

1. Troilit oder Magnetkies. Er ist wie in dem Meteoreisen von tombakbrauner Farbe, nur derb, und auch eigentlich nur auf den angeschliffenen Flächen zu erkennen; er findet sich gewöhnlich nur in kleinen Partien, aber in allen Pallasiten, am größten noch in dem von Krasnojarsk und Brahin, wo er doch Körner von 3 bis 4 Linien Durchmesser bildet. Bei einem Stücke des P. von Krasnojarsk des Berliner Museums findet sich ein Olivin-Korn, das ganz von Troilit umschlossen ist, bei einem andern ein anderes, das zu 3/4 des Umfangs von Troilit und nur zu 1/4 von Meteoreisen umschlossen ist. Zuweilen sind einzelne Körner, wie sie öfter von haarbreiten Lagen von Meteoreisen getrennt sind, auch durch solche Lagen von Troilit voneinander getrennt.

2. Chromeisenerz von samtschwarzer Farbe, unvollkommenem Metallglanz, braunem Strich, und vor dem Lötrohr mit Phosphorsalz ein smaragdgrünes Glas gebend. Es ist in der Regel vollkommen in dem Meteoreisen eingewachsen und grenzt an dasselbe in geraden Flächen, ist also kristallisiert; zuweilen grenzt es aber auch an den Olivin, es bildet dann mit ihm eine unregelmäßige Grenze und nimmt Eindrücke von diesem an, scheint also später als dieser kristallisiert zu sein. Man erkennt das Chromeisenerz auch nur auf geschliffenen Flächen; es scheint aber überall nur sparsam vorzukommen; ich habe es bestimmt gesehen nur in den Pallasiten von Brahin und Atacama. Doch muss es auch in den übrigen, wenigstens in dem von Krasnojarsk und Steinbach vorkommen, weil Laugier in dem ersteren und Stromeyer in dem Steinbach- (Grimma-) Olivin etwas Chromoxyd angibt, was doch wahrscheinlich von eingemengtem Chromeisenerz herrührt.\footnote{In den angeschliffenen Stücken des Pallas-Eisens von der Berliner Sammlung ist das Chromeisenerz nicht sichtbar, und in der Analyse des Pallas-Eisens von Berzelius wird auch kein Chrom angegeben. Indessen ist beides doch kein Grund, dass nicht Chromeisenerz in dem Pallas-Eisen vorkommen kann, da die angeschliffenen Flächen der Berliner Stücke nicht groß sind, und es zufällig auf diesen fehlen kann, und Berzelius das Pallas-Eisen vor der Analyse gehämmert und dadurch alle spröden Gemengteil, wie den Olivin und also auch das etwa vorhandene Chromeisenerz, entfernt hat.}
\subsection{Mesosiderit.}
\paragraph{}
Ein körniges Gemenge von Meteoreisen, Troilit, Olivin und Augit. Es gehören hierher die Eisenmeteorit der Sierra de Chaco und von Hainholz.

1. Sierra de Chaco in der Wüste Atacama, N. von Chile 1862. Ein ursprünglich 28,87 Loth schweres, mit natürlicher Oberfläche und Bruch begrenztes Stück, Geschenk des Prof. Domeyko in St. Yago in Chile.\footnote{Vergl. Monatsberichte der k. Akad. der Wissensch. 1863 S. 30.} Davon wurde ein Teil abgeschnitten; die Sammlung enthält jetzt noch ein großes Stück von 23,90 und ein kleines von 1,4 Loth. Es sieht im Bruche körnig und im Allgemeinen grünlichschwarz und glanzlos aus; man erkennt nur einzelne größere Körner von rötlichgelbem Olivin und kleinere schwärzlichgrüne von Augit; das überall fein eingesprengte Eisen ist hier fast gar nicht wahrzunehmen. Vollkommen aber unterscheiden sich die Gemengteil auf einer geschliffenen und polierten Fläche; das Eisen tritt nun gleich durch seine stahlgraue Farbe und seinen starken Metallglanz hervor, und man sieht nun erst, in welcher Menge es vorhanden ist. Es ist in feinen Teilen überall mit kleinen Teilen der Silicate gemengt, die überall mit ganz unregelmäßigen, eckigen und zackigen Oberflächen ineinandergreifen, und zwischen deren der Troilit überall, aber in noch feineren Teilen, durch seine tombakbraune Farbe kenntlich, enthalten ist. Dazwischen treten nun in einzelnen größeren Körnern Nickeleisen, Olivin und Augit auf. Geätzt zeigen die größeren Körner des Nickeleisens sehr feine und zierliche Widmanstättensche Figuren von einem eigentümlichen Verhalten; man erkennt darin nicht ein System von Streifen, die einem aus schaligen Zusammensetzungsstücken parallel den Flächen des Oktaeders bestehenden Individuum entsprechen, sondern stets mehrere; bei einem am Rande des kleinen Stückes befindlichen Korne von 4 Linien Durchmesser, das aber nur zum Teil auf dem Stücke enthalten ist, sind deren drei zu erkennen, die durch eine halbe Linie dickes nicht gestreiftes Nickeleisen getrennt sind, in welchen nur hier und da kleine Körner oder körnige Partien von Augit liegen. Die kleineren Körner des Nickeleisens zeigen keine Widmanstättensche Figuren, sondern enthalten in ihrer Mitte nur unregelmäßig gestaltete Teile von der in verdünnter Salpetersäure nicht angegriffenen Substanz. Nach einer vorläufig mitgeteilten Nachricht von Hrn. Domeyko enthält dasselbe nach seinen Untersuchungen 88,55 pC. Eisen und 11,5 Nickel, ist also an dem letzteren Bestandteil sehr reich. Der Olivin ist von grünlichgelber bis rötlichgelber und brauner Farbe und zuweilen von beträchtlicher Größe; auf der äußern Fläche befindet sich ein Korn von 3/4 Zoll im Durchmesser. Er ist zerklüftet und nimmt im Allgemeinen keine so gute Politur an, wie das Augit, vielleicht weil er schon etwas zersetzt ist. Er schmilzt und verändert sich vor dem Lötrohr nicht, ist also wie der gewöhnlich in den Meteoriten vorkommende Olivin nicht Eisenreich. Das Augit ist olivengrün, auf der geschliffenen und polierten Fläche ganz schwarz und glänzend, in sehr dünnen Splittern aber doch mit grünlichweissem Lichte durchsichtig; er ist deutlich spaltbar nach den Flächen des vertikalen Prismas und seiner Quer- und Längsfläche, und so vollkommen, dass sich die Spaltungsflächen ziemlich genau mit dem Reflexionsgoniometer messen lassen. An deutlichsten ließ sich bei einem Splitter die Neigung der Fläche des vertikalen Prismas zur Längsfläche messen, ich fand sie zu 136° 4’. Vor dem Lötrohr ist dieses Augit nur in dünnen Splittern an den äußersten Kanten zu einem schwarzen Glase schmelzbar; mit Phosphorsalz bildet er unter Abscheidung der Kieselsäure ein Glas, das, solange es heiß ist, grünlichweiß ist, das aber beim Erkalten ganz ausblasst. Nickeleisen wie auch in geringerer Menge Troilit kommen in diesem Augit wie auch in dem Olivin gewöhnlich in sehr feinen Teilen eingemengt vor, wie man auf der geschliffenen Fläche des Meteoriten, wenn man sie mit der Lupe betrachtet, ganz deutlich sehen kann, daher man zu den Lötrohrversuchen diese Silicate erst pulvern, und die anziehbaren Teile mit dem Magnete ausziehen muss. Troilit ist in größeren Körnern in dem Meteorit nicht eingemengt.

Die natürliche Oberfläche ist nur wenig uneben; das Nickeleisen ist hier wohl etwas mit braunem Eisenoxydhydrat bedeckt, doch nicht sehr stark; die großen Körner von Augit und Olivin sind ganz deutlich zu erkennen, wenn sie auch aus der Oberfläche nicht hervortreten.

2. Hainholz, Reg.-Bez. Minden, Preußen, 1856. Zwei größere platte und zwei kleinere Stücke; erstere durch Zerschneiden eines einzigen Stückes erhalten, sind außer den Schnittflächen mit natürlicher Oberfläche begrenzt. Dem vorigen sehr ähnlich; das Nickeleisen findet sich jedoch nicht in so großen Körnern, und diese zeigen auch beim Ätzen keine Wid. Fig., sondern sind matt, und enthalten wohl glänzend gebliebene Teile, die aber nicht in der Mitte der Körner liegen, sondern ganz an den Rand derselben gedrängt sind. Der Troilit ist ferner nicht in so großer Menge und das Augit nicht in so großen Körnern enthalten, dagegen kommt der Olivin hier in noch größeren Individuen vor. An dem einen Stücke der Sammlung findet sich ein Kristall von etwas über Zolllänge, und Reichenbach gibt an, dass ist einem in seinen Besitz befindlichen Stücke sich ein Kristall von 1 3/4 Zoll Länge und 1 1/2 Zoll Breite befindet.\footnote{Vgl. Poggendorffs Ann. 1857, B. 101, S. 312.} Reichenbach beobachtete ferner in den Stücken seiner Sammlung Kugeln und Knollen, die, ohne sich in der Beschaffenheit zu unterscheiden, abgesondert in der Masse vorkommen\footnote{A. a. O. S. 312.}; bei den Stücken der Berliner Sammlung finden sie sich nicht, doch kommen sie vielleicht auch bei dem Eisen der Sierra de Chaco vor, da man auf der Bruchfläche bei diesem ein Viertel bis einen halben große rundliche Vertiefungen sieht, die wie Eindrücke von solchen Kugeln aussehen.

An der Oberfläche ist dieser Meteorit schon stärker zersetzt, besonders stellenweise. Dieselbe ist im Allgemeinen braun, einzelne Eisenkörnchen ragen daraus hervor, auch erkennt man auf ihr zum Teil noch die großen Olivin-Kristalle. Als das Stück der Sammlung durchschnitten wurde, bildeten sich nach nicht langer Zeit auf gewissen Stellen kleine Bläschen von Eisenchlorid. Ich ließ daher die Stücke auf den Rat des Baron Reichenbach einige Zeit in reinem Wasser liegen. Dadurch wurde das Eisen an diesen Stellen noch stärker oxydiert und braun, aber obgleich dies schon vor länger als anderthalb Jahren geschah, so hat doch das Rosten seit der Zeit nicht weiter zugenommen. Bei dem Eisen der Sierra de Chaco zeigt sich keine Neigung zum Rosten.
\clearpage
\section{Steinmeteorit.}
\subsection{Chondrit.}
\paragraph{}
Diese Art ist unter den verschiedenen Arten der Steinmeteorit die bei weiten zahlreichste und zugleich diejenige, mit der sich Berzelius vorzugsweise beschäftigt hat, der, wenn er auch vorzugsweise nur die chemische Beschaffenheit ermittelte, doch dadurch zugleich die wichtigsten Anhaltspunkte für die Beurteilung der mineralogischen Beschaffenheit gegeben hat.
\begin{center}
Äußere Beschaffenheit.
\end{center}
\paragraph{}
Diese Meteoriten sind besonders durch ihre kugliche Struktur ausgezeichnet, worauf sich ihr Name bezieht. Sie bestehen nämlich aus einer mehr oder weniger feinkörnigen Grundmasse, in der mehr oder weniger häufig kleine Kugeln neben vielen andern Gemengteilen, wie Olivin, Nickeleisen, Magnetkies, Chromeisen und anderen schwarzen Körnern liegen.

Die Grundmasse ist teils bedeutend fest, wie bei dem Chondrit von Erxleben, Chantonnay etc., teils weniger fest und fast zerreiblich, wie bei dem von Mauerkirchen, Iowa, Bachmut etc. Sie hat am häufigsten eine lichte aschgraue Farbe, die einesteils ins graulichweiße bis schneeweiße, auf der andern Seite doch seltener ins dunkelgraue und selbst graulichschwarze übergeht, ist aber selten gleichmäßig gefärbt; in der Regel kommen Massen von graulichweißer und von aschgrauer oder selbst graulichschwarzer Farbe an einem und demselben Stücke vor, und grenzen dann aneinander teils mit unbestimmter, ineinander, wenn auch schnell verlaufender, teils mit sehr bestimmter scharfer Grenze. Das erstere findet z. B. bei den Chondriten von Chantonnay und Güterslohe, das letztere bei den von Ensisheim und Weston, ganz besonders aber bei einem Stücke von Siena (aus der Klaprothschen Sammlung stammend) statt, von dem in Fig. 9 Taf. II eine Zeichnung einer angeschliffenen Fläche desselben gegeben ist, und bei dem auch der Unterschied in der Farbe recht stark hervortritt. Zuweilen kommen scharfe und unbestimmte Grenzen an einem und demselben Stücke vor, wie bei dem Ch. von Bremervörde. Die verschiedenfarbigen Massen liegen teils in größeren Partien nebeneinander (Ch. von Güterslohe und Chantonnay), teils durchzieht die eine wie in Adern die andere; die weiße die schwarze in dem Ch. von Ensisheim, die schwarze die weiße in dem von Agen; die weiße breitet sich stellenweise aus und schließt dann Stücke der schwarzen ein, wie in dem Ch. von Ensisheim, so dass das Gestein hier ein anscheinend breccienartiges Ansehen erhält. Wo die Grundmasse fest ist, ist sie auch so hart, dass sie sich mit dem Messer nicht ritzen lässt; auch erhält sie in diesem Fall schon einigen Glanz, der ihr sonst fehlt.

Die Kugeln, die in dieser Grundmasse auf eine ähnliche Weise wie in den Varioliten oder vielen roten Porphyren eingewachsen vorkommen und die die kuglige Struktur dieser Meteorit bedingen, sind gewöhnlich nur so klein wie Schrotkörner oder Hirsekörner, zuweilen aber auch grösser, selbst 3 bis 4 Linien groß, wie bei dem Ch. von Ausson und New-Concord.\footnote{In dem von Mezö-Madaras beobachtete Reichenbach sogar eine Kugel eines halben Zolls Durchmesser (Poggendorffs Ann. 1860 B. 111, S. 366).} Sie sind ferner mit Ausnahme der größeren, die mehr unregelmäßig gerundet und in die Länge gezogen sind, gewöhnlich regelmäßig gerundet, sind aber selten an einem Stücke von gleicher oder ungefähr gleicher Größe, besonders wenn darunter solche von der Größe wie in dem Ch. von Ausson und New-Concord vorkommen. Ihre Oberfläche ist rau und selbst drusig (Richmond), seltener glatt (Poltava), und im Bruche erscheinen sie teils uneben, teils fasrig, im letzteren Fall jedoch stets nur sehr feinfasrig, indessen doch immer bestimmt erkennbar fasrig, besonders unter der Lupe; was mir aber dabei sehr bemerkenswert scheint und sie von den Kugeln der irdischen Gebirgsarten, namentlich der Diorit unterscheidet, nie radial, sondern immer exzentrisch fasrig; so bei den Ch. von Erxleben, Stauropol, Forsyth, Bachmut, Ausson etc. Ihre Farbe ist wie die der Grundmasse, unterscheidet sich aber doch immer etwas von ihr; sie sind gewöhnlich grünlichgrau oder braun, dabei von einem nur geringen Glanze, der etwas fettartig ist und stets nur äußerst schwach an den Kanten durchscheinend, fast undurchsichtig. Sie sind wie die Grundmasse bald heller, bald dunkler, und gewöhnlich finden sich beide Arten zusammen in einem und demselben Meteoriten, wo denn bald die helleren, bald die dunkleren vorherrschen. Das erstere, was der gewöhnlichere Fall ist, findet z. B. statt bei dem Ch. von Mezö Madaras, Okniny, Cabarras, das letztere bei den von Güterslohe, Ausson. In dem Ch. von Krasnoi-Ugol sah ich auch eine graue Kugel eine kleinere weiße einschließen,\footnote{Etwas Ähnliches beobachtete Reichenbach auch bei Kugeln von Tabor und Parma (Pongendorffs Ann. 1860 B. 111, S. 377).} und in dem von Mauerkirchen, wo meistenteils nur lichtere Kugeln vorkommen, sind dieselben doch öfter nach der Oberfläche zu etwas dunkler gefärbt. Sie sind ferner mit der Grundmasse mehr oder weniger fest verwachsen, und ersteres gewöhnlich da, wo die Grundmasse fester, letzteres, wo sie zerreiblich ist. Im ersteren Fall fallen sie beim Zerschlagen des Gesteins nicht heraus, und man sieht dann auf der Bruchfläche des Gesteins auch ihren Bruch, in letzteren Fall fallen sie heraus oder lösen sich zum Teil von der Grundmasse, so dass man auf dem Bruch die halbkugelförmigen Höhlungen oder Erhabenheit der sich herausgelösten oder sitzengebliebenen Kugeln sieht. Die Kugeln finden sich in der Regel häufig, in manchen Fällen jedoch nur sparsam, wie in dem Ch. von Erxleben, Klein-Wenden, Ensisheim und Chantonnay, in andern aber wiederum so häufig, dass sie ganz gedrängt nebeneinander liegen und sich öfter gegenseitig in der Ausbildung stören, wie in dem Ch. von Timochin, Richmond, Benares und Mezö-Madaras.\footnote{Hausmann hält die lichtern Kugeln in dem Bremervörde für undeutliche Kristalle, deren Form nicht näher zu bestimmen ist (Göttinger Nachrichten von 1856, S. 151). „Nach den Durchschnitten derselben, welche selten die Größe von ein paar Linien erreichen, zu urteilen” sagt er, „scheinen sie teils rechteckige teils irregulär sechsseitige Prismen, zu sein, wonach auf ein trimetrisches Kristallisationssystem zu schließen sein dürfte.” Ich habe von Kristallformen bei diesen Mineralen nichts bemerkt und kann auch einer anderen Angabe von Hausmann, dass es vor dem Lötrohr „ruhig und nicht eben schwer zu Email schmelze” nicht beistimmen. Ich fand es vor dem Lötrohr unschmelzbar und stimme nur darin mit Hausmann überein, dass es beim Erhitzen dunkelbraun gefärbt wird. Ich kann daher auch nicht das Mineral, wie Hausmann weiter unten in seiner Abhandlung getan hat, für feldspatartig halten.}

Der Olivin ist nur selten in der Grundmasse erkennbar und findet sich dann in kleinen, nur höchstens liniengroßen, gelblichgrünen, durchsichtigen Körnern, wie in dem Ch. von Erxleben, Klein-Wenden, Krasnoi-Ugol, Timochin und Pultava. Durchschnitte von ausgebildeten Kristallen, und zwar von rektangulärer Form, habe ich nur etwa bei dem Olivin in dem Ch. von Pultava und Erxleben gesehen.

Das Nickeleisen ist dagegen ein beständiger und häufiger Gemengteil. Es findet sich in den Meteorsteinen gewöhnlich nur fein eingesprengt, doch kommen unter den feineren Körnern mitunter einzelne größere vor; so sieht man in einem der Stücke des Ch. von Barbotan ein längliches Korn von 3 Linien Länge,\footnote{Partsch spricht von linsen- und höhnen großen Körnern in den Stücken dieses Meteoriten in dem Wiener Mineralien-Kabinette (Meteoriten, S. 77).} ein ähnliches ragt bei dem von Klein-Wenden aus der Oberfläche hervor, und andere ähnliche, wenn auch nicht ganz so große Körner enthalten die Ch. von Lucé (Toulouse?), Seres und Macao. Die feinen Körner sind zackig und eckig, die größeren haben meistenteils eine rundliche Oberfläche, kristallisiert findet es sich in den Stücken der Berliner Sammlung nicht, daher ich auch die unvollkommenen Hexaeder, die Partsch bei dem Nickeleisen der Wiener Stücke des Ch. von Barbotan beobachtet haben will (Meteoriten, S. 77), nur für solche rundliche Körner erklären möchte. Indessen Individuen sind diese Körner doch, denn das oben erwähnte Korn in dem Ch. von Barbotan, wie auch ein anderes in dem von Ausson zeigen geschliffen und geätzt die Linien des Meteoreisens von Braunan.\footnote{Partsch (Meteoriten, S. 82) und Reichenbach (Pongendorffs Annalen B. 111, S. 365) erhielten bei den Eisenkörnern in den Meteoriten von Macao und Blansko selbst Widmanstättensche Figuren.} Es umschließt auch, wie man auf der geschliffenen Fläche sehen kann, kleine Teile der Grundmasse. Die Menge des eingesprengten Eisens ist oft sehr beträchtlich, wie in dem Ch. von Erxleben und Kl. Wenden; wie groß aber die Menge desselben eigentlich ist, erkennt man erst, wenn die Stücke angeschliffen und poliert sind, wo sein Metallglanz und seine stahlgraue in die silberweiße übergehende Farbe erst recht stark hervortreten und es auf diese Weise kenntlich machen. Die ganz feinen Körner sind da überhaupt erst zu erkennen, und man sieht dann, dass solche auch in den Kugeln enthalten sind, wo in solchen feinen Teilen das Nickeleisen nie fehlt, wenn es auch gewöhnlich nur in sehr geringer Menge vorhanden ist In der Umgebung der Kugeln ist es dagegen häufig in größerer Menge angehäuft, wie z. B. in dem Ch. von Mezö-Madaras, Krasnoi-Ugol und Ausson. Der feuchten Luft ausgesetzt, oxydiert sich das Nickeleisen leicht und das gebildete Eisenoxyd färbt die Umgebung braun, wie man dies bei so vielen Stücken in den Sammlungen sehen kann.\footnote{Wie schnell diese Oxydation wo sich geht, zeigen die beiden Stücke von dem Ch. von Güterslohe des Berliner Museums. Von den gefallenen Steinen wurde einer schon am folgenden Tage gefunden, ein anderer erst ein Jahr später, und Stücke von beiden, der erstere fast vollständig, wurden von Hrn. Dr. Stohlmann in Güterslohe durch gütige Vermittlung des Hrn. Prof. Dove dem Berliner mineralogischen Museum verehrt. Das erste Stück ist im Bruch aber vollkommen frisch, das andre dagegen durch und durch voller Rostflecke. Hätte es in der feuchten Erde noch länger gelegen, so würde es ganz zerfallen oder unkenntlich geworden sein. Diese schnelle Verwitterung ist auch der Grund, weshalb man noch nie einen Meteorstein gefunden hat, den man nicht hat, fallen sehen, während man doch so viele von den nur so selten fallenden Eisenmassen zufällig gefunden hat oder noch findet, die durch die entstehende oxydierte Rinde vor weiterem Angriff der Atmosphäre geschützt werden (siehe oben S. 42).}

Magnetkies\footnote{Vergl. oben S. 40.} ist ebenfalls ein nie fehlender Gemengteil dieser Meteorit. Er findet sich wie das Nickeleisen gewöhnlich fein eingesprengt, doch nicht in solcher Menge als dieses,\footnote{Nach Partsch ist in Richmond mehr Magnetkies als Nickeleisen; bei dem kleinen Stücke der Berliner Sammlung ist dies nicht der Fall; es enthält wohl Magnetkies, aber nur in geringer Menge.} seltener kommt er in größeren Körnern vor, doch auf diese Weise selbst noch häufiger als das Nickeleisen. In solchen findet er sich in den Ch. von Barbotan, Parma, Zaborcica, besonders aber in dem von Grüneberg, wo in dem größeren Stücke des Berliner Museums ein Korn von ihm enthalten ist, das einen halben Zoll im Durchmesser hat. Wo er fein eingesprengt vorkommt, ist er auf der Bruchfläche des Gesteins kaum oder schwer zu erkennen, aber recht gut auf der geschliffenen Fläche, wo ihn tombakbraune Farbe und geringerer Glanz gleich vor dem Nickeleisen auszeichnen. Man sieht dann auch, dass der Magnetkies bald mit dem Nickeleisen verbunden ist, bald in getrennten Körnern vorkommt. In dem Stücke von Zaborcica der Berliner Sammlung schließt ein erbsengroßes Korn von Magnetkies ein Korn von Nickeleisen ein und in einem Stücke von Grüneberg sowie von Krasnoi-Ugoi wird umgekehrt ein Korn Magnetkies von Nickeleisen vollständig umschlossen. Der Magnetkies dieser Meteorit ist gar nicht oder nur äußerst schwach magnetisch.

In geringer Menge, aber doch vielleicht überall kommen in diesen Meteoriten schwarze Körner vor, die aber gewöhnlich nur sehr klein, und nur zuweilen etwas grösser, und so groß wie etwa ein Hirsekorn sind, so dass man sie herausnehmen und besonders untersuchen kann; sie erweisen sich dann als Chromeisenerz, da sie zerrieben ein braunes Pulver und mit Borax oder Phosphorsalz vor dem Lötrohr geschmolzen ein chromgrünes Glas geben. Ob aber nun sämtliche schwarze Körner aus Chromeisenerz bestehen oder neben diesen noch andere schwarze Körner vorkommen, muss ich dahingestellt sein lassen. Die größten schwarzen Körner beobachtete ich in dem Ch. von Château Renard, doch sind sie auch noch deutlich erkennbar in dem von Erxleben, Klein-Wenden, Tabor, Richmond. Wöhler beobachtete sie auch in dem von Bremervörde. Wo sie sehr klein sind, erkennt man sie am besten auf einer angeschliffenen Fläche, denn auf einer Bruchfläche des Gesteins können sie leicht mit dem Nickeleisen verwechselt werden, da dies auch schwarz erscheint, wenn man es nicht gerade im vollen reflektieren Lichte betrachtet. Sie sind auch teils mit dem Nickeleisen verbunden, teils nicht. Das Chromeisenerz der Meteorit ist überall nicht magnetisch.\footnote{Andre Gemengteil als die angegebenen habe ich nicht bemerkt, dennoch sind aber solche von anderen Beobachtern öfter beschrieben. Hausmann nahm an, dass der Bremervörde aus einem „feldspatartigen Körper (vergl. oben S. 86), einem Körper der Pyroxen-Substanz und Olivin” bestehe, und Wöhler fügte diesen Gemengteilen außer Chromeisenerz noch etwas Grafit in feinen Blättchen hinzu (Göttinger Nachrichten 1856, S. 153), welcher letztere wohl darin enthalten sein kann, da derselbe ja auch in den Meteoreisen vorkommt. Eichwald führt bei der Beschreibung des Ch. von Lixna an: „Von den nicht metallischen Körnern könnte man die helleren, fast weißen für kleine abgerundete Kristalle von Anorthit oder Labrador, die gelblichbraunen für Olivin oder sehr kleine Granatkristalle halten und die viel Größeren und Selteneren für Augit nehmen. Diese letzteren sind etwa 3/4 Linien groß und dennoch zehn Mal grösser als die Kristalle des Olivins und Anorthits. Sie bilden das nicht metallische Gemenge des Meteorsteins, in dem die Augitkristalle deutlich eingesprengt erscheinen, während die anderen kleinen Kristalle seine Hauptmasse ausmachen” (Pongendorffs Ann. 1852 B. 85, S. 577). Dufrenoy spricht bei dem Ch. von Château Renard von einem unvollkommen blättrigen Mineral, das in einigen Punkten analoge Streifen zeigt, wie sie den hemitropen Massen von Albit oder Labrador eigen sind. Das darin vorkommende Chromeisenerz hält er für ein dem Perlit ähnliches Mineral (Poggend. Ann. 1841 B. 53, S. 413). Abich nennt das Gefüge des Ch. von Stauropol psammitisch und spricht bei der Beschreibung auch von einem Mineral, das sich auf Labrador (oder Saussurit) zurückführen ließe. Es hätte eine grünlichgraue Färbung, ließe bei günstiger Zerspaltung deutlichen Blätterdurchgang erkennen und finde sich in rundlichen, aber auch stumpfkantig vorkommenden Fragmenten von gewöhnlich 2, zuweilen aber auch von 8 Millimetern Größe. Eine auf einer Bruchfläche sichtbare Labradormasse von 14 Millimeter im Durchmesser enthielte in ihrem Innern ein fremdartiges Aggregat, aus einem durch Zersetzung unkenntlichen weißlichen Mineral, feinen Teilen von Meteoreisen und kleinen, weißgelblichen, mehr fett- als glasglänzenden Kristallfragmenten eines besonderen Minerals bestehend (bulletin de l'acad. imp. des sc. d. St. Petersbourg 1860 t. II, p. 412).}

Äußerlich hat der Meteorit dieser Abteilung wie die übrigen eine durch Schmelzung der Oberfläche entstandene dünne schwarze Rinde, die indessen hier stets matt und öfters durch das hervorragende, schwer schmelzbare Nickeleisen höckerig ist, wie z. B. bei Aigle u. s. w. Nicht selten sind sie auch im Innern mit Sprüngen durchsetzt, auf welchen etwas von der geschmolzenen Oberfläche während des Zuges durch die Atmosphäre durch den Druck der Luft hineingepresst ist,* wie bei den Ch. von Lissa, Ensisheim, Politz, Château Renard u. s. w. Auch finden sich öfter glänzende schwarze Ablösungs- oder Rutschflächen, auf welchen das Eisen breit gefletscht ist, wie z. B. bei den Ch. von Lixna und Aigle.

Ich will hier nur noch die Beschreibung einiger ausgezeichneten Abänderungen dieser Abteilung von Meteoriten folgen lassen, die gewissermaßen als Typen von ganzen Gruppen von Abänderungen in dieser Abteilung betrachtet werden können, und wähle dazu diejenigen, die in guten Exemplaren in der Berliner Sammlung vertreten sind.

1. Der Chondrit von Erxleben, gef. d. 15. April 1812. Er gehört zu den am deutlichsten kristallinischen der Sammlung, und ist daher allen übrigen voranzustellen. Derselbe hat eine lichte, graulichweiße, feinkörnige Grundmasse, die glänzend von Glasglanz, hart und fest ist, und daher geschliffen eine gute Politur annimmt. Darin liegen ziemlich häufig ungefähr liniengroße Kugeln, die gewöhnlich mit der festen Grundmasse fest, in manchen Fällen doch auch weniger stark verwachsen sind und sich von ihr beim Zerschlagen des Gesteins ablösen, so dass man auf der Bruchfläche, wenn auch gewöhnlich ihren Bruch, doch auch zuweilen einen Teil ihrer kugligen Oberfläche oder die konkaven Höhlungen, worin sie gesessen haben, sieht. Die Oberfläche der Kugeln, sowie auch die der Vertiefungen ist feinkörnig, die Farbe der Kugeln teils etwas grünlichgrau und etwas dunkler als die der Grundmasse, teils gelblichgrau und lichter als diese. Die Kugeln der ersteren Art sind häufiger und im Bruche uneben, die der letzteren weniger häufig und feinfasrig. Die Farben treten noch besser als im Bruche auf einer geschliffenen Fläche hervor. Olivin kommt in den Stücken der Sammlung mehr oder weniger deutlich vor, zuweilen noch auf der Bruchfläche des Gesteins geradlinige Umrisse zeigend, doch ist er im Allgemeinen nicht häufig. Nickeleisen ist fein eingesprengt in ziemlich großer Menge in dem Meteorstein enthalten und findet sich in sehr feinen Körnern auch in den Kugeln, besonders den dunklen, wie man auf der geschliffenen Fläche sehen kann. Auf dieser erkennt man auch den Magnetkies, der in feinen Teilen zum Teil mit Nickeleisen verwachsen vorkommt und im frischen Bruch gar nicht erkannt werden kann; ebenso sieht man auch erst auf der geschliffenen Fläche die schwarzen Körner, die teils einzeln, teils mit dem Nickeleisen verwachsen darin vorkommen, doch muss man Acht haben, sie hier nicht mit den feinen Löchern, die sich auf der Schlifffläche finden und dadurch entstehen, dass Teilchen von Nickeleisen beim Schleifen aus der Fläche herausgerissen werden, wie auch mit den Körnern von Nickeleisen und Magnetkies, welche alle bei gewisser Beleuchtung schwarz aussehen, zu verwechseln.

Dem Chondrit von Erxleben ist der von Kl.-Wenden (1843, Sept. 16) so ähnlich, dass man beide nicht voneinander unterscheiden kann. Ebenso auch der von Abich beschriebene Chondrit von Stauropol im Kaukasus (1857, März 25), der von Grewingk und Schmidt beschriebene Ch. von Pillistfer in Livland (1863, April 15) und einige andere.

2. Ensisheim (1492, Nov. 7). Eine feste, feinkörnige Masse, die teils schwärzlichgrau und nur schimmernd, teils graulichweiß und etwas stärker glänzend ist. Beide Massen enthalten eingewachsene Kugeln, die aber stark mit der Grundmasse verwachsen und stets etwas dunkler und glänzender als die Grundmasse gefärbt sind. Die graulichweiße Masse durchzieht die schwarze wie Adern nach allen Richtungen und teilt sie dadurch in eine Menge eckiger Stücke, während sie selbst, da wo sie breiter wird, wieder eine Menge kleiner eckiger Stücke der schwärzlichgrauen Masse einschließt, so dass dadurch der ganze Meteorit ein breccienartiges Ansehen erhält. Die schwarze und die graulichweiße Masse schneiden scharf aneinander ab. Die schwarze Masse ist indessen bei weitem vorherrschend, und manche kleineren Bruchstücke enthalten nur sie und nichts von der weißen.

Nickeleisen ist in der ganzen Masse eingesprengt, aber nicht so gleichmäßig und in solcher Menge wie bei den Ch. von Erxleben und Klein-Wenden, die Körner sind aber meistenteils sehr fein und dann nur auf einer geschliffenen Fläche zu erkennen; sie werden nur hier und da etwas grösser. Magnetkies ist viel seltener, findet sich aber in einzelnen größeren Körnern.

Der Stein ist von vielen glänzenden und schwarzen Ablösungsflächen durchzogen.

Der Ch. von Chantonnay (1812, Aug. 5) ist dem von Ensisheim ähnlich, die Masse hat hier ebenfalls stellenweise eine verschiedene Farbe, eine dunklere und hellere, und erstere ist noch schwärzer als bei dem Ch. von Ensisheim, aber beide liegen in großen Partien aneinander und schneiden nicht so scharf ab. Der Stein nimmt wie der von Ensisheim geschliffen eine gute Politur an.

3. Grüneberg (1841, März 22). Lichte aschgraue feinkörnige und feste Grundmasse, worin eine große Menge von Kugeln von Schrot- und Hirsekorngröße mit ihr fest verwachsen liegen, so dass sie auf der Bruchfläche des Gesteins auch ihre Bruchflächen zeigen. Sie sind größtenteils von grünlichgelber, doch zuweilen auch von schwärzlichgrauer Farbe und im Bruch uneben. Viel eingesprengtes Nickeleisen, das meistenteils in kleinen feinen Partien, doch zuweilen auch in 2 bis 3 Linien langen Körnern darin enthalten ist; auch verhältnismäßig viel Magnetkies, der meistenteils in einzelnen noch größeren Partien als das Nickeleisen, seltener in feinen Teilen vorkommt; beide öfter miteinander verbunden; an dem kleineren Stücke der Sammlung wird ein Schrotkorn-großes Korn von Magnetkies von Nickeleisen fast ganz umschlossen. Die Menge, in der sich der Magnetkies findet, zeichnet diesen Meteoriten ganz besonders aus. Er hat Ablösungsflächen, die aber nur zum Teil schwarz sind.

Dem Ch. von Grüneberg sehr ähnlich sind eine große Menge von Meteoriten, wenngleich nicht alle so viel Magnetkies enthalten, wie die von Cabarras-County, Mezö-Madaras, Tabor, Toulouse, Barbotan, Tipperary, Seres, Krasnoi-Ugol, Wessely u. s. w.

4. Mauerkirchen (1768, Nov. 20). Lichte graulichweiße zerreibliche Grundmasse mit sehr vielen eingewachsenen Kugeln von fast gleicher oder nur etwas mehr gelblicher Farbe als die Grundmasse. Die Kugeln sind untereinander meistens von gleicher lichter Farbe, nur zuweilen finden sich solche, die nach der Oberfläche etwas dunkler gefärbt sind; ganz gleichmäßig gefärbte dunklere Kugeln kommen nicht vor. Ungeachtet der nur zerreiblichen Grundmasse sieht man auf der Bruchfläche doch meistens nur den Bruch der Kugeln; dann fein eingesprengtes Nickeleisen, auch etwas ebenso beschaffenen Magnetkies.

Dem Ch. von Mauerkirchen sehr ähnlich sind die von Iowa (nicht zu unterscheiden), Linum, Apt, Bachmut, Lissa, Politz, New-Concord, Slobodka u. s. w.
\begin{center}
Mikroskopische Untersuchung.
\end{center}
\paragraph{}
Wenn man von diesen Meteoriten möglichst dünne Platten schleift, so werden oft mehrere der Gemengteil so durchsichtig, dass man unter dem Mikroskop ihre Struktur und Form erkennen kann. Diess gelingt zwar nur vollkommen bei den festeren, wie bei den Ch. von Erxleben, Klein-Wenden, Stauropol; bei andern, die weniger fest und mehr bröcklig sind, wie bei den Ch. von Aigle, Mauerkirchen, Timochin, nur teilweise, indem nur einzelne ihrer Gemengteil wie die Kugeln durchsichtig oder zum Teil durchsichtig werden; dennoch sind auch diese belehrend. Ich will daher das Aussehen einiger dieser Platten beschreiben und die Resultate angeben, die diese Art der Untersuchung ergeben hat.

1. Erxleben. Eine kleine, dünn geschliffene Plätte\footnote{Sie ist Fig. 7 Taf. III in ihrer natürlichen Größe dargestellt; die Fig. 1, 2, 3 sind Abbildungen einzelner Stellen derselben bei 140-maliger Vergrößerung. Ich verdanke das Plättchen noch dem verstorbenen Dr. Oschatz, er diese Art von Präparaten mit großer Geschicklichkeit anfertigte. Indessen werden sie jetzt von seinem Nachfolger, dem Optiker Krieg, hier in Berlin schon von gleicher Güte gemacht.} ließ schon bei schwacher Vergrößerung erkennen: eine Grundmasse, wasserhelle Kristalle, schwarze undurchsichtige Körner, eine grau aussehende Kugel und andere graue Partien. Die durchsichtigen Kristalle wie \emph{a} und \emph{b} in Fig. 1 und 2 Taf. III sind nur hier und da sichtbar; wo sie an die Grundmasse angrenzen, sind sie gewöhnlich nicht scharf und regelmäßig begrenzt; nur einmal auf der kleinen Beobachtungsplatte bildet ihr Umriss ein ziemlich regelmäßiges Sechseck \emph{a} in Fig. 1; wo sie dagegen an die schwarzen Körner angrenzen, da ist die Begrenzung gewöhnlich ganz scharf und geradlinig. Sie sind größtenteils ungefärbt, nur stellenweise sind sie, wie auch die Grundmasse, etwas olivengrün gefärbt, doch verläuft sich die Farbe nach allen Seiten und scheint mehr von außen eingedrungen zu sein. Die Grundmasse ist im Ansehen von diesen Kristallen wenig verschieden, sie unterscheidet sich eigentlich nur dadurch, dass ihre körnigen Zusammensetzungsstücke weniger groß als die Kristalle sind, doch gehen sie in der Größe in diese über. Die Zusammensetzungsflächen sind überall unregelmäßig laufende krumme Flächen, schwarz oder grau, wie dies aber auch der Fall ist bei den Zusammensetzungsflächen der Körner des Marmors von Carara, wenn man eine dünn geschliffene Platte dieser Substanz unter dem Mikroskop betrachtet. Dies rührt in diesem Fall teils von der schrägen Lage der Zusammensetzungsflächen gegen die Schlifffläche her, teils aber von einer großen Menge sehr Kleiner schwarzer Körner, die mehr oder weniger zusammenliegen und bei starker Vergrößerung deutlich zu sehen sind.\footnote{Siehe Fig. 4, die den Kristall \emph{b} in Fig. 1 bei 360-maliger Vergrößerung darstellt.}

Der schwarze Gemengteil findet sich in kleineren und größeren Körnern und Partien, die größeren mit unregelmäßigen, oft ganz zackigen Umrissen, die kleineren von mehr rundlicher Form. Letztere liegen in großer Menge neben den größeren in der Grundmasse, zum Teil sind sie auch in den durchsichtigen Kristallen eingewachsen.\footnote{Neben diesen finden sich auch zuweilen kleine Höhlungen mit einer Flüssigkeit und einer Luftblase angefüllt, wie in irdischen Kristallen. Vergl. Fig. 3, welche den Kristall \emph{a} Fig. 1 bei 360-maliger Vergrößerung darstellt.} Dieser schwarze Gemengteil zeigt, wenn man die Platte bei durchgelassenem Lichte betrachtet, keine oder eine nur sehr undeutliche Verschiedenartigkeit der Teile; schließt man aber dieses ab und betrachtet man ihn nur bei zurückgeworfenem Lichte, am besten bei hellem Sonnenlichte und bei stärkerer, etwa 140 maliger Vergrößerung, so sieht man, dass namentlich die größeren Körner häufig Gemenge von zwei bis drei Substanzen sind, einmal von einer, die auch jetzt noch schwarz erscheint, einer nun fast bleigrau und metallisch glänzenden, die das Nickeleisen ist (n Fig. 1 und 2) und einer dritten, die schwärzlichbraun matt, aber voller Sprünge oder wenigstens Stellen ist, die das Licht sehr stark reflektieren, und nach Vergleichung mit der direkt gesehenen polierten Platte, Magnetkies ist (m, Fig. 1 u. 2). Im reflektierten Lichte ohne Sonnenschein ist er unter dem Mikroskop von der schwarzen Masse kaum zu unterscheiden. Nickeleisen findet sich häufiger und in größeren Partien als Magnetkies, der im Ganzen selten solche bildet, wie in Fig. 2. Wo das Nickeleisen mit der schwarzen Substanz gemengt ist, sitzt diese gewöhnlich in einzelnen kleinen Partien an der Außenseite, und man findet wohl kaum größere Nickeleisenpartien ohne Verwachsung mit der schwarzen Masse, wenn auch größere schwarze Körner ohne damit verwachsene Eisen vorkommen. Regelmäßige Begrenzung zeigt Nickeleisen, Magnetkies und der schwarze Körper nicht; wo sie aber an die durchsichtigen Kristalle angrenzen, zeigen alle, wie angeführt, den regelmäßigen Eindruck der Kristalle, sind also alle später wie diese Kristallisiert.

Was dieser schwarze Körper sei, und ob er überhaupt einerlei ist, kann auch mit Bestimmtheit noch nicht ausgesprochen werden; denn wenn auch die meisten dieser Körner, und überall die kleineren, schwarz und völlig undurchsichtig sind, so scheinen doch einzelne größere ein schwaches dunkelgrünes Licht durchzulassen.\footnote{Eigentümlich ist der schwarze Körper \emph{c} in Fig. 1 Taf. III. Er zeigt nur an einer Stelle Nickeleisen, die ganze übrige Masse ist bräunlichgrün und ohne Metallglanz; dreht man aber die Platte um, so sieht man hier nur Nickeleisen. Das Korn zeigt also auf der einer Seite die schwarze, nicht metallische Masse und auf der andern Seite Nickeleisen, und die Grenze zwischen beiden muss also zufällig gerade der Schlifffläche parallel gehen.} Wird dadurch eher Verschiedenheit angedeutet? Es wäre möglich, und dann könnten vielleicht die schwarzen undurchsichtigen Körner Chromeisenerz sein, da in dem Ch. von Erxleben durch die Analyse etwas Chromoxyd nachgewiesen ist, das wahrscheinlich von dem Chromeisenerz herrührt, und dieses stets auch in sehr feinen Teilen undurchsichtig ist. Wofür dann aber die schwärzlichgrün durchscheinenden Teile zu halten sind, ist schwerer zu entscheiden. Am meisten würde man vielleicht geneigt sein, sie für Augit zu halten, da dieser durch die Analyse wahrscheinlich gemacht ist und auch ziemlich allgemein als Gemengteil der Chondrit angenommen wird, ohne ihn irgend bestimmt beobachtet zu haben; aber wo der Augit in den Meteorsteinen erkannt ist, wie in dem Eukrit von Juvenas, ist er in dünngeschliffenen Platten immer stark durchscheinend und braun; die sämtlichen schwarzen Körner aber für Chromeisenerz zu nehmen, dafür scheinen sie für die geringe Menge von Chromoxyd, die die Analyse stets nachgewiesen hat, in zu großer Menge vorhanden zu sein. Ich muss also die Deutung dieser schwärzlichgrünen Körner noch dahingestellt sein lassen.

Es bleiben nun noch die grauen Partien, die wahrscheinlich Durchschnitte von den Kugeln sind, wiewohl nur eine dieser Partien in dem untersuchten Plättchen (die Partie \emph{a} in Fig. 7)\footnote{Sie ist in Fig. 7 bei 140maliger Vergrößerung und zum Teil in Fig. 6 bei 360maliger Vergrößerung gezeichnet.} rund und scharf begrenzt ist; die andern unbestimmt begrenzten (\emph{d} in Fig. 1) scheinen Durchschnitte von nebeneinander liegenden Kugeln zu sein. In der einzelnen runden, scharf begrenzten Partie sieht man graue, untereinander parallele Streifen, die in einer durchsichtigen ungefärbten Masse enge nebeneinander liegen. Die grauen Streifen (Fig. 5) scheinen aus lauter grauen runden Körnern mit unbestimmten Umrissen zu bestehen, die in Reihen nebeneinander liegen, und zwischen denen dann wieder einzelne größere, schwarze, scharf begrenzte Körner von Chromeisenerz und feinere von Nickeleisen enthalten sind. Bei starker (360-maliger) Vergrößerung sieht es aber aus, als wären diese grauen Körner nur durch lauter neben- und übereinander liegende krumme Sprünge entstanden. Bei den andern grauen Partien \emph{d} in Fig. 1 liegen die grauen Körner unregelmäßiger, was wahrscheinlich davon herrührt, dass die Schnittfläche durch diese Kugeln nach anderen Richtungen als in Fig. 5 geht; sie sind aber ebenso mit den schwarzen Körnern gemengt. Was es für eine Bewandtnis mit diesen grauen Streifen hat, lasse ich unentschieden. Die späteren Beobachtungen zeigen, dass die Kugeln aus fasrigen Zusammensetzungstücken bestehen, die auch hier angedeutet zu sein scheinen.

Eine dünn geschliffene Platte von dem Ch. von Klein-Wenden war größer, man konnte mehr sehen; sie verhielt sich aber sonst wie die von dem Ch. von Erxleben; scharfe und geradlinige Umrisse der durchsichtigen Kristalle, wo sie an die schwarzen Partien, sei es an das Nickeleisen oder den Magnetkies oder die schwarze Substanz angrenzen, fanden sich auch hier. Kugeln mit grauen Streifen wie Fig. 5 Taf. III waren gerade nicht zu sehen, nur solche, die wie \emph{d} in Fig. 1 dunkle Flecken mit schwarzen Körnern von Chromeisenerz und Nickeleisen enthielten; außerdem fanden sich aber andere, die im Allgemeinen durchsichtig und ungefärbt, aber mit vielen schwarzen Sprüngen durchsetzt waren, deren zusammenhängende Schwärze sich aber bei starker Vergrößerung wie bei Fig. 4 in kleine, voreinander getrennte, schwarze Körner auflöste. Eine solche sehr große Kugel hatte das Ansehen von Fig. 8, die schwarzen dicken Sprünge waren untereinander ziemlich parallel; sie wiederholten sich dabei schnell und schlossen, durch Querrisse verbunden, die scharf begrenzten, durchsichtigen Teile ein.

Eine noch größere Platte von dem Ch. von Stauropol zeigte außer den gewöhnlichen Erscheinungen eine noch größere Menge von Olivin-Kristallen und von Kugeln. Die ersteren hatten zum Teil eine regelmäßige Form wie Fig. 3 u. 4 Taf. IV; sie waren durchsichtig, ungefärbt, mit schwarzen Sprüngen durchsetzt und hatten auch schwarze Körner und Partien eingeschlossen. Die Kugeln waren hier recht deutlich zweierlei Art, und diese zwei Arten schon ganz bestimmt mit bloßen Augen oder mit der Lupe auf der dünn geschliffenen Platte zu unterscheiden, da die einen stellenweise durchsichtig, die andern nur grau durchscheinend waren. Auf einem Stücke mit nur angeschliffener Fläche konnte man diesen Unterschied nicht wahrnehmen. Die durchsichtigen Kugeln schlossen sich ganz den Olivin-Kristallen an, sie schienen nur runde Kristalle oder runde Zusammenhäufungen von unausgebildeten oder mehr oder weniger ausgebildeten Kristallen zu sein. Die ersteren waren mit schwarzen, untereinander ungefähr parallelen Sprüngen durchsetzt, ähnlich wie bei Fig. 8 Taf. III,\footnote{Oder mehr wie bei Fig. 10 Taf. IV, einer Kugel aus dem Ch. von Timochin, bei welchem die Sprünge in den Kugeln gewöhnlich weitläufiger sind.} die andern hatten unregelmäßigere Sprünge, und bei den dritten waren die Kristalle durch die übrigen Gemengteil, wie Grundmasse, Nickeleisen, Magnetkies und die schwarzen Körner verbunden (wie Fig. 5 Taf. IV, wo die dazwischen eingeschlossene, metallische Substanz zufällig fast nur Magnetkies war, der sonst im Ch. von Stauropol nicht sehr häufig vorhanden ist). Diese letzteren Kugeln waren auch viel grösser als die andern, daher Fig. 5 (wie auch Fig. 3 und 4) nur bei 9Omaliger Vergrößerung gezeichnet sind. Diese Ansicht bestätigte die Untersuchung im polarisierten Licht, da die ersten Kugeln überall eine gleichmäßige Färbung, die zweiten und dritten an verschiedenen Zusammensetzungsstücken und Kristallen verschiedene, überall an den Grenzen scharf abschneidende Färbung zeigten. Die anderen Kugeln, welche mit bloßen Augen betrachtet, grau und nur durchscheinend erschienen, wie die Kugel in dem Ch. von Erxleben Fig. 5 Taf. III, waren auch unter dem Mikroskop wie diese mit grauen parallelen Streifen gestreift, nur deutlicher (ähnlich wie die Kugel in dem Ch. von Krasnoi-Ugol, Fig. 8 Taf. IV) und an den verschiedenen Stellen nach 2 Richtungen, die gegeneinander einen spitzen Winkel machten und aneinander scharf abschnitten. Andere Kugeln zeigten nur graue, mehr unbestimmt verlaufende Flecken mit schwarzen, scharf begrenzten Körnern. Zuweilen waren diese so dunkel und undurchsichtig, dass die hellen Räume dazwischen nur sehr klein waren. Eine solche ganz runde Kugel ist Fig. 6 Taf. IV dargestellt, sie ist noch dadurch ausgezeichnet, dass sie mit einem durchsichtigen, von schwarzen Flecken stellenweise ganz freien Ringe umgeben ist. Dicht neben ihr befindet sich eine kleinere, mehr unregelmäßig begrenzte, mit lichtern grauen und geraden parallelen Streifen (Fig. 7).

Die Patte von dem Ch. von Krasnoi-Ugol war im Allgemeinen sehr undurchsichtig, zeigte indessen doch die Kugeln sehr deutlich. Diese waren wiederum von der doppelten Art, die grau gestreiften waren aber hier in der Mehrzahl vorhanden. Sie lagen dann teils ganz einzeln in der Masse, teils waren sie zu mehreren zusammengehäuft, sich gegenseitig in der Ausbildung störend. Eine solche einzelne, recht runde Kugel, deren Streifen auch recht scharf begrenzt sind, ist die vorhin erwähnte, in Fig. 8 Taf. IV dargestellte Kugel; ihre Streifen haben die angegebene Lage, und äußerlich ist sie wieder mit einer ganz hellen Hülle umgeben. Neben ihr liegt eine Kugel der ersten Art, die aber viel kleiner ist. Eine Zusammenhäufung von Kugeln ist Fig. 9 dargestellt.

Auch die Platte von dem Ch. von Ensisheim ist sehr dunkel und schwarz. Hierin fand sich ein sehr großer, ziemlich regelmäßig begrenzter Kristall, der mit von Zeit zu Zeit wiederkehrenden, ziemlich parallelen Sprüngen nach zwei untereinander rechtwinkligen Richtungen durchsetzt ist, wie dies auch bei den irdischen Olivin-Kristallen öfter der Fall ist, bei denen dann die Sprünge parallel der Quer- und Längsfläche gehen.

Die Platten von den Ch. von Timochin, Ausson, Aigle und Mauerkirchen nahmen schon keine vollständige Politur an; nur die Kugeln und das Nickeleisen erschienen glänzend, und nur die ersteren waren durchscheinend, die übrige Masse gar nicht. Auch war es sehr schwer, Platten von einiger Größe zu erhalten, namentlich bei denen, der wie der Ch. von Mauerkirchen nur wenig und sehr fein verteiltes Eisen enthalten. Die Kugeln waren aber überall von der doppelten Art; eine Kugel der ersten Art, die sich in dem Ch. von Timochin fand, und in ihrer Art recht ausgezeichnet war, ist in Fig. 10 dargestellt. Die Ch. von New-Concord und Dharamsala gaben wieder Platten mit vollständigerer Politur, zeigten aber weiter nichts Neues.

Aus der mikroskopischen Untersuchung der dünn geschliffenen Platten ergibt sich also, dass die unter dem Mikroskop schwarz erscheinende Masse außer Nickeleisen, Magnetkies und Chromeisenerz wahrscheinlich noch eine andere schwarze Substanz enthält, deren Natur noch nicht gekannt ist, und ferner, dass die eingewachsenen Kugeln bestimmt zweierlei Art sind, teils solche, die nur runde, zerklüftete Kristalle und offenbar Olivin-Kristalle, ähnlich denen in dem Pallas-Eisen oder Zusammenhäufungen derselben sind, teils solche, die aus fasrigen Zusammensetzungsstücken bestehen, die, wie auch die Beobachtung dieser Kugeln auf der Bruchfläche der Meteoriten mit bloßen Augen oder mit der Lupe gelehrt hat, immer exzentrisch fasrig, nie radial fasrig sind.
\begin{center}
Chemische Beschaffenheit.
\end{center}
\paragraph{}
Kleine Splitter sowohl von der Grundmasse als von den Kugeln in der Platinzange gehalten und vor dem Lötrohr erhitzt, verändern wohl die Farbe und werden schwarz, schmelzen aber nicht.\footnote{Man hat wohl oft von einer Schmelzbarkeit der Masse der Chondrit gesprochen, Hausmann bei dem Ch. von Bremervörde, Dufrenoy bei dem von Château Renard, Damour bei dem von Montrejeau (Ausson); ich habe dies nie gefunden.} Nur das sehr fein geriebene Pulver schmilzt an den äußersten dünnen Rändern zu einer grünlichgrauen oder graulichgrünen Schlacke.\footnote{Das Pulver wird dazu bekanntlich befeuchtet, auf der Kohle zu einer dünnen Platte ausgebreitet, mit dem Lötrohr erhitzt und die nun zusammenhängende Platte mit der Platinzange vorsichtig gefasst, und in der Lötrohrflamme geglüht. Um das Pulver recht fein reiben zu können, wurde aus dem zuerst erhaltenen gröblich Pulver des Meteoriten das darin enthaltene Nickeleisen mit den Magneten ausgezogen.}

Wenn man kleine Stückchen dieser Meteoriten in Chlorwasserstoffsäure einige Zeit liegen lässt, so werden Nickeleisen und Magnetkies unter Entwickelung von Wasserstoff und Schwefelwasserstoff und rötlichgelber Färbung der Säure aufgelöst und die den Meteoriten bildenden Silicate zersetzt. Nach Verlauf von einigen Tagen ist die Säure schleimig geworden, und es hat sich ein Absatz von Kieselsäure am Boden des Gefäßes und auf den Stücken gebildet und auf der Oberfläche der Säure eine geringe Menge Schwefel abgeschieden.\footnote{Ich habe diese Abscheidung bei mehreren dieser Meteoriten bestimmt wahrgenommen, was die Anwesenheit von Magnetkies in diesen Meteorsteinen beweist. Ebenso sah sie auch Harris bei dem Ch. von Montrejeau (Ann. d. Chem. u. Pharm. 1859 B. 109, S. 183).} Wäscht man den Absatz von den Stücken ab, so erscheinen dieselben sehr bröcklig und porös; sie sind weiß und erdig geworden, und die Kugeln, die von der Säure weniger angegriffen werden, ragen daraus hervor; sie haben noch ihre Form und ihre graue Farbe behalten, wenn sie vorher so gefärbt waren, wie bei den Ch. von Erxleben und Aigle, und könnten bei gehöriger Vorsicht wohl ganz isoliert werden. Bei längerer Einwirkung der Säure wird aber der ganze Stein so mürbe, dass man ihn mit den Fingern gänzlich zu einem feinen Pulver zerdrücken kann. Dasselbe erscheint nun ganz weiß und enthält nur einzelne feine schwarze Teile, die man durch Feinreiben und Schlämmen wohl etwas konzentrieren, aber von den weißen Teilen nicht vollständig trennen kann. Sie sind wahrscheinlich nur Chromeisenerz; denn, wenn man den Rückstand vor dem Lötrohr mit Phosphorsalz schmilzt, so erhält man ein Glas, das, solange es heiß ist, durchsichtig ist und ein Kieselskelett einschließt, beim Erkalten aber opalisiert und eine wenn auch nur schwache doch deutliche chromgrüne Farbe erhält.

Betrachtet man die mit kalter Chlorwasserstoffsäure digerierten und abgewaschenen Bruchstücke unter der Lupe, so erkennt man auf der Oberfläche einzelne kleine, aber stark demantglänzende Kristalle. Ich habe diese bei vielen dieser Meteorit, ganz besonders aber bei Stücken von Aigle gesehen, die Monate lang in Chlorwasserstoffsäure gelegen hatten. Unter dem Mikroskop bei mäßiger Vergrößerung kann man, wenn man sie im reflektierten Lichte betrachtet, auch die Gestalt einzelner Flächen, von denen gerade das Licht zurückgestählt wird, erkennen, man sieht dann ganz scharf begrenzte Rhomben oder Deltoide, auch Rechtecke u. s. w., doch lässt sich danach die Form der Kristalle mit Sicherheit nicht bestimmen. Kocht man die Stücke mit Wasser oder auch mit kohlensaurem Natron, so verändern sich die Kristalle nicht im Geringsten; aber es gelang mir nicht, die Kristalle zu isolieren. Schabte man die Kristalle von der Oberfläche, auf der sie sitzen, ab, so waren sie in dem erhaltenen Pulver nicht wieder zu erkennen. Ich habe so leider über ihre Natur nichts weiter ausmachen können. Wenn man kleine Stückchen des Ch. von Aigle mit Chlorwasserstoffsäure kocht und dann sowie vorhin behandelt, so waren die Kristalle viel unvollkommener, es scheint also, dass sie nur in kalter Chlorwasserstoffsäure in der angegebenen Vollkommenheit erhalten werden können.\footnote{Ich habe diese Kristalle mehreren meiner Freunde gezeigt, die nie einen Zweifel darüber hatten, dass es Kristalle wären.}

Die chemische Zusammensetzung dieser Meteorit ist ungeachtet ihres oft verschiedenen Ansehens, doch sehr ähnlich, wie dies schon Berzelius gefunden hat, der sich, wie oben angeführt, mit dieser Abteilung der Meteorit besonders beschäftigt hat, und wie dies auch alle späteren Analysen ergeben haben. Rammelsberg hat in seiner Mineralchemie (S. 924) diese Analysen neu berechnet und sehr gut zusammengestellt. Sie sind sämtlich nach der zuerst von Berzelius eingeschlagenen Methode, wonach durch Digestion des feinen Pulvers mit heißer Chlorwasserstoffsäure die davon zersetzbaren Gemengteil von den davon nicht zersetzbaren getrennt werden, angestellt. Eine vollkommene Trennung der Gemengteil lässt sich, wie der Erfolg gezeigt hat, auf diese Weise nicht erreichen, da keiner der Gemengteil, das Chromeisenerz ausgenommen, in Chlorwasserstoffsäure vollkommen unlöslich, der eine nur mehr oder minder als der andere löslich ist, also eine stärkere oder schwächere Chlorwasserstoffsäure oder eine längere oder kürzere Digestion damit bei höherer oder geringerer Temperatur oft sehr bemerkenswerte Unterschiede in der Analyse eines Meteoriten hervorbringen können; dennoch gewährt diese Methode Anhaltspunkte, die für die Deutung der Resultate von Wichtigkeit sind, und ist daher, wenn auch unvollkommen, immer schätzbar. Ich will von den vorhandenen Analysen hier nach den Werken von Rammelsberg\footnote{Mineralchemie S. 925 u. s. f.} nur einige derselben, und zwar die folgenden 4 anführen, die Meteoriten der vier verschiedenen Gruppen betreffen:

1. Klein-Wenden nach Rammelsberg.

2. Chantonnay nach Berzelius.

3. Blansko nach Berzelius.

4. Kakowa nach Harris.

Allgemeine Zusammensetzung.

1|2\footnote{Das Verhältnis der metallischen zu den nicht metallischen Gemengteilen ist bei dieser Analyse von Berzelius nicht angegeben.}|3|4  
Nickeleisen 22,90 - 20,13 23,20  
Schwefeleisen\footnote{Das Schwefeleisen ist von Rammelsberg als Sulphuret Fe berechnet. Die Annahme als Magnetkies Fe$^{5}$Fe (Fe$^{7}$S$^{8}$) ändert die Rechnung nur wenig.} 5,61 - 2,97 13,14  
Chromeisenerz\footnote{Das Chromeisenerz ist als FeCr berechnet.} 1,04 - 0,63 0,56  
Silicate 70,45 - 76,27 63,21

Zusammensetzung des Nickeleisens.

Eisen 88,98 - 90,9 92,24  
Nickel 10,35 - 9,1 7,76  
Kobalt 10,35 - - -  
Kupfer 0,21 - - -  
Zinn 0,35 - - -  
Phosphor 0,11 - - -  
100. 100. 100.

Zusammensetzung der Silicate.

Magnesia 32,75 27,79 32,49 27,06  
Kalk 3,66 1,52 1,22 1,52  
Eisenoxydul 10,92 19,47 11,45 24,40  
Manganoxydul 0,08 0,76 0,59 -  
Nickeloxyd - 0,90 - 0,20  
Kupferoxyd 0,21 - - -  
Natron 0,41 1,24 0,98 1,31
Kali 0,53 1,24 0,24 0,21  
Tonerde 5,23 2,95 2,94 2,61  
Kieselsäure 46,18 44,16 48,95 44,68  
99,97 98,79 98,96 98,86

Diese Silicate bestehen in 100 Teilen aus:

|1|2|3|4| 
A. Zersetzbaren S.: 42,23 51,12 46,13 56,7  
B. Unzersetzbaren S.: 57,77 48,88 53,87 43,3

A. Zusammensetzung der zersetzbaren Silicate.

|Kl.-Wenden|Chantonnay|Blansko|Kakova|  
Magnesia 20,00 17,56 19,89 11,2  
Kalk 0,89 - - 0,7  
Eisenoxydul 4,53 14,72 6,88 24,4  
Manganoxydul 0,08 0,42 0,25 -  
Nickeloxyd - 0,23 - 0,2  
Natron - 0,50 0,48 -  
Kali - 0,50 0,24 -  
Tonerde - - 0,15 -  
Kieselsäure 16,72 16,67 18,20 19,5  
|42,22 50,10 46,09 56,0

B. Zusammensetzung der unzersetzbaren Silicate.

Magnesia 12,75 10,23 12,60 15,86  
Kalk 2,77 1,52 1,22 0,81  
Eisenoxydul 6,39 4,75 4,57 -  
Manganoxydul - 0,34 0,34 -  
Nickeloxyd 0,21 0,67 - -  
Natron 0,41 0,49 0,50\footnote{Kalihaltig.} 1,92  
Kali 0,53 0,25 - 0,26  
Tonerde 5,23 2,95 2,79 2,46  
Kieselsäure 29,46 27,49 30,75 21,74  
57,75 48,69 52,77 43,05

Sauerstoff 9,94\footnote{Ohne den Sauerstoff des Nickeloxyds und so auch bei den folgenden.}

Hieraus ergibt sich:

Dass das Nickeleisen der Chondrit in seiner chemischen Zusammensetzung mit dem für sich vorkommenden in dem Meteoreisen übereinstimmt, 

dass die Silicate als Basen vorzugsweise Magnesia und Eisenoxydul enthalten, und alle andern, wie Tonerde, Kalk, Kali, Natron sich nur in sehr geringer Menge finden,

dass die Menge der zersetzbaren und unzersetzbaren Gemengteil in allen Chondriten ungefähr gleich ist,

dass bei den zersetzbaren Silicaten die Basen fast allein aus Magnesia und Eisenoxydul in etwas gegeneinander verschiedenen Mengen bestehen, während die nicht zersetzbaren dieselben Basen mit kleinen Mengen von Natron, Kali, Kalk und besonders von Tonerde enthalten.

Bei den zersetzbaren Silicaten ist der Sauerstoff der Basen als gleich mit dem Sauerstoff der Kieselsäure anzunehmen. Wenn in der Tat der Sauerstoff der Basen ein klein wenig grösser als der der Kieselsäure ist, so erklären dies Berzelius und Rammelsberg gewiss sehr richtig dadurch, dass wohl immer noch etwas Nickeleisen in diesem Teile der Chondrit zurückgeblieben ist, der mit dem Magnete nicht vollständig hat ausgezogen werden können, und dass durch die Säure auch schon eine gewisse Menge von dem doch immer nur schwer, nicht völlig unzersetzbaren Teile zersetzt ist, dessen Basen, wie die kleinen Mengen von Tonerde, Kalk und die Alkalien beweisen, nun hier auftreten, während die analytische Methode zur Folge hat, dass immer etwas Kieselsäure von dem zersetzbaren Teile bei dem unzersetzbaren bleibt.

Bei dem unzersetzbaren Gemengteil ist der Sauerstoff der Basen meistenteils ungefähr halb so groß als der der Säure.

Berzelius schloss daraus, dass der zersetzbare Gemengteil Olivin sei. Das Verhältnis des Eisenoxyduls zur Magnesia ist bei dem Olivin der verschiedene Chondrit etwas verschieden, doch ist dies auch bei dem terrestrischen Olivin der verschiedenen Fundörter der Fall. Den unzersetzbaren Gemengteil hielt er wiederum für ein Gemenge und zwar eines Leucitartigen Minerals, welches die Tonerde und die Alkalien, und eines Augit-artigen, welches die andern Basen, Magnesia, Kalk und Eisenoxydul enthielt,\footnote{Pongendorffs Ann. 1836 B. 15, S. 221.} für welchen ersteren indessen Rammelsberg, wegen seiner Auflöslichkeit in Säuren ein feldspatartiges Mineral und zwar den Labrador annahm,\footnote{Mineralchemie S. 931.} der zwar auch, aber schwerer als der Leucit zersetzbar ist und dessen Sauerstoffverhältnis sich nicht viel von dem des Leucits entfernt.\footnote{Der Sauerstoff der Basen zu dem der Kieselsäure ist bei ihm wie 1:1 1/2.}

Dieser Annahme ist man nun später teils gefolgt, Teils hat man wieder neue Kombinationen gemacht. So halten es Chancel und Moitessier\footnote{\emph{Comptes rendus}, 1859 t. 47, p. 267 et p. 479.} bei der Analyse des Ch. von Montrejeau (Ausson) für wahrscheinlich, dass darin nicht Labrador und Augit, sonders Oligoklas und Hornblende enthalten sei, Sartorius\footnote{Ber. d. Wiener Akad. 1859, B. 34 (S. 3).} berechnet in dem unzersetzbaren Gemengteil des Ch. von Kakova nach der Analyse von Harris 82,17 Magnesia-Wollastonit und 17,4 Anorthit, und Abich\footnote{Bulletin de l'acad. d. sciences de St. Petersbourg 1860, t. 2, p. 404.} nimmt an, dass die Silicate in dem Ch. von Stauropol Hyalosiderit, Olivin und Labrador seien und ersterer den zersetzbaren, letztere den unzersetzbaren Gemengteil ausmachen.

Zu allen diesen Annahmen gibt aber die mineralogische Untersuchung, wie das eben Angeführte dartut, nicht das mindeste Anhalten. Labrador, Oligoklas und Anorthit mit ihren tafelartigen Kristallen und der so charakteristischen Streifung auf den Spaltungsflächen des Querbruchs zeigt sich nie, Augit und Hornblende müssten bei ihrer Unersetzbarkeit durch Säuren doch nach Auflösung des zersetzbaren Gemengteil in dem unzersetzbaren zu finden sein, in welchem sich wohl etwas Chromeisenerz, aber nichts anderes erkennen lässt. Magnesia- Wollastonit ist noch nie beobachtet, müsste aber wie das Wollastonit durch Säuren zersetzbar sein, und könnte sich nicht, ebenso wenig wie der Anorthit in dem unzersetzbaren Gemengteil finden, was schon Wöhler bemerkt hat. Olivin kann bei seiner leichten Zersetzbarkeit auch nicht in dem unzersetzbaren Gemengteil vorkommen und ihn außerdem in einem Gemenge mit Hyalosiderit anzunehmen, ist nicht zu billigen, da beide isomorph sind und isomorphe Körper in einem Gemenge nie beobachtet sind und auch nicht vorkommen können, da man nicht einsieht, warum sich nicht beide hätten, verbinden und in diesem Falle einen Olivin von mittlerem Eisengehalt hätten bilden sollen.\footnote{Der Beweis, den Abich für die Richtigkeit seiner Annahme aus dem Umstand entnimmt, dass die Summe der spezifischen Gewichte der angenommenen Gemengteil mit den spezifischen Gewichten des Steins stimmt, ist scheinbar, da dergleichen Berechnungen bei den Gebirgsarten aus vielen Gründen noch nie sichere Resultate gegeben haben.}

Man sieht, zu welcher Willkür es führt, wenn man Mineralien aus der Analyse eines Mineralgemenges berechnet, die man nicht gesehen hat, und noch dazu das chemische Verhalten dieser Mineralien gar nicht berücksichtigt. Die einzigen Gemengteil, die man mit Sicherheit in diesen Meteoriten annehmen kann, weil man sie sehen kann, sind außer dem Nickeleisen und Magnetkies nur Olivin und Chromeisenerz. Woraus die Grundmasse besteht, ob der schwarze Gemengteil nur Chromeisenerz oder ein Gemenge desselben mit einer anderen Substanz sei, wie die mikroskopische Untersuchung wahrscheinlich macht, und woraus ganz besonders die Kugeln bestehen, wissen wir nicht. Freilich sind diese schon analysiert worden, aber nur in einer frühen Zeit von Howard,\footnote{Philos. transactions 1802 und daraus in Gilberts Annalen B. 13, S. 311.} dem wir überhaupt die erste Analyse eines Meteoriten verdanken. Derselbe trennte bei der Analyse des Ch. von Benares die Kugeln von der Grundmasse und untersuchte beide besonders, nachdem er aus den letzteren das Nickeleisen mit den Magneten ausgezogen hatte, und fand nun beide ungefähr gleich zusammengesetzt, nämlich:

||in den Kugeln|in der Grundmasse|  
Magnesia 15,00 18,00  
Nickeloxyd 2,5 2,5  
Eisenoxyd 34,00 34,00  
Kieselsäure 50,00 48,00  

Später sind sie bei der Analyse wenig beachtet. Berzelius stellte bei der Analyse des Ch. von Blansko aus Mangel an Material nur einige Versuche mit ihnen an, aus denen er indessen glaubte denselben Schluss wie Howard ziehen zu können, dass sie mit der übrigen Masse des Meteoriten gleich zusammengesetzt wären.\footnote{Pongendorffs Annalen 1834, B. 33, S. 21.} Er fand dabei, dass ein Teil ihres Pulvers gelatiniere, ein anderer von der Säure gar nicht verändert werde. In der neusten Zeit glaubt zwar Damour, dem wir auch eine sorgfältige Analyse des an diesen Kugeln so reichen Meteoriten von Montrejeau (Ausson) verdanken,\footnote{\emph{Comptes rendus} 1859 t. 49, p. 31.} dass er eine Analyse der Kugeln gegeben habe, indem er der Meinung ist, dass daraus der ganze von Säuren nicht oder schwer zersetzbare Teil des Meteoriten bestehe, den er besonders untersucht hat; dies ist aber doch noch zu beweisen, da er die Kugeln für die Analyse nicht besonders ausgesucht hat. Diesen unzersetzbaren Teil, den er allein als aus den Kugeln bestehend annimmt und den er ähnlich wie bei den oben angeführten Analysen zusammengesetzt fand, hält er aber nicht für ein einfaches Mineral, sondern für ein Gemenge von Augit und Feldspat, berechnet somit wieder Mineralien, die gar nicht in den Kugeln zu erkennen sind und noch dazu ein Gemenge, das nie beobachtet ist, was ihm auch Leymerie\footnote{\emph{Comptes rendus} 1859, t. 49, p. 247.} vorwirft, der aus der Analyse von Damour nur den Schluss ziehen zu können glaubt, dass die Kugeln ein Silicat von Magnesia und Eisenoxydul mit höherem Kieselsäuregehalt als im Olivin seien, wenn nicht dieser höhere Kieselsäuregehalt von anhängender Grundmasse herzuleiten sei.

Aus der mikroskopischen Untersuchung der Kugeln geht hervor, dass dieselben zweierlei Art ist, die einen vielleicht nur Olivin, die andern mit fasriger Struktur davon verschieden, ein wahrscheinlich Tonerde-haltiges Doppelsilicat. Es bleiben demnach von wirklich bestimmten Gemengteilen in den Chondriten immer nur noch Nickeleisen, Magnetkies, Olivin und Chromeisenerz, von nicht bestimmten die fragliche Grundmasse, der schwarze Gemengteil neben dem Chromeisenerz und die Kugeln mit fasriger Struktur, wenn die übrigen Olivin sind.
\clearpage
\section{Abbildungen.}
\clearpage
\setlength\intextsep{0pt}
\pagestyle{fancy}
\fancyhf{}
\rhead{Tafel 1.}
\cfoot{\thepage}
\begin{figure}[p]
\includegraphics[scale=0.5,keepaspectratio]{Figures/Table1/Fig1.png}\tiny 1
\includegraphics[scale=0.5,keepaspectratio]{Figures/Table1/Fig2.png}\tiny 2
\includegraphics[scale=0.5,keepaspectratio]{Figures/Table1/Fig3.png}\tiny 3
\includegraphics[scale=0.5,keepaspectratio]{Figures/Table1/Fig5.png}\tiny 5
\tiny   6
\includegraphics[scale=0.5,keepaspectratio]{Figures/Table1/Fig6.png}
\includegraphics[scale=0.5,keepaspectratio]{Figures/Table1/Fig8.png}\tiny 8
\includegraphics[scale=0.5,keepaspectratio]{Figures/Table1/Fig9.png}\tiny 9
\includegraphics[scale=0.5,keepaspectratio]{Figures/Table1/Fig10.png}\tiny 10
\includegraphics[scale=0.5,keepaspectratio]{Figures/Table1/Fig4.png}\tiny 4
\includegraphics[scale=0.5,keepaspectratio]{Figures/Table1/Fig7.png}\tiny 7
\includegraphics[scale=0.5,keepaspectratio]{Figures/Table1/Fig11.png}\tiny 11
\includegraphics[scale=0.5,keepaspectratio]{Figures/Table1/Fig12.png}\tiny 12
\end{figure}
\clearpage
\rhead{Tafel 2.}
\cfoot{\thepage}
\begin{figure}[p]
\tiny 1
\includegraphics[scale=0.5,keepaspectratio]{Figures/Table2/Fig1.png}
\includegraphics[scale=0.5,keepaspectratio]{Figures/Table2/Fig4.png}\tiny 4
\includegraphics[scale=0.5,keepaspectratio]{Figures/Table2/Fig2.png}\tiny 2
\includegraphics[scale=0.5,keepaspectratio]{Figures/Table2/Fig3.png}\tiny 3
\tiny 7
\includegraphics[scale=0.5,keepaspectratio]{Figures/Table2/Fig7.png}
\includegraphics[scale=0.5,keepaspectratio]{Figures/Table2/Fig6.png}\tiny 6
\includegraphics[scale=0.5,keepaspectratio]{Figures/Table2/Fig5.png}\tiny 5
\includegraphics[scale=0.5,keepaspectratio]{Figures/Table2/Fig8.png}\tiny 8
\includegraphics[scale=0.5,keepaspectratio]{Figures/Table2/Fig9.png}\tiny 9
\end{figure}
\clearpage
\rhead{Tafel 3.}
\cfoot{\thepage}
\begin{figure}[p]
\includegraphics[scale=0.5,keepaspectratio]{Figures/Table3/Fig1.png}\tiny 1
\includegraphics[scale=0.5,keepaspectratio]{Figures/Table3/Fig2.png}\tiny 2
\includegraphics[scale=0.5,keepaspectratio]{Figures/Table3/Fig3.png}\tiny 3
\includegraphics[scale=0.5,keepaspectratio]{Figures/Table3/Fig7.png}\tiny 7
\includegraphics[scale=0.5,keepaspectratio]{Figures/Table3/Fig4.png}\tiny 4
\includegraphics[scale=0.5,keepaspectratio]{Figures/Table3/Fig5.png}\tiny 5
\includegraphics[scale=0.5,keepaspectratio]{Figures/Table3/Fig6.png}\tiny 6
\includegraphics[scale=0.5,keepaspectratio]{Figures/Table3/Fig8.png}\tiny 8
\includegraphics[scale=0.5,keepaspectratio]{Figures/Table3/Fig9.png}\tiny 9
\includegraphics[scale=0.5,keepaspectratio]{Figures/Table3/Fig10.png}\tiny 10
\includegraphics[scale=0.5,keepaspectratio]{Figures/Table3/Fig11.png}\tiny 11
\includegraphics[scale=0.5,keepaspectratio]{Figures/Table3/Fig12.png}\tiny 12
\end{figure}
\clearpage
\rhead{Tafel 4.}
\cfoot{\thepage}
\begin{figure}[p]
\includegraphics[scale=0.5,keepaspectratio]{Figures/Table4/Fig1.png}\tiny 1
\includegraphics[scale=0.5,keepaspectratio]{Figures/Table4/Fig2.png}\tiny 2
\includegraphics[scale=0.5,keepaspectratio]{Figures/Table4/Fig3.png}\tiny 3
\includegraphics[scale=0.5,keepaspectratio]{Figures/Table4/Fig4.png}\tiny 4
\includegraphics[scale=0.5,keepaspectratio]{Figures/Table4/Fig5.png}\tiny 5
\includegraphics[scale=0.5,keepaspectratio]{Figures/Table4/Fig6.png}\tiny 6
\includegraphics[scale=0.5,keepaspectratio]{Figures/Table4/Fig7.png}\tiny 7
\includegraphics[scale=0.5,keepaspectratio]{Figures/Table4/Fig8.png}\tiny 8
\includegraphics[scale=0.5,keepaspectratio]{Figures/Table4/Fig9.png}\tiny 9
\includegraphics[scale=0.5,keepaspectratio]{Figures/Table4/Fig10.png}\tiny 10
\end{figure}
\clearpage
\end{document}
