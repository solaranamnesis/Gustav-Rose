\documentclass[a4paper, 11pt, oneside]{article}
\usepackage[utf8]{inputenc}
\usepackage[T1]{fontenc}
\usepackage[ngerman]{babel}
\usepackage{fbb} %Derived from Cardo, provides a Bembo-like font family in otf and pfb format plus LaTeX font support files
\usepackage{booktabs}
\setlength{\emergencystretch}{15pt}
\usepackage{fancyhdr}
\usepackage{graphicx}
\graphicspath{ {./} }
\begin{document}
\begin{titlepage} % Suppresses headers and footers on the title page
	\centering % Centre everything on the title page
	%\scshape % Use small caps for all text on the title page

	%------------------------------------------------
	%	Title
	%------------------------------------------------
	
	\rule{\textwidth}{1.6pt}\vspace*{-\baselineskip}\vspace*{2pt} % Thick horizontal rule
	\rule{\textwidth}{0.4pt} % Thin horizontal rule
	
	\vspace{1.5\baselineskip} % Whitespace above the title
	
	{\scshape\LARGE Beschreibung und Einteilung der Meteoriten\\auf Grund der Sammlung im Mineralogischen Museum zu Berlin.}
	
	\vspace{1\baselineskip} % Whitespace above the title

	\rule{\textwidth}{0.4pt}\vspace*{-\baselineskip}\vspace{3.2pt} % Thin horizontal rule
	\rule{\textwidth}{1.6pt} % Thick horizontal rule
	
	\vspace{1\baselineskip} % Whitespace after the title block
	
	%------------------------------------------------
	%	Subtitle
	%------------------------------------------------
	
	{\scshape Von Gustav Rose.} % Subtitle or further description
	
	\vspace*{1\baselineskip} % Whitespace under the subtitle
	
    {\scshape\small Aus den Abhandlungen der Königliche Akademie der Wissenschaften zu Berlin 1863.\\Mit vier Kupfertafeln.} % Subtitle or further description
    
	%------------------------------------------------
	%	Editor(s)
	%------------------------------------------------
    \vspace*{\fill}

	\vspace{1\baselineskip}

	{\small\scshape Berlin. 1864.}
	
	{\small\scshape{In Kommission bei F. Dümmlers Verlags-Buchhandlung Harrwitz und Gossmann.}}
	
	\vspace{0.5\baselineskip} % Whitespace after the title block

    \scshape Internet Archive Online Edition  % Publication year
	
	{\scshape\small Namensnennung Nicht-kommerziell Weitergabe unter gleichen Bedingungen 4.0 International} % Publisher
\end{titlepage}
\setlength{\parskip}{1mm plus1mm minus1mm}
\clearpage
\tableofcontents
\clearpage
\section{Geschichte der Meteoritensammlung in dem mineralogischen Museum von Berlin auf Grund deren die neue Einteilung der Meteoriten gemacht ist.}
\paragraph{}
Die Meteoritensammlung macht einen besonderen Teil des mineralogischen Museums der Berliner Universität aus. Als bei der Gründung der Universität im Jahre 1810 auch das mineralogische Museum durch Übernahme der Mineraliensammlung der früheren General-Bergbau-Direktion gegründet wurde, waren die wenigen Meteoriten, die sich in derselben befanden, noch nicht getrennt und mit den übrigen Mineralien vereinigt. Wie viele Meteoriten sich schon damals in ihr befanden, lässt sich nicht angeben, da darüber die Nachweisungen fehlen, indessen enthielt sie doch schon manche kostbare Stücke, wie ein großes prachtvolles Exemplar von dem Pallas-Eisen, das in einer Sammlung Russischer Mineralien enthalten war, die Kaiser Alexander I dem Könige Friedrich Wilhelm III im Jahre 1803 zum Geschenk gemacht hatte, so wie ein großes Stück von dem Durango-Eisen, welches Al. von Humboldt aus Mexiko mitgebracht und dem damaligen Direktor Dietrich Karsten für die Sammlung übergeben hatte. Weiß, der nach Karstens Tode Direktor des mineralogischen Museums wurde, hatte ein großes Interesse für die Meteoriten und ließ keine Gelegenheit vorübergehen, die sich ihm zur Erwerbung von Meteoriten darbot, doch fand sich dieselbe im Anfang, wo das Interesse für die Meteoriten überhaupt noch nicht so lebhaft war wie jetzt, nicht häufig. Die erste größere Bereicherung erhielt das Museum erst durch den Ankauf der Mineraliensammlung von Klaproth nach dessen im Jahre 1817 erfolgten Tode, indem sich darin nach Weglassung aller Arten, die sich später als unecht erwiesen haben, Steinmeteorit von 12 Fundörtern und Eisenmeteorit von 5 Fundörtern befanden, und mehrere derselben in mehreren Exemplaren vertreten waren. In dem im Jahre 1826 vollendeten Kataloge des Museums sind Meteorit von 31 Fundörtern aufgeführt, und zwar Steinmeteorit von 21 und Eisenmeteorit von 9 Fundörtern. Aber schon im nächsten Jahre vermehrte sich die Sammlung um mehr als das Doppelte durch das Vermächtnis Chladni's, wodurch die ganze berühmte Meteoritensammlung dieses um die Meteoritenkunde so verdienten Gelehrten dem Berliner Museum zufiel.\footnote{Chladni starb den 3. April 1827 auf einer Reise in Breslau. Er stand in den freundschaftlichsten Beziehungen zu dem damaligen Direktor des Berliner Museums, wie überhaupt zu den Berliner Gelehrten, und dies Verhältnis hatte ihn bei dem Wunsche seine Sammlung gemeinnützig zu machen, den er in seine Testamente ausdrücklich ausgesprochen hatte, wohl besonders bewogen, seine Sammlung dem Berliner Museum zu vermachen.} Sie bestand in Steinmeteoriten von 31 und Eisenmeteoriten von 10 verschiedenen Fundörtern, unter denen 18 neue Meteorit sich befanden.

Durch den Ankauf der Sammlung des Medizinal-Raths Bergemann im Jahre 1837 erhielt das Museum einen Zuwachs an Steinmeteoriten von 9 und von Eisenmeteoriten von 4 Fundörtern, doch waren darunter nur 2 neue Fundörter. Die späteren Erwerbungen geschahen nun nur durch Kauf, Tausch oder Schenkung einzelner Meteorit, wobei vor Allem das große Verdienst hervorzuheben ist, welches sich Al. von Humboldt durch die Schenkung so vieler ausgezeichneter Meteorit um das Museum erworben hat. Bei dem Tode des Professor Weiß im Jahre 1856 belief sich die Zahl der verschiedenen Meteoriten auf 90; sie ist seit dieser Zeit auf 176 gestiegen.\footnote{Wobei die mir zweifelhaften Eisenmeteoriten von Scriba, Hemalga, Newstead, Livingstone und Melrose, die in den Katalogen von der Wiener, Göttinger und Londoner Sammlung aufgeführt werden, nicht gerechnet sind.} Einen großen Zuwachs erhielt sie noch in der neuesten Zeit\footnote{Noch vor der Lesung des dritten Teils dieser Abhandlung.} durch den Ankauf einer ganzen Meteoritensammlung vom Prof. Shepard in New Haven in den Vereinigten Staaten, zu welchem die Akademie auf das liberalste die Mittel bewilligte.\footnote{Vergl. die Monatsberichte der Akademie von 1862, S. 644.} Die Sammlung stammte zum Teil aus der großen Meteoritensammlung des Prof. Lawrence Smith in Louisville, V. St., und enthielt neben vielem Neuen einzelne Stücke von bedeutender Größe wie ein fast vollständiges Exemplar von dem M. von New Concord von 26 Pfund und 24,3 Loth und eine große Platte von dem Toluca-Eisen mit vielen Einschlüssen, die über einen Fuß lang und einen halben Fuß breit ist.

Von vielen Seiten aufgefordert, ein Verzeichnis der in dem Berliner mineralogischen Museum befindlichen Meteoriten bekannt zu machen, schien es mir zweckmäßig in diesem Verzeichnis nicht, wie man gewöhnlich zu tun pflegt, die Eisen- und Steinmeteorit in der zufälligen Ordnung ihrer Fall- oder Fundzeit aufzuführen, sondern das Gleichartige zusammenstellend, sie nach ihrer mineralogischen Beschaffenheit zu ordnen. Ich fand zu einer solchen Anordnung umso mehr Veranlassung, als ich beabsichtigte, mit einem solchen Verzeichnis eine neue Aufstellung der Meteoriten des Berliner Museums vorzunehmen. Ich habe deshalb sämtliche Meteoriten des Museums genau untersucht und dazu sämtliche Stein- und Eisenmeteorit anschleifen lassen, und letztere geätzt, da man nur auf diese Weise bei den ersteren einen Überblick über die Gemengteil erhalten, bei den letzteren die Struktur erkennen kann, eine Arbeit, die lange aufhielt. Außerdem hatte ich von einem großen Teil der ersteren dünne Platten schleifen lassen, und von den geätzten Flächen der letzteren Hausenblasenabdrücke gemacht, um sie unter dem Mikroskop zu beobachten, und bin nun dadurch zu den Resultaten gelangt, die ich mir erlaube, hiermit der Akademie vorzulegen.

Systematische Anordnungen der Meteoriten sind schon von Partsch,\footnote{Die Meteoriten oder die vom Himmel gefallenen Steine und Eisenmassen im k. k. Hof-Mineralien-Kabinette zu Wien 1843, S. 162.} Shepard\footnote{\emph{Report on American meteorites} (from the Amer. Journ. of Science and arts, 2. Ser). New Haven 1848 p. 16.} und in der neusten Zeit von Reichenbach\footnote{Anordnung und Einteilung der Meteoriten in Pongendorffs Annalen 1859, B. 107, S. 155.} versucht worden. Partsch teilt die Meteoriten ein zuerst in Stein- und Eisenmeteorit, und letztere in normale und anomale; eine Einteilung und Benennung, die Reichenbach sehr tadelt, da man bei Meteoriten von normal und anomal nicht reden könne. Vergleicht man aber die Reihung selbst, so ist diese sehr naturgemäß. Zu den anomalen rechnet er nur 4, die von Alais, Capland, Chassigny und Simonod, von denen die drei ersteren allerdings auch von besonderer Art sind. Den von Simonod kenne ich nicht, er wird von Reichenbach als Meteorit ganz verworfen.\footnote{A. a. O. S. 163.} Die normalen werden in 2 Abteilungen geteilt, in solche die kein metallisches Eisen enthalten, wie a) die von Juvenas, Stannern, Konstantinopel, Jonzac, die nach der damaligen Annahme aus Augit und Labrador bestehen und b) die von Bialystock, Loutolax, Nobleborough und Mässing, die außerdem noch Olivin enthalten und ein breccienartiges Ansehen haben, worauf dann die große Schaar derer folgt, die Eisen eingesprengt enthalten. Hätte Partsch die ganze erstere Abteilung noch zu den anomalen gerechnet, so wäre die Einteilung noch passender gewesen, und hätte er die Meteoriten der ersteren Abteilung ungewöhnliche, die der letzteren gewöhnliche genannt, so würden diese Ausdrücke Reichenbach vielleicht weniger Gelegenheit zum Anstoß gegeben haben. Aber die Einteilung ist doch immer noch nicht näher gerechtfertigt und zu unbestimmt.

Das Meteoreisen teilt Partsch in ästiges und derbiges; das erstere erhält durch eingemengten Olivin eine schwammige Gestalt, das letztere hat eine unbestimmte Form und geringe Beimengungen. Zu den ersteren gehören die Meteoriten von Atacama, Krasnojarsk (das Pallas-Eisen) und von Brahin, zu den letzteren alle übrigen, die nun noch weiter, je nachdem sie durch Ätzung mehr oder weniger deutliche oder auch gar keine Widmanstättenschen Figuren geben, eingeteilt werden.

Die Einteilung von Shepard bezieht sich zwar hauptsächlich nur auf den amerikanischen Meteoriten, nimmt aber doch auch auf einige ausländische Rücksicht. Er teilt den Meteoriten ebenfalls zuerst ein in Eisen- und Steinmeteorit und beide dann weiter wie folgt:

I. Klasse: Eisenmeteorit.

1. Ordnung: Dehnbare und gleichartige.

1. Sekt.: Reine (Scriba, Walker County).

2. Sekt.: Legierte.

a) Feinkörnige (Green County, Texas, Dickson County, Burlington).

b) Grobkörnige (De Kalb, Ashville, Guildford, Carthago).

2. Ordnung: Dehnbare und ungleichartige.

1. Sekt.: Blasig-olivinige (Krasnojarsk).

2. Sekt.: Blasig-pyritische (Cambria).

3. Sekt.: Pyritisch-graphitische (Cocke County).

3. Ordnung: Spröde.

1. Sekt.: Reine (Redford County, Randolph County).

2. Sekt.: Legierte (Otsego County).

II. Classe: Steinmeteorit.

1. Ordnung: Trachytische.

1. Sekt.: Olivinige.

a) Grobkörnige (Weston, Richmond).

b) Feinkörnige (Nobleboro, Little Piney).

2. Sekt.: Augitisch (Juvenas).

3. Sekt.: Chadnitisch (Bishopville).

4. Sekt.: Kohligen (Cold Bokkeveld).

2. Ordnung: Trappartige.

1. Sekt.: Gleichartige (Chantonnay).

2. Sekt.: Porphyrartige (Renazzo).

3. Ordnung: Bimmsteinartige (Waterville).\footnote{Nicht meteorisch.}

Reichenbach gibt wohl den, meiner Meinung nach einzig richtigem Weg zur Einteilung der Meteoriten an, befolgt ihn aber selbst nicht. Er sagt, es wäre am natürlichsten, der Meteorit nach der Verschiedenheit der Mineralspezies, die sie enthalten, einzuteilen, da wir diese aber noch zu wenig kennen, so müssen wir es so machen wie die Botaniker bei den natürlichen Systemen der Pflanzen und die Meteoriten nach der allgemeinen Ähnlichkeit reihen. Er teilt demnach dieselben in 9 Sippen und jede wieder in verschiedene Gruppen, indem er mit den eisenfreien Steinmeteoriten von Langres (Chassigny), Bishopville, Jonzac, Juvenas, Stannern, Konstantinopel anfängt, durch die eisenhaltigen zu den Eisenmeteoriten fortgeht, die noch Olivin enthalten, und mit den aus fast reinem Eisen bestehenden Meteoriten schließt. Es kann nicht fehlen, dass ein so scharfblickender Kenner der Meteoriten, wie Baron Reichenbach, nicht eine Menge neuer interessanter Zusammenstellungen macht und Beziehungen zwischen Meteorsteinen hervorhebt, die früher nicht beachtet waren; sein System ist aber doch nur, wie er selbst das Verfahren der Botaniker nennt, ein „geistreiches Tatonnement“, es ist ihm derselbe Vorwurf zu machen, den er dem Partschen Systeme macht, es fehlt ihm ein Einteilungsprinzip; wir scheinen doch hinreichend in der Kenntnis der Meteoriten vorgerückt, um vollständig die strengen Grundsätze in Anwendung bringen zu können, die uns bei der Einteilung der Gebirgsarten, mit denen die Meteoriten doch vollständig zu vergleichen sind, leiten. Wie man dort aus einem jeden selbstständigen Gemenge eine besondere Gebirgsart macht, so muss man es auch hier tun, und wenn man allerdings auch noch nicht vollständig alle Gemengteil der Meteoriten genau kennt, so weiß man davon doch so viel, um das Zusammengehörige zusammenstellen zu können.

Ich behalte zuerst die alte Einteilung in Eisen- und Steinmeteorit bei, je nachdem der Meteorit nur oder vorzugsweise aus Eisen, und zwar Nickeleisen, oder vorzugsweise aus einem Gemenge von Silicaten bestehen, in denen das Nickeleisen nur untergeordnet oder gar nicht enthalten ist.

I. Die Eisenmeteorit machen 3 Arten aus, Meteoritenarten kann man sie nennen, wie man in der Petrographie Gebirgsarten oder Felsarten sagt.

Die 1. Art besteht aus Nickeleisen, das nur in geringer Menge mit einigen Eisenverbindungen gemengt ist; ich nenne sie Meteoreisen.

Die 2. Art besteht aus demselben Meteoreisen, worin Kristalle von Olivin porphyrartig eingewachsen sind. Der von Pallas am Jenisei gefundene Eisenmeteorit war der erste der Art, den man kennen lernte; er ist bekannt unter dem Namen Pallas-Eisen und bildet noch immer einen Hauptrepräsentanten dieser Art; ich schlage daher vor, die ganze Art Pallasit zu nennen.

Die 3. Art ist ein körniges Gemenge von Meteoreisen und Magnetkies mit Olivin und Augit. Ich nenne sie Mesosiderit von $\mu\varepsilon\sigma$o$\varsigma$ in der Mitte stehend und $\sigma\iota\delta\eta\varrho$o$\varsigma$ Eisen, da sie aus einer ziemlich gleichen Menge von metallischen Eisenverbindungen und Silicaten besteht und so gewissermaßen in der Mitte zwischen den Eisen- und Steinmeteoriten steht.

II. Die Steinmeteorit sind in 7 Arten zu teilen, für die ich die folgenden Namen vorschlage:


1. Chondrit (von $\chi$o$\nu\delta\varrho$o$\varsigma$, die kleine Kugel). Sie ist die erste und hauptsächlichste Art, die den größten Teil der Steinmeteorit enthält. Sie ist durch kleine Kugeln ausgezeichnet, die aus einem noch nicht bestimmten Magnesia-Silicat bestehen und in einem feinkörnigen Gemenge eingemengt sind, das aus Olivin, Chromeisenerz, einer schwarzen noch zu bestimmenden Substanz, sowie aus Nickeleisen und Magnetkies besteht.

2. Howardit zu Ehren Howards benannt, dem wir die erste Analyse eines Meteorsteins verdanken; ein feinkörniges Gemenge von Olivin mit einem weißen Silicat, möglicher Weise Anorthit, und mit einer geringen Menge von Chromeisenerz und Nickeleisen. Sie enthält die Meteorsteine von Loutolax, Bialystock und Mässing u. s. w.

3. Chassignit, von Chassigny, dem Fallorte des einzigen bekannten Meteoriten dieser Art; ein kleinkörniger eisenreicher Olivin mit sparsam eingemengten kleinen Körnern von Chromeisenerz.

4. Chladnit nach Chladni benannt; ein Gemenge von Shepardit (Mg$_{2}$Si$_{3}$) mit einem noch näher zu bestimmenden Tonerde-haltigen Silicate mit geringen Mengen von Nickeleisen, Magnetkies und einigen anderen noch zu bestimmenden Substanzen. Hierher gehört auch nur ein Meteorit, der von Bishopville.\footnote{Mit Chladnit hat zwar Shepard, der diesen Meteorit zuerst untersucht und beschrieben hat, schon das in ihm vorkommende Magnesia-Silicat bezeichnet, doch schien es mir zweckmäßiger, nach Chladni, der sich um die Meteoritenkunde so viele Verdienste erworben hat, einen Meteoriten, als ein Mineral zu benennen, wenn sich dieses auch bis jetzt nur in einem Meteoriten gefunden hat. Ich möchte dann weiter vorschlagen, den bisherigen Chladnit: Shepardit zu nennen, der zwar auch schon einer in diesen Meteoriten sparsam vorkommenden Substanz gegeben ist, die nach Shepard Schwefelchrom ist, die indessen doch in ihren Eigenschaften noch erst sehr wenig gekannt ist; Vorschläge, mit denen Hr. Shepard selbst sich einverstanden gegen mich erklärt hat.}

5. Shalkit, der Meteorstein von Shalka, ein körniges Gemenge von vorwaltendem Olivin mit Shepardit und Chromeisenerz.

6. Die kohligen Meteorit, wie von Bokkeveld und Alais, die ich nicht genauer untersucht habe und für die ich daher noch einen eigenen Nannen aus Setze.

7. Eukrit von $\varepsilon\nu\kappa\varrho\iota\tau$o$\varsigma$ wohl bestimmbar, da die mineralogische Beschaffenheit dieser Art bis auf einige Nebendinge ganz klar ist, und ihre wesentlichen Gemengteil vollkommen bestimmbar sind. Ein hauptsächlich aus Augit und Anorthit bestehendes körniges Gemenge mit einer geringen Menge Magnetkies, meistens noch geringerer Menge Nickeleisen, zuweilen mit kleinen, näher zu bestimmenden tafelartigen Kristallen (Juvenas) und mit etwas Olivin (Petersburg, V. St.). Es gehören hierhin die Meteoriten von Juvenas, Stannern, Jonzac und Petersburg.
\section{Eisenmeteorit.}
\paragraph{}
Die ersten bestimmten Angaben über die Natur des Meteoreisens haben wir von Howard im Jahre 1802 erhalten, der bei Gelegenheit der chemischen Analyse des 1798 bei Benares in Bengalen gefallenen Meteorsteins die merkwürdige Entdeckung machte, dass das in demselben eingesprengte Eisen Nickel enthalte.\footnote{Philosophical transactions von 1802 und daraus in Gilberts Annalen.} Er fand darin 35 pC. und einen ähnlichen Gehalt in andern gediegenen Eisenmassen, die für meteorisch angesehen wurden, in dem Eisen von Otumpa in Brasilien, von Krasnojarsk in Sibirien (dem Pallas-Eisen), dem Eisen aus Böhmen und vom Senegal.

Klaproth bestätigte später den Nickelgehalt bei der Untersuchung des Meteoreisens von Agram und Ellbogen, wenngleich er die Menge darin weit geringer fand (nur 3,5 und 2,5 pC.), und man war nun seit der Zeit gewohnt, den Nickelgehalt als ein charakteristisches Kennzeichen des Meteoreisens zu betrachten, und bei zufällig auf der Oberfläche der Erde gefundenen Eisenmassen ihren meteorischen Ursprung erst dann anzunehmen, wenn die chemische Untersuchung einen Gehalt an Nickel nachgewiesen hatte, eine Annahme, wozu man auch jetzt noch berechtigt ist, da man noch nie ein Meteoreisen ohne Nickelgehalt gefunden hat. Tellurisches gediegenes Eisen enthält keinen Nickel, ist überhaupt eine außerordentliche Seltenheit und wohl zufällig nur durch einen Reduktionsprozess entstanden.

Später fand Stromeyer in dem Meteoreisen neben dem Nickel etwas Kobalt, das ja auch in den tellurischen Mineralien so häufig das Nickel zu begleiten pflegt und darauf auch etwas Kupfer, und Laugier in dem Meteoreisen von Krasnojarsk und Brahin etwas Chrom, 0,5 pC., das aber wie das in den Meteorsteinen vorkommende, wo es Laugier schon früher gefunden hatte, von eingemengtem Chromeisenerz herrührt.

Am meisten bereichert wurde unsere Kenntnis von der chemischen Beschaffenheit des Meteoreisens durch die genauen Analysen derselben von Berzelius, die er zuerst auf Veranlassung des Grafen Caspar Sternberg mit dem in Böhmen aufgefundenen Eisen von Bohumilitz\footnote{Pongendorffs Annalen von 1833 B. 27, S. 118.} und dann bei seiner großen Arbeit über die Meteoriten überhaupt, die durch die Übersendung der 1833 bei Blansko in Mähren gefallenen Meteorsteine durch Baron von Reichenbach veranlasst wurde, mit dem Meteoreisen von Krasnojarsk (dem Pallas-Eisen) und dem von Elbogen\footnote{A. a. O. von 1834 B. 33, S. 123.} angestellt hatte. Er schied durch Behandlung des Meteoreisens mit verdünnter Salpetersäure einen darin löslichen und einen anderen darin unlöslichen Teil, und fand auf diese Weise bei dem Eisen von Bohumilitz (a), Krasnojarsk (b), und Elbogen (c):

 | \emph{a} | \emph{b} | \emph{c}  
Eisen | 93,775 | 88,042 | 88,231  
Nickel | 3,812 | 10,732 | 7,517  
Kobalt | 0,213 | 0,455 | 0,762  
Magnesium | - | 0,050 | 0,279  
Mangan | - | 0,132 | Spur  
Zinn und Kupfer | - | 0,066 | Spur  
Kohle | - | 0,043 | -  
Schwefel | - | Spur | Spur  
Unlösliches | 2,200 | 0,480 | 2,211  
 | 100,000 | 100,000 | 100,000  

Der in verdünnter Salpetersäure unlösliche Rückstand bestand aus metallischen Körnern und Schüppchen, die schwer zu Boden liegen, und aus einer feinen verteilten schwarzen kohleähnlichen Masse, die sich leicht in der Flüssigkeit aufschlämmen lässt. Die ersteren waren merkwürdiger Weise Phosphormetalle und bestanden aus:

Eisen | 65,987 | 48,67 | 68,11  
Nickel | 15,008 | 18,33 | 17,72  
Magnesium | – | 9,66 | 17,72  
Phosphor | 14,023 | 18,47 | 14,17  
Kiesel | 2,037 | - | -  
Kohle | 1,422 | - | -  
98,477 | 95,13 | 100,00  

Die letztere, die beim Erhitzen rauchte und sodann verglimmt, wurde nur bei dem Pallas-Eisen quantitativ untersucht und bestand hier aus:

Eisen 57,18
Nickel 34,00
Magnesium 4,52
Zinn und Kupfer 3,75
Kohle 0,55

Eine Spur von Phosphor, die Berzelius fand, glaubt er umschlossenen Teilen der Phosphorverbindung zu schreiben zu müssen. Bei dem Rückstand aus dem Bohumilitz-Eisen wurden auch noch etwas Kiesel und Chromeisenerz gefunden. Dieser feinere Teil des Rückstands ist daher von dem schwereren wesentlich verschieden zusammengesetzt.

Durch Berzelius wurden also in dem Meteoreisen 6 neue Stoffe aufgefunden: Phosphor, Zinn, Mangan, Magnesium, Kiesel und Kohle, von denen der Phosphor ganz besonders bemerkenswert ist, da solche Phosphormetalle, wie sie in dem Meteoreisen hiernach enthalten sind, unter den tellurischen Mineralien nicht bekannt sind.

Nach Berzelius wurden nun noch Analysen von anderen Meteoreisenmassen und von andern Chemikern nach denselben oder ähnlichen Methoden gemacht, die aber ganz ähnliche Resultate gegeben haben.\footnote{Vergl. die Aufzählung derselben in Rammelsbergs Mineralchemie, S. 902.} In allen wurde ein in verdünnter Säure unlöslicher, hauptsächlich aus Phosphornickeleisen bestehender Rückstand erhalten, derselbe war wie bei Berzelius stets nur in sehr geringer und in sehr veränderlicher Menge enthalten, und außerdem waren die Verhältnisse von Phosphor gegen Eisen und Nickel so verschieden, dass sich eine gemeinschaftliche Formel für die chemische Zusammensetzung dieser Verbindung nicht aufstellen lässt.\footnote{Es geht dies aus den Berechnungen von Rammelsberg hervor, wonach bei den verschiedenen Analysen auf 1 Atom Phosphor 2, 3 1/2, 5, 6, 8, 14, 15, 18, 30 Atome Metall kommen. (A. a. O. S. 948) Es scheint aber nicht, dass man die beiden Arten des Rückstandes, die Berzelius wohl unterschieden, getrennt hat.} Diesem unlöslichen Rückstand hat Haidinger bei Gelegenheit der von Patera ausgeführten Analyse des Meteoreisens von Arva, worin derselbe in verhältnismäßig großer Menge enthalten ist, den Namen Schreibersit gegeben\footnote{Österreich. Blätter für Lit. 1847, N. 175, S. 644 und N. Jahrbuch für Min. von 1848, S. 698.} zu Ehren des früheren Direktors des kaiserlichen Mineralienkabinetts in Wien, der sich um die Meteoritenkunde durch die Herausgabe seiner Beiträge zur Geschichte und Kenntnis meteorischer Stein- und Metallmassen verdient gemacht hat.

Aber schon viel früher, als Berzelius durch seine chemischen Untersuchungen die gemengte Beschaffenheit des Meteoreisens dartat, hatte sie von Widmanstätten in Wien auf eine andere Weise bewiesen. Derselbe zeigte nämlich schon 1808, dass wenn man an dem Meteoreisen angeschliffene und polierte Flächen mit einer schwachen Säure ätzt, gewisse Figuren hervortreten, die man seitdem die Widmanstättenschen Figuren genannt hat. Die Fläche, die vor dem Ätzen ganz gleichartig aussieht, oder nur bei höchster Politur und nach dem Anhauchen schwache Andeutungen der Figuren gibt, erscheint nun überall mit schmalen, glanzlosen, unter einander parallelen Streifen bedeckt, die nach verschiedenen Richtungen gehend, sich unter verschiedenen schiefen Winkeln durchneiden, von dünnen, hervortretenden, metallisch glänzenden Leisten eingefasst werden und dunklere matte Felder einschließen, was alles eine sehr komplizierte Struktur des Meteoreisens anzeigt. Besonders schön fielen diese Figuren auf dem großen Stücke aus, das 1812 von der Elbogener Eisenmasse abgeschnitten und nach Wien gebracht war. Da die schmalen einfassenden Leisten, die wenig oder gar nicht von der verdünnten Säure angegriffen werden, bei der Ätzung aus der übrigen Masse etwas hervortreten, so kam Widmanstätten auf die Idee, die geätzten Eisenmassen wie einen Schriftsatz in der Buchdruckerpresse abdrucken zu lassen, was auch vollkommen gelang. Er konnte dadurch vollkommen naturgetreue Abbildungen liefern, wie sie die Kunst nicht darzustellen vermag. V. Schreibers beschrieb in dem eben genannten Werke\footnote{Beiträge etc. S. 70, vergl. auch Partsch Meteoriten, S. 100.} die einzelnen Teile des Meteoreisens, die Streifen, Einfassungsleisten und Zwischenfelder, und gab auch einen Abdruck von der geätzten Fläche der großen Elbogener Masse des Wiener Mineralien-Kabinetts; die tieferen Stellen sind darin weiß und nur die höheren Stellen, die Leisten, schwarz, zum Teil auch die Zwischenfelder, die oft wieder gestreift erscheinen. Auch von andern Eisenmassen ließ er Abdrücke machen, die später herausgegeben werden sollten, wozu es aber nicht gekommen ist, und die dann nur an einzelne Personen verteilt wurden. Nach v. Schreibers wurden dergleichen Abdrücke nun auch von andern hergestellt, von Partsch in seine Werke die Meteoriten, von Haidinger in den Sitzungsberichten der Wiener Akademie, von mir selbst in Pongendorffs Annalen u. s. w. Außerordentlich schön sind die Abdrücke in den neusten Abhandlungen von Haidinger, die die Figuren des Meteoreisens von Sarepta und Arva darstellen.\footnote{Sitzungsberichte der kaiserl. Akad. d. Wiss. vom 24. Juli 1862.}

Indessen geben nicht alle Eisenmeteorit Widmanstättenschen Figuren, und namentlich ist dies der Fall bei der im Jahre 1847 bei Braunau gefallenen Eisenmasse, bei welcher Haidinger\footnote{Berichte der Versammlungen der Freunde der Naturwissenschaften in Wien, 1847 und daraus in Pongendorffs Ann. 1847 B. 72, S. 580.} die merkwürdige Entdeckung machte, dass sie in ihrer ganzen Masse nach denselben drei untereinander rechtwinkligen Richtungen parallel den Flächen des Hexaeders spaltbar sei. Es waren zwei Massen gefallen, die beide in die Hände des Prälaten vom Kloster zu Braunau Hrn. Rotter gelangten, der die größere, 42 Pfd. 6 Lth. schwere Masse zerschneiden ließ und einzelne Stücke davon den verschiedenen Museen als Geschenk übersandte.\footnote{Auch das Berliner Museum erhielt auf diese Weise ein ausgezeichnet schönes Stück, 2 Pfund 21,3 Loth schwer.} An dem Stücke, welches das k. Mineralien-Kabinett in Wien erhielt, machte Haidinger die obige Beobachtung. Es war wie die übrigen zum Teil durchschnitten und die weitere Trennung durch Zerreißung hervorgebracht, so dass also stellenweise der natürliche Bruch sichtbar war. Da die Spaltungsflächen auf der ganzen Bruchfläche und demnach auch wahrscheinlich durch das ganze Stück in gleicher Richtung fortgehen, so ist das ganze Stück und so auch die ganze Masse, von der es abgeschnitten, ein Stück eines Individuums, eines Krystalls, dessen äußere Form nicht mehr wahrgenommen werden kann, weil er beim Durchzuge durch die Luft zerplatzt und die einzelnen Stücke an der Oberfläche abgeschmolzen sind, dessen innere Struktur in den Stücken aber erhalten ist. Haidinger ließ das erhaltene Stück anschleifen und ätzen; es entstanden nun keine Widmanstättenschen Figuren, aber andere gerade und untereinander parallele Linien nach mehreren Richtungen wurden sichtbar, die nachher von Neumann\footnote{Naturwissenschaftliche Abhandl. gesammelt von Haidinger 1849 B. 3, Abt. 2, S. 45.} ihrer Richtung nach sorgfältig beschrieben und gedeutet wurden, worauf ich später zurückkommen werde.

Sehr wichtige und interessante Untersuchungen über die Struktur des Meteoreisens hat nun in der letzten Zeit der Baron von Reichenbach,\footnote{Pongendorffs Ann. 1861 B. 114, S. 99, 250, 264,477.} gemacht. Er unterscheidet bei den Eisenmeteoriten, die Widmanstättenschen Figuren geben, vier Gemengteil, die durch die Ätzung einer angeschliffenen Fläche sichtbar werden und die er mit dem Namen Balkeneisen oder Kamazit, Bandeisen oder Tänit, Fülleisen oder Plessit und Glanzeisen oder Lamprit bezeichnet. Das Balkeneisen bildet auf der geätzten Fläche die unter einander parallelen Streifen, die sich unter schiefen Winkeln (von 30, 60 und 120 Graden) durchschneiden und nimmt somit den größten Raum ein; es wird durch Ätzung grau und glanzlos und zeigt sich nun mit einer Menge unter einander paralleler Linien nach Art des Braunauer Eisens bedeckt, die Reichenbach Schraffirungslinien nennt und für Andeutungen von Spaltungsflächen hält; in vielen Fällen erscheint es aber selbst wieder körnig, wie namentlich in dem Eisen von Ruffs mountain. Das Bandeisen fasst die Streifen des Balkeneisens ein und bedeckt sie in papierdünnen Blättern zu beiden Seiten; es wird von der verdünnten Säure schwach rötlichgelb gefärbt, sonst wenig oder gar nicht angegriffen, und ragt daher auf der geätzten Fläche über dem Balkeneisen leistenartig etwas hervor. Das Fülleisen erfüllt die drei oder vierseitigen Felder, die von dem Balkeneisen eingeschlossen werden; es wird von der Ätzung wie das Balkeneisen angegriffen und erhält dabei eine noch dunkle graue Farbe, wie dieses. Es ist in manchen Abänderungen wie in dem Eisen von Ruffs mountain gar nicht vorhanden, füllt auch häufig die Felder nicht allein aus, sondern enthält oft noch eine große Menge Blättchen von Bandeisen, die in untereinander paralleler Richtung enge nebeneinander und bei vierseitigen Feldern gewöhnlich zwei parallelen Seiten, oft aber auch zum Teil den beiden andern parallel liegen, in welchem letzteren Fall die Blätter in einer Diagonale des Vierecks aneinandergrenzen. Reichenbach nennt diese die Zwischenfelder ausfüllenden Blätter des Bandeisens Kämme. Das Glanzeisen liegt in einzelnen länglichen Körnern und Streifen in der Mitte des Balkeneisens; es wird durch die verdünnte Säure gar nicht angegriffen und behält den vollen Glanz und die lichte, stahlgraue fast zinnweiße Farbe, die es durch die Politur der Fläche erhalten hat. Es findet sich nicht in allen Eisenmeteoriten, sehr ausgezeichnet in dem von Lenarto und Arva.

Reichenbach prüfte diese 4 Eisenarten noch weiter nach einer Methode, die schon Widmanstätten angewandt hatte, durch das Anlaufen in der Hitze. Er zeigte, dass das Balkeneisen zuerst anläuft, dann das Fülleisen und zuletzt das Band- und Glanzeisen. Da nun auch das erstere von der Säure am leichtesten, die letzteren am schwersten angegriffen werden, so sieht man, dass die Wirkungen der Hitze und der Säure gleichen Schritt halten, wie denn auch beide Erscheinungen auf stärkerer oder schwächerer Verwandtschaft zum Sauerstoff beruhen. Bei einer Hitze, bei welcher das Balkeneisen schon dunkelblau geworden ist, erscheint das Fülleisenbläulichrot und das Bandeisen goldgelb. Stahl läuft aber bekanntlich bei 230° C. gelb, bei 263° purpurrot, bei 290° blau an. Die Hitze also, die das Balkeneisen schon blau macht, färbt erst das Fülleisen purpurrot und das Bandeisen goldgelb.

Eine vollständige Trennung sämtlicher Gemengteil für die chemische Untersuchung konnte Reichenbach nicht bewerkstelligen, doch glückte es ihm wenigstens einigermaßen für einen derselben, für das Bandeisen. Manche dieser Eisenmeteoriten, wie namentlich der von Cosby Creek, die vor ihrer Auffindung vielleicht lange Zeit in der feuchten Erde gelegen haben, sind nämlich an der Oberfläche sehr stark oxydiert und zerteilen sich hier parallel den Blättern des Bandeisens in Platten, welche Zerteilung durch leises Hämmern noch vollständiger bewirkt werden kann. Die oxydierten Platten des Balkeneisens sind aber hier mit papierdünnen Blättern des Bandeisens bedeckt, die sich nun mit Leichtigkeit von dem Balkeneisen ablösen und so in hinreichender Menge zur Analyse gewinnen lassen.\footnote{Diese Blättchen von Bandeisen können so zuweilen von bedeutender Größe erhalten werden; so beschreibt Reichenbach ein Stück von dem Cosby-Eisen in seiner Sammlung das mit einem Blatte Bandeisen bedeckt ist, das eine Länge von 3 Zoll bei einer Breite von 2 Zoll hat.} Reichenbach verfuhr so mit dem Eisen von Cosby; das gesammelte Bandeisen untersuchte er zuerst in Rücksicht des spezifischen Gewichtes, er fand dasselbe 7,428, etwas größer als das spezifische Gewicht der ganzen Masse, das 7,260 beträgt, es wurde sodann von seinem Sohne Reinold v. Reichenbach analysiert, der zugleich auch eine Analyse der ganzen Masse machte. Er fand\footnote{Vergl. Pongendorffs Ann. 1861 B. 114, S. 258.} in dem Bandeisen (a) und in der ganzen Masse nach 2 Analysen (b) und (c):

a | b | c  
Eisen | 85,714 | 90,125 | 89,324  
Nickel | 13,215 | 9,786 | 10,123  
Kobalt | 0,550 | 9,786 | 0,422  
Schwefel | 0,226 | Spur | Spur  
Phosphor | 0,295 | 0,089 | 0,131  
 | 100 | 100 | 100  

Die Analyse gab also in dem Bandeisen einen etwas größer Nickelgehalt, auch etwas mehr Schwefel und Phosphor und dafür weniger Eisen als in der ganzen Masse, welches Verhältnis gegen die übrigen Gemengteil sich noch etwas größer stellen würde, wenn man bei der Analyse der ganzen Masse das Bandeisen hätte entfernen können. Reichenbach glaubte indessen durch diese Analyse noch keine völlige Aufklärung über die chemische Beschaffenheit des Bandeisens erhalten zu haben, da er bei der Besichtigung desselben unter dem Mikroskop fand, dass noch eine Menge anders gearteter Körperchen in demselben eingelagert waren.

In den Eisenmeteoriten, die keine Widmanstättenschen Figuren geben, hat nach Reichenbach die eine oder die andere dieser Eisenarten überhandgenommen und die andern kommen dann nur ganz unregelmäßig und untergeordnet und zum Teil auch gar nicht darin vor. So besteht das Eisen von Braunau fast nur aus Balkeneisen, und das Eisen vom Cap der guten Hoffnung und von Rasgatà ist Reichenbach geneigt, als fast ganz aus Fülleisen bestehend anzunehmen.

Die meisten Abänderungen des Meteoreisens enthalten aber nun noch einen andern Gemengteil, feine nadelförmige Krystalle oder Nadeln, wie sie Reichenbach kurzweg nennt. Wöhler\footnote{Annalen der Chem. u. Pharm. B. 81, S. 254.} beobachtete sie zuerst bei dem Meteoreisen von einem unbekannten Fundort\footnote{Reichenbach hält dies Meteoreisen für das von Santa Rosa in Columbien, da es aber Widmanstättenschen Figuren gibt, stimmt es wenigstens nicht mit dem überein, welches Boussingault von dort mitgebracht und an v. Humboldt geschenkt hat.} sowohl auf dessen polierter und geätzter Fläche, als auch in dem Rückstande bei seiner Behandlung mit verdünnter Salpetersäure, wo sie unter dem Mikroskop kenntlich wurden. Reichenbach zeigte,\footnote{Vergl. Poggendorffs 1862, B. 115, S. 148.} dass sie in den meisten Eisenmeteoriten enthalten sind und bei der Ätzung einer polierten Fläche derselben zum Vorschein kommen, wobei sie einen ausgezeichneten Parallelismus durch die ganze Masse zeigen. Ihre Länge überschreitet selten 2 Linien. Reichenbach hält sie für eine vollkommenere Ausbildung des Bandeisens.\footnote{Zu diesen Einmengungen würden auch noch die kleinen Eisenkügelchen zu zählen sein, die ich zuerst und dann ausführlich Reichenbach beschrieben (Pongendorffs Ann. 1861 B. 113, S. 187 und B. 115, S. 152), und die auch auf der geschliffenen Fläche in ihren Durchschnitten sichtbar werden sollen. Die Annahme von solchen Kügelchen beruht aber, wie ich mich jetzt überzeugt habe, auf einem Irrtum. Die angeblichen runden Kugeln sind nichts anderes als Stellen, die beim Ätzen durch eine ansitzende Luftblase vor dem Angriff der Säure geschützt waren. Die Luftblase bildete sich durch die Art, wie ich das Meteoreisen ätzte und die darin bestand, dass ich dasselbe mit der polierten wohl gereinigten Fläche in die verdünnte Säure tauchte und darin eine halbe bis eine ganze Minute hielt, wobei dann öfter eine Luftblase an der Fläche sitzen blieb, die den Angriff der Säure verhinderte. Wenn man die Fläche vorher mit Wasser nass macht oder die Fläche nicht mit einem Male unter die Oberfläche der Säure bringt, sondern erst mit einer Seite und sie dann mehr und mehr neigt, bis sie ganz in Wasser eingetaucht ist, so bleiben keine Luftblasen hängen. Daher kommt es, dass, wie Reichenbach erwähnt, er die Eisenkügelchen nur bei den Stücken des Berliner Museums und nicht in den Stücken seiner eigenen Sammlung gesehen hatte.}

Außer den genannten, vorzugsweise aus metallischem Eisen bestehenden Einmengungen kommen in den Eisenmeteoriten noch andere, teils gröbere teils feinere, mehr oder weniger häufig vor. Zu den ersteren gehören Schwefeleisen, Grafit und besonders Olivin.

Das Schwefeleisen ist nach den Untersuchungen von Smith und Rammelsberg kein Magnetkies, wie man bisher angenommen hatte, sondern einfach Schwefeleisen, FeS, Troilit, wie es Haidinger zu nennen vorgeschlagen hat\footnote{Zur Erinnerung an den Berichterstatter des Meteoritenfalls von Albareto bei Modena 1766, Domenico Troilit, der schon lange vor Chladni die Tatsächlichkeit der Meteoritenfälle bewies, freilich ohne seiner Meinung Geltung verschaffen zu können. Vergl. Sitzungsbericht d. k. Akad. der Wiss. März 1863.}; also eine Verbindung, die unter den Mineralien der Erde bisher noch nicht bekannt ist. Es besteht nach Rammelsberg\footnote{Monatsber. d. k. Pr. Akad. d. Wiss. 1864, S. 29.} in zwei Abänderungen aus dem Eisen von Seeläsgen (a) und Sevier County (b) und nach der Berechnung nach der Formel (c) aus:

 | a | b | c  
Eisen | 63,41 | 62,22 | 63,64  
Nickel | - | 1,76 | -  
Mangan | 0,64 | - | -  
Schwefel | 35,91 | 36,01 | 36,36  
 | 99,96 | 99,99 | 100,0  

Es enthält also bald Nickel (Schwefelnickel) bald ist es, obgleich mitten in dem nickelhaltigen Meteoreisen vorkommend, davon frei, wie der Olivin in dem Pallas-Eisen (s. weiter unten), doch scheint das erstere häufiger zu sein, da auch Smith in dem Troilit aus dem Meteoreisen von Tazewell etwas Nickel angibt. 

Der Troilit ist bis jetzt in dem Meteoreisen nur derb vorgekommen, in mehr oder weniger großen Körnern und unregelmäßigen Massen, zuweilen in der Form von Zylindern von mehr als 1 Zoll Größe und mehreren Linien Dicke, wie Reichenbach beobachtet hat. Er ist im Bruch uneben, zeigt aber öfter dünnschalige Zusammensetzungsstücke, wie dies öfter beim Magnetkies, z. B. von Bodenmais vorkommt; tombakbraun, metallisch glänzend, spezifisches Gewicht 4,787 (Seeläsgen), 4,817 (Sevier County). Diess hohe spezifische Gewicht unterscheidet ihn von dem Magnetkiese, dessen Gewicht nicht über 4,623 hinauszieht. Ebenso unterscheidet er sich durch sein Verhalten gegen Chlorwasserstoffsäure, indem er sich darin ohne einen Rückstand von Schwefel auflöst. Nickeleisen, Chromeisenerz und Grafit kommen öfter in ihm eingemengt vor, doch ist er zuweilen auch davon ganz frei, wie die Analyse des Troilit von Seeläsgen durch Rammelsberg beweist.

Wenn es so erwiesen ist, dass einfach Schwefeleisen in dem Meteoreisen vorkommt, so bleibt es doch noch auszumachen übrig, ob alles Schwefeleisen in demselben von derselben Art sei oder ob neben diesen nicht auch Magnetkies vorkommt. In den Steinmeteoriten ist, wie bekannt, das Vorkommen dieses letzter nicht zweifelhaft, da, wenn darüber auch noch keine chemischen Untersuchungen angestellt sind, das Schwefeleisen in dem Meteorstein von Juvenas kristallisiert vorkommt, und an der Krystallform als Magnetkies erkannt werden kann. Ist es daher möglich, dass dieser auch in den Eisenmeteoriten vorkommt, so ist dies doch noch nicht erwiesen, wie auf der anderen Seite auch das Vorkommen von einfach Schwefeleisen in den Steinmeteoriten nicht bewiesen ist; ich werde daher bis auf Weiteres das Schwefeleisen der Eisenmeteorit, auch wo es noch nicht untersucht ist, als Troilit und das der Steinmeteorit als Magnetkies aufführen. Der Grafit findet sich in kleinen abgerundeten, im Innern aus dicht zusammengehäuften Schüppchen bestehenden Parthien bis zu der Größe einer Haselnuss oder Walnuss, zuweilen aber auch, wie Haidinger bei dem Eisen von Arva beobachtete\footnote{Pongendorffs Ann. 1846 B. 67, S. 437.} in Pseudomorphosen.\footnote{Haidinger glaubte darin die Form einer Kombination des Hexaeders mit dem Pentagondodekaeder zu erkennen und nimmt daher an, dass die Pseudomorphosen aus Eisenkies entstanden wären, eine Ansicht, die ich jedoch nicht teilen möchte, da Eisenkies mit Sicherheit in den Meteoriten bis jetzt nicht beobachtet ist, und die Pseudomorphosen selbst, die Hr. Haidinger die Güte hatte, mir zur Ansicht zu schicken, mir mehr die Form eines Hexaeders mit zu geschärften als mit schief abgestumpften Kanten zu haben schienen. Man kann nun aber fragen, woraus die Pseudomorphosen dann entstanden wären? Am nächsten liegt hier nun wohl die Annahme, dass dies der Diamant gewesen sei; wenn aber auch diese Annahme durch die Form der Pseudomorphose und die Möglichkeit der Bildung gerechtfertigt wird, so findet sie doch darin eine große Schwierigkeit, dass eben Diamanten in den Meteoriten bisher noch nicht beobachtet sind.} Der Olivin findet sich in einzelnen abgerundeten Körnern in manchen Abänderungen in großer Menge, wie namentlich in dem Meteoreisen, das Pallas 1776 am Jenisei im östlichen Sibirien gefunden hatte. Biot\footnote{Bulletin des sciences, par la soc. philomatique. 1820 p. 89.} schloss aus dem optischen Verhalten der Olivin-Körner, dass dieselben wirkliche Krystalle wären, und ich beobachtete,\footnote{Pongendorffs Ann. 1825 B. 4, S. 186.} dass die Körner, wie wohl meistenteils, ganz rund, wo sie frei im Eisen liegen, doch schon einzelne sehr glänzende Flächen und in seltenen Fällen sogar in großer Menge enthalten. Außerdem Pallas-Eisen enthalten auch noch das Meteoreisen von Brahin (Gouv. Minsk) und von der Wüste Atacama und andere solche Krystalle.

Wo aber diese Einschlüsse vorkommen, sind sie stets, wie Reichenbach hervorhob, von einer Hülle von Balkeneisen umgeben, und wenn sie sich in einem Meteoriten, der Widmanstättenschen Figuren gibt, finden, so fangen diese immer erst in einer gewissen Entfernung, die 1 bis 3 Linien beträgt, an, sich regelmäßig zu entwickeln. Dies zeigt sich besonders schön bei den Olivin-Einschlüssen. Sind sie in großer Menge vorhanden, wie in Pallas-Eisen und in dem Eisen von Brahin und Atacama, so dass sie oft nur wenig Raum zwischen sich lassen, so wird dieser von dem Balken-, Band- und Fülleisen meistenteils ganz ausgefüllt, und zwar so, dass zuerst an dem Olivin sich eine dünne Lage von Balkeneisen anlegt, dann eine viel dünnere Lage von dem Bandeisen folgt und zuletzt das Fülleisen den inneren Raum einnimmt, wie man dies auf einer durch ein solches Meteoreisen gelegten Schnittfläche, die man geätzt hat, sehr gut sehen kann. Sind die Räume zwischen den Olivinkristallen größer, so bilden sich in dem Fülleisen die Widmanstättenschen Figuren; bei den genannten Meteoriten sieht man jedoch diese nur selten, aber bei dem Eisen von Steinbach und Rittersgrün, wo die Olivine kleiner sind und die Eisenmasse zwischen ihnen größer ist, sind auch die Widmanstättenschen Figuren größer, umso mehr als auch nun die Einfassung des Olivins durch das Balkeneisen schmaler ist.

Die Widmanstättenschen Figuren waren früher in den Olivin-haltigen Eisenmeteoriten ganz übersehen, bis sie Partsch in dem Eisen von Steinbach entdeckte,\footnote{Die Meteoriten, S. 91.} der dadurch auf den gleichen Ursprung von vielen Stücken Meteoreisen, die in den verschiedenen Sammlungen mit der Angabe von verschiedenen Fundörtern aufgeführt waren, schloss. Sie wurden nachher auch von Reichenbach beschrieben.

Zu den feiner Einmengungen, die sich in den Eisenmeteoriten finden, gehören eine Menge kleiner mikroskopischer, meist farbloser, doch auch farbiger glänzender Steinchen von mehr als Quarzhärte, die Wöhler außer dem Schreibersit als Rückstand bei der Auflösung des Eisens von Rasgatà erhielt, jedoch nicht weiter untersuchte und einige kleine Quarzkrystalle, die ich in dem Eisen von Toluca beobachtete\footnote{Pongendorffs Ann. 1861 B. 113, S. 184.} und die noch so groß und glänzend waren, dass ich ihre Winkel mit Genauigkeit bestimmen konnte. Sie steckten zwar nur in der äußern oxydierten Rinde, doch so, dass man nicht daran zweifeln konnte, dass sie sich in dem Meteoreisen gebildet hatten und nicht erst später hineingekommen waren.

Alle diese Eisenmeteorit, die man zufällig auf der Oberfläche der Erde findet, sind mit solcher Rinde von Eisenoxydhydrat umgeben, die sich erst durch Oxydation gebildet hat. Diese Oxydation geht aber in der Regel nicht weit, und die so entstandene Rinde schützt die Eisenmeteorit vor ihrer Zerstörung und bewirkt, dass sie sich Jahrtausende weiter unversehrt erhalten. Sie ist die Ursache, dass die Eisenmeteorit, die nur so selten fallen, dass man nur die Fallzeit von dreien kennt, doch häufig gefunden werden, so dass man jetzt in den Sammlungen mehr als halb so viel Eisen- wie Steinmeteoriten hat. Die Eisenmeteoriten, welche man hat fallen sehen und unmittelbar nach ihrem Falle gesammelt hat, wie die von Agram und Braunau, haben keine solche Rinde von Brauneisenerz; die rundlichen Erhabenheit und Vertiefungen, die sich überall an der Oberfläche derselben finden, sind, wie Reichenbach gezeigt hat, mit einem dünnen Überzuge von Magneteisenerz bedeckt, der sich bei dem Durchzuge durch die Luft an der Oberfläche durch Schmelzung und Oxydation bildet,\footnote{Pongendorffs Ann. 1858 B. 103, S. 637. Reichenbach spricht hier von Eisenoxydul, was wohl nur heißen soll Oxydoxydul oder Magneteisenerz.} und so äußerst dünn ist, weil das gebildete Magneteisenerz beim Schmelzen abtropft und nur das wenigste durch Adhäsion haften bleibt. Unter diesem Überzuge findet sich dann eine 1 bis 1 Linien dicke Lage, in welcher das Eisen ganz körnig geworden ist, wie auch schon Reichenbach bei dem Eisen von Braunau beobachtet hat\footnote{A. a. O. 1862 B. 115, S. 135.} und auch in den Abdrücken dieses Eisens von Haidinger zu sehen ist,\footnote{Sitzungsberichte der math.-naturw. Classe d. k. Akad. der Wissenschaft. 1855 B. 15, S. 354, Fig. 5.} was beweist, dass das Eisen vor der Oxydation seinen Aggregatzustand ändert. Dass sich auf der Oberfläche dieser Eisenmassen beim Liegen in und auf der feuchten Erde nicht bloß Eisenoxydhydrat, sondern auch Magneteisenerz bildet, hat Krantz gezeigt,\footnote{Pongendorffs Ann. 1857, B. 101, S. 152.} der auf der Oberfläche des Eisens von Toluca oktaedrische Krystalle dieser Substanz beobachtet hat.
\clearpage
\section{}
\section{Abbildungen}
\clearpage
\setlength\intextsep{0pt}
\pagestyle{fancy}
\fancyhf{}
\rhead{Tafel 1}
\cfoot{\thepage}
\begin{figure}[p]
\includegraphics[scale=0.5,keepaspectratio]{Figures/Table1/Fig1.png}\tiny 1
\includegraphics[scale=0.5,keepaspectratio]{Figures/Table1/Fig2.png}\tiny 2
\includegraphics[scale=0.5,keepaspectratio]{Figures/Table1/Fig3.png}\tiny 3
\includegraphics[scale=0.5,keepaspectratio]{Figures/Table1/Fig5.png}\tiny 5
\tiny   6
\includegraphics[scale=0.5,keepaspectratio]{Figures/Table1/Fig6.png}
\includegraphics[scale=0.5,keepaspectratio]{Figures/Table1/Fig8.png}\tiny 8
\includegraphics[scale=0.5,keepaspectratio]{Figures/Table1/Fig9.png}\tiny 9
\includegraphics[scale=0.5,keepaspectratio]{Figures/Table1/Fig10.png}\tiny 10
\includegraphics[scale=0.5,keepaspectratio]{Figures/Table1/Fig4.png}\tiny 4
\includegraphics[scale=0.5,keepaspectratio]{Figures/Table1/Fig7.png}\tiny 7
\includegraphics[scale=0.5,keepaspectratio]{Figures/Table1/Fig11.png}\tiny 11
\includegraphics[scale=0.5,keepaspectratio]{Figures/Table1/Fig12.png}\tiny 12
\end{figure}
\clearpage
\rhead{Tafel 2}
\cfoot{\thepage}
\begin{figure}[p]
\tiny 1
\includegraphics[scale=0.5,keepaspectratio]{Figures/Table2/Fig1.png}
\includegraphics[scale=0.5,keepaspectratio]{Figures/Table2/Fig4.png}\tiny 4
\includegraphics[scale=0.5,keepaspectratio]{Figures/Table2/Fig2.png}\tiny 2
\includegraphics[scale=0.5,keepaspectratio]{Figures/Table2/Fig3.png}\tiny 3
\tiny 7
\includegraphics[scale=0.5,keepaspectratio]{Figures/Table2/Fig7.png}
\includegraphics[scale=0.5,keepaspectratio]{Figures/Table2/Fig6.png}\tiny 6
\includegraphics[scale=0.5,keepaspectratio]{Figures/Table2/Fig5.png}\tiny 5
\includegraphics[scale=0.5,keepaspectratio]{Figures/Table2/Fig8.png}\tiny 8
\includegraphics[scale=0.5,keepaspectratio]{Figures/Table2/Fig9.png}\tiny 9
\end{figure}
\clearpage
\rhead{Tafel 3}
\cfoot{\thepage}
\begin{figure}[p]
\includegraphics[scale=0.5,keepaspectratio]{Figures/Table3/Fig1.png}\tiny 1
\includegraphics[scale=0.5,keepaspectratio]{Figures/Table3/Fig2.png}\tiny 2
\includegraphics[scale=0.5,keepaspectratio]{Figures/Table3/Fig3.png}\tiny 3
\includegraphics[scale=0.5,keepaspectratio]{Figures/Table3/Fig7.png}\tiny 7
\includegraphics[scale=0.5,keepaspectratio]{Figures/Table3/Fig4.png}\tiny 4
\includegraphics[scale=0.5,keepaspectratio]{Figures/Table3/Fig5.png}\tiny 5
\includegraphics[scale=0.5,keepaspectratio]{Figures/Table3/Fig6.png}\tiny 6
\includegraphics[scale=0.5,keepaspectratio]{Figures/Table3/Fig8.png}\tiny 8
\includegraphics[scale=0.5,keepaspectratio]{Figures/Table3/Fig9.png}\tiny 9
\includegraphics[scale=0.5,keepaspectratio]{Figures/Table3/Fig10.png}\tiny 10
\includegraphics[scale=0.5,keepaspectratio]{Figures/Table3/Fig11.png}\tiny 11
\includegraphics[scale=0.5,keepaspectratio]{Figures/Table3/Fig12.png}\tiny 12
\end{figure}
\clearpage
\rhead{Tafel 4}
\cfoot{\thepage}
\begin{figure}[p]
\includegraphics[scale=0.5,keepaspectratio]{Figures/Table4/Fig1.png}\tiny 1
\includegraphics[scale=0.5,keepaspectratio]{Figures/Table4/Fig2.png}\tiny 2
\includegraphics[scale=0.5,keepaspectratio]{Figures/Table4/Fig3.png}\tiny 3
\includegraphics[scale=0.5,keepaspectratio]{Figures/Table4/Fig4.png}\tiny 4
\includegraphics[scale=0.5,keepaspectratio]{Figures/Table4/Fig5.png}\tiny 5
\includegraphics[scale=0.5,keepaspectratio]{Figures/Table4/Fig6.png}\tiny 6
\includegraphics[scale=0.5,keepaspectratio]{Figures/Table4/Fig7.png}\tiny 7
\includegraphics[scale=0.5,keepaspectratio]{Figures/Table4/Fig8.png}\tiny 8
\includegraphics[scale=0.5,keepaspectratio]{Figures/Table4/Fig9.png}\tiny 9
\includegraphics[scale=0.5,keepaspectratio]{Figures/Table4/Fig10.png}\tiny 10
\end{figure}
\clearpage
\end{document}
